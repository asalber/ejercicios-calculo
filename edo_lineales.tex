% Version control information:
\svnidlong
{$HeadURL: https://ejercicioscalculo.googlecode.com/svn/trunk/edo_separables.tex $}
{$LastChangedDate: 2010-01-28 20:28:03 +0100 (jue, 28 ene 2010) $}
{$LastChangedRevision: 11 $}
{$LastChangedBy: asalber $}
%\svnid{$Id: edo_separables.tex 11 2010-01-28 19:28:03Z asalber $
%
\newproblem{edolin-1}{gen}{}
%ENUNCIADO
{Integrar las siguientes ecuaciones diferenciales lineales:
\begin{enumerate}
\item $y'-2y=4$.
\item $y'-6xy=x$.
\item $\frac{dz}{dt}+\frac{3z}{10+3t}=6$ con la condición inical $z(2)=100$.
\item $y'+y\cos x=\sen x\cos x$ con la condición inicial $y(0)=1$.
\end{enumerate}
}
%SOLUCIÓN
{
\begin{enumerate}
\item $y=Ce^{2x}-2$.
\item $y=Ce^{3x^2}-\frac{1}{6}$.
\item $z=\dfrac{9t^2+60t+1444}{3t+10}$.
\item $y=2e^{-\sen x}+\sen x -1$.
\end{enumerate}
}
%RESOLUCIÓN
{}


\newproblem{edolin-2}{gen}{}
%ENUNCIADO
{Un tanque de 50 litros contiene inicialmente 10 litros de agua. En el instante inicial se vierte al tanque una disolución salina que
contiene 100 gr de sal por cada litro de agua, a razón de 4 litros por minuto, mientras que la mezcla bien agitada abandona el tanque a un
ritmo de 2 litros por minuto. ¿Cuánto tiempo transcurrirá hasta que se llene el depósito? En dicho instante, ¿qué cantidad de sal habrá en
el depósito?}
%SOLUCIÓN
{El depósito se llenará a los 20 minutos y contendrá $4.8$ kg de sal. 
}
%RESOLUCIÓN
{}


{}