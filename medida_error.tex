% Version control information:
\svnidlong
{$HeadURL: https://ejercicioscalculo.googlecode.com/svn/trunk/edo_separables.tex $}
{$LastChangedDate: 2008-07-09 20:02:39 +0200 (mié, 09 jul 2008) $}
{$LastChangedRevision: 9 $}
{$LastChangedBy: asalber $}
%\svnid{$Id: edo_separables.tex 9 2008-07-09 18:02:39Z asalber $
%
\newproblem{err-1}{gen}{}
%ENUNCIADO
{Si al medir el volumen de un objeto se obtiene como valor más probable $23618.87543$ cm$^3$ con un error de $302.432$
cm$^3$, ¿cuál será la expresión correcta del resultado de dicha medida?     
}
%SOLUCIÓN
{$23620\pm310$ cm$^3$.
}
%RESOLUCIÓN
{Redondeando el error por exceso a la segunda cifra significativa tenemos $310$ cm$^3$, y redondeando la medida a la
primera cifra afectada por el error, es decir a las decenas, tenemos $23620$ cm$^3$, luego la expresión se la medida
será $23620\pm310$ cm$^3$.}


\newproblem{err-2}{qui}{}
%ENUNCIADO
{Se ha medido el tiempo de caída de una esfera en un líquido cuatro veces, obteniéndose los valores: $30.5$; $29.6$;
$30.1$; $30.8$. ¿Cuál será el resultado del experimento?}
%SOLUCIÓN
{$\bar t=20.25$ s y $s_{\bar t}=0.2598$, de manera que suponiendo que la incertidumbre en el aparato de medida es
$0.1$ s, el tiempo de caída es $30.25\pm0.36$.
}
%RESOLUCIÓN
{}


\newproblem{err-3}{far}{}
%ENUNCIADO
{En el análisis de la orina de una atleta, en un control antidopaje, se detectan $16.5 \pm 0.8$ $\mu$g de cafeína por
cada cm$^3$ de orina. El resultado del contra-análisis es igual a $13.5 \pm 0.5$ $\mu$g de cafeína por cm$^3$. ¿Son
ambos resultados coincidentes o necesariamente ha existido algún fallo en uno de ellos? Sabiendo que el límite máximo
permitido es de $12$ $\mu$g de cafeína por cm$^3$, ¿ha dado o no positivo dicho atleta?
}
%SOLUCIÓN
{Según el análisis la concentración de cafeina está en el intervalo $(14.1\,,\,18.9)$ $\mu$g/cm$^3$ con un $99.7\%$ de
confianza y según el contra-análisis está en el intervalo $(12,15)$ $\mu$g/cm$^3$ con la misma confianza, luego como
ambos intervalos se solapan se puede decir que son medidas coincidentes. Como ambos intervalos están por encima de 12
$\mu$g/cm$^3$ si se puede decir que el atleta ha dado positivo.
}
%RESOLUCIÓN
{}


\newproblem{err-4}{gen}{}
%ENUNCIADO
{Mediante una cinta métrica dividida en milímetros se ha obtenido que la longitud de una mesa es $1.5250$ m y que tiene
una anchura de $82.00$ cm. ¿Cuál es el área de dicha mesa?
}
%SOLUCIÓN
{$12505\pm12$ cm$^2$.
}
%RESOLUCIÓN
{}


\newproblem{err-5}{gen}{}
%ENUNCIADO
{Supongamos que nos piden calcular la densidad de una pieza homogénea de forma cónica sabiendo que su masa es $m=
300.23\pm 0.05$ g, su altura $h=12.3 \pm 0.1$ cm, y el radio de la base $r=7.44 \pm 0.01$ cm. Teniendo en cuenta que el
volumen de un cono es igual $\pi r^2 h/3$, calcular cuánto vale la densidad de la pieza cónica.
}
%SOLUCIÓN
{$0.4211\pm0.0047$ g/cm$^3$.
}
%RESOLUCIÓN
{}


\newproblem{err-6}{gen}{}
%ENUNCIADO
{Si suponemos dos cilindros concéntricos, cuyos radios son $r$ el interno y $R$ el externo, y los dos con altura $h$, y
consideramos el volumen de la pieza que queda entre los dos cilindros, calcular el volumen de dicha pieza estimando
correctamente el error cometido teniendo en cuenta que se han realizado 6 medidas diferentes de $r$, $R$ y $h$:
\[
\begin{array}{|c|c|c|c|}
\hline
\mbox{Medida} & \mbox{R (mm)} & \mbox{r (mm)} & \mbox{h(cm)} \\
\hline
1 & 48.51 & 43.42 & 29.12 \\
\hline
2 & 47.39 & 42.94 & 29.14 \\
\hline
3 & 48.81 & 42.59 & 28.99 \\
\hline
4 & 47.52 & 43.11 & 29.13 \\
\hline
5 & 47.93 & 42.45 & 29.13 \\
\hline
6 & 47.88 & 42.11 & 29.06 \\
\hline
\end{array}
\]
}
%SOLUCIÓN
{$\bar R=48.0067$ mm y $s_{\bar R}=0.2264$ mm con lo que se obtiene la medida $R=48.01\pm 0.24$mm.\\
$\bar r = 42.77$ mm y $s_{\bar r}=0.19471$ mm con lo que se obtiene la medida $r=42.77\pm 0.21$mm.\\
$\bar h= 292.45$ mm y $s_{\bar h}=1.44839$ mm con lo que se obtiene la medida $h=292.5\pm 1.6$ mm.
La medida indirecta del volumen es $440\pm 27$ cm$^3$. 
}
%RESOLUCIÓN
{}


\newproblem{err-7}{fis}{}
%ENUNCIADO
{En una fuente se ha llenado completamente de agua un recipiente de base cuadrada de lado $l$ y altura $h$ en un tiempo
$t$, y se quiere medir el caudal de agua $q$ que mana de la fuente ($q=(l^2h)/t$). Además, el experimento lo han
realizado consecutivamente 4 alumnos obteniendo:
\[
\begin{array}{|c|c|c|c|}
\hline
\mbox{Alumno} & l & h & t \\
\hline
1 & 220.4 \pm 0,1 \mbox{ mm} & 535.3 \pm 0.1 \mbox{ mm} & 314.6 \pm 0.1 \mbox{ s} \\
\hline
2 & 220.6 \pm 0.1 \mbox{ mm} & 535.2 \pm 0.1 \mbox{ mm} & 313.9 \pm 0.1 \mbox{ s} \\
\hline
3 & 220.8 \pm 0.1 \mbox{ mm} & 535.9 \pm 0.1 \mbox{ mm} & 314.2 \pm 0.1 \mbox{ s} \\
\hline
4 & 221.0 \pm 0.1 \mbox{ mm} & 535.6 \pm 0.1 \mbox{ mm} & 314.8 \pm 0.1 \mbox{ s} \\
\hline
\end{array}
\]
}
%SOLUCIÓN
{$q_1=82650\pm 120$ mm$^3$/s, $q_2=82970\pm 120$ mm$^3$/s, $q_3=83150\pm 120$ mm$^3$/s y $q_4=83010\pm 120$ mm$^3$/s.
La medida final del caudal es $q=82969\pm 59$ mm$^3$/s.
}
%RESOLUCIÓN
{}


\newproblem{err-8}{gen}{*}
%ENUNCIADO
{El volumen de un cono es $V(r,h)=\frac{1}{3}\pi r^2 h$ donde $r$ es el radio de la base y $h$ la altura. Si se ha medido un cono y se ha
obtenido un radio de 2 m y una altura de 1 m, se pide:
\begin{enumerate}
\item Calcular el gradiente del volumen del cono anterior. ¿Cómo se interpretaría?
\item Calcular el hessiano del volumen del cono.
\item Si el radio se ha medido una única vez con un error de $\pm 0,01$ m y la altura también una única vez con un error de $\pm 0,001$ m,
dar la expresión de la medida del volumen con su error.
\end{enumerate}
}
%SOLUCIÓN
{\begin{enumerate}
\item $\grad V(r,h)=(\frac{2}{3}\pi r h,\frac{1}{3}\pi r^2)$, $\grad V(2,1) = (\frac{4}{3}\pi,\frac{4}{3}\pi)$.
\item $H V(r,h)=\left(
\begin{array}{cc}
\frac{2}{3}\pi h & \frac{2}{3}\pi r\\
\frac{2}{3}\pi r & 0
\end{array}
\right)$,
$H V(2,1)=\left(
\begin{array}{cc}
\frac{2}{3}\pi & \frac{4}{3}\pi\\
\frac{4}{3}\pi r & 0
\end{array}
\right)$
y
$|H V(2,1)| = -\frac{16}{9}\pi^2$.
\item $V=4.189\pm 0.047$ $m^3$.
\end{enumerate}
}
%RESOLUCIÓN
{}


\newproblem{err-9}{gen}{*}
%ENUNCIADO
{Una magnitud física $m$, que se mide de forma indirecta, depende de otras dos $x$ e $y$ según la fórmula:
\[
m(x,y) = \sqrt{xy}\log _4 \left(\frac{x}{y}\right)
\]
Si se han medido directamente $x$ e $y$ con una imprecisión igual en ambos casos $\triangle x=\triangle y=0.1$, y después de un tratamiento
estadístico de los datos se han obtenido las siguientes medias y errores típicos de las medias: $\bar{x}=10$, $\bar{y}=15$,
$\sigma_{\bar{x}}=0,2$ y $\sigma_{\bar{y}}=0,3$, se pide:

\begin{enumerate}
\item Calcular el gradiente de la magnitud $m(x,y)$.
\item ¿Cuánto valdría el error cometido en la medida de $m(x,y)$?
\item ¿Cómo se expresaría el resultado obtenido para $m$?
\end{enumerate}
}
%SOLUCIÓN
{\begin{enumerate}
\item $\grad m(x,y)=\left(\dfrac{\sqrt y}{\sqrt x \ln 4}\left(\dfrac{\ln x-\ln y}{2}+1\right), \dfrac{\sqrt x}{\sqrt y \ln
4}\left(\dfrac{\ln x-\ln y}{2}-1\right) \right)$
\item $\varepsilon = 0.35347$.
\item $m=-3.58\pm 0.36$.
\end{enumerate}
}
%RESOLUCIÓN
{}
