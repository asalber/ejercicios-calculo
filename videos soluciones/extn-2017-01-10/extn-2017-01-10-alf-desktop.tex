% Author: Alfredo Sánchez Alberca (asalber@ceu.es}
% !TEX program = xelatex
\documentclass[aspectratio=149,10pt,t]{beamer}
%-------------------------------------------------------------------------------
% GENERAL PACKAGES
%-------------------------------------------------------------------------------
% Language
\usepackage{polyglossia}
\setmainlanguage{spanish}
% Maths
\usepackage{amsmath} % Math symbols and environments
\usepackage{amsfonts}
\usepackage{amssymb}
% Tables
\usepackage{array}
\usepackage{multirow}
\usepackage{booktabs}
% Graphics
\usepackage{graphicx}
\usepackage{tikz}
\usetikzlibrary{positioning}


% Colors
\definecolor{blueceu}{RGB}{0,164,227}
\definecolor{greenceu}{RGB}{194,205,24}
\definecolor{redceu}{RGB}{238,50,36}
\definecolor{purpleceu}{RGB}{169,78,145}
\definecolor{greyceu}{RGB}{117,117,97}
\definecolor{darkgrey}{RGB}{40,40,50}
\definecolor{softblueceu}{RGB}{193,225,246}
\setbeamercolor{structure}{fg=blueceu}
\setbeamercolor{normal text}{fg=darkgrey}
\hypersetup{colorlinks, urlcolor=purpleceu}

%-------------------------------------------------------------------------------
% FONTS
%-------------------------------------------------------------------------------
\usepackage{fontspec}
\setmainfont[Ligatures=TeX]{TeX Gyre Pagella}
\usepackage{unicode-math}
\setmathfont[math-style=ISO,bold-style=ISO,vargreek-shape=TeX]{TeX Gyre Pagella Math}
% Creative common icons
\usepackage[scale=1.5]{ccicons}

%-------------------------------------------------------------------------------
% CONFIGURATION
%-------------------------------------------------------------------------------
\setbeamersize{text margin left=.5cm, text margin right=.5cm} % Defines margin sizes
\beamertemplatenavigationsymbolsempty % Hide navitation bar
\usefonttheme[onlymath]{serif} % Math text in serif
\setbeamertemplate{blocks}[rounded] % Blocks with rounded corners
%\setbeamercolor{block title}{bg=RoyalBlue!10} % Color of block title
%\setbeamercolor{block body}{bg=RoyalBlue!10} % Color of block body

%-------------------------------------------------------------------------------
% COMMANDS
%-------------------------------------------------------------------------------
\newcommand{\sen}{\operatorname{sen}}
\newcommand{\tg}{\operatorname{tg}}
\newcommand{\arcsen}{\operatorname{arcsen}}
\newcommand{\arctg}{\operatorname{arctg}}

%-------------------------------------------------------------------------------
% DOCUMENT
%-------------------------------------------------------------------------------
\begin{document}
%---------------------------------------------------------------------SLIDE----
\begin{frame}[c]
\vspace{2cm}

\begin{center}
\structure{\LARGE {\textbf{Ejercicios de Cálculo}}}
\bigskip

\large
\begin{tabular}{rl}
Temas: & \structure{Derivadas en $n$ variables: Extremos relativos}\\
Titulaciones: & \structure{Química, Farmacia, Biotecnología}
\end{tabular}

\bigskip
Alfredo Sánchez Alberca\\
\url{asalber@ceu.es}\\
\url{http://aprendeconalf.es}\\

\includegraphics[scale=0.2]{../img/logo_uspceu}

\bigskip
{\color{darkgrey}\ccbyncsaeu}
\end{center}
\end{frame}

%---------------------------------------------------------------------SLIDE----
\begin{frame}[c]
\Large
En un proceso químico, la temperatura depende de la cantidad de dos sustancias $x$ e $y$ de acuerdo a la fórmula \[T(x,y)=4x^3+y^3-3xy.\]
Teniendo en cuenta que las cantidades de $x$ e $y$ no pueden ser negativas, estudiar los extremos relativos y los puntos de silla de la temperatura.
\end{frame}


%------------------------------------------------------------------SLIDE----
\begin{frame}
\begin{columns}
\begin{column}[T]{0.65\textwidth}
En un proceso químico, la temperatura depende de la cantidad de dos sustancias $x$ e $y$ de acuerdo a la fórmula $T(x,y)=4x^3+y^3-3xy$.
Teniendo en cuenta que las cantidades de $x$ e $y$ no pueden ser negativas, estudiar los extremos relativos y los puntos de silla de la temperatura.
\end{column}
\begin{column}[T]{0.35\textwidth}
\structure{Datos}\\
$T(x,y)=4x^3+y^3-3xy$
\end{column}
\end{columns}
\end{frame}


%------------------------------------------------------------------SLIDE----
\begin{frame}
\begin{columns}
\begin{column}[T]{0.6\textwidth}
En un proceso químico, la temperatura depende de la cantidad de dos sustancias $x$ e $y$ de acuerdo a la fórmula $T(x,y)=4x^3+y^3-3xy$.
Teniendo en cuenta que las cantidades de $x$ e $y$ no pueden ser negativas, estudiar los extremos relativos y los puntos de silla de la temperatura.
\end{column}
\begin{column}[T]{0.4\textwidth}
\structure{Datos}\\
$T(x,y)=4x^3+y^3-3xy$\\
$\nabla T(x,y)=(12x^2-3y, 3y^2-3x)$\\
Puntos críticos: $(0,0)$ y $(\frac{\sqrt[3]{4}}{4}, \frac{\sqrt[3]{2}}{2})$
\end{column}
\end{columns}
\end{frame}

\end{document}
