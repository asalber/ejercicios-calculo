\documentclass[aspectratio=149,10pt,xcolor=dvipsnames,t]{beamer}
%---------------------------------------------------------------------------
% GENERAL PACKAGES
%---------------------------------------------------------------------------
\usepackage[utf8x]{inputenc} % Sets UTF8 codification
\usepackage[T1]{fontenc}
\usepackage[spanish]{babel} % Sets spanish language
\usepackage{amsmath} % Math symbols and environments
\usepackage{amsfonts}
\usepackage{amssymb}

\usepackage{array}
\usepackage{multirow}
\usepackage{graphicx}
%\usepackage{url}
\usepackage{textcomp}

%---------------------------------------------------------------------------
% FONTS
%---------------------------------------------------------------------------
\usepackage{mathpazo} % Palatino

%---------------------------------------------------------------------------
% CONFIGURATION
%---------------------------------------------------------------------------
\setbeamersize{text margin left=.5cm, text margin right=.5cm} % Defines margin sizes 
\beamertemplatenavigationsymbolsempty % Hide navitation bar
\usefonttheme[onlymath]{serif} % Math text in serif
\setbeamertemplate{blocks}[rounded] % Blocks with rounded corners
%\setbeamercolor{block title}{bg=RoyalBlue!10} % Color of block title
%\setbeamercolor{block body}{bg=RoyalBlue!10} % Color of block body

\begin{document}
%---------------------------------------------------------------------SLIDE----
\begin{frame}[c]
\vspace{2cm}

\begin{center}
\structure{\LARGE {\textbf{Ejercicios de Cálculo}}}
\bigskip

\large
\begin{tabular}{rl}
Temas: & \structure{Ecuaciones Diferenciales de Primer Orden}\\
Titulaciones: & \structure{Química, Biotecnología}
\end{tabular}

\bigskip
Alfredo Sánchez Alberca (\texttt{asalber@ceu.es})

\includegraphics[scale=0.2]{img/logo_uspceu}

\biskip
\includegraphics[scale=0.07]{img/cc-logo}
\includegraphics[scale=0.2]{img/cc-by}
\includegraphics[scale=0.2]{img/cc-e}
\includegraphics[scale=0.2]{img/cc-c}
\end{center}
\end{frame}

%---------------------------------------------------------------------SLIDE----
\begin{frame}[c]
\begin{enumerate}
\item En una reacción química, una sustancia $A$ se transforma en otra $B$ con una velocidad del doble de la
cantidad de sustancia $A$. 
Si en el instante inicial la cantidad de $A$ es de $5$ gr/dl, ¿qué cantidad de sustancia $A$ habrá a los
2 segundos? 

\item Si en esa misma reacción, la sustancia $B$, a su vez, se transforma en otra $C$ a una velocidad del triple de la
cantidad de $B$, sabiendo que al comienzo de la reacción la cantidad de sustancia $B$ era nula, ¿qué cantidad de $B$
habrá a los 2 segundos?
\end{enumerate}
\end{frame}


%------------------------------------------------------------------SLIDE----
\begin{frame}
\begin{columns}
\begin{column}[T]{0.65\textwidth}
\begin{enumerate}
\item En una reacción química, una sustancia $A$ se transforma en otra $B$ con una velocidad del doble de la
cantidad de sustancia $A$. 
Si en el instante inicial la cantidad de $A$ es de $5$ gr/dl, ¿qué cantidad de sustancia $A$ habrá a los
2 segundos?
\end{enumerate}
\end{column}
\begin{column}[T]{0.35\textwidth}
\structure{Datos}\\
$A(t)=$ Cantidad de sustancia $A$ en el instante $t$\\
$A(0)=5$ gr/dl
\end{column}
\end{columns}
\end{frame}


%------------------------------------------------------------------SLIDE----
\begin{frame}
\begin{columns}
\begin{column}[T]{0.65\textwidth}
\begin{enumerate}
\setcounter{enumi}{1}
\item Si en esa misma reacción, la sustancia $B$, a su vez, se transforma en otra $C$ a una velocidad del triple de la
cantidad de $B$, sabiendo que al comienzo de la reacción la cantidad de sustancia $B$ era nula, ¿qué cantidad de $B$
habrá a los 2 segundos?
\end{enumerate}
\end{column}
\begin{column}[T]{0.35\textwidth}
\structure{Datos}\\
$A(t)=$ Cantidad de sustancia $A$ en el instante $t$\\
$A(t)=5e^{-2t}$\\
$B(t)=$ Cantidad de sustancia $B$ en el instante $t$\\
$B(0)=0$ gr/dl
\end{column}
\end{columns}
\end{frame}

\end{document}