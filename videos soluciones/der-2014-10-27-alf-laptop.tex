\documentclass[aspectratio=149,10pt,xcolor=dvipsnames,t]{beamer}
%---------------------------------------------------------------------------
% GENERAL PACKAGES
%---------------------------------------------------------------------------
\usepackage[utf8x]{inputenc} % Sets UTF8 codification
\usepackage[T1]{fontenc}
\usepackage[spanish]{babel} % Sets spanish language
\usepackage{amsmath} % Math symbols and environments
\usepackage{amsfonts}
\usepackage{amssymb}

\usepackage{array}
\usepackage{multirow}
\usepackage{graphicx}
%\usepackage{url}
\usepackage{textcomp}

% Colors
\definecolor{blueceu}{RGB}{0,164,227} 
\definecolor{greenceu}{RGB}{194,205,24} 
\definecolor{redceu}{RGB}{238,50,36} 
\definecolor{purpleceu}{RGB}{169,78,145} 
\definecolor{greyceu}{RGB}{117,117,97} 
\definecolor{darkgrey}{RGB}{40,40,50}
\definecolor{softblueceu}{RGB}{193,225,246} 
\setbeamercolor{structure}{fg=blueceu}
\setbeamercolor{normal text}{fg=darkgrey}
\hypersetup{colorlinks, urlcolor=purpleceu}

%---------------------------------------------------------------------------
% FONTS
%---------------------------------------------------------------------------
\usepackage{mathpazo} % Palatino

%---------------------------------------------------------------------------
% CONFIGURATION
%---------------------------------------------------------------------------
\setbeamersize{text margin left=.5cm, text margin right=.5cm} % Defines margin sizes 
\beamertemplatenavigationsymbolsempty % Hide navitation bar
\usefonttheme[onlymath]{serif} % Math text in serif
\setbeamertemplate{blocks}[rounded] % Blocks with rounded corners
%\setbeamercolor{block title}{bg=RoyalBlue!10} % Color of block title
%\setbeamercolor{block body}{bg=RoyalBlue!10} % Color of block body

\begin{document}
%---------------------------------------------------------------------SLIDE----
\begin{frame}[c]
\vspace{1.5cm}

\begin{center}
\structure{\LARGE {\textbf{Ejercicios de Cálculo}}}
\bigskip

\large
\begin{tabular}{rl}
Temas: & \structure{Derivadas: Aproximación mediante el diferencial}\\
Titulaciones: & \structure{Química, Farmacia}
\end{tabular}

\bigskip
Alfredo Sánchez Alberca\\
\url{asalber@ceu.es}\\
\url{http://aprendeconalf.es}\\

\includegraphics[scale=0.2]{img/logo_uspceu}

\biskip
\includegraphics[scale=0.07]{img/cc-logo}
\includegraphics[scale=0.2]{img/cc-by}
\includegraphics[scale=0.2]{img/cc-e}
\includegraphics[scale=0.2]{img/cc-c}
\end{center}
\end{frame}

%---------------------------------------------------------------------SLIDE----
\begin{frame}[c]
\Large
En una reacción química, la concentración de una sustancia $c$ depende de la concentración de otras dos sustancias $a$ y $b$
según la ecuación $c=\sqrt[3]{ab^2}$. 
Si en un determinado instante en el que la concentración de $a=b=2$ mg/mm$^3$ se empieza a aumentar la concentración de $a$ a razón de $0.2$ mg$\cdot$ mm$^{-3}$/s y la de $b$ a razón de $0.4$ mg$\cdot$ mm$^{-3}$/s, ¿con qué velocidad cambia la
concentración de $c$? 
¿Cuál será la concentración aproximada de $c$ a los 2 segundos?
\end{frame}


%------------------------------------------------------------------SLIDE----
\begin{frame}
\begin{columns}
\begin{column}[T]{0.75\textwidth}
En una reacción química, la concentración de una sustancia $c$ depende de la concentración de otras dos sustancias $a$ y $b$
según la ecuación $c=\sqrt[3]{ab^2}$. 
Si en un determinado instante en el que la concentración de $a=b=2$ mg/mm$^3$ se empieza a aumentar la concentración de $a$ a razón de $0.2$ mg$\cdot$ mm$^{-3}$/s y la de $b$ a razón de $0.4$ mg$\cdot$ mm$^{-3}$/s, ¿con qué velocidad cambia la
concentración de $c$? 
¿Cuál será la concentración aproximada de $c$ a los 2 segundos?
\end{column}
\begin{column}[T]{0.25\textwidth}
\structure{Datos}\\
$c=\sqrt[3]{ab^2}$\\
$a=b=2$ mg/mm$^3$\\
$a'= 0.2$ mg$\cdot$ mm$^{-3}$/s\\
$b'= 0.4$ mg$\cdot$ mm$^{-3}$/s
\end{column}
\end{columns}
\end{frame}


%------------------------------------------------------------------SLIDE----
\begin{frame}
\begin{columns}
\begin{column}[T]{0.75\textwidth}
En una reacción química, la concentración de una sustancia $c$ depende de la concentración de otras dos sustancias $a$ y $b$
según la ecuación $c=\sqrt[3]{ab^2}$. 
Si en un determinado instante en el que la concentración de $a=b=2$ mg/mm$^3$ se empieza a aumentar la concentración de $a$ a razón de $0.2$ mg$\cdot$ mm$^{-3}$/s y la de $b$ a razón de $0.4$ mg$\cdot$ mm$^{-3}$/s, ¿con qué velocidad cambia la
concentración de $c$? 
¿Cuál será la concentración aproximada de $c$ a los 2 segundos?
\end{column}
\begin{column}[T]{0.25\textwidth}
\structure{Datos}\\
$c=\sqrt[3]{ab^2}$\\
$a=b=2$ mg/mm$^3$\\
$a'= 0.2$ mg$\cdot$ mm$^{-3}$/s\\
$b'= 0.4$ mg$\cdot$ mm$^{-3}$/s
\end{column}
\end{columns}
\end{frame}


\end{document}