\documentclass[aspectratio=169,10pt,xcolor=dvipsnames,t]{beamer}
%---------------------------------------------------------------------------
% GENERAL PACKAGES
%---------------------------------------------------------------------------
\usepackage[utf8]{inputenc} % Sets UTF8 codification
\usepackage[T1]{fontenc}
\usepackage[spanish]{babel} % Sets spanish language
\usepackage{amsmath} % Math symbols and environments
\usepackage{amsfonts}
\usepackage{amssymb}

\usepackage{array}
\usepackage{multirow}
\usepackage{graphicx}
\usepackage{textcomp}
\usepackage[
    type={CC},
    modifier={by-nc-sa},
    version={3.0},
    imagemodifier={-eu}
]{doclicense}

% Colors
\definecolor{blueceu}{RGB}{0,164,227} 
\definecolor{greenceu}{RGB}{194,205,24} 
\definecolor{redceu}{RGB}{238,50,36} 
\definecolor{purpleceu}{RGB}{169,78,145} 
\definecolor{greyceu}{RGB}{117,117,97} 
\definecolor{darkgrey}{RGB}{40,40,50}
\definecolor{softblueceu}{RGB}{193,225,246} 
\setbeamercolor{structure}{fg=blueceu}
\setbeamercolor{normal text}{fg=darkgrey}
\hypersetup{colorlinks, urlcolor=purpleceu}

%---------------------------------------------------------------------------
% FONTS
%---------------------------------------------------------------------------
\usepackage{mathpazo} % Palatino

%---------------------------------------------------------------------------
% CONFIGURATION
%---------------------------------------------------------------------------
\setbeamersize{text margin left=.5cm, text margin right=.5cm} % Defines margin sizes 
\beamertemplatenavigationsymbolsempty % Hide navitation bar
\usefonttheme[onlymath]{serif} % Math text in serif
\setbeamertemplate{blocks}[rounded] % Blocks with rounded corners
%\setbeamercolor{block title}{bg=RoyalBlue!10} % Color of block title
%\setbeamercolor{block body}{bg=RoyalBlue!10} % Color of block body

\begin{document}
%---------------------------------------------------------------------SLIDE----
\begin{frame}[c]
  \vspace{1.5cm}

  \begin{center}
    \structure{\LARGE {\textbf{Ejercicios de Cálculo}}}
    \bigskip

    \large
    \begin{tabular}{rl}
      Temas:        & \structure{Extremos y curvatura de una función}          \\
      Titulaciones: & \structure{Farmacia}
    \end{tabular}

    \bigskip
    Alfredo Sánchez Alberca\\
    \url{asalber@ceu.es}\\
    \url{https://aprendeconalf.es}\\

    \includegraphics[scale=0.2]{img/logo_uspceu}

    \bigskip
    \doclicenseIcon
  \end{center}
\end{frame}

%---------------------------------------------------------------------SLIDE----
\begin{frame}[c]
  \Large
  Se administra una medicina a un enfermo y $t$ horas después la concentración en sangre del principio activo viene dada por la función $c(t) = t^2e^{-t/2}$ en miligramos por mililitro.
  Se pide:
  \begin{enumerate}
    \item Calcular el valor máximo de la concentración de principio activo e indicar en qué momento se alcanza dicho valor máximo.
    \item Estudiar la concavidad y calcular los puntos de inflexión de la concentración de principio activo.
  \end{enumerate}
\end{frame}


%------------------------------------------------------------------SLIDE----
\begin{frame}
  \begin{columns}
    \begin{column}[T]{0.7\textwidth}
      \begin{enumerate}
        \item Calcular el valor máximo de la concentración de principio activo e indicar en qué momento se alcanza dicho valor máximo.
      \end{enumerate}
    \end{column}
    \quad
    \begin{column}[T]{0.3\textwidth}
      \structure{Datos}\\
      Concentración principio activo: $c(t) = t^2e^{-t/2}$
    \end{column}
  \end{columns}
\end{frame}


%------------------------------------------------------------------SLIDE----
\begin{frame}
  \begin{columns}
    \begin{column}[T]{0.7\textwidth}
      \begin{enumerate}
        \item[2] Estudiar la concavidad y calcular los puntos de inflexión de la concentración de principio activo.
      \end{enumerate}
    \end{column}
    \quad
    \begin{column}[T]{0.3\textwidth}
      \structure{Datos}\\
      Concentración principio activo: $c(t) = t^2e^{-t/2}$\\
      $c''(t) = e^{-t/2}\left(\dfrac{t^2}{4}-2t+2\right)$
    \end{column}
  \end{columns}
\end{frame}

\end{document}