\documentclass[aspectratio=149,10pt,xcolor=dvipsnames,t]{beamer}
%---------------------------------------------------------------------------
% GENERAL PACKAGES
%---------------------------------------------------------------------------
\usepackage[utf8x]{inputenc} % Sets UTF8 codification
\usepackage[T1]{fontenc}
\usepackage{lmodern}
\usepackage{amsmath} % Math symbols and environments
\usepackage{amsfonts}
\usepackage{amssymb}

\usepackage{array}
\usepackage{multirow}
\usepackage{graphicx}
\usepackage{textcomp}

\usepackage[spanish]{babel} % Sets spanish language

% Colors
\definecolor{blueceu}{RGB}{0,164,227} 
\definecolor{greenceu}{RGB}{194,205,24} 
\definecolor{redceu}{RGB}{238,50,36} 
\definecolor{purpleceu}{RGB}{169,78,145} 
\definecolor{greyceu}{RGB}{117,117,97} 
\definecolor{darkgrey}{RGB}{40,40,50}
\definecolor{softblueceu}{RGB}{193,225,246} 
\setbeamercolor{structure}{fg=blueceu}
\setbeamercolor{normal text}{fg=darkgrey}
\hypersetup{colorlinks, urlcolor=purpleceu}


%---------------------------------------------------------------------------
% FONTS
%---------------------------------------------------------------------------
\usepackage{mathpazo} % Palatino

%---------------------------------------------------------------------------
% CONFIGURATION
%---------------------------------------------------------------------------
\setbeamersize{text margin left=.5cm, text margin right=.5cm} % Defines margin sizes 
\beamertemplatenavigationsymbolsempty % Hide navitation bar
\usefonttheme[onlymath]{serif} % Math text in serif
\setbeamertemplate{blocks}[rounded] % Blocks with rounded corners
%\setbeamercolor{block title}{bg=RoyalBlue!10} % Color of block title
%\setbeamercolor{block body}{bg=RoyalBlue!10} % Color of block body

\begin{document}
%---------------------------------------------------------------------SLIDE----
\begin{frame}[c]
\vspace{1.5cm}

\begin{center}
\structure{\LARGE {\textbf{Ejercicios de Cálculo}}}
\bigskip

\large
\begin{tabular}{rl}
Temas: & \structure{Derivadas en $n$ variables: Extremos relativos}\\
Titulaciones: & \structure{Todas}
\end{tabular}

\bigskip
Alfredo Sánchez Alberca\\
\url{asalber@ceu.es}\\
\url{http://aprendeconalf.es}\\

\includegraphics[scale=0.2]{img/logo_uspceu}

\biskip
\includegraphics[scale=0.07]{img/cc-logo}
\includegraphics[scale=0.2]{img/cc-by}
\includegraphics[scale=0.2]{img/cc-e}
\includegraphics[scale=0.2]{img/cc-c}
\end{center}
\end{frame}

%---------------------------------------------------------------------SLIDE----
\begin{frame}[c]
\Large
Dada la función $f(x, y) = \dfrac{ax^3}{3} + \dfrac{by^3}{3}-4ax-4by$ con $a,b>0$ dos parámetros, estudiar la
existencia de extremos relativos y puntos de silla de $f$.
\end{frame}


%------------------------------------------------------------------SLIDE----
\begin{frame}
\begin{columns}
\begin{column}[T]{0.6\textwidth}
Dada la función $f(x, y) = \dfrac{ax^3}{3} + \dfrac{by^3}{3}-4ax-4by$ con $a,b>0$ dos parámetros, estudiar la
existencia de extremos relativos y puntos de silla de $f$.
\end{column}
\begin{column}[T]{0.4\textwidth}
\structure{Datos}\\
$f(x, y) = \dfrac{ax^3}{3} + \dfrac{by^3}{3}-2ax-2by$\\
$a,b>0$
\end{column}
\end{columns}
\end{frame}



%------------------------------------------------------------------SLIDE----
\begin{frame}
\begin{columns}
\begin{column}[T]{0.6\textwidth}
Dada la función $f(x, y) = \dfrac{ax^3}{3} + \dfrac{by^3}{3}-4ax-4by$ con $a,b>0$ dos parámetros, estudiar la
existencia de extremos relativos y puntos de silla de $f$.
\end{column}
\begin{column}[T]{0.4\textwidth}
\structure{Datos}\\
$f(x, y) = \dfrac{ax^3}{3} + \dfrac{by^3}{3}-2ax-2by$\\
$a,b>0$\\
$\nabla f(x,y)=(ax^2-4a,by^2-4b)$\\
Puntos críticos: $(-2,-2)$, $(-2,2)$, $(2,-2)$, $(2,2)$
\end{column}
\end{columns}
\end{frame}

\end{document}