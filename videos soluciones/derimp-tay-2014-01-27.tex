\documentclass[aspectratio=149,10pt,xcolor=dvipsnames,t]{beamer}
%---------------------------------------------------------------------------
% GENERAL PACKAGES
%---------------------------------------------------------------------------
\usepackage[utf8x]{inputenc} % Sets UTF8 codification
\usepackage[T1]{fontenc}
\usepackage[spanish]{babel} % Sets spanish language
\usepackage{amsmath} % Math symbols and environments
\usepackage{amsfonts}
\usepackage{amssymb}

\usepackage{array}
\usepackage{multirow}
\usepackage{graphicx}
%\usepackage{url}
\usepackage{textcomp}

%---------------------------------------------------------------------------
% FONTS
%---------------------------------------------------------------------------
\usepackage{mathpazo} % Palatino

%---------------------------------------------------------------------------
% CONFIGURATION
%---------------------------------------------------------------------------
\setbeamersize{text margin left=.5cm, text margin right=.5cm} % Defines margin sizes 
\beamertemplatenavigationsymbolsempty % Hide navitation bar
\usefonttheme[onlymath]{serif} % Math text in serif
\setbeamertemplate{blocks}[rounded] % Blocks with rounded corners
%\setbeamercolor{block title}{bg=RoyalBlue!10} % Color of block title
%\setbeamercolor{block body}{bg=RoyalBlue!10} % Color of block body

\begin{document}
%---------------------------------------------------------------------SLIDE----
\begin{frame}[c]
\vspace{2cm}

\begin{center}
\structure{\LARGE {\textbf{Ejercicios de Cálculo}}}
\bigskip

\large
\begin{tabular}{rl}
Temas: & \structure{Derivadas implícitas y polinomios de Taylor}\\
Titulaciones: & \structure{Todas}
\end{tabular}

\bigskip
Alfredo Sánchez Alberca (\texttt{asalber@ceu.es})

\includegraphics[scale=0.2]{img/logo_uspceu}

\biskip
\includegraphics[scale=0.07]{img/cc-logo}
\includegraphics[scale=0.2]{img/cc-by}
\includegraphics[scale=0.2]{img/cc-e}
\includegraphics[scale=0.2]{img/cc-c}
\end{center}
\end{frame}

%---------------------------------------------------------------------SLIDE----
\begin{frame}[c]
Tres variables se hallan relacionadas mediante la expresión $e^{x^2y-z}+2xyz=3$. 
Suponiendo esta relación define a $z$ como función de $x$ e $y$ ($z=f(x,y)$) en el entorno del punto $(1,1,1)$.
Se pide:
\begin{enumerate}
\item ¿En qué dirección se produce la máxima variación de $z$ a partir del punto $(1,1)$?
\item ¿Cómo varía $z$ en el punto $(1,1)$ si $x$ tiene a variar el doble que $y$?
\item Calcular el polinomio de Taylor de segundo grado de la función $F(x,y,z)=e^{x^2y-z}+2xyz$ en el punto $(1,1,1)$.
\end{enumerate}
\end{frame}


%------------------------------------------------------------------SLIDE----
\begin{frame}
\begin{columns}
\begin{column}[T]{0.7\textwidth}
\begin{enumerate}
\item ¿En qué dirección se produce la máxima variación de $z$ a partir del punto $(1,1)$?
\end{enumerate}
\end{column}
\begin{column}[T]{0.3\textwidth}
\structure{Datos}\\
$e^{x^2y-z}+2xyz=3$\\
$z=f(x,y)$
\end{column}
\end{columns}
\end{frame}


%------------------------------------------------------------------SLIDE----
\begin{frame}
\begin{columns}
\begin{column}[T]{0.7\textwidth}
\begin{enumerate}
\item[2] ¿Cómo varía $z$ en el punto $(1,1)$ si $x$ tiene a variar el doble que $y$?
\end{enumerate}
\end{column}
\begin{column}[T]{0.3\textwidth}
\structure{Datos}\\
$e^{x^2y-z}+2xyz=3$\\
$z=f(x,y)$\\
$\nabla z(1,1)=(-4,-3)$
\end{column}
\end{columns}
\end{frame}


%------------------------------------------------------------------SLIDE----
\begin{frame}
\begin{columns}
\begin{column}[T]{0.7\textwidth}
\begin{enumerate}
\item[3] Calcular el polinomio de Taylor de segundo grado de la función $F$ en el punto $(1,1,1)$.
\end{enumerate}
\end{column}
\begin{column}[T]{0.3\textwidth}
\structure{Datos}\\
$F(x,y,z)=e^{x^2y-z}+2xyz$\\
Punto $(1,1,1)$\\
$\frac{\partial F}{\partial x}=e^{x^2y-z}2xy+2yz$\\
$\frac{\partial F}{\partial x}(1,1,1)=4$\\
$\frac{\partial F}{\partial y}=e^{x^2y-z}x^2+2xz$\\
$\frac{\partial F}{\partial y}(1,1,1)=3$\\
$\frac{\partial F}{\partial z}=-e^{x^2y-z}+2xy$\\
$\frac{\partial F}{\partial z}(1,1,1)=1$
\end{column}
\end{columns}
\end{frame}


%------------------------------------------------------------------SLIDE----
\begin{frame}
\begin{columns}
\begin{column}[T]{0.7\textwidth}
\begin{enumerate}
\item[3] Calcular el polinomio de Taylor de segundo grado de la función $F$ en el punto $(1,1,1)$.
\begin{align*}
P(x,y,z)=F(1,1,1)+\nabla F(1,1,1)(x-1,y-1,z-1)+\\
+\frac{1}{2}((x-1,y-1,z-1)\nabla^2F(1,1,1)(x-1,y-1,z-1))
\end{align*}
\end{enumerate}
\end{column}
\begin{column}[T]{0.3\textwidth}
\structure{Datos}\\
$F(x,y,z)=e^{x^2y-z}+2xyz$\\
Punto $(1,1,1)$\\
$F(1,1,1)=3$\\
$\nabla F(1,1,1)=(4,3,1)$\\
$\nabla^2F(1,1,1)=\left(
\begin{array}{ccc}
6 & 6 & 0\\
6 & 1 & 1\\
0 & 1 & 1\\
\end{array}
\right)$
\end{column}
\end{columns}
\end{frame}

\end{document}