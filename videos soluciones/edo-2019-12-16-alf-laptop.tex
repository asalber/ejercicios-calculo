\documentclass[aspectratio=169,10pt,xcolor=dvipsnames,t]{beamer}
%---------------------------------------------------------------------------
% GENERAL PACKAGES
%---------------------------------------------------------------------------
\usepackage[utf8]{inputenc} % Sets UTF8 codification
\usepackage[T1]{fontenc}
\usepackage[spanish]{babel} % Sets spanish language
\usepackage{amsmath} % Math symbols and environments
\usepackage{amsfonts}
\usepackage{amssymb}

\usepackage{array}
\usepackage{multirow}
\usepackage{graphicx}
\usepackage{textcomp}
\usepackage[
    type={CC},
    modifier={by-nc-sa},
    version={3.0},
    imagemodifier={-eu}
]{doclicense}

% Colors
\definecolor{blueceu}{RGB}{0,164,227} 
\definecolor{greenceu}{RGB}{194,205,24} 
\definecolor{redceu}{RGB}{238,50,36} 
\definecolor{purpleceu}{RGB}{169,78,145} 
\definecolor{greyceu}{RGB}{117,117,97} 
\definecolor{darkgrey}{RGB}{40,40,50}
\definecolor{softblueceu}{RGB}{193,225,246} 
\setbeamercolor{structure}{fg=blueceu}
\setbeamercolor{normal text}{fg=darkgrey}
\hypersetup{colorlinks, urlcolor=purpleceu}

%---------------------------------------------------------------------------
% FONTS
%---------------------------------------------------------------------------
\usepackage{mathpazo} % Palatino

%---------------------------------------------------------------------------
% CONFIGURATION
%---------------------------------------------------------------------------
\setbeamersize{text margin left=.5cm, text margin right=.5cm} % Defines margin sizes 
\beamertemplatenavigationsymbolsempty % Hide navitation bar
\usefonttheme[onlymath]{serif} % Math text in serif
\setbeamertemplate{blocks}[rounded] % Blocks with rounded corners
%\setbeamercolor{block title}{bg=RoyalBlue!10} % Color of block title
%\setbeamercolor{block body}{bg=RoyalBlue!10} % Color of block body

\begin{document}
%---------------------------------------------------------------------SLIDE----
\begin{frame}[c]
\vspace{1.5cm}

\begin{center}
\structure{\LARGE {\textbf{Ejercicios de Cálculo}}}
\bigskip

\large
\begin{tabular}{rl}
Temas: & \structure{Ecuaciones diferenciales ordinarias}\\
Titulaciones: & \structure{Química, Ciencias Ambientales}
\end{tabular}

\bigskip
Alfredo Sánchez Alberca\\
\url{asalber@ceu.es}\\
\url{http://aprendeconalf.es}\\

\includegraphics[scale=0.2]{img/logo_uspceu}

\biskip
\doclicenseIcon
\end{center}
\end{frame}

%---------------------------------------------------------------------SLIDE----
\begin{frame}[c]
\Large
Una laguna contaminada con nitratos contiene 1000 toneladas de nitratos disueltos en 6 millones de metros cúbicos de agua. 
Para descontaminar la laguna se empieza a introducir agua pura a razón de 100000 metros cúbicos por día y se saca la misma cantidad de agua contaminada. 
Suponiendo que la concentración de nitratos se mantiene uniforme en la laguna, ¿cuál será la cantidad de nitratos en la laguna después de 2 semanas? 
Si la concentración máxima de nitratos para no considerar el agua contaminada es de $0.1$ kg/m$^3$, ¿cuándo se puede considerar que la laguna está descontaminada?
\end{frame}


%------------------------------------------------------------------SLIDE----
\begin{frame}
\begin{columns}
\begin{column}[T]{0.7\textwidth}
Una laguna contaminada con nitratos contiene 1000 toneladas de nitratos disueltos en 6 millones de metros cúbicos de agua. 
Para descontaminar la laguna se empieza a introducir agua pura a razón de 100000 metros cúbicos por día y se saca la misma cantidad de agua contaminada. 
Suponiendo que la concentración de nitratos se mantiene uniforme en la laguna, ¿cuál será la cantidad de nitratos en la laguna después de 2 semanas? 
\end{column}
\quad
\begin{column}[T]{0.3\textwidth}
\structure{Datos}\\
$n(t)$: Cantidad de nitratos en el instante $t$\\
Volumen: $6\cdot 10^6$ m$^3$\\
Cantidad inicial nitratos: $n(0)= 10^6$ kg\\
Velocidad entrada: $10^5$ m$^3$/día\\
Velocidad salida: $10^5$ m$^3$/día
\end{column}
\end{columns}
\end{frame}


%------------------------------------------------------------------SLIDE----
\begin{frame}
\begin{columns}
\begin{column}[T]{0.7\textwidth}
Si la concentración máxima de nitratos para no considerar el agua contaminada es de $0.1$ kg/m$^3$, ¿cuándo se puede considerar que la laguna está descontaminada?
\end{column}
\begin{column}[T]{0.3\textwidth}
\structure{Datos}\\
$n(t)=10^6e^{-t/60}$\\
Concentración máxima nitratos: $0.1$ kg/m$^3$
\end{column}
\end{columns}
\end{frame}

\end{document}