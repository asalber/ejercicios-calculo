% Author: Alfredo Sánchez Alberca (asalber@ceu.es}
% !TEX program = xelatex
\documentclass[aspectratio=169,10pt,t]{beamer}
%-------------------------------------------------------------------------------
% GENERAL PACKAGES
%-------------------------------------------------------------------------------
% Language
\usepackage{polyglossia}
\setmainlanguage{spanish}
% Maths
\usepackage{amsmath} % Math symbols and environments
\usepackage{amsfonts}
\usepackage{amssymb}
% Tables
\usepackage{array}
\usepackage{multirow}
\usepackage{booktabs}

% Graphics
\usepackage{graphicx}
\usepackage{tikz}
\usetikzlibrary{positioning}

% Colors
\definecolor{blueceu}{RGB}{0,164,227}
\definecolor{greenceu}{RGB}{194,205,24}
\definecolor{redceu}{RGB}{238,50,36}
\definecolor{purpleceu}{RGB}{169,78,145}
\definecolor{greyceu}{RGB}{117,117,97}
\definecolor{darkgrey}{RGB}{40,40,50}
\definecolor{softblueceu}{RGB}{193,225,246}
\setbeamercolor{structure}{fg=blueceu}
\setbeamercolor{normal text}{fg=darkgrey}
\hypersetup{colorlinks, urlcolor=purpleceu}

% Boxes
\usepackage[most]{tcolorbox}
\usepackage{setspace}
\newtcolorbox{datos}{
  enhanced,
  colback=blueceu!10, 
  colframe=blueceu, 
  fonttitle=\bfseries, 
  left=3pt, 
  right=3pt, 
  boxrule=0.5pt,
  code={\setstretch{1.4}},
  title={Datos},
}



%-------------------------------------------------------------------------------
% FONTS
%-------------------------------------------------------------------------------
\usepackage{fontspec}
\setmainfont[Ligatures=TeX]{TeX Gyre Pagella}
\usepackage{unicode-math}
\setmathfont[math-style=ISO, bold-style=ISO]{TeX Gyre Pagella Math}
% Creative common icons
\usepackage[
    type={CC},
    modifier={by-nc-sa},
    version={3.0},
    imagemodifier={-eu}
]{doclicense}

%-------------------------------------------------------------------------------
% CONFIGURATION
%-------------------------------------------------------------------------------
\setbeamersize{text margin left=.5cm, text margin right=.5cm} % Defines margin sizes
\beamertemplatenavigationsymbolsempty % Hide navitation bar
\usefonttheme[onlymath]{serif} % Math text in serif
\setbeamertemplate{blocks}[rounded] % Blocks with rounded corners
%\setbeamercolor{block title}{bg=RoyalBlue!10} % Color of block title
%\setbeamercolor{block body}{bg=RoyalBlue!10} % Color of block body

%-------------------------------------------------------------------------------
% COMMANDS
%-------------------------------------------------------------------------------
% \newcommand{\sen}{\operatorname{sen}}
% \newcommand{\tg}{\operatorname{tg}}
% \newcommand{\arcsen}{\operatorname{arcsen}}
% \newcommand{\arctg}{\operatorname{arctg}}

%-------------------------------------------------------------------------------
% DOCUMENT
%-------------------------------------------------------------------------------
\begin{document}
%---------------------------------------------------------------------SLIDE----
\begin{frame}[c]
\vspace{2cm}

\begin{center}
\structure{\LARGE {\textbf{Ejercicios de Cálculo}}}
\bigskip

\large
\begin{tabular}{rl}
Temas: & \structure{Derivadas en $n$ variables}\\
Titulaciones: & \structure{Química, Farmacia, Biotecnología}
\end{tabular}

\bigskip
Alfredo Sánchez Alberca\\
\url{asalber@ceu.es}\\
\url{https://aprendeconalf.es}\\

\includegraphics[scale=0.2]{../img/logo_uspceu}

\bigskip
\doclicenseIcon
\end{center}
\end{frame}

%---------------------------------------------------------------------SLIDE----
\begin{frame}[c]
\Large
La función $f(x,y)=ye^{-x^2-\frac{1}{2}y^2}$, expresa la cantidad de una sustancia $z=f(x,y)$ en función de otras dos $x$ e $y$ en una reacción química.
\begin{enumerate}
  \item Calcular el valor máximo local de $z$ teniendo en cuenta que $x\geq 0$ e $y\geq 0$.
  \item ¿Cuál será la variación de $z$ cuando $x=1$ e $y=0$, si comenzamos a aumentar la cantidad de $x$ al doble de ritmo de la de $y$?
  \item Calcular el polinomio de Taylor de segundo grado de $f$ en el punto $(1,0)$.
\end{enumerate}
\end{frame}


%------------------------------------------------------------------SLIDE----
\begin{frame}
\begin{columns}
\begin{column}[T]{0.75\textwidth}
\begin{enumerate}
\item Calcular el valor máximo local de $z=f(x,y)$ teniendo en cuenta que $x\geq 0$ e $y\geq 0$.
\end{enumerate}
\end{column}
\begin{column}[T]{0.25\textwidth}
\begin{datos}
$f(x,y)=ye^{-x^2-\frac{1}{2}y^2}$
\end{datos}
\end{column}
\end{columns}
\end{frame}


%------------------------------------------------------------------SLIDE----
\begin{frame}
\begin{columns}
\begin{column}[T]{0.53\textwidth}
\begin{enumerate}
  \item Calcular el valor máximo local de $z=f(x,y)$ teniendo en cuenta que $x\geq 0$ e $y\geq 0$.
\end{enumerate}
\end{column}
\begin{column}[T]{0.47\textwidth}
\begin{datos}
$f(x,y)=ye^{-x^2-\frac{1}{2}y^2}$\\
$\nabla f(x,y)=\left(-2xye^{-x^2-\frac{1}{2}y^2}, (1-y^2)e^{-x^2-\frac{1}{2}y^2}\right)$\\
Puntos críticos: $(0,1)$
\end{datos}
\end{column}
\end{columns}
\end{frame}


%------------------------------------------------------------------SLIDE----
\begin{frame}
\begin{columns}
\begin{column}[T]{0.53\textwidth}
\begin{enumerate}
  \item[2.] ¿Cuál será la variación de $z$ cuando $x=1$ e $y=0$, si comenzamos a aumentar la cantidad de $x$ al doble de ritmo de la de $y$?
\end{enumerate}
\end{column}
\begin{column}[T]{0.47\textwidth}
\begin{datos}
$f(x,y)=ye^{-x^2-\frac{1}{2}y^2}$\\
$\nabla f(x,y)=\left(-2xye^{-x^2-\frac{1}{2}y^2}, (1-y^2)e^{-x^2-\frac{1}{2}y^2}\right)$\\
Punto $P=(1,0)$\\
Dirección $\mathbf{v}=(2,1)$
\end{datos}
\end{column}
\end{columns}
\end{frame}


%------------------------------------------------------------------SLIDE----
\begin{frame}
\begin{columns}
\begin{column}[T]{0.42\textwidth}
\begin{enumerate}
  \item[3.] Calcular el polinomio de Taylor de segundo grado de $f$ en el punto $(1,0)$.
\end{enumerate}
\end{column}
\begin{column}[T]{0.61\textwidth}
\begin{datos}
$f(x,y)=ye^{-x^2-\frac{1}{2}y^2}$\\
$\nabla f(x,y)=\left(-2xye^{-x^2-\frac{1}{2}y^2}, (1-y^2)e^{-x^2-\frac{1}{2}y^2}\right)$\\
$\nabla^2f(x,y)=\left(
\begin{array}{cc}
  (4x^2y-2y)e^{-x^2-\frac{1}{2}y^2} & (2xy^2-2x)e^{-x^2-\frac{1}{2}y^2}\\
  (2xy^2-2x)e^{-x^2-\frac{1}{2}y^2} & (y^3-3y)e^{-x^2-\frac{1}{2}y^2}
\end{array}
\right)$\\
Punto $P=(1,0)$
\end{datos}
\end{column}
\end{columns}
\end{frame}

\end{document}
