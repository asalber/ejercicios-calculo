\documentclass[aspectratio=149,10pt,xcolor=dvipsnames,t]{beamer}
%---------------------------------------------------------------------------
% GENERAL PACKAGES
%---------------------------------------------------------------------------
\usepackage[utf8x]{inputenc} % Sets UTF8 codification
\usepackage[T1]{fontenc}
\usepackage[spanish]{babel} % Sets spanish language
\usepackage{amsmath} % Math symbols and environments
\usepackage{amsfonts}
\usepackage{amssymb}

\usepackage{array}
\usepackage{multirow}
\usepackage{graphicx}
%\usepackage{url}
\usepackage{textcomp}

%---------------------------------------------------------------------------
% FONTS
%---------------------------------------------------------------------------
\usepackage{mathpazo} % Palatino

%---------------------------------------------------------------------------
% CONFIGURATION
%---------------------------------------------------------------------------
\setbeamersize{text margin left=.5cm, text margin right=.5cm} % Defines margin sizes 
\beamertemplatenavigationsymbolsempty % Hide navitation bar
\usefonttheme[onlymath]{serif} % Math text in serif
\setbeamertemplate{blocks}[rounded] % Blocks with rounded corners
%\setbeamercolor{block title}{bg=RoyalBlue!10} % Color of block title
%\setbeamercolor{block body}{bg=RoyalBlue!10} % Color of block body

\begin{document}
%---------------------------------------------------------------------SLIDE----
\begin{frame}[c]
\vspace{2cm}

\begin{center}
\structure{\LARGE {\textbf{Ejercicios de Cálculo}}}
\bigskip

\large
\begin{tabular}{rl}
Temas: & \structure{Derivadas en $n$ variables: Extremos, Polinomios de Taylor}\\
Titulaciones: & \structure{Todas}
\end{tabular}

\bigskip
Alfredo Sánchez Alberca (\texttt{asalber@ceu.es})

\includegraphics[scale=0.2]{img/logo_uspceu}

\biskip
\includegraphics[scale=0.07]{img/cc-logo}
\includegraphics[scale=0.2]{img/cc-by}
\includegraphics[scale=0.2]{img/cc-e}
\includegraphics[scale=0.2]{img/cc-c}\end{center}
\end{frame}


%---------------------------------------------------------------------SLIDE----
\begin{frame}[c]
Dado el campo escalar
\[
h(x,y) = xy+\frac{xy^2}{2}-2x^2,
\]
se pide:
\begin{enumerate}
\item Determinar sus extremos relativos y sus puntos de silla.
\item Obtener el polinomio de Taylor de segundo grado en el punto $(1,2)$ y utilizarlo para dar una aproximación de $h(1.04,\,1.98)$.
\end{enumerate}
\end{frame}


%------------------------------------------------------------------SLIDE----
\begin{frame}
\begin{columns}
\begin{column}[T]{0.65\textwidth}
\begin{enumerate}
\item Determinar sus extremos relativos y sus puntos de silla.
\end{enumerate}
\end{column}
\begin{column}[T]{0.35\textwidth}
\structure{Datos}\\
$h(x,y) = xy+\dfrac{xy^2}{2}-2x^2$
\end{column}
\end{columns}
\end{frame}


%------------------------------------------------------------------SLIDE----
\begin{frame}
\begin{columns}
\begin{column}[T]{0.55\textwidth}
\begin{enumerate}
\item Determinar sus extremos relativos y sus puntos de silla.
\end{enumerate}
\end{column}
\begin{column}[T]{0.47\textwidth}
\structure{Datos}\\
$h(x,y) = xy+\dfrac{xy^2}{2}-2x^2$\\
$\nabla h(x,y)=\left(\dfrac{y^2}{2}-4x+y, x(1+y)\right)$\\
Puntos críticos: $(0,0)$, $(0,-2)$, $(-\frac{1}{8},-1)$
\end{column}
\end{columns}
\end{frame}


%------------------------------------------------------------------SLIDE----
\begin{frame}
\begin{columns}
\begin{column}[T]{0.57\textwidth}
\begin{enumerate}
\setcounter{enumi}{1}
\item Obtener el polinomio de Taylor de segundo grado en el punto $(1,2)$ y utilizarlo para dar una aproximación de $h(1.04,\,1.98)$.
\end{enumerate}
\end{column}
\begin{column}[T]{0.43\textwidth}
\structure{Datos}\\
$h(x,y) = xy+\dfrac{xy^2}{2}-2x^2$\\
$\nabla h(x,y)=\left(\dfrac{y^2}{2}-4x+y, x(1+y)\right)$\\
$\nabla^2 h(x,y)=\left(
\begin{array}{cc}
-4 & 1+y\\
1+y & x
\end{array}
\right)$
\end{column}
\end{columns}
\end{frame}

\end{document}