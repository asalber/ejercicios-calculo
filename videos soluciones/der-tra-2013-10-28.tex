\documentclass[aspectratio=149,10pt,xcolor=dvipsnames,t]{beamer}
%---------------------------------------------------------------------------
% GENERAL PACKAGES
%---------------------------------------------------------------------------
\usepackage[utf8x]{inputenc} % Sets UTF8 codification
\usepackage[T1]{fontenc}
\usepackage[spanish]{babel} % Sets spanish language
\usepackage{amsmath} % Math symbols and environments
\usepackage{amsfonts}
\usepackage{amssymb}

\usepackage{array}
\usepackage{multirow}
\usepackage{graphicx}
%\usepackage{url}
\usepackage{textcomp}

%---------------------------------------------------------------------------
% FONTS
%---------------------------------------------------------------------------
\usepackage{mathpazo} % Palatino

%---------------------------------------------------------------------------
% CONFIGURATION
%---------------------------------------------------------------------------
\setbeamersize{text margin left=.5cm, text margin right=.5cm} % Defines margin sizes 
\beamertemplatenavigationsymbolsempty % Hide navitation bar
\usefonttheme[onlymath]{serif} % Math text in serif
\setbeamertemplate{blocks}[rounded] % Blocks with rounded corners
%\setbeamercolor{block title}{bg=RoyalBlue!10} % Color of block title
%\setbeamercolor{block body}{bg=RoyalBlue!10} % Color of block body

\begin{document}
%---------------------------------------------------------------------SLIDE----
\begin{frame}[c]
\vspace{2cm}

\begin{center}
\structure{\LARGE {\textbf{Ejercicios de Cálculo}}}
\bigskip

\large
\begin{tabular}{rl}
Temas: & \structure{Derivadas de trayectorias}\\
Titulaciones: & \structure{Todas}
\end{tabular}

\bigskip
Alfredo Sánchez Alberca (\texttt{asalber@ceu.es})

\includegraphics[scale=0.2]{img/logo_uspceu}

\biskip
\includegraphics[scale=0.07]{img/cc-logo}
\includegraphics[scale=0.2]{img/cc-by}
\includegraphics[scale=0.2]{img/cc-e}
\includegraphics[scale=0.2]{img/cc-c}
\end{center}
\end{frame}

%---------------------------------------------------------------------SLIDE----
\begin{frame}[c]
Una mosca se mueve por una habitación siguiendo la trayectoria $g(t)=(\log t, t, t^2)$. ¿Cón qué
velocidad se mueve la mosca en el instante en que pasa por el punto $(0,1,1)$? Si en ese instante decide cambiar de
trayectoria y sigue la trayectoria de la recta tangente a la trayectoria $C$ en ese punto, manteniendo uniforme su
velocidad, ¿en qué posición estará en el instante $t=2$?
\end{frame}


%------------------------------------------------------------------SLIDE----
\begin{frame}
\begin{columns}
\begin{column}[T]{0.65\textwidth}
¿Cón qué velocidad se mueve la mosca en el instante en que pasa por el punto $P$?
\end{column}
\begin{column}[T]{0.35\textwidth}
\structure{Datos}\\
Trayectoria $g(t)=(\log t, t, t^2)$\\
Punto $P=(0,1,1)$
\end{column}
\end{columns}
\end{frame}


%------------------------------------------------------------------SLIDE----
\begin{frame}
\begin{columns}
\begin{column}[T]{0.65\textwidth}
Si en ese instante decide cambiar de trayectoria y sigue la trayectoria de la recta tangente a la trayectoria $g(t)$ en el punto $P$, manteniendo uniforme su velocidad, ¿en qué posición estará en el instante $t=2$?
\end{column}
\begin{column}[T]{0.35\textwidth}
\structure{Datos}\\
Trayectoria $g(t)=(\log t, t, t^2)$\\
Punto $P=(0,1,1)$\\
Velocidad $g'(1)=(1,1,2)$
\end{column}
\end{columns}
\end{frame}

\end{document}