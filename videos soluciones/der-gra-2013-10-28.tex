\documentclass[aspectratio=149,10pt,xcolor=dvipsnames,t]{beamer}
%---------------------------------------------------------------------------
% GENERAL PACKAGES
%---------------------------------------------------------------------------
\usepackage[utf8x]{inputenc} % Sets UTF8 codification
\usepackage[T1]{fontenc}
\usepackage[spanish]{babel} % Sets spanish language
\usepackage{amsmath} % Math symbols and environments
\usepackage{amsfonts}
\usepackage{amssymb}

\usepackage{array}
\usepackage{multirow}
\usepackage{graphicx}
%\usepackage{url}
\usepackage{textcomp}

%---------------------------------------------------------------------------
% FONTS
%---------------------------------------------------------------------------
\usepackage{mathpazo} % Palatino

%---------------------------------------------------------------------------
% CONFIGURATION
%---------------------------------------------------------------------------
\setbeamersize{text margin left=.5cm, text margin right=.5cm} % Defines margin sizes 
\beamertemplatenavigationsymbolsempty % Hide navitation bar
\usefonttheme[onlymath]{serif} % Math text in serif
\setbeamertemplate{blocks}[rounded] % Blocks with rounded corners
%\setbeamercolor{block title}{bg=RoyalBlue!10} % Color of block title
%\setbeamercolor{block body}{bg=RoyalBlue!10} % Color of block body

\begin{document}
%---------------------------------------------------------------------SLIDE----
\begin{frame}[c]
\vspace{2cm}

\begin{center}
\structure{\LARGE {\textbf{Ejercicios de Cálculo}}}
\bigskip

\large
\begin{tabular}{rl}
Temas: & \structure{Derivadas en $n$ variables: Gradiente y derivada direccional}\\
Titulaciones: & \structure{Química}
\end{tabular}

\bigskip
Alfredo Sánchez Alberca (\texttt{asalber@ceu.es})

\includegraphics[scale=0.2]{img/logo_uspceu}

\biskip
\includegraphics[scale=0.07]{img/cc-logo}
\includegraphics[scale=0.2]{img/cc-by}
\includegraphics[scale=0.2]{img/cc-e}
\includegraphics[scale=0.2]{img/cc-c}
\end{center}
\end{frame}

%---------------------------------------------------------------------SLIDE----
\begin{frame}[c]
 La función 
\[s(x,y,z)=\dfrac{\log(xy)}{z}\]
expresa la concentración de una sustancia $s$ en función de
las concentraciones de otras tres sustancias $x$, $y$, $z$ en una reacción química. Si en un determinado
instante las concentraciones $x$, $y$ y $z$ valen 1, se pide:
\begin{enumerate}
\item ¿En qué dirección aumentará lo más rápidamente posible la concentración de $s$?
\item Si empezamos a cambiar las concentraciones de $x$, $y$ y $z$ en la dirección del vector
$(2,1,0)$, es decir, $x$ crece el doble de $y$, y $z$ se mantiene constante, ¿cuánto cambiará la concentración de $s$?
\end{enumerate}
\end{frame}


%------------------------------------------------------------------SLIDE----
\begin{frame}
\begin{columns}
\begin{column}[T]{0.65\textwidth}
\begin{enumerate}
\item ¿En qué dirección aumentará lo más rápidamente posible la concentración de $s$?
\end{enumerate}
\end{column}
\begin{column}[T]{0.35\textwidth}
\structure{Datos}\\
$s(x,y,z)=\dfrac{\log(xy)}{z}$\\
Punto $P=(1,1,1)$
\end{column}
\end{columns}
\end{frame}


%------------------------------------------------------------------SLIDE----
\begin{frame}
\begin{columns}
\begin{column}[T]{0.65\textwidth}
\begin{enumerate}
\setcounter{enumi}{1}
\item Si empezamos a cambiar las concentraciones de $x$, $y$ y $z$ en la dirección del vector
$(2,1,0)$, es decir, $x$ crece el doble de $y$, y $z$ se mantiene constante, ¿cuánto cambiará la concentración de $s$?
\end{enumerate}
\end{column}
\begin{column}[T]{0.35\textwidth}
\structure{Datos}\\
$s(x,y,z)=\dfrac{\log(xy)}{z}$\\
Punto $P=(1,1,1)$\\
$\nabla s(1,1,1)=(1,1,0)$
\end{column}
\end{columns}
\end{frame}

\end{document}