% Author: Alfredo Sánchez Alberca (asalber@ceu.es}
% !TEX program = xelatex
\documentclass[aspectratio=149,10pt,t]{beamer}
%-------------------------------------------------------------------------------
% GENERAL PACKAGES
%-------------------------------------------------------------------------------
% Language
\usepackage{polyglossia}
\setmainlanguage{spanish}
% Maths
\usepackage{amsmath} % Math symbols and environments
\usepackage{amsfonts}
\usepackage{amssymb}
% Tables
\usepackage{array}
\usepackage{multirow}
\usepackage{booktabs}
% Graphics
\usepackage{graphicx}
\usepackage{tikz}
\usetikzlibrary{positioning}


% Colors
\definecolor{blueceu}{RGB}{0,164,227}
\definecolor{greenceu}{RGB}{194,205,24}
\definecolor{redceu}{RGB}{238,50,36}
\definecolor{purpleceu}{RGB}{169,78,145}
\definecolor{greyceu}{RGB}{117,117,97}
\definecolor{darkgrey}{RGB}{40,40,50}
\definecolor{softblueceu}{RGB}{193,225,246}
\setbeamercolor{structure}{fg=blueceu}
\setbeamercolor{normal text}{fg=darkgrey}
\hypersetup{colorlinks, urlcolor=purpleceu}

%-------------------------------------------------------------------------------
% FONTS
%-------------------------------------------------------------------------------
\usepackage{fontspec}
\setmainfont[Ligatures=TeX]{TeX Gyre Pagella}
\usepackage{unicode-math}
\setmathfont[math-style=ISO,bold-style=ISO,vargreek-shape=TeX]{TeX Gyre Pagella Math}
% Creative common icons
\usepackage[scale=1.5]{ccicons}

%-------------------------------------------------------------------------------
% CONFIGURATION
%-------------------------------------------------------------------------------
\setbeamersize{text margin left=.5cm, text margin right=.5cm} % Defines margin sizes
\beamertemplatenavigationsymbolsempty % Hide navitation bar
\usefonttheme[onlymath]{serif} % Math text in serif
\setbeamertemplate{blocks}[rounded] % Blocks with rounded corners
%\setbeamercolor{block title}{bg=RoyalBlue!10} % Color of block title
%\setbeamercolor{block body}{bg=RoyalBlue!10} % Color of block body

%-------------------------------------------------------------------------------
% COMMANDS
%-------------------------------------------------------------------------------
\newcommand{\sen}{\operatorname{sen}}
\newcommand{\tg}{\operatorname{tg}}
\newcommand{\arcsen}{\operatorname{arcsen}}
\newcommand{\arctg}{\operatorname{arctg}}

%-------------------------------------------------------------------------------
% DOCUMENT
%-------------------------------------------------------------------------------
\begin{document}
%---------------------------------------------------------------------SLIDE----
\begin{frame}[c]
\vspace{1.5cm}

\begin{center}
\structure{\LARGE {\textbf{Ejercicios de Cálculo}}}
\bigskip

\large
\begin{tabular}{rl}
Temas: & \structure{Derivadas en $n$ variables: Polinomios de Taylor}\\
Titulaciones: & \structure{Ciencias Ambientales, Medicina}
\end{tabular}

\bigskip
Alfredo Sánchez Alberca\\
\url{asalber@ceu.es}\\
\url{https://aprendeconalf.es}\\

\includegraphics[scale=0.2]{../img/logo_uspceu}

\bigskip
{\color{darkgrey}\ccbyncsaeu}
\end{center}
\end{frame}

%---------------------------------------------------------------------SLIDE----
\begin{frame}[c]
\Large
Un modelo ecológico explica el número de individuos de una población mediante la función
\[f(x,t)=\dfrac{e^t}{x}\]
donde $t$ es el tiempo y $x$ el número de predadores en la región.
Calcular el valor aproximado de individuos en la población para $t=0.1$ y $x=0.9$ utilizando el polinomio de Taylor de segundo grado de la función $f$ en el punto $(1,0)$.
\end{frame}


%------------------------------------------------------------------SLIDE----
\begin{frame}
\begin{columns}
\begin{column}[T]{0.8\textwidth}
Cálculo del polinomio de Taylor de segundo grado de la función $f$ en el punto $(1,0)$.
\end{column}
\begin{column}[T]{0.2\textwidth}
\structure{Datos}\\
$f(x,t)=\dfrac{e^t}{x}$\\
Punto $P=(1,0)$
\end{column}
\end{columns}
\end{frame}


%------------------------------------------------------------------SLIDE----
\begin{frame}
\begin{columns}
\begin{column}[T]{0.6\textwidth}
Calcular el valor aproximado de individuos en la población para $t=0.1$ y $x=0.9$ utilizando el polinomio de Taylor de segundo grado de la función $f$ en el punto $(1,0)$.
\end{column}
\begin{column}[T]{0.4\textwidth}
\structure{Datos}\\
$f(x,t)=\dfrac{e^t}{x}$\\
Punto $P=(1,0)$\\
$P^2_{f,P}(x,t)=3-3x+2t+x^2+\frac{t^2}{2}-xt$
\end{column}
\end{columns}
\end{frame}

\end{document}
