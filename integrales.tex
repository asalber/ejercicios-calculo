% Version control information:
\svnidlong
{$HeadURL: https://ejercicioscalculo.googlecode.com/svn/trunk/integrales.tex $}
{$LastChangedDate: 2008-02-19 18:25:16 +0100 (mar, 19 feb 2008) $}
{$LastChangedRevision: 4 $}
{$LastChangedBy: asalber $}
%\svnid{$Id: integrales.tex 4 2008-02-19 17:25:16Z asalber $
%
\newproblem*{int-1}{gen}{}
%ENUNCIADO
{Calcular por cambio de variable las integrales indefinidas siguientes:
\begin{multicols}{2}
\begin{enumerate}
\item $\dint e^{4x}\, dx$
\item $\dint \dfrac{x^{3}}{2+x^{8}}\, dx$
\item $\dint \dfrac{e^{\arcsen x}}{\sqrt{1-x^{2}}}\, dx$
\item $\dint \dfrac{1}{x\log x}\, dx$
\end{enumerate}
\end{multicols}
}


\newproblem*{int-2}{gen}{}
%ENUNCIADO
{Calcular las primitivas de las siguientes funciones:
\begin{multicols}{2}
\begin{enumerate}
\item $\dfrac{x+3}{\left( x^{2}+6x\right) ^{1/3}}$
\item $\dfrac{\arcsen x+x}{\sqrt{1-x^{2}}}$
\end{enumerate}
\end{multicols}
}


\newproblem*{int-3}{gen}{}
%ENUNCIADO
{Calcular las siguientes integrales por partes:
\begin{multicols}{2}
\begin{enumerate}
\item $\dint x^{5}\log x\, dx$
\item $\dint e^{x}\cos x\, dx$
\end{enumerate}
\end{multicols}
}


\newproblem*{int-4}{gen}{}
%ENUNCIADO
{Calcular las integrales:
\begin{multicols}{2}
\begin{enumerate}
\item $\dint x\sen 3x\,dx$
\item $\dint \dfrac{x}{\cos^{2}x}\,dx$
\end{enumerate}
\end{multicols}
}


\newproblem*{int-5}{gen}{}
%ENUNCIADO
{Calcular las siguientes integrales de funciones racionales:
\begin{multicols}{2}
\begin{enumerate}\setlength{\itemsep}{3mm}
\item $\dint \dfrac{x+1}{x^{2}-4x+8}\,dx$
\item $\dint \dfrac{x^{4}}{x^{4}-1}\,dx$
\item $\dint \dfrac{x}{x^{6}-1}\,dx$ 
\end{enumerate}
\end{multicols}
\textbf{Nota}: Para el apartado (c) hacer previamente la sustitución $x^{2}=t$.
}


\newproblem*{int-6}{gen}{}
%ENUNCIADO
{Calcular las integrales trigonométricas siguientes:
\begin{multicols}{2}
\begin{enumerate}\setlength{\itemsep}{3mm}
\item $\dint \dfrac{\sen x}{3\sen x+4\cos x}\,dx$
\item $\dint \sen^{4}x\,dx$
\item $\dint \dfrac{\tg x}{1+\cos x}\,dx$ 
\end{enumerate}
\end{multicols}
}


\newproblem*{int-7}{gen}{}
%ENUNCIADO
{Calcular las primitivas de las siguientes funciones:
\begin{multicols}{2}
\begin{enumerate}\setlength{\itemsep}{3mm}
\item $f(x)=x^3-3x^2+3$
\item $g(x)=\dfrac{x}{x^2-1}$
\item $h(x)=\dfrac{e^{1/x}}{x^2}$
\item $i(x)=\tg x$
\item $j(x)=\dfrac{x+3}{\sqrt{1-x^2}}$
\item $k(x)=\dfrac{x^2}{\sqrt{1-x^2}}$
\item $l(x)=(x^2-2x+5)e^{-x}$
\item $m(x)=\dfrac{\log x}{\sqrt x}$
\item $n(x)=3^x\cos x$
\item $o(x)=\sen(\log x)$
\item $p(x)=\dfrac{1}{x^3-4x^2+5x+2}$
\item $q(x)=\dfrac{1}{x^3+1}$
\end{enumerate}
\end{multicols}
}


\newproblem*{int-8}{gen}{}
%ENUNCIADO
{La función $e^{-x^2}$ no tiene una primitiva conocida. Calcular de manera aproximada $\dint_{-1/2}^{1/2} e^{-x^2}\, dx$ mediante aproximaciones de Taylor.
}


\newproblem*{int-9}{gen}{*}
%ENUNCIADO
{Dada la función
\[
f(x) = \ln x\left( {x^3  + 2x + 1} \right)
\]
Calcular $\dint_1^2 {f(x)\,dx}$.
}
