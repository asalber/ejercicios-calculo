\documentclass[aspectratio=149,10pt,xcolor=dvipsnames,t]{beamer}
%---------------------------------------------------------------------------
% GENERAL PACKAGES
%---------------------------------------------------------------------------
\usepackage[utf8x]{inputenc} % Sets UTF8 codification
\usepackage[T1]{fontenc}
\usepackage[spanish]{babel} % Sets spanish language
\usepackage{amsmath} % Math symbols and environments
\usepackage{amsfonts}
\usepackage{amssymb}

\usepackage{array}
\usepackage{multirow}
\usepackage{graphicx}
%\usepackage{url}
\usepackage{textcomp}

%---------------------------------------------------------------------------
% FONTS
%---------------------------------------------------------------------------
\usepackage{mathpazo} % Palatino

%---------------------------------------------------------------------------
% CONFIGURATION
%---------------------------------------------------------------------------
\setbeamersize{text margin left=.5cm, text margin right=.5cm} % Defines margin sizes 
\beamertemplatenavigationsymbolsempty % Hide navitation bar
\usefonttheme[onlymath]{serif} % Math text in serif
\setbeamertemplate{blocks}[rounded] % Blocks with rounded corners
%\setbeamercolor{block title}{bg=RoyalBlue!10} % Color of block title
%\setbeamercolor{block body}{bg=RoyalBlue!10} % Color of block body

\begin{document}
%---------------------------------------------------------------------SLIDE----
\begin{frame}[c]
\vspace{2cm}

\begin{center}
\structure{\LARGE {\textbf{Ejercicios de Cálculo}}}
\bigskip

\large
\begin{tabular}{rl}
Temas: & \structure{Derivadas en $n$ variables: Tangentes y Gradientes}\\
Titulaciones: & \structure{Todas}
\end{tabular}

\bigskip
Alfredo Sánchez Alberca (\texttt{asalber@ceu.es})

\includegraphics[scale=0.2]{img/logo_uspceu}

\biskip
\includegraphics[scale=0.07]{img/cc-logo}
\includegraphics[scale=0.2]{img/cc-by}
\includegraphics[scale=0.2]{img/cc-e}
\includegraphics[scale=0.2]{img/cc-c}\end{center}
\end{frame}

%---------------------------------------------------------------------SLIDE----
\begin{frame}[c]
La presión en la posición $(x,y,z)$ de un espacio es 
\[
f(x,y,z)= x^2+y^2-z^3
\]
y la trayectoria de un observador $A$ es 
\[
\begin{cases}
x=t\\
y=1\\
z=1/t
\end{cases}
t>0.
\]
Se pide:
\begin{enumerate}
\item Calcular la ecuación de la recta tangente a la trayectoria de $A$ en el punto $(1,1,1)$.
\item ¿Es la dirección de esta trayectoria al pasar por el punto $(1,1,1)$ aquella en la que el crecimiento de $f$
es máximo? 
Justificar la respuesta. 
\end{enumerate}
\end{frame}


%------------------------------------------------------------------SLIDE----
\begin{frame}
\begin{columns}
\begin{column}[T]{0.65\textwidth}
\begin{enumerate}
\item Calcular la ecuación de la recta tangente a la trayectoria de $A$ en el punto $(1,1,1)$.
\end{enumerate}
\end{column}
\begin{column}[T]{0.35\textwidth}
\structure{Datos}\\
$A:g(t)=(t,1,1/t)$ $t>0$
\end{column}
\end{columns}
\end{frame}


%------------------------------------------------------------------SLIDE----
\begin{frame}
\begin{columns}
\begin{column}[T]{0.65\textwidth}
\begin{enumerate}
\setcounter{enumi}{1}
\item ¿Es la dirección de esta trayectoria al pasar por el punto $(1,1,1)$ aquella en la que el crecimiento de $f$
es máximo? 
Justificar la respuesta. 
\end{enumerate}
\end{column}
\begin{column}[T]{0.35\textwidth}
\structure{Datos}\\
$A:g(t)=(t,1,1/t)$ $t>0$\\
$g'(1)=(1,0,-1)$\\
$f(x,y,z)= x^2+y^2-z^3$\\
\end{column}
\end{columns}
\end{frame}

\end{document}