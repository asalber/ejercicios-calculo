% Version control information:
%$HeadURL: https://ejercicioscalculo.googlecode.com/svn/trunk/derivadas_parametricas.tex $
%$LastChangedDate: 2010-01-28 20:28:03 +0100 (jue, 28 ene 2010) $
%$LastChangedRevision: 11 $
%$LastChangedBy: asalber $

\newproblem{derpar-1}{gen}{}
%ENUNCIADO
{Una partícula se mueve a lo largo de una curva $y=\cos(2x+1)$, siendo $x=t^2+1$ y $t$ el tiempo. ¿Con qué velocidad está desplazándose respecto a las direcciones vertical y horizontal cuando $t=2$?
}
%SOLUCIÓN
{Velocidad horizontal: $\frac{dx}{dt} = 2t$ y en el instante $t=2$, $\frac{dx}{dt}(t=2)=4$.\\
Velocidad vertical: $\frac{dy}{dt}=-\sen(2t^2+3)4t$ y en el instante $t=2$, $\frac{dy}{dt}=-8\sen 11$. 
}
%RESOLUCIÓN
{
}


\newproblem{derpar-2}{gen}{}
%ENUNCIADO
{Un punto se mueve en el plano siguiendo una trayectoria
\[ \left\{
\begin{array}{rcl}
    x & = & \sen t,\\
    y & = & t^2-1.
\end{array} \right.\]
Se pide:
\begin{enumerate}
    \item  Hallar la derivada de la función $y(x)$ (es decir,
    $\dfrac{dy}{dx}$) para los puntos $t=0$ y $t=2$.

    \item  Hallar la tangente a la trayectoria en el punto (0,-1).
\end{enumerate}
}
%SOLUCIÓN
{\begin{enumerate}
\item $\frac{dy}{dx} = \frac{2t}{\cos t}$. $\frac{y}{dx}(t=0) = 0$ y $\frac{dy}{dx}(t=2) = 4/\cos 2$.
\item Tantenge: $y=-1$. 
\end{enumerate}
}
%RESOLUCIÓN
{
}


\newproblem*{derpar-3}{gen}{}
%ENUNCIADO
{Una partícula se mueve a lo largo de la curva
\[ \left\{
 \begin{array}{rcl}
   x & = & 2\sen t,  \\
   y & = & \sqrt{3}\cos t,
 \end{array}\right.
  \]
donde $x$ e $y$ están medidos en metros y el tiempo $t$ en
segundos.
\begin{enumerate}
   \item  Hallar la ecuación de la recta tangente a la trayectoria
   en el punto (1,3/2).

   \item  ¿Con qué velocidad se mueve la partícula respecto a
   las direcciones vertical y horizontal en dicho punto?
\end{enumerate}
}


\newproblem{derpar-4}{gen}{*}
%ENUNCIADO
{Las coordenadas paramétricas de un punto material lanzado bajo un ángulo respecto al horizonte son
\[
\left\{
  \begin{array}{ll}
    x=v_0t \\
    y=-\frac{1}{2}gt^2
  \end{array}
\right.
\]
donde $t$ es el tiempo contado a partir del instante en que el punto llega a la posición más alta, $v_0$ es la velocidad horizontal en el instante $t=0$ y $g=9.8$ m$^2$/s es la aceleración de la gravedad. ¿En qué instante la magnitud de la velocidad horizontal será igual a la de la velocidad vertical? ¿Cuánto debería valer $v_0$ para que en dicho instante el punto haya recorrido 100 m horizontalmente? Calcular la ecuación de la recta tangente en dicho instante con el valor de $v_0$ calculado.
}
%SOLUCION
{
Las velocidades serán iguales en el instante $ t=\frac{v_0}{9.8}$. Para que en dicho instante el punto haya recorrido 10  m horizontalmente, la velocidad inicial debería ser $v_0 = 31.3$ m/s.

La ecuación de la recta tangente en dicho instante es $y =-x+50.14$.
}
%RESOLUCIÓN
{La velocidad horizontal es la derivada del espacio recorrido horizontalmente (componente $x$) con respecto al tiempo, es decir,
\[
\frac{dx}{dt} = \frac{d}{dt}(v_0t)=v_0.
\]
Del mismo modo, la velocidad vertical es la derivada del espacio recorrido verticalmente (componente $y$) en relación al tiempo,
\[
\frac{dy}{dt} = \frac{d}{dt}(-\frac{1}{2}gt^2)=-gt
\]
Para ver en qué instante ambas magnitudes serán iguales, las igualamos y resolvemos la ecuación:
\[
|\frac{dx}{dt}|=|\frac{dy}{dt}| \Leftrightarrow v_0 = gt \Leftrightarrow t=\frac{v_0}{g}=\frac{v_0}{9.8}.
\]

Para que en dicho instante el punto haya recorrido 100 m horizontalmente, debe cumplirse que $x(v_0/9.8)=100$, de lo que se deduce:
\[
x(v_0/9.8)=v_0\frac{v_0}{9.8} = \frac{v_0^2}{9.8}=100 \Leftrightarrow v_0^2 = 980 \Leftrightarrow v_0 = +\sqrt{980}= 31.3.
\]
Por tanto, el instance en cuestión es $t=v_0/9.8= 31.3/9.8 = 3.19$.

Por último, la ecuación de la recta tangente en dicho instante, para el valor de $v_0$ calculado es:
\[
y = y(3.19) + \frac{dy}{dx}(3.19) (x-x(3.19))
\]
Ya hemos visto que $x(3.19)=100$, y que en dicho instante la velocidad horizontal y vertical coinciden, de manera que
\[
\frac{dy}{dx}(3.19)=\frac{dy/dt}{dx/dt}=-1,
\]
de modo que sólo nos queda calcular el espacio vertical recorrido en dicho instante, que es
\[
y(3.19)=-\frac{1}{2}9.8\cdot 3.19^2= -49.86.
\]
Sustituyendo en la ecuación anterior llegamos a la recta tangente:
\[
y = -49.86-(x-100) \Leftrightarrow y=-x+50.14.
\]
}


\newproblem{derpar-5}{gen}{*}
%ENUNCIADO
{Dada la función paramétrica
\[
\left(
    x =\frac{(t-2)^2}{t^2+1},\, y=\dfrac{2t}{t^2+1}
\right)
\]
Calcular los valores máximos y mínimos de $x$ y de $y$. ¿En qué instante la tasa de crecimiento de $y$ coincide con la de $x$?
}
%SOLUCIÓN
{$\frac{dx}{dt}=\frac{4t^2-6t-4}{(t^2+1)^2}$. Puntos críticos: $t=-1/2$ (máximo) y $t=2$ (mínimo).\\
$\frac{dy}{dt}=\frac{-2t^2+2}{(t^2+1)^2}$. Puntos críticos: $t=-1$ (mínimo) y $t=1$ (máximo).\\
$\frac{dx}{dt}=\frac{dy}{dt}$ en los puntos $t=\frac{1-\sqrt 5}{2}$ y $t=\frac{1+\sqrt 5}{2}$.
}
%RESOLUCIÓN
{
}


\newproblem*{derpar-6}{gen}{*}
%ENUNCIADO
{Una mosca se mueve en un plano siguiendo la trayectoria
\[
\left\{
\begin{array}{lll}
x & = & \sen t
\; ,
\\
y & = & \cos t + t^2 - 1
\; .
\end{array}
\right.
\]
Se pide
\begin{enumerate}
\item Hallar la derivada de la función $y(x)$, es decir $dy/dx$,
en los puntos $t=0$ y $t=\pi/2$.
\item Hallar la ecuación de la recta tangente y normal a la trayectoria
en el punto $(x,y)=(0,0)$.
\end{enumerate}
}


\newproblem*{derpar-7}{gen}{*}
%ENUNCIADO
{Dadas las siguientes ecuaciones paramétricas:
\[
\left\{
\begin{array}{l}
x(t)=e^{at}t \\
y(t)=\ln t\cos (t-1)
\end{array}
\right.
\] 
calcular la ecuación de la recta tangente a la gráfica de $y$ como función de $x$ en el punto que corresponde a $t=1$.
}


\newproblem*{derpar-8}{gen}{*}
%ENUNCIADO
{La cantidad de árboles en un ecosistema, $a$, depende del tiempo según la expresión:
\[
a(t)=100\ln(t^2+1)
\]
Y la cantidad de un determinado parásito de los árboles, $p$, que también depende del tiempo, viene dada por:
\[
p(t) = \sqrt[3]{{t^2  + 2}}
\]
Y se pide:
\begin{enumerate}
\item Calcular el número de parásitos cuando el número de árboles sea 500.
\item La derivada del número de parásitos con respecto al número de árboles cuando el número de parásitos sea 3.
\end{enumerate}
}


\newproblem*{derpar-9}{gen}{*}
%ENUNCIADO
{Supongamos un ecosistema en el que hay una especie ``presa", $p$, y otra ``depredador", $d$, y que la cantidad de individuos de una y otra dependen del tiempo, en años, según las siguientes expresiones ($t>0$):
\[
\renewcommand{\arraystretch}{2.2}
\begin{array}{*{20}c}
   {p(t) = \dfrac{{\ln (t^2  + 1)}}{{t + 1}}}  \\
   {d(t) = te^{ - 2t} }  \\
\end{array}
\]
\begin{enumerate}
\item Calcular el número de presas y depredadores para tiempos muy grandes.
\item Calcular la derivada del número de presas con respecto a los depredadores cuando $d=2/e^4$.
\end{enumerate}
}


\newproblem{derpar-10}{gen}{*}
%ENUNCIADO
{Un punto se mueve en el plano siguiendo una trayectoria 
\[
\begin{cases}
x = \tg t,  \\ 
y = t^2-2t+3. \\
\end{cases}
\]

\begin{enumerate}
\item  Hallar $\frac{\partial y}{\partial x}$ en $t=0$.
\item  Hallar la tangente a la trayectoria en el punto $(0,3)$.
\end{enumerate}
}
%SOLUCIÓN
{\begin{enumerate}
\item $\dfrac{\partial y}{\partial x}(t) = \frac{2t-2}{1+\tg^2t}$ y $\dfrac{\partial y}{\partial x}(0) = -2$.
\item $y = 3-2x$.
\end{enumerate}
}
%RESOLUCIÓN
{Se trata de la ecuación de una trayectoria en coordenadas paramétricas.
\begin{enumerate}
\item  Aplicando la regla de la cadena se tiene que 
\[
\dfrac{\partial y}{\partial t} = \dfrac{\partial y}{\partial x}\dfrac{\partial x}{\partial t},
\] 
en consecuencia,
\[
\dfrac{\partial y}{\partial x}(t) = \frac{\partial y/\partial t}{\partial x/\partial t}(t)=\frac{2t-2}{1+\tg^2t}.
\]
En el punto $t=0$ tendremos
\[
\dfrac{\partial y}{\partial x}(0) = \frac{-2}{1+\tg^20} = -2.
\]

\item  La ecuación de la recta tangente a la trayectoria en el punto $(x(t_0),y(t_0))$ correspondiente al instante $t_0,$ viene dada por la expresión
\[
y-y(t_0) = \dfrac{\partial y}{\partial x}(t_0)(x-x(t_0)).
\]
Como el punto $(0,3)$ se alcanza precisamente en el instante $t=0$ tenemos que la ecuación de la recta tangente a la trayectoria en dicho instante es:
\[
y-y(0) = \dfrac{\partial y}{\partial x}(0)(x-x(0)),
\]
es decir,
\[
y-3 = -2(x-0),
\]
y simplificando obtenemos:
\[
y = 3-2x.
\]
\end{enumerate}
}



\newproblem{derpar-11}{gen}{}
%ENUNCIADO
{Hallar las ecuaciones de la rectas tangente y normal a la curva $C$ en el punto $P$ en cada uno de los casos siguientes:
\begin{enumerate}
\item $C: y=x^2$, $P=(0,0)$
\item $C: \begin{cases}
x=2\cos t,\\
y=2\sen t,
\end{cases}
$ $0\leq t\leq 2\pi$, $P=(0,2)$
\item $C:x^2+y^2=1$, $P=(\sqrt{2}/2),\sqrt{2}/2)$
\item $C:(x-1)^2+y^2=4$, $P=(3,0)$
\item $C:x^2-y^2=1$, $P=(1,0)$
\item $C:\begin{cases}
x=e^t\cos t,\\
y=e^t\sen t,
\end{cases}
$, $t\in \mathbb{R}$, $P=(1,0)$
\end{enumerate}
}
%SOLUCIÓN
{
}
%RESOLUCIÓN
{
}


\newproblem{derpar-12}{gen}{}
%ENUNCIADO
{Hallar al ecuación de la recta tangente y el plano normal a la línea
\[
C: 
\begin{cases}
x=\cos t \\
y=\sen t\\
z= t,
\end{cases}
\quad t\in \mathbb{R},
\]
en el punto $P=(1,0,0)$.
}
%SOLUCIÓN
{
}
%RESOLUCIÓN
{
}


\newproblem{derpar-13}{gen}{}
%ENUNCIADO
{Una trayectoria pasa por el punto $(3,6,5)$ en el instante $t=0$ con velocidad $\mathbf{i}-\mathbf{k}$. 
Hallar la ecuación del plano normal y de la recta tangente en ese instante.
}
%SOLUCIÓN
{
}
%RESOLUCIÓN
{
}


\newproblem{derpar-14}{gen}{}
%ENUNCIADO
{Una partícula sigue la trayectoria
\[
\begin{cases}
x=e^t,\\
y=e^{-t},\\
z=\cos t,
\end{cases}
\quad t\in \mathbb{R}
\]
hasta que se sale por la tangente en el instante $t=1$. ¿Dónde estará en el instante $t=3$?
}
%SOLUCIÓN
{
}
%RESOLUCIÓN
{
}