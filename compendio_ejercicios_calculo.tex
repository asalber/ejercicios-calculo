% Desarrollado por Alfredo Sánchez Alberca (asalber@gmail.com)
\documentclass[a4paper,titlepage]{article}
%%%%%%%%%%%%%%%%%%%%%%%%%%%%
\usepackage{svn-multi}
% Version control information:
\svnidlong
{$HeadURL: https://ejercicioscalculo.googlecode.com/svn/trunk/compendio_ejercicios_calculo.tex $}
{$LastChangedDate: 2010-01-28 20:28:03 +0100 (jue, 28 ene 2010) $}
{$LastChangedRevision: 11 $}
{$LastChangedBy: asalber $}
%\svnid{$Id: compendio_ejercicios_calculo.tex 11 2010-01-28 19:28:03Z asalber $}

\usepackage[spanish]{babel}
\usepackage[utf8x]{inputenc}
\usepackage{amsmath}
\usepackage{amssymb}
\usepackage{macros}
\usepackage{graphicx}
\usepackage{eurosym}
\usepackage{multicol}
\usepackage{fancybox}
\usepackage{enumitem}
\usepackage{probsoln-alf}
%\usepackage{times}
\usepackage[colorlinks=true]{hyperref}
\hypersetup{pdfautor={Alfredo S\'anchez Alberca (asalber@ceu.es)}, pdftitle={Ejercicios de C\'alculo}} 
\usepackage{url}
\usepackage[top=3cm, bottom=3cm, left=2.54cm, right=2.54cm, marginparwidth=2mm]{geometry}
\usepackage{fancyhdr}
\pagestyle{fancy}

\lhead{\textsc{\textcolor[rgb]{0.00,0.00,0.50}{Universidad San Pablo CEU}}} \rhead{\textsl{\textsf{\textcolor[rgb]{0.00,0.00,0.50}{Departamento de Métodos Cuantitativos}}}}
\renewcommand{\headrulewidth}{0pt}
\renewcommand{\floatpagefraction}{.8}
\renewcommand{\textfraction}{.1}

\pdfinfo{/CreationDate (D:\svnpdfdate)}
\svnRegisterAuthor{alf}{Alfredo Sánchez Alberca}

\makeatletter
\let\savees@listquot\es@listquot
\def\es@listquot{\protect\savees@listquot}
\makeatletter

\reversemarginpar
%\showshortanswers
\showanswers
%\PSNrandseed{2007}


\begin{document}
\sloppy

\begin{titlepage}
\vspace*{5cm}
\begin{center}
{\huge \bf COMPENDIO DE EJERCICIOS DE CÁLCULO\par}
\vspace{0.5cm}
{\large \noindent \textbf{Autores}: \\
   Santiago Angulo Díaz-Parreño (\url{sangulo@ceu.es})\\
   Anselmo Romero Limón (\url{arlimon@ceu.es})\\
   Alfredo Sánchez Alberca (\url{asalber@ceu.es})
}

\vspace{1cm}
 \includegraphics[scale=0.3]{img/logo_uspceu_01}
\end{center}
 \vfill
 \flushleft\sffamily
 Version control information:\\
 Head URL: \url{\svnkw{HeadURL}}\\
 Last changed date: \svndate\\
 Last changes revision: \svnrev\\
 Version: \svnFullRevision*{\svnrev}\\
 Last changed by: \svnFullAuthor*{\svnauthor}\\
\end{titlepage}

\setcounter{tocdepth}{2}
\tableofcontents
\newpage

\section{Domínios e Imágenes}
\begin{enumerate}[leftmargin=*]
\selectallproblems{dominios}
\end{enumerate}

\section{Límites}
\begin{enumerate}[leftmargin=*]
\selectallproblems{limites}
\end{enumerate}

\section{Continuidad}
\begin{enumerate}[leftmargin=*]
\selectallproblems{continuidad}
\end{enumerate}

\section{Derivadas}
\begin{enumerate}[leftmargin=*]
\selectallproblems{derivadas}
\end{enumerate}

\section{Diferencial}
\begin{enumerate}[leftmargin=*]
\selectallproblems{diferencial}
\end{enumerate}

\section{Derivadas Implícitas}
\begin{enumerate}[leftmargin=*]
\selectallproblems{derivadas_implicitas}
\end{enumerate}

\section{Derivadas Paramétricas}
\begin{enumerate}[leftmargin=*]
\selectallproblems{derivadas_parametricas}
\end{enumerate}

\section{Teorema de Rolle y del Valor Medio}
\begin{enumerate}[leftmargin=*]
\selectallproblems{rolle_valor_medio}
\end{enumerate}

\section{Crecimiento, Concavidad y Extremos Relativos}
\begin{enumerate}[leftmargin=*]
\selectallproblems{extremos}
\end{enumerate}

\section{Polinomios de Taylor}
\begin{enumerate}[leftmargin=*]
\selectallproblems{taylor}
\end{enumerate}

\section{Derivadas Parciales}
\begin{enumerate}[leftmargin=*]
\selectallproblems{derivadas_parciales}
\end{enumerate}

\section{Integrales}
\begin{enumerate}[leftmargin=*]
\selectallproblems{integrales}
\end{enumerate}

\section{Integrales Impropias}
\begin{enumerate}[leftmargin=*]
\selectallproblems{integrales_impropias}
\end{enumerate}

\section{Aplicaciones de las integrales}
\begin{enumerate}[leftmargin=*]
\selectallproblems{integrales_aplicaciones}
\end{enumerate}

\section{Ecuaciones Diferenciales Ordinarias de Variables Separables}
\begin{enumerate}[leftmargin=*]
\selectallproblems{edo_separables}
\end{enumerate}

\section{Medida y Error}
\begin{enumerate}[leftmargin=*]
\selectallproblems{medida_error}
\end{enumerate}

\vspace{2cm}

\textsc{Nota}: Los problemas marcados con una estrella ($\bigstar$) son problemas de
exámenes de otros años.
\end{document}
