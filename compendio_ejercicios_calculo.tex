%!TEX program = pdflatex
% Author Alfredo Sánchez Alberca (asalber@ceu.es)

\documentclass[a4paper,titlepage]{article}
%===============================================
\usepackage[spanish]{babel}
\usepackage[utf8]{inputenc}
\usepackage[top=3cm, bottom=3cm, left=2.54cm, right=2.54cm, marginparwidth=2mm]{geometry}

% COLORS
\usepackage[table]{xcolor}
\definecolor{color1}{RGB}{5,161,230} % Light blue
\definecolor{color2}{RGB}{238,50,36} % Red
\definecolor{color3}{RGB}{0,205,0} % Light Green
\definecolor{ocre}{RGB}{243,102,25} % Define the orange color used for highlighting throughout the book
\definecolor{blueceu}{RGB}{5,161,230} % Blue color of CEU logo
\definecolor{greenceu}{RGB}{185,209,16} % Green color of CEU logo
\definecolor{redceu}{RGB}{238,50,36} % Red color of CEU logo
\definecolor{grayceu}{RGB}{111,107,83} % Gray color of CEU logo
\definecolor{chaptergrey}{RGB}{5,161,230} % Blue color of CEU logo

% MATH
\usepackage{amsmath}
\usepackage{amssymb}
\usepackage{amsthm}
\DeclareMathOperator{\sen}{sen}
\DeclareMathOperator{\arcsen}{arcsen}
\DeclareMathOperator{\tg}{tg}
\DeclareMathOperator{\arctg}{arctg}

% GRAPHICS
\usepackage{graphicx}
\usepackage{tikz}
\usepackage{pgfplots}
\usetikzlibrary{shapes, calc, decorations, quotes, angles, rightangles}
\usepackage{tkz-euclide}

\usepackage{multicol}
\usepackage[inline]{enumitem}
\usepackage{fancyhdr}
\pagestyle{fancy}
\lhead{\textsc{\textcolor{blueceu}{Universidad CEU San Pablo}}}
\rhead{\textsl{\textsf{\textcolor{blueceu}{Departamento de Matemática Aplicada y Estadística}}}}
\renewcommand{\headrulewidth}{0pt}

\usepackage{booktabs}


% SECTIONS
\usepackage{titlesec}
\titleformat*{\section}{\normalfont\Large\bfseries\color{color1}}


% % SOLUTIONS
% \newif\ifsolution
% % \solutiontrue  % Comment to hide solutions
%
% \newtheoremstyle{solution} % Theorem style name
% {-5pt} % Space above
% {7pt} % Space below
% {\normalfont} % Body font
% {-28pt} % Indent amount
% {\bf} % Theorem head font
% {\kern-11.5pt} % Punctuation after theorem head
% {19pt} % Space after theorem head
% {\begin{tikzpicture}
% \draw (0,0) node [fill=color2, xshift=4mm, inner
% sep=2pt]{\includegraphics[scale=0.3]{img/bulb}};
% \end{tikzpicture}}
%
% \theoremstyle{solution}
% \newtheorem{solutionT}{Solution}
%
% \RequirePackage[framemethod=default]{mdframed}
%
% % Solution box
% \newmdenv[skipabove=7pt,
% skipbelow=10pt,
% rightline=false,
% leftline=true,
% topline=false,
% bottomline=false,
% linecolor=color2,
% backgroundcolor=black!5,
% innerleftmargin=5pt,
% innerrightmargin=5pt,
% innertopmargin=4pt,
% innerbottommargin=5pt,
% leftmargin=0pt,
% rightmargin=0pt,
% linewidth=4pt]{solBox}
%
% \usepackage{comment}
% \ifsolution
%   \newenvironment{sol}{\ifsolution\begin{solBox}\begin{solutionT}}{\end{solutionT}\end{solBox}\fi}
% \else
%   \excludecomment{sol}
% \fi

% PDF
\usepackage[colorlinks=true]{hyperref}
\hypersetup{pdfauthor={Alfredo Sánchez Alberca (asalber@ceu.es)}, pdftitle={Ejercicios de Cálculo} }
\usepackage{url}

\renewcommand{\floatpagefraction}{.8}
\renewcommand{\textfraction}{.1}

% SOLUTIONS SETTINGS
\usepackage{probsoln-alf}
\reversemarginpar
\showshortanswers
%\showanswers
%\PSNrandseed{2007}

\begin{document}
\sloppy

\title{\vskip 2cm
\Huge \textbf{\textsf{\quad \textcolor{blueceu}{EJERCICIOS DE CÁLCULO}\quad}}\\
   \vskip 1cm
\Large \sffamily
\begin{tabular}{rl}
\textcolor{blueceu}{Asignatura:} & Matemáticas\\
\textcolor{blueceu}{Curso:} & Primero\\
\textcolor{blueceu}{Grado:} &  Todos\\
\textcolor{blueceu}{Año:} & 2020-2021\\
\textcolor{blueceu}{Autor:} & Alfredo S\'anchez Alberca (\url{asalber@ceu.es})
\end{tabular}
}

\author{}
\date{\includegraphics[scale=0.3]{img/logo_uspceu_01}}

\maketitle
\newpage
\tableofcontents
\newpage

% \section{Domínios e imágenes}
% \begin{enumerate}[leftmargin=*]
% \selectallproblems{capitulos/dominios}
% \end{enumerate}

% \section{Límites}
% \begin{enumerate}[leftmargin=*]
% \selectallproblems{capitulos/limites}
% \end{enumerate}

% \section{Continuidad}
% \begin{enumerate}[leftmargin=*]
% \selectallproblems{capitulos/continuidad}
% \end{enumerate}

\section{Derivadas}
\begin{enumerate}[leftmargin=*]
\selectallproblems{capitulos/derivadas}
\end{enumerate}

\section{Diferencial}
\begin{enumerate}[leftmargin=*]
\selectallproblems{capitulos/diferencial}
\end{enumerate}

% \section{Derivadas paramétricas}
% \begin{enumerate}[leftmargin=*]
% \selectallproblems{capitulos/derivadas_parametricas}
% \end{enumerate}

% \section{Trayectorias}
% \begin{enumerate}[leftmargin=*]
% \selectallproblems{capitulos/derivadas_trayectorias}
% \end{enumerate}

% \section{Teorema de Rolle y del valor medio}
% \begin{enumerate}[leftmargin=*]
% \selectallproblems{capitulos/rolle_valor_medio}
% \end{enumerate}

\section{Crecimiento, concavidad y extremos relativos}
\begin{enumerate}[leftmargin=*]
\selectallproblems{capitulos/extremos}
\end{enumerate}

% \section{Polinomios de Taylor}
% \begin{enumerate}[leftmargin=*]
% \selectallproblems{capitulos/taylor}
% \end{enumerate}

\section{Derivadas parciales}
\begin{enumerate}[leftmargin=*]
\selectallproblems{capitulos/derivadas_parciales}
\end{enumerate}

% \section{Superficies}
% \begin{enumerate}[leftmargin=*]
% \selectallproblems{capitulos/derivadas_superficies}
% \end{enumerate}

% \section{Derivadas implícitas}
% \begin{enumerate}[leftmargin=*]
% \selectallproblems{capitulos/derivadas_implicitas}
% \end{enumerate}

% \section{Derivadas implícitas en varias variables}
% \begin{enumerate}[leftmargin=*]
% \selectallproblems{capitulos/derivadas_parciales}
% \end{enumerate}

% \section{Extremos en varias variables}
% \begin{enumerate}[leftmargin=*]
% \selectallproblems{capitulos/extremos_n_variables}
% \end{enumerate}

% \section{Polinomios de Taylor en varias variables}
% \begin{enumerate}[leftmargin=*]
% \selectallproblems{capitulos/taylor_n_variables}
% \end{enumerate}

% \section{Integrales}
% \begin{enumerate}[leftmargin=*]
% \selectallproblems{capitulos/integrales}
% \end{enumerate}

% \section{Integrales impropias}
% \begin{enumerate}[leftmargin=*]
% \selectallproblems{capitulos/integrales_impropias}
% \end{enumerate}

% \section{Aplicaciones de las integrales}
% \begin{enumerate}[leftmargin=*]
% \selectallproblems{capitulos/integrales_aplicaciones}
% \end{enumerate}

\section{Ecuaciones diferenciales ordinarias de variables separables}
\begin{enumerate}[leftmargin=*]
\selectallproblems{capitulos/edo_separables}
\end{enumerate}

% \section{Ecuaciones diferenciales ordinarias homogéneas}
% \begin{enumerate}[leftmargin=*]
% \selectallproblems{capitulos/edo_homogeneas}
% \end{enumerate}

% \section{Ecuaciones diferenciales ordinarias lineales}
% \begin{enumerate}[leftmargin=*]
% \selectallproblems{capitulos/edo_lineales}
% \end{enumerate}

% \section{Medida y error}
% \begin{enumerate}[leftmargin=*]
% \selectallproblems{capitulos/medida_error}
% \end{enumerate}

\vspace{2cm}

\textsc{Nota}: Los problemas marcados con una estrella ($\bigstar$) son problemas de
exámenes de otros años.
\end{document}
