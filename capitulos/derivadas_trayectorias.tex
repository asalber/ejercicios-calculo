% Autor: Alfredo Sánchez Alberca (asalber@ceu.es)

\newproblem{dertray-1}{gen}{*}
%ENUNCIADO
{La presión en la posición $(x,y,z)$ de un espacio es 
\[
f(x,y,z)= x^2+y^2-z^3
\]
y la trayectoria de un observador $A$ es 
\[
\begin{cases}
x=t\\
y=1\\
z=1/t
\end{cases}
t>0.
\]
Se pide:
\begin{enumerate}
\item Calcular la ecuación de la recta tangente a la trayectoria de $A$ en el punto $(1,1,1)$.
\item ¿Es la dirección de esta trayectoria al pasar por el punto $(1,1,1)$ aquella en la que el crecimiento de $g$
es máximo? 
Justificar la respuesta. 
\end{enumerate}
}
%SOLUCIÓN
{\begin{enumerate}
\item $l:(1+t,1,1-t)$.
\item La dirección de la trayectoria $A$ en $(1,1,1)$ es $(1,0,-1)$ y la dirección de máximo crecimiento de $g$ en
$(1,1,1)$ es $(2,2,-3)$, luego no coinciden. 
\end{enumerate}
}
%RESOLUCIÓN
{
}
