% Autor: Alfredo Sánchez Alberca (asalber@ceu.es)

\newproblem{geo-1}{gen}{}
%ENUNCIADO
{Calcular la distancia entre los puntos del plano $P=(6,-2)$ y $Q=(3,2)$.
}
%SOLUCIÓN
{5.
}
%RESOLUCIÓN
{
}


\newproblem{geo-2}{gen}{}
%ENUNCIADO
{Calcular las ecuaciones vectorial, analítica y general de la recta que pasa por el punto $P=(3,-2)$ con la dirección del vector $\mathbb{v}=(2,5)$.
}
%SOLUCIÓN
{Ecuación vectorial: $(3+2t, -2+5t)$.\\
Ecuación analítica: $y=\frac{5}{2}x-\frac{19}{2}$.\\
Ecuación general: $2y-5x+19=0$.
}
%RESOLUCIÓN
{
}


\newproblem{geo-3}{gen}{}
%ENUNCIADO
{Calcular la ecuación de la recta que pasa por el punto $P=(4,-1)$ y es paralela al eje $X$.
}
%SOLUCIÓN
{$y=-1$.
}
%RESOLUCIÓN
{
}


\newproblem{geo-4}{gen}{}
%ENUNCIADO
{Calcular la ecuación de la recta que corta al eje $X$ en 2 y al eje $Y$ en $-4$.
}
%SOLUCIÓN
{$y=2x-4$.
}
%RESOLUCIÓN
{
}


\newproblem{geo-5}{gen}{}
%ENUNCIADO
{Calcular la distancia del punto $P=(2,-3)$ a la recta de ecuación $3x-4y+2=0$.
}
%SOLUCIÓN
{4.
}
%RESOLUCIÓN
{
}


\newproblem{geo-6}{gen}{}
%ENUNCIADO
{Calcular el área del triángulo de vértices $A=(0,-3)$, $B=(5,0)$ y $C=(0,3)$.
}
%SOLUCIÓN
{15.
}
%RESOLUCIÓN
{
}
