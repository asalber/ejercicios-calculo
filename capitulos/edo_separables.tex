% Autor: Alfredo Sánchez Alberca (asalber@ceu.es)

\newproblem{edosep-1}{gen}{}
%ENUNCIADO
{Integrar las siguientes ecuaciones de variables separables:
\begin{enumerate}
\item $x\sqrt{1-y^2}+y\sqrt{1-x^2}y'=0$ con la condición inicial $y(0)=1$.
\item $(1+e^x)yy'=e^y$ con la condición inicial $y(0)=0$.
\item e$^y(1+x^2)y'-2x(1+\mbox{e}^y)=0$.
\item $y-xy'=a(1+x^2y')$.
\end{enumerate}
}
%SOLUCIÓN
{
\begin{enumerate}
\item $-\sqrt{1-y^2}=\sqrt{1-x^2}-1$.
\item $e^{-y}(y+1)=\log(1+e^x)-x-\log 2+1$.
\item $y=\log(C(1+x^2)-1)$.
\item $y=C\frac{x}{ax+1}+a$.
\end{enumerate}
}
%RESOLUCIÓN
{}


\newproblem{edosep-2}{qui}{}
%ENUNCIADO
{La desintegración radioactiva está regida por la ecuación
diferencial
\[
\frac{\partial x}{\partial t}+ax=0,
\]
donde $x$ es la masa, $t$ el tiempo y $a$ es una constante positiva. La vida media $T$ es el tiempo durante el cual la
masa se desintegra a la mitad de su valor inicial. Expresar $T$ en función de $a$ y evaluar $a$ para el isótopo de
uranio $U^{238}$, para el cual $T=4'5\cdot10^9$ años. } 
%SOLUCIÓN
{$T = \frac{\log 2}{a}$ y $a=1.54\cdot 10^{-10}$ años$^{-1}$.
}
%RESOLUCIÓN
{}



\newproblem{edosep-3}{qui}{}
%ENUNCIADO
{El azúcar se disuelve en el agua con una velocidad proporcional a la cantidad que queda por disolver. Si inicialmente
había 13.6 kg de azúcar y al cabo de 4 horas quedan sin disolver 4.5 kg, ¿cuánto tardará en disolverse el 95\% del
azúcar contando desde el instante inicial? }
%SOLUCIÓN
{$C(t)= 13.6e^{-0.276 t}$ y el instante en que se habrá disuelto el 95\% del azúcar es $t_0=10.854$ horas.
}
%RESOLUCIÓN
{}



\newproblem{edosep-4}{qui}{*}
%ENUNCIADO
{Una reacción química sigue la siguiente ecuación diferencial
\[
y'-2y=4,
\]
donde $y=f(t)$ es la concentración de oxígeno en el instante $t$ (medido en segundos). Si la concentración de oxígeno
al comienzo de la reacción era nula, ¿cuál será la concentración (mg/lt) a los 3 segundos? ¿En qué instante la
concentración de oxígeno será de 200 mg/lt?}
%SOLUCIÓN
{$y(t)=2e^{2t}-2$. La concentración a los tres segundos será $y(3)=804$ mg/lt y el instante en que la concentración de
oxígeno será de 200 mg/lt es $t_0=2.3076$ s.}
%RESOLUCIÓN
{}



\newproblem{edosep-5}{med}{}
%ENUNCIADO
{La sala de disección de un forense se mantiene fría a una temperatura constante de $5^\circ C$. Mientras se encontraba
realizando la autopsia de una víctima de asesinato, el forense es asesinado y el cuerpo de la víctima robado. A las 10
de la mañana el ayudante el forense descubre su cadáver a una temperatura de $23^\circ C$ y llama a la policía. A medio
día llega ésta y comprueba que la temperatura del cadáver es de $18'5^\circ C$. Supuesto que el forense tenía en vida
una temperatura normal de $37^\circ C$, ¿a qué hora fue asesinado?}
%SOLUCIÓN
{Fue asesinado a las 6 de la mañana aproximadamente.
}
%RESOLUCIÓN
{}



\newproblem{edosep-6}{qui}{*}
%ENUNCIADO
{Sea la siguiente ecuación diferencial que relaciona la temperatura y el tiempo en un determinado sistema físico:
\[
x't^2-x't+x'-2xt+x=0,
\]
siendo $x$ la temperatura expresada en Kelvins y $t$ el tiempo en segundos. 

Sabiendo que la temperatura en el instante inicial del experimento es 100 K, calcular la temperatura en función del tiempo, y dar la temperatura del sistema físico tres segundos después de comenzar el experimento.}
%SOLUCIÓN
{$x(t)=100(t^2-t+1)$ y la temperatura del sistema a los tres segundos de comenzar el experimento es $x(3)=700$ K.
}
%RESOLUCIÓN
{En primer lugar, intentamos separar las variables para ver si se trata de una ecuación de variables separables:
\[\renewcommand{\arraystretch}{2}
\begin{array}{c}
x't^2  - x't + x' - 2xt + x = 0 \Leftrightarrow x'(t^2-t+1)+x(-2t+1)=0 \Leftrightarrow\\
\Leftrightarrow \dfrac{dx}{dt}(t^2-t+1)=x(2t-1) \Leftrightarrow \dfrac{dx}{x}=\dfrac{2t-1}{t^2-t+1} dt
\end{array}
\]
Así pues, se trata de una ecuación diferencial ordinaria de variables separables. Integándo en ambos lados de la
ecuación tenemos  
\[\renewcommand{\arraystretch}{2}
\begin{array}{c}
\dint \dfrac{dx}{x}=\dint \dfrac{2t-1}{t^2-t+1}\,dt \Leftrightarrow \log |x|= \log |t^2-t+1|+C \Leftrightarrow \\
\Leftrightarrow \exp(\log |x| )= \exp(\log |t^2-t+1|+C) \Leftrightarrow x=(t^2-t+1)e^C,
\end{array}
\]
Y renombrando $e^C$ como una constante $C$, llegamos a la solución general de la ecuación
\[
x(t)=C(t^2-t+1).
\]

Imponiendo ahora la condición inicial $x(0)=100 K$, tenemos
\[
x(0)=C(0^2-0+1)=C=100,
\]
de manera que la solución particular es
\[
x(t)=100(t^2-t+1).
\]

Por último, la temperatura del sistema a los 3 segundos de comenzar el experimento será
\[
x(3)=100(3^2-3+1)=700\textrm{ K}.
\]
}


\newproblem{edosep-7}{far}{*}
%ENUNCIADO
{Se tiene un medicamento en un frigorífico a 2ºC, y se debe administrar a 15ºC. A las 9 h se saca el medicamento del
frigorífico y se coloca en una habitación que se encuentra a 22ºC. A las 10 h se observa que el medicamento está a
10ºC. Suponiendo que la velocidad de calentamiento es proporcional a la diferencia entre la temperatura del medicamento
y la del ambiente, ¿en qué hora se deberá administrar dicho medicamento?}
%SOLUCIÓN
{A las $11.06$ horas.
}
%RESOLUCIÓN
{La ecuación diferencial que rige el enfriamiento de los cuerpos es
\[
\frac{dT}{dt}k(T-T_a),
\]
donde $T$ es la temperatura del cuerpo, $t$ es el tiempo, $T_a$ es la
temperatura del medio que se supone constante y en este caso es 22ºC, y $k$ es
una constante de proporcionalidad.

Como se trata de una ecuación de variables separables, procedemos a separar las
variables:
\[
\frac{dT}{dt}=k(T-22) \Leftrightarrow \frac{dT}{T-22}=kdt,
\]
e integrar:
\[
\int \frac{dT}{T-22}=\int kdt \Leftrightarrow \log|T-22|=kt+C \Leftrightarrow
T-22 = e^{kt+C}=e^{kt}e^C
\]
y reescribiendo $e^C$ como una constante $C$ llegamos a la solución general:
\[T(t)=Ce^{kt}+22.\]

Imponemos ahora las condiciones iniciales para llegar a la solución particular.
En primer lugar, sabemos que en el instante en que se saca el fármaco
del frigorífico la temperatura del mismo era de 2ºC. Fijaremos dicho instante
como el instante inicial $t=0$ (que en realidad son las 9 h). Así pues, se
tiene:
\[
T(0)=2 \Leftrightarrow Ce^{k\cdot 0}+22 = 2 \Leftrightarrow C = -20
\]

En segundo lugar, transcurrida una hora del instante inicial ($t=1$), la
temperatura del fármaco era de 10ºC, de manera que se tiene:
\[
T(1)=10 \Leftrightarrow -20e^{k\cdot 1}+22 =10 \Leftrightarrow -20e^k = -12
\Leftrightarrow e^k = 12/20 \Leftrightarrow k=\log (12/20)=-0.51.
\]
Por consiguiente, llegamos a la solución particular
\[
T(t)= 22-20e^{-0.51t}
\]

Para terminar calculamos el tiempo que debe transcurrir hasta que el medicamento
alcance los 15ºC a que debe administrarse:
\[
T(t)=15 \Leftrightarrow 22-20e^{-0.51t}=15 \Leftrightarrow
e^{-0.51t}=\frac{22-15}{20} \Leftrightarrow t=\frac{\log(7/20)}{-0.51}=2.06
\mbox{ h}.
\]
Por tanto, debe administrarse unas $2.06$ horas después del instante inicial,
aproximadamente a las $11.06$ h.
}


\newproblem{edosep-8}{qui}{*}
%ENUNCIADO
{Una cámara de 500 l está llena de aire en condiciones normales
cuando comienza a entrar oxí­geno puro a razón de 5 litros por minuto. 
Al mismo tiempo se extrae la misma cantidad de la mezcla uniforme. ¿Qué
concentración de oxí­geno habrá a los 10 minutos? Suponiendo que una
concentración de oxí­geno en el aire superior a 0.5 gr/l puede ser perjudicial,
¿cuándo será peligroso respirar el aire de la cámara? 

\textbf{Nota}: La concentración de oxí­geno en el aire en condiciones normales es
de $0.15$ gr/l, mientras que en el oxí­geno puro es de $0.71$ gr/l. La ecuación
diferencial que explica el fenómeno es
\[
\frac{dx}{dt}=c_ev_e-c_sv_s
\]
donde $x$ es la cantidad de oxí­geno en la cámara en el instante $t$, $c_e$ y
$c_s$ son las concentraciones de oxí­geno en el aire que entra y sale
respectivamente, y $v_e$ y $v_s$ son las velocidades de entrada y salida del
aire. 
}
%SOLUCIÓN
{
}
%RESOLUCIÓN
{}



\newproblem{edosep-9}{qui}{*}
%ENUNCIADO
{Sabiendo que el núcleo del Polonio 210 es radiactivo y que su tiempo de semidesintegración (tiempo necesario para que
la cantidad inicial se reduzca a la mitad) es de 138 días:
\begin{enumerate}
\item ¿Qué cantidad inicial de Polonio 210 teníamos si al cabo de 100 días nos quedan 20 gramos?

\item ¿Qué tiempo tendrá que transcurrir para que se desintegre un 10\% de la masa inicial?
\end{enumerate}
}
%SOLUCIÓN
{
}
%RESOLUCIÓN
{}


\newproblem{edosep-10}{amb}{*}
%ENUNCIADO
{Estudios científicos han demostrado que la longitud en función
del tiempo de muchas especies, entre ellas las de gran variedad de
peces, viene dada por la ecuación de Bertalanffy:
\[
\frac{{dL}}{{dt}} = k\left( {L_f  - L(t)} \right)
\]
donde $L_f$ es la longitud de la especie al final del periodo de
crecimiento, y $k$ es una constante. Suponiendo que la longitud de
una especie de peces al final de su periodo de crecimiento es de un
metro, y que con uno y dos meses mide, respectivamente, 20 y 40 cm:
\begin{enumerate}
\item ¿Cuál será la longitud de esa especie para todo tiempo $t$?
\item ¿Cuánto tiempo debe transcurrir desde su nacimiento hasta que la longitud sea de 95 cm?
\end{enumerate}
}
%SOLUCIÓN
{\begin{enumerate}
\item $L(t)=-1.0667e^{-0.2877t}+1$.
\item $t_0=10.637$ años. 
\end{enumerate}
}
%RESOLUCIÓN
{}


\newproblem{edosep-11}{qui}{*}
%ENUNCIADO
{La cantidad de masa de un determinado reactivo de una reacción
química, $M$, en gramos, es función del tiempo, en segundo, y se
rige mediante la siguiente ecuación diferencial:
\[
M' - (a + b)M = 0
\]
donde $a$ y $b$ son constantes. Si inicialmente tenemos 20 gramos de reactivo, al cabo de 10 segundos tenemos 40 gramos, calcular:
\begin{enumerate}
\item La cantidad de reactivo para todo tiempo $t$.
\item La cantidad de reactivo al cabo de medio minuto.
\item ¿Cuando será la cantidad de reactivo 100 g?
\end{enumerate}
}
%SOLUCIÓN
{\begin{enumerate}
\item $M(t) = 20\;e^{\frac{{\ln 2}}{{10}}t}.$
\item $M(30) = 160$ gr.
\item $t_0  = 23.22$ s.
\end{enumerate}
}
%RESOLUCIÓN
{
\begin{enumerate}
\item Para calcular la masa $M$ para todo tiempo $t$ debemos
resolver la ecuación diferencial separable del enunciado.
Procediendo a su separación obtenemos:
\[
M' - (a + b)M = 0 \Leftrightarrow \frac{{dM}}{{dt}} = (a + b)M
\Leftrightarrow \frac{{dM}}{M} = (a + b)dt
\]
e integrando la ecuación separada:
\[
\int {\frac{{dM}}{M}}  = \int {(a + b)dt}  \Leftrightarrow \ln M =
(a + b)t + C_0
\]
donde $C_0$ es una constante de integración.

Por último, tomando exponenciales en ambos miembros de la ecuación
integrada, y teniendo en cuenta que la exponencial de una constante
es una nueva constante a la que llamamos $C$, nos queda:
\[
M(t) = e^{(a + b)t + C_0 }  = e^{(a + b)t} e^{C_0 }  = Ce^{(a + b)t}
\]
Como, además, tenemos 2 datos iniciales, podemos calcular los
valores tanto de $C$ como de la suma $a+b$:
\[
M(0) = 20 = Ce^{(a + b)0}  = C
\]
\[
M(10) = 40 = Ce^{(a+b)10}=20e^{(a + b)10}  \Leftrightarrow e^{(a +
b)10} = 2 \Leftrightarrow a + b = \frac{{\ln 2}}{{10}}
\]
Por lo tanto, la masa $M$ para todo tiempo $t$ vale:
\[
M(t) = 20\;e^{\frac{{\ln 2}}{{10}}t}
\]

\item Una vez que tenemos la masa para todo tiempo $t$, a los 30 s
tendremos:
\[
M(30) = 20\;e^{\frac{{\ln 2}}{{10}}30}  = 20e^{3\ln 2}  = 160
\]
donde la cantidad viene dada en gramos.

\item Para calcular el tiempo $t_0$ que debe transcurrir hasta que
tengamos 100 g de masa, sustituimos de nuevo en la solución general:
\[
M(t_0 ) = 100 = 20\;e^{\frac{{\ln 2}}{{10}}t_0 }  \Leftrightarrow
\ln 5 = \frac{{\ln 2}}{{10}}t_0  \Leftrightarrow t_0  = \frac{{10\ln
5}}{{\ln 2}} = 23.22
\]
donde el tiempo viene dado en segundos.
\end{enumerate}
}


\newproblem{edosep-12}{qui}{*}
%ENUNCIADO
{Se sabe que en una reacción química una sustancia se transforma en otra a una velocidad  proporcional a la cantidad
sin transformar. Si a las 2 horas del comienzo de la reacción había 20 gr. de la sustancia original y a las 3 horas
quedaban 10 gr., ¿qué cantidad es sustancia había al comienzo de la reacción? ¿Cuándo se habrá transformado el 90\% de
la sustancia?}
%SOLUCIÓN
{La cantidad original de sustancia era $x(0)=80$  gr y el tiempo que tiene que pasar para que se transforme el $90\%$
es $3.32$ horas.  }
%RESOLUCIÓN
{Llamemos $x(t)$ a la función que mide la cantidad de sustancia original en el instante $t$. Según el enunciado, la
transformación química responde a la ecuación diferencial 
\[
\frac{dx}{dt}=kx.
\]
Se trata de una ecuación diferencial de variables separables, así que, para resolverla separamos las variables e
integramos: 
\[
\frac{dx}{dt}=kx \Leftrightarrow \frac{dx}{x}=kdt \Leftrightarrow \int \frac{dx}{x} = \int k\,dt \Leftrightarrow
\log|x| = kt+C \Leftrightarrow x(t)=Ce^{kt}, 
\]
que es la solución general de la ecuación.

Para determinar las constantes imponemos las condiciones inicales:
\begin{align*}
x(2)=20 &\Leftrightarrow Ce^{2k} = 20 \Leftrightarrow e^{2k} = 20/C \Leftrightarrow 2k =
\log(20/C) \Leftrightarrow k=\log(20/C)/2,\\
x(3)=10 &\Leftrightarrow Ce^{3k} = 10 \Leftrightarrow Ce^{\frac{3}{2}\log(20/C)} = Ce^{\log(20/C)^{3/2}}=
C\left(\frac{20}{C}\right)^{3/2}=10 \Leftrightarrow C^{1/2}=\frac{20^{3/2}}{10} \Leftrightarrow C = 80. 
\end{align*}
de donde se deduce $k=\log(20/80)/2 = -\log 2$, y en consecuencia, la solución particular de la ecuación es
\[
x(t)=80e^{-\log2\cdot t}.
\]

Según esta ecuación, la cantidad original de sustancia en el instante inicial $(t=0)$ era
\[
x(0)=80e^{-\log2\cdot 0} = 80 \mbox{ gr},
\]
y el tiempo necesario para que se transforme el 90\% de la sustancia, es decir, que quede el 10\% será
\[
x(t_{0})=80*0.1=8 \Leftrightarrow 80e^{-\log2\cdot t_{0}}=8 \Leftrightarrow 
e^{-\log2\cdot t_{0}}=8/80=0.1 \Leftrightarrow t_{0}=-\frac{\log0.1}{\log2}=3.32 \mbox{ horas.}
\]
}


\newproblem{edosep-13}{qui}{}
%ENUNCIADO
{Un depósito contiene 5 kg de sal disueltos en 500 litros de agua en el instante en que comienza entrar una solución
salina con 0.4 kg de sal por litro a razón de 10 litros por minuto. Si la mezcla se mantiene uniforme mediante
agitación y sale la misma cantidad de litros que entra, ¿cuánta sal quedará en el depósito después de 5 minutos? ¿y
después de 1 hora?   

\noindent\textbf{Nota:} La tasa de variación de la cantidad de sal en el tanque es la diferencia entre la cantidad de
sal que entra y la que sale del tanque en cada instante.}
%SOLUCIÓN
{$C(t)=-195e^{-t/50}+200$. La cantidad de sal a los 5 minutos será $C(5)=23.557$ kg y a la hora $C(60)=141.267$ kg.
}
%RESOLUCIÓN
{}


\newproblem{edosep-14}{qui}{*}
%ENUNCIADO
{En una reacción química, un compuesto se transforma en otra sustancia a un ritmo proporcional al cuadrado de la
cantidad no transformada. Si había inicialmente 20 gr de la sustancia original y tras 1 hora queda la mitad, ¿en qué
momento se habrá transformado el 75\% de dicho compuesto?}
%SOLUCIÓN
{$C(t)=\frac{20}{t+1}$ y el instante en que se habrá transformado el 75\% de la cantidad inicial es $t_0=3$ horas.
}
%RESOLUCIÓN
{}


\newproblem{edosep-15}{gen}{}
%ENUNCIADO
{Cuando el movimiento se produce en un medio en el que hay cierta resistencia, como en el aire, aparece una fuerza
proporcional a la velocidad que se opone al mismo. En este caso, las leyes de Newton conducen a la siguiente ecuación
diferencial para la velocidad de caída en el medio:
\[
m\frac{{dv}} {{dt}} =  - kv - mg
\]
donde $v$ es la velocidad, $m$ es la masa, $g$ es la gravedad, y $k$ es la constante de proporcionalidad.

Si se dispara un móvil directamente hacia arriba al nivel del suelo, con velocidad inicial $100$ m/s, una masa de
$0.05$ kg, una constante $k$ de $0,002$ kg/s y $g$ de $10$ m/s$^2$, ¿cuál será la máxima altura del móvil y cuándo la
alcanzará? ¿Cuándo y con qué velocidad golpeará el móvil en el suelo?}
%SOLUCIÓN
{
}
%RESOLUCIÓN
{}


\newproblem{edosep-16}{amb}{*}
%ENUNCIADO
{La cantidad de masa, $M$, expresada en Kg, de sustancias contaminantes en un depósito de aguas residuales, cumple la
ecuación diferencial:
\[
\frac{{dM}} {{dt}} =  - 0.5M + 1000
\]
donde $k$ es una constante y $t$ es el tiempo expresado en días (podemos imaginar que el depósito está conectado a una
depuradora que elimina sustancia contaminante con un ritmo proporcional a la propia cantidad de contaminante, lo cual
explicaría el sumando $-0.5M$, y que también hay un aporte constante de contaminante de 1000 kg/día, que puede provenir
de un desagüe, lo cual explicaría el sumando constante $+1000$).

Si la cantidad inicial de contaminante es de 10000 Kg:
\begin{enumerate}
\item ¿Cuál será la cantidad de contaminante para todo tiempo $t$?
\item ¿Cuál será la cantidad de contaminante al cabo de una semana?
\end{enumerate}
}
%SOLUCIÓN
{\begin{enumerate}
\item $M(t)=8000e^{-0.5t}+2000$.
\item $M(7)=2241.579$ kg. 
\end{enumerate}
}
%RESOLUCIÓN
{}


\newproblem{edosep-17}{gen}{}
%ENUNCIADO
{Si tenemos en cuenta que cualquier onda sonora que atraviesa un medio sufre un proceso de amortiguamiento, y que su
Intensidad $I$ (cantidad de energía por unidad de área y tiempo que atravesaría una superficie colocada de forma
perpendicular a la dirección de desplazamiento de la onda, en w/m$^2$) viene dada por la ley de Lamber-Beer:
\[
\frac{{dI}}{{dx}} =  - \alpha I
\]
donde $\alpha$ es el coeficiente de absorción, y suponemos una onda sonora que llega a una pared con una intensidad de 1 w/m$^2$, y atraviesa 10 cm de pared con un coeficiente de absorción del material de la pared de $0,1$ cm$^-1$. En estas condiciones:
\begin{enumerate}
\item ¿Cuál es la intensidad que llega al otro lado de la pared?
\item Teniendo en cuenta que en ondas sonoras más que la intensidad misma se utiliza el nivel de intensidad $\beta$, cuya unidad es el decibelio, que viene dado por:
\[
\beta  = 10\log _{10} \frac{I}{{I_0 }}
\]
donde $I_0$ es una intensidad de referencia asociada con la intensidad más débil que se puede oír e igual a $10^{-12}$ W/m$^2$, calcular cuál es el nivel de intensidad de la onda entrante en la pared, y cuál el de la saliente.
\end{enumerate}
}
%SOLUCIÓN
{
}
%RESOLUCIÓN
{}


\newproblem{edosep-18}{med}{}
%ENUNCIADO
{El plasma sanguíneo se conserva a 4ºC. Para poder utilizarse en una transfusión el plasma tiene que alcanzar la
temperatura del cuerpo (37ºC). Sabemos que se tardan 45 minutos en alcanzar dicha temperatura en un horno a 50ºC.
¿Cuánto se tardará si aumentamos la temperatura del horno a 60º?}
%SOLUCIÓN
{Con el horno a 50ºC se tiene $T(t)=-46e^{-0.02808t}+50$, con el horno a 60ºC se tiene $T(t)=-56e^{-0.02808t}+60$ y en
este horno tardará $31.69$ min.}
%RESOLUCIÓN
{}


\newproblem{edosep-19}{gen}{}
%ENUNCIADO
{Hallar las curvas tales que en cada punto $(x,y)$ la pendiente de la recta tangente sea igual al cubo de la abscisa en
dicho punto. ¿Cuál de estas curvas pasa por el origen?}
%SOLUCIÓN
{$y=x^4/4$.
}
%RESOLUCIÓN
{}


\newproblem{edosep-20}{med}{}
%ENUNCIADO
{Al introducir glucosa por vía intravenosa a velocidad constante, el cambio de concentración global de glucosa  con
respecto al tiempo $c(t)$ se explica mediante la siguiente ecuación diferencial 
\[
\frac{dc}{dt}=\frac{G}{100V}-kc,
\]
donde $G$ es la velocidad constante a la que se suministra la glucosa, $V$ es el volumen total de la sangre en el
cuerpo y $k$ es una constante positiva que depende del paciente. Se pide calcular $c(t)$.}
%SOLUCIÓN
{$c(t)=De^{kt}+\frac{G}{100Vk}$
}
%RESOLUCIÓN
{}


\newproblem{edosep-21}{amb}{}
%ENUNCIADO
{La temperatura $T$ de una habitación en un día de invierno varía con el tiempo de acuerdo a la ecuación:
\[
\frac{dT}{dt}=
\begin{cases}
40-T, & \mbox{si la calefacción está encendida;} \\
-T, & \mbox{si la calefacción está apagada.}
\end{cases}
\]
Suponiendo que la temperatura del aula es de 5ºC  a las 9:00 de la mañana, y que a esa hora se enciende la calefacción, pero que debido a una avería la calefacción permanece apagada de 11:00 a 12:00, ¿qué temperatura habrá en la habitación a las 13:00?}
%SOLUCIÓN
{De 9 a 11 la temperatura es $T(t)=-35e^{-t}+40$ y la temperatura a las $11$ será de $35.263$ºC.\\
De 11 a 12 la temperatura es $T(t)=35.263e^{-t}$ y la temperatura a las $12$ será de $12.973$ºC.\\
De 12 a 13 la temperatura es $T(t)=-27.027e^{-t}+40$ y la temperatura a las $13$ será de $30.057$ºC.
}
%RESOLUCIÓN
{}


\newproblem{edosep-22}{amb}{}
%ENUNCIADO
{Se considera que la población de una determinada ciudad, $P(t)$, con índices constantes de natalidad y mortalidad,
$\beta$ y $\gamma$ respectivamente, pero en la que también ingresan por inmigración $I$ personas al año, sigue la
ecuación diferencial:
\[
\frac{{dP}} {{dt}} = \left( {\beta  - \gamma } \right)P + I
\]
Suponiendo que dicha población tenía $1,5$ millones de habitantes en $1980$, que la diferencia entre los índices de
natalidad y mortalidad es de $0.01$ (es decir, crece un $1\%$ anual), y también que absorbe $40000$ inmigrantes al año,
¿cuál será la población en el año $2005$?}
%SOLUCIÓN
{
}
%RESOLUCIÓN
{}


\newproblem{edosep-23}{qui}{}
%ENUNCIADO
{Una reacción química se comporta según la siguiente ecuación diferencial:
\[
y\sqrt {2x} \,dy - 2y^2 \,dx = 0
\]
donde $y$ es la energía liberada (en Kj) y $x$ es la cantidad de una determinada sustancia (en gr). Sabiendo que para 2
gr la energía liberada es de 50 Kj, ¿cuánta cantidad habrá que utilizar para obtener 1000 Kj?}
%SOLUCIÓN
{$6.12$  gr.
}
%RESOLUCIÓN
{Se trata de una ecuación diferencial ordinaria de variables separables, así que, para resolverla primero separamos las variables
\[
y\sqrt {2x} \,dy - 2y^2 \,dx = 0
\Leftrightarrow
y\sqrt {2x} \,dy =  2y^2 \,dx 
\Leftrightarrow
\frac{y}{y^2}dy =  \frac{2}{\sqrt{2x}}dx 
\Leftrightarrow
\frac{1}{y}dy =  \frac{2}{\sqrt{2x}}dx 
\]
y ahora integramos ambos miembros de la ecuación
\begin{align*}
\int \frac{1}{y}dy &= \ln y +C, \\
\int \frac{2}{\sqrt{2x}}dx &= 2\sqrt{2x}+C.
\end{align*}
Por tanto, la solución general de la ecuación es
\[
\ln y = 2\sqrt{2x}+C 
\Leftrightarrow
y(x) = e^{2\sqrt{2x}+C} = e^{2\sqrt{2x}}e^C = C e^{2\sqrt{2x}},
\]
renonbrando $e^C$ como una constante $C$. 

Para llegar a una solución particular, imponemos la condición inicial que nos dan, que es $y(2)=50$.
\[
y(2) = C e^{2\sqrt{2\cdot 2}} = C e^4 = 50  
\Leftrightarrow
C = \frac{50}{e^4}
\]
Así pues, la solución particular es
\[
y(t) = \frac{50}{e^4} e^{2\sqrt{2x}} = 50  e^{2\sqrt{2x}-4}.
\]

Por último, para ver la masa $x_0$ necesaria para generar 1000 Kj, sustituimos en la solución particular
\[
\renewcommand{\arraystretch}{2}
\begin{array}{c}
y(x_0) = 50  e^{2\sqrt{2x_0}-4} = 1000 
\Leftrightarrow
e^{2\sqrt{2x_0}-4} = 	\dfrac{1000}{50} = 20
\Leftrightarrow \\
\Leftrightarrow 
2\sqrt{2x_0}-4 = 	\ln 20 = 2.9957
\Leftrightarrow
\sqrt{2x_0} = \dfrac{2,9957+4}{2} = 3.4979
\Leftrightarrow
x_0 = \dfrac{3,4979 ^2}{2} = 6.12 \mbox{ gr}.
\end{array}
\]
}


\newproblem{edosep-24}{gen}{*}
%ENUNCIADO
{Resolver el problema del valor inicial
\[
\left\{
  \begin{array}{l}
    y\sqrt{2x}dy-2y^2dx=0\\
    y(0)=5
  \end{array}
\right.
\]
¿Para qué valor de $x$, se obtiene $y=1000$?
}
%SOLUCIÓN
{$y(x) = 5 e^{2\sqrt{2x}}$. El valor de $x$ para el que $y=1000$ es $x=3.509$.
}
%RESOLUCIÓN
{Se trata de una ecuación diferencial ordinaria de variables separables, por lo que, para resolverla primero debemos separar las variables
\[
y\sqrt{2x}dy-2y^2dx=0 \Leftrightarrow  y\sqrt{2x}dy = 2y^2dx \Leftrightarrow \frac{y}{y^2}dy = \frac{2}{\sqrt{2x}}dx \Leftrightarrow \frac{1}{y}dy = \sqrt{2}x^{-1/2}dx
\]
Una vez separadas las variables integramos ambos lados de la ecuación
\[
\int \frac{1}{y}dy = \int \sqrt{2}x^{-1/2}dx \Leftrightarrow \log y = 2\sqrt{2x} +C
\]
y despejando $y$  obtenemos la solución general de la ecuación
\[
y(x) = e^{2\sqrt{2x}+C} = Ce^{2\sqrt{2x}}.
\]
Para obtener la solución particular imponemos la condición inicial $y(0)=5$,
\[
y(0) = Ce^{2\sqrt{2\cdot 0}} = 5 \Leftrightarrow C e^{0} = 5 \Leftrightarrow C = 5,
\]
de modo que la solución del problema del valor inicial es
\[
y(x) = 5 e^{2\sqrt{2x}}.
\]

Finalmente, calculamos el valor $x$ para el que $y=1000$:
\[
y(x) = 5 e^{2\sqrt{2x}} = 1000 \Leftrightarrow e^{2\sqrt{2x}} = \frac{1000}{5}=200 \Leftrightarrow 2\sqrt{2x} = \log 200 \Leftrightarrow x = \frac{(\log 200/2)^2}{2} = 3.509.
\]
}


\newproblem{edosep-25}{amb}{*}
%ENUNCIADO
{La velocidad de aumento del número de bacterias en un cultivo es proporcional al número de bacterias presentes, siguiendo la ecuación: 
\[
\frac{dx}{dt}=ax
\]
siendo $x$ el número de bacterias presentes y $t$ el tiempo.
\begin{enumerate}
\item  ¿Por cuánto se habrá multiplicado el número de bacterias al cabo de $5$ horas, si se duplicó al cabo de $3$ horas?
\item  Si al cabo de $4$ horas hay $10000$ bacterias,  ¿cuántas había al principio?
\end{enumerate}
}
%SOLUCIÓN
{La solución general de la ecuación es $x(t)=Ce^{at}$.
\begin{enumerate}
\item $k=3.17$.
\item Al principio había $3968$ bacterias.
\end{enumerate}
}
%RESOLUCION
{Antes de contestar a los apartados resolvemos la ecuación diferencial que plantea el problema. Se trata de una ecuación diferencial de variables separadas que se resuelve fácilmente: 
\[
\frac{dx}{dt}=ax \Longleftrightarrow \frac{dx}{x}=a\,dt \Longleftrightarrow 
\int \frac{dx}{x} = \int a\,dt \Longleftrightarrow \ln x = at+C \Longleftrightarrow
e^{\ln x}=e^{at+C},
\]
y, aplicando la función exponencial a ambos lados de la última igualdad para simplificar, obtenemos la solución general 
\[
e^{\ln x}=e^{at+C} \Longleftrightarrow x=e^{at}e^{C} \Longleftrightarrow x=Ce^{at}.
\]
donde, para simplificar, hemos reescrito $C=e^C$ al ser una constante.
\begin{enumerate}
\item Para resolver el primer apartado, llamamos $x(0)$ al número inicial de bacterias en el cultivo. Como al cabo de 3 horas se había duplicado el número de bacterias en el cultivo, tenemos la ecuación $x(3)=2x(0),$ que al revolverla nos lleva a 
\[
x(3)=2x(0) \Longleftrightarrow Ce^{3a}=2Ce^{0a}=2C \Longleftrightarrow
e^{3a}=2 \Longleftrightarrow 3a=\ln 2 \Longleftrightarrow a=\frac{\ln 2}{3}.
\]

Para saber por cuanto se habrá multiplicado el número de bacterias al cabo de 5 horas, planteamos igual que antes la ecuación $x(5)=kx(0),$ donde $k$ es el factor de multiplicación. Al resolver esta ecuación obtenemos 
\[
x(5)=kx(0)\Longleftrightarrow Ce^{5\frac{\ln 2}3}=kCe^{0\frac{\ln 2}{3}}=kC \Longleftrightarrow e^{5\frac{\ln 2}{3}}=k \Longleftrightarrow k=3.17.
\]
Luego al cabo de 5 horas habrá aproximadamente tres veces más bacterias que al comienzo

\item El número de inicial de bacterias es
\[
x(0)=Ce^{0\frac{\ln 2}3}=C.
\]

Ahora bien, como al cabo de 4 horas había 1000 bacterias, planteamos la ecuación $x(4)=10000$, que al resolverla, nos proporciona el valor de $%
C.$
\[
x(4)=Ce^{4\frac{\ln 2}{3}}=10000 \Longleftrightarrow C=\frac{10000}{e^{4\frac{\ln 2}3}} = 3968.5.
\]
\end{enumerate}
}


\newproblem{edosep-26}{gen}{*}
%ENUNCIADO
{Dada la ecuación diferencial: $yy'+ e^{x^2}x = 2xy^2e^{x^2}$, calcular el valor de $y(1)$ sabiendo que $y(0)=-1$.}
%SOLUCIÓN
{$y(1)=4.005$.}
%RESOLUCIÓN
{}


\newproblem*{edosep-27}{gen}{*}
%ENUNCIADO
{Dos figuras de cerámicas del mismo material se ponen en un horno para su cocción a $1000^\circ$C. 
Si en el instante en que se meten al horno la primera está a $40^\circ$C y la segunda a $5^\circ$C, y al minuto la temperatura de la primera ha aumentado hasta los $200^\circ$C, ¿cuales serán sus temperaturas a los 5 minutos?  
}
%SOLUCIÓN
{
}
%RESOLUCIÓN
{}


\newproblem*{edosep-28}{qui}{*}
%ENUNCIADO
{El átomo de radio se desintegra dando helio y una emanación gaseosa, radón, que también es radioactiva. 
Sabiendo que la velocidad de desintegración es proporcional a la masa ($m$) en cada instante, se pide:
\begin{enumerate}
\item Resolver la ecuación diferencial que explica la desintegración del radio.
\item Calcular la constante de desintegración sabiendo que la masa del radio disminuye un $0.043\%$ cada año.
\item Calcular el periodo del radio, que es el instante $T$ tal que  $m(t+T)=\frac{1}{2}m(t)$ $\forall t\geq 0$.
\end{enumerate} 
}
%SOLUCIÓN
{
}
%RESOLUCIÓN
{}


\newproblem{edosep-29}{gen}{*}
%ENUNCIADO
{Obtener la ecuación de la curva que pasa por el punto $P=(1,1)$, tal que la pendiente de la tangente en cada punto coincida con el cuadrado de su ordenada.
}
%SOLUCIÓN
{$y=\frac{-1}{x-2}$.
}
%RESOLUCIÓN
{}


\newproblem{edosep-30}{med}{*}
%ENUNCIADO
{Un investigador constata que, tras una inyección intravenosa de glucosa, la tasa de glucosa en sangre $g(t)$ en cada instante $t$ sigue la ecuación diferencial
\[
g'+kg=0,
\] 
donde $k>0$ es una constante conocida como \emph{coeficiente de asimilación}. Se pide:
\begin{enumerate}
\item Resolver la ecuación diferencial para un sujeto cuya tasa de glucosa en el instante de aplicar la inyección es 80 mg/dl.
\item Si el valor del coeficiente de asimilación varía de $1.06\cdot 10^{-2}$ a $2.42\cdot 10^{-2}$ en los sujetos normales, estudiar si los resultados del sujeto anterior son normales tiendo en cuenta que a los 30 minutos la tasa de glucosa era de $1.2$ mg/dl.
\end{enumerate}
}
%SOLUCIÓN
{
}
%RESOLUCIÓN
{}


\newproblem*{edosep-31}{gen}{*}
%ENUNCIADO
{El carbono contenido en la materia viva incluye una ínfima proporción del isótopo radioactivo $C^{14}$, que proviene de los rayos cósmicos de la parte superior de la atmósfera.
Gracias a un proceso de intercambio complejo, la materia viva mantiene una proporción constante de $C^{14}$ en su carbono total (esencialmente constituido por el isótopo estable $C^{12}$).
Después de morir, ese intercambio cesa y la cantidad de carbono radioactivo disminuye: pierde $1/8000$ de su masa al año.
Estos datos permiten determinar el año en que murió un individuo. 
Se pide:
\begin{enumerate}
\item Si el análisis de los fragmentos de un esqueleto de un hombre de Neandertal mostró que la proporción de $C^{14}$ era de $6.24\%$ de la que hubiera tenido al estar vivo.
¿Cuándo murió el individuo?
\item Calcular la vida media del carbono $C^{14}$, es decir, el tiempo a partir del cual se ha desintegrado la mitad del carbono inicial.  
\end{enumerate}
}
%SOLUCIÓN
{
}
%RESOLUCIÓN
{}


\newproblem{edosep-32}{amb}{}
%ENUNCIADO
{Una colonia de salmones vive tranquilamente en una zona costera.
La tasa de natalidad es del 2\% por día y la de mortalidad del 1\% por día. 
En el instante inicial, la colonia tiene 1000 salmones y ese día llega un tiburón a esa zona costera que se come 15 salmones todos los días.
¿Cuánto tiempo tarda el tiburón en hacer desaparecer a la colonia de salmones?
}
%SOLUCIÓN
{Aproximadamente 110 días.
}
%RESOLUCIÓN
{}


\newproblem{edosep-33}{far}{*}
%ENUNCIADO
{El número de bacterias en un determinado cultivo crece a una velocidad proporcional al número de bacterias presente. 
Al cabo de dos días, el número de bacterias se ha duplicado y un día más tarde había 1000 bacterias. 
¿Cuántas bacterias había al principio? 
}
%SOLUCIÓN
{353.55 bacterias.
}
%RESOLUCIÓN
{}





