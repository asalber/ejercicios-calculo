% Autor: Alfredo Sánchez Alberca (asalber@ceu.es)

\newproblem{extn-1}{gen}{*}
%ENUNCIADO
{Hallar los extremos relativos y los puntos de silla de la función:
\[
f(x,y) = (x^2+y^2)^2-2a^2(x^2-y^2),
\]
con $a\neq 0$.
}
%SOLUCIÓN
{No tiene máximos relativos. Mínimos relativos en $(-a,0)$ y $(a,0)$. Punto de silla en $(0,0)$.
}
%RESOLUCIÓN
{
}


\newproblem{extn-2}{gen}{*}
%ENUNCIADO
{Dado el campo escalar
\[
h(x,y) = xy+\frac{xy^2}{2}-2x^2,
\]
determinar sus extremos relativos y sus puntos de silla.
}
%SOLUCIÓN
{Máximo relativo en $(-1/8,-1)$. No tiene mínimos relativos. Puntos de silla en $(0,0)$ y $(0,-2)$.
}
%RESOLUCIÓN
{
}

\newproblem{extn-3}{gen}{}
%ENUNCIADO
{Estudiar los extremos y los puntos de silla de $f$ en los siguientes casos:
\begin{enumerate}
\item $f(x,y) = x^2+y^2$.
\item $f(x,y) = x^2-y^2$.
\item $f(x,y) = x^2-2xy+y^2$.
\item $f(x,y) = \log(x^2+y^2+1)$.
\end{enumerate}
}
%SOLUCIÓN
{\begin{enumerate}
\item Mínimo en $(0,0)$.
\item Punto de silla en $(0,0)$.
\item No se puede saber con el hessiano.
\item Mínimo en $(0,0)$.
\end{enumerate}
}
%RESOLUCIÓN
{
}


\newproblem{extn-4}{gen}{}
%ENUNCIADO
{La función
\[
f(x,y) = \frac{x^3}{3}-x-\left(\frac{y^3}{3}-y\right)
\]
tiene un máximo, un mínimo y dos puntos de silla. Encontrarlos.
}
%SOLUCIÓN
{Máximo en $(-1,1)$, mínimo en $(1,-1)$ y puntos de silla en $(1,1)$ y $(-1,-1)$.
}
%RESOLUCIÓN
{
}


\newproblem{extn-5}{gen}{}
%STATEMENT
{Si suponemos que el rendimiento de una cosecha, $R$, depende de las concentraciones de nitrógeno, $n$, y fósforo, $p$, presentes en el suelo según la función:
\[
R(n,p) = n \cdot p \cdot e^{ - (n + p)}
\]
\begin{enumerate}
\item Calcular todas las derivadas parciales de primer y segundo orden de la función $R(n,p)$.
\item Teniendo en cuenta que una condición necesaria para que una función de varias variables presente un máximo en un
punto es que todas las derivadas parciales de primer orden se anulen en dicho punto, ¿cuánto deben valer las
concentraciones de nitrógeno y fósforo para que el rendimiento de la cosecha sea máximo?
\end{enumerate}
}
%SOLUCION
{El rendimiento de la cosecha será máximo para $n=p=1$.
}
%RESOLUTION
{
}


\newproblem{extn-6}{gen}{*}
%STATEMENT
{Dada la función $f(x,y)=\dfrac{ax^3}{3} + \dfrac{by^3}{3}-4ax-4by$, con $a$ y $b$ dos parámetros positivos, estudiar la existencia de extremos relativos y puntos de silla de $f$.
}
%SOLUTION
{Máximo relativo en $(-2,-2)$, mínimo relativo en $(2,2)$ y puntos de silla en $(-2,2)$ y $(2,-2)$.
}
%RESOLUTION
{
}
