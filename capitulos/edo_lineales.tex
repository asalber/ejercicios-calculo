% Autor: Alfredo Sánchez Alberca (asalber@ceu.es)

\newproblem{edolin-1}{gen}{}
%ENUNCIADO
{Integrar las siguientes ecuaciones diferenciales lineales:
\begin{enumerate}
\item $y'-2y=4$.
\item $y'-6xy=x$.
\item $\frac{dz}{dt}+\frac{3z}{10+3t}=6$ con la condición inicial $z(2)=100$.
\item $y'+y\cos x=\sen x\cos x$ con la condición inicial $y(0)=1$.
\end{enumerate}
}
%SOLUCIÓN
{
\begin{enumerate}
\item $y=Ce^{2x}-2$.
\item $y=Ce^{3x^2}-\frac{1}{6}$.
\item $z=\dfrac{9t^2+60t+1444}{3t+10}$.
\item $y=2e^{-\sen x}+\sen x -1$.
\end{enumerate}
}
%RESOLUCIÓN
{}


\newproblem{edolin-2}{gen}{}
%ENUNCIADO
{Un tanque de 50 litros contiene inicialmente 10 litros de agua. En el instante inicial se vierte al tanque una disolución salina que
contiene 100 gr de sal por cada litro de agua, a razón de 4 litros por minuto, mientras que la mezcla bien agitada abandona el tanque a un
ritmo de 2 litros por minuto. ¿Cuánto tiempo transcurrirá hasta que se llene el depósito? En dicho instante, ¿qué cantidad de sal habrá en
el depósito?}
%SOLUCIÓN
{El depósito se llenará a los 20 minutos y contendrá $4.8$ kg de sal. 
}
%RESOLUCIÓN
{}


\newproblem{edolin-3}{qui}{*}
%ENUNCIADO
{En una reacción química, una sustancia $A$ se transforma en otra $B$ con una velocidad del doble de la cantidad de
sustancia $A$.
Si en el instante inicial la cantidad de $A$ es de $5$ gr/dl, ¿qué cantidad de sustancia $A$ habrá a los 2 segundos? 

Si en esa misma reacción, la sustancia $B$, a su vez, se transforma en otra $C$ a una velocidad del triple de la
cantidad de $B$, sabiendo que al comienzo de la reacción la cantidad de sustancia $B$ era nula, ¿qué cantidad de $B$
habrá a los 2 segundos?
}
%SOLUCIÓN
{\begin{enumerate}
\item Cantidad de sustancia $A$ a los 2 segundos: $0.0916$ gr/dl.
\item Cantidad de sustancia $B$ a los 2 segundos: $0.1584$ gr/dl.
\end{enumerate}
}
%RESOLUCIÓN
{}