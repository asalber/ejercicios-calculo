% Autor: Alfredo Sánchez Alberca (asalber@ceu.es)

\newproblem{derimpn-1}{gen}{}
%ENUNCIADO
{Suponiendo que $z$ es función de $x$ e $y$ ($z=f(x,y)$), a partir de la ecuación $F(x,y,z)=0$, deducir que 
\[
\frac{\partial z}{\partial x} = \frac{-\dfrac{\partial F}{\partial x}}{\dfrac{\partial F}{\partial z}}
\quad \mbox{y} \quad
\frac{\partial z}{\partial y} = \frac{-\dfrac{\partial F}{\partial y}}{\dfrac{\partial F}{\partial z}}.
\]

Aplicarlo para obtener $\dfrac{\partial f}{\partial x}(2,1)$, sabiendo que $x^2yz=4$ y que $f(2,1)=1$.
}
%SOLUCIÓN
{$\frac{\partial f}{\partial x}(2,1)=-1.$
}
%RESOLUCIÓN
{
}

\newproblem{derimpn-2}{gen}{*}
%ENUNCIADO
{La ecuación 
\[
x\log y+\frac{2e^{y^2+z}}{x} - \frac{x}{z^2} = -1
\] 
define a $z$ como función de $x$ e $y$ alrededor del punto $(2,1,-1)$. 
Calcular el vector gradiente de $z$ en ese punto e interpretarlo.
}
%SOLUCIÓN
{$\nabla z(2,1,-1) = (-1/2,4/3)$.
}
%RESOLUCIÓN
{
}