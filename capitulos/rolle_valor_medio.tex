% Autor: Alfredo Sánchez Alberca (asalber@ceu.es)

\newproblem*{rol-1}{gen}{}
%ENUNCIADO
{Estudiar si se puede aplicar el teorema de Rolle a las funciones siguientes en los intervalos indicados:
\begin{enumerate}
    \item  $f(x)=\left\{
    \begin{array}{ccl}
        x^3 &  & \mbox{si } x<0,  \\
        x^2 &  & \mbox{si } x\geq 0,
    \end{array}\right.$
    \quad en [-1,1].

    \item  $g(x)=\left\{
    \begin{array}{ccl}
        x+3 &  & \mbox{si } x\leq 0,  \\
        x^2+1 &  & \mbox{si } x> 0,
    \end{array}\right.$
    \quad en [-1,1].

    \item  $h(x)=x^3-x$ \quad en [0,1] o [-1,0].

    \item  $k(x)=1-e^{\sen x}$ \quad  en $[0,\pi]$.
\end{enumerate}
En caso afirmativo, encontrar un punto en el que la derivada se anule.
}


\newproblem*{rol-2}{gen}{}
%ENUNCIADO
{Demostrar que la ecuación $e^x=1+x$ tiene exactamente una solución real.
}


\newproblem*{rol-3}{gen}{}
%ENUNCIADO
{Demostrar que para cualquier valor de $k\in \mathbb{R}$ la ecuación $x^3-3x+k=0$ no puede tener dos raíces en el intervalo $(0,1)$.
}


\newproblem*{rol-4}{gen}{}
%ENUNCIADO
{En el movimiento uniformemente acelerado de ecuación $x=8t^2-2t+5$ se puede asegurar, durante el intervalo de tiempo [0,1], que la velocidad media coincide con la velocidad instantánea en un cierto momento $t_{0}$. Calcular $t_{0}$. ¿Qué propiedad nos asegura la certeza de la proposición anterior?
}


\newproblem*{rol-5}{gen}{}
%ENUNCIADO
{A las cuatro de la tarde un coche pasa, a una velocidad de 70 km/h por el punto kilométrico 400 de la autopista $A4$. Diez minutos después pasa, circulando a una velocidad de 80 km/h, por el punto kilométrico 425 de la citada autopista. Le para la policía y le pone una multa por exceso de velocidad. ¿Tenía razón la policía?

\noindent  \textbf{Nota}: Velocidad máxima permitida 120 km/h.
}


\newproblem*{rol-6}{gen}{}
%ENUNCIADO
{Dados $a,b \in \R$ con $a<b$, determinar los puntos que verifican el teorema del valor medio de Lagrange en el intervalo $[a,b]$ para las siguiente funciones:
\begin{multicols}{2}
\begin{enumerate}
    \item  $f(x)=x^3$.

    \item  $g(x)=e^x$.
\end{enumerate}
\end{multicols}
}
