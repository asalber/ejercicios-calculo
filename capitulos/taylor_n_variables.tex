% Autor: Alfredo Sánchez Alberca (asalber@ceu.es)

\newproblem{tayn-1}{gen}{*}
%ENUNCIADO
{Dada la función $f(x,y)=\sqrt{xy}$, se pide:
\begin{enumerate}
\item Calcular el polinomio de Taylor de primer grado centrado en el punto $(4,9)$.
\item Calcular el valor aproximado de $f(4.01,\,8.99)$ a partir del polinomio anterior. 
\end{enumerate}
}
%SOLUCIÓN
{\begin{enumerate}
\item $P(x,y)= \frac{3}{4}x+\frac{1}{3}y$.
\item $f(4.01,\,8.99)\approx 6.00416667$.
\end{enumerate}
}
%RESOLUCIÓN
{
}


\newproblem{tayn-2}{gen}{}
%ENUNCIADO
{Obtener el desarrollo de Taylor de segundo orden de $f$ alrededor del punto $P$ en cada uno de los casos siguientes:
\begin{enumerate}
\item $f(x,y)=\sen(x+2y)$, $P=(0,0)$.
\item $f(x,y)=e^x\cos y$, $P(0,0)$.
\item $f(x,y)=\sen(xy)$, $P(1,\pi/2)$.
\end{enumerate}
}
%SOLUCIÓN
{\begin{enumerate}
\item $P^2_{f,P}(x,y)= x+2y$.
\item $P^2_{f,P}(x,y)= 1+x+\frac{x^2}{2}-\frac{y^2}{2}$.
\item $P^2_{f,P}(x,y)= 1+\frac{1}{2}\left(-\frac{\pi^2}{4}(x-1)^2-\pi(x-1)(y-\pi/2)-(y-\pi/2)^2\right)$.
\end{enumerate}
}
%RESOLUCIÓN
{
}


\newproblem{tayn-3}{gen}{}
%ENUNCIADO
{Dada la función 
\[
f(x,y,z)=e^x\sqrt{yz},
\]
estimar el valor de $f(0.01,24.8,1.02)$ mediante un desarrollo de Taylor lineal alrededor del punto $P=(0,25,1)$.
}
%SOLUCIÓN
{$5.08$.
}
%RESOLUCIÓN
{
}


\newproblem{tayn-4}{gen}{}
%ENUNCIADO
{Hallar las aproximaciones lineal y cuadrática de la expresión
\[
\frac{(3.98-1)^2}{(5.97-3)^2}
\]
usando desarrollos de Taylor. Comparar el resultado con el valor exacto.
}
%SOLUCIÓN
{La aproximación lineal es $1.00666666$ y la cuadrática $1.006744438$.
}
%RESOLUCIÓN
{
}