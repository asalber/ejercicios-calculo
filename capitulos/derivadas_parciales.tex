% Autor: Alfredo Sánchez Alberca (asalber@ceu.es)

\newproblem{par-1}{gen}{}
%ENUNCIADO
{Calcular las siguientes derivadas parciales:
\begin{multicols}{2}
\begin{enumerate}
\item $\dfrac{\partial}{\partial x}\ln \dfrac{x}{y}$.
\item $\dfrac{\partial}{\partial v}\dfrac{nRT}{v}$.
%\item $\dfrac{\partial^2}{\partial x \partial y}\left(e^{x+y}\sen\dfrac{x}{y}\right)$.
%\item $\dfrac{\partial^2}{\partial y \partial x}\left(e^{x+y}\sen\dfrac{x}{y}\right)$.
\end{enumerate}
\end{multicols}
}
%SOLUCIÓN
{\begin{enumerate}

\item $\frac{\partial}{\partial x}\,\log \left(\frac{x}{y}\right) = \frac{1}{x}$.
\item $\frac{\partial}{\partial v}\,\left(\frac{n\,R\,T}{v}\right) = -\frac{n\,R\,T}{v^2}$.
%\item $\frac{\partial^2}{\partial x \partial y}\,\left(\sin \left(\frac{x}{y}\right)\,e^{y+x}\right) = \frac{\left(\sen \left(\frac{x}{y}\right)\,y^3+\cos \left(\frac{x}{y}\right)\,y^2-x\,\cos \left(\frac{x}{y}\right)\,y-\cos \left(\frac{x}{y}\right)\,y+x\,\sen \left(\frac{x}{y}\right)\right)\,e^{y+x}}{y^3}$
%\item $\frac{\partial^2}{\partial y \partial x}\,\left(\sin \left(\frac{x}{y}\right)\,e^{y+x}\right) = \frac{\left(\sen \left(\frac{x}{y}\right)\,y^3+\cos \left(\frac{x}{y}\right)\,y^2-x\,\cos \left(\frac{x}{y}\right)\,y-\cos \left(\frac{x}{y}\right)\,y+x\,\sen \left(\frac{x}{y}\right)\right)\,e^{y+x}}{y^3}$
\end{enumerate}
}
%RESOLUCIÓN
{
}


\newproblem{par-2}{gen}{}
%ENUNCIADO
{Calcular el vector gradiente y la matriz Hessiana de las siguientes funciones:
\begin{multicols}{2}
\begin{enumerate}
\item $e^{x^2+y^2+z^2}$
\item $\sen((x^2-y^2)z)$
\end{enumerate}
\end{multicols}
}
%SOLUCIÓN
{\begin{enumerate}
\item $\nabla e^{x^2+y^2+z^2} = \left( 2\,x\,e^{z^2+y^2+x^2} , 2\,y\,e^{z^2+y^2+x^2} , 2\,z\,e^{z^2 +y^2+x^2} \right)$,\\
$
H e^{x^2+y^2+z^2} =
\left(
\begin{array}{ccc}
(4x^2+2)e^{x^2+y^2+z^2} & 4xye^{x^2+y^2+z^2} & 4xze^{x^2+y^2+z^2} \\
4xye^{x^2+y^2+z^2} & (4y^2+2)e^{x^2+y^2+z^2} & 4yze^{x^2+y^2+z^2} \\
4xze^{x^2+y^2+z^2} & 4yze^{x^2+y^2+z^2} & (4z^2+2)e^{x^2+y^2+z^2}
\end{array}
\right).
$
\item $\nabla \sen((x^2-y^2)z) = \left( 2\,x\,z\,\cos \left(\left(x^2-y^2\right)\,z\right) , -2\,y\, z\,\cos \left(\left(x^2-y^2\right)\,z\right) , \left(x^2-y^2\right) \,\cos \left(\left(x^2-y^2\right)\,z\right) \right) $\\
$H \sen((x^2-y^2)z) =$\\
\resizebox{\linewidth}{!}{
$
\left(
\begin{array}{ccc}
4x^2\sen((x^2-y^2)z)+2\cos((x^2-y^2)z) & 4xy\sen((x^2-y^2)z) & -2x(x^2-y^2)\sen((x^2-y^2)z) \\
4xy\sen((x^2-y^2)z) & -4y^2\sen((x^2-y^2)z)-2\cos((x^2-y^2)z) & 2y(x^2-y^2)\sen((x^2-y^2)z) \\
-2x(x^2-y^2)\sen((x^2-y^2)z) & 2y(x^2-y^2)\sen((x^2-y^2)z) & -(x^2-y^2)^2\sen((x^2-y^2)z)
\end{array}
\right).
$
}
\end{enumerate}
}
%RESOLUCIÓN
{
}


\newproblem{par-3}{gen}{*}
%ENUNCIADO
{Calcular el gradiente de la función
\[ f(x,y,z)=\log \frac{\sqrt{x}}{yz}+\arcsen (xz). \]
}
%SOLUCIÓN
{$\nabla f(x,y,z) = \left( \frac{z}{\sqrt{1-x^2z^2}}+\frac{1}{2x} ,\frac{-1}{y} , \frac{x}{\sqrt{1-x^2\,z^2}}-\frac{1}{z} \right) $.
}
%RESOLUCIÓN
{
}


\newproblem{par-4}{gen}{}
% ENUNCIADO
{Una nave espacial está en problemas cerca del sol.
Se encuentra en la posición $(1,1,1)$ y la temperatura de la nave cuando está en la posición $(x,y,z)$ viene dada por
$T(x,y,z)=\mbox{e}^{-x^2-2y^2-3z^2}$ donde $x,y,z$ se miden en metros.
¿En qué dirección debe moverse la nave para que la temperatura decrezca lo más rápidamente posible? }
%SOLUCIÓN
{Debe moverse en la dirección $-\nabla f(1,1,1)=e^{-6}(2,4,6)$.
}
%RESOLUCIÓN
{
}

\newproblem{par-5}{gen}{*}
%ENUNCIADO
{Dada la función
\[
f(x,y,z)=\log \sqrt{xy-\frac{z^2}{xy}}
\]
\begin{enumerate}
\item Hallar el vector gradiente.
\item Hallar un punto en el que el vector gradiente sea paralelo a la bisectriz del plano $XY$, y calcular el vector gradiente en dicho punto.
\end{enumerate}
}
%SOLUCIÓN
{\begin{enumerate}
\item $\nabla f(x,y,z) = \left( -\frac{z^2+x^2y^2}{2xz^2-2x^3y^2} , -\frac{z^2+x^2y^2}{2yz^2-2x^2y^3} , \frac{z}{z^2-x^2y^2}  \right) $.
\item El vector gradiente es paralelo a la bisectriz del plano $XY$ en cualquier punto de la forma $(a,a,0)$ con $a\in \mathbb{R}$.\\
$\nabla f(1,1,0) = \left(\frac{1}{2},\frac{1}{2},0\right)$.
\end{enumerate}
}
%RESOLUCIÓN
{
}


\newproblem{par-6}{far}{*}
%ENUNCIADO
{La cantidad $C$ de cierta toxina en sangre (en mg/dl) depende del número de bacterias, $b$ (bacterias/dl), del número de linfocitos, $l$ (linfocitos/dl), y del tiempo, $t$ (horas), según la ecuación:
\[
C(b,l,t) = \frac{{t^2  \cdot e^{3b + 2} }}{{l^2 }} - \frac{1}{{\log
(b \cdot l)}}
\]
\begin{enumerate}
\item Calcular su gradiente.

\item Comprobar que se cumple: $\dfrac{{\partial ^2 C}}{{\partial t\partial b}} = \dfrac{{\partial ^2 C}}{{\partial b\partial t}}$.
\end{enumerate}
}
%SOLUCIÓN
{
\begin{enumerate}
\item $\nabla C(b,l,t)=\left( \frac{{3t^2 \cdot e^{3b + 2} }}{{l^2 }}+\frac{1}{{b\log^2
(b \cdot l)}}, \frac{{-2t^2 \cdot e^{3b + 2} }}{{l^3 }}+\frac{1}{{l\log^2
(b \cdot l)}}, \frac{{2t \cdot e^{3b + 2} }}{{l^2 }} \right)$.

\item $\frac{\partial ^2 C}{\partial t \partial b}  = \frac{{6t \cdot e^{3b + 2} }}{{l^2 }}$.
\end{enumerate}
}
%RESOLUCIÓN
{
\begin{enumerate}
  \item La fórmula del gradiente es
\begin{equation}
\label{e:gradiente}
\nabla C(b,l,t)=\left(\frac{\partial C}{\partial b}, \frac{\partial C}{\partial l},\frac{\partial C}{\partial t}\right),
\end{equation}
de modo que necesitamos calcular las tres primeras derivadas parciales:
\begin{align*}
\frac{\partial C}{\partial b} &= \frac{\partial}{\partial b}\left(\frac{{t^2
\cdot e^{3b + 2} }}{{l^2 }}\right)-\frac{\partial}{\partial b}\left(\frac{1}{{\log
(b \cdot l)}}\right)= \frac{{3t^2 \cdot e^{3b + 2} }}{{l^2 }}+\frac{1}{{b\log^2
(b \cdot l)}}\\
\frac{\partial C}{\partial l} &= \frac{\partial}{\partial l}\left(\frac{{t^2
\cdot e^{3b + 2} }}{{l^2 }}\right)-\frac{\partial}{\partial l}\left(\frac{1}{{\log
(b \cdot l)}}\right)= \frac{{-2t^2 \cdot e^{3b + 2} }}{{l^3 }}+\frac{1}{{l\log^2
(b \cdot l)}}\\
\frac{\partial C}{\partial t} &= \frac{\partial}{\partial t}\left(\frac{{t^2
\cdot e^{3b + 2} }}{{l^2 }}\right)-\frac{\partial}{\partial t}\left(\frac{1}{{\log
(b \cdot l)}}\right)= \frac{{2t \cdot e^{3b + 2} }}{{l^2 }}\\
\end{align*}

Así que, sustituyendo en la fórmula \ref{e:gradiente} tenemos:
\[
\nabla C(b,l,t)=\left( \frac{{3t^2 \cdot e^{3b + 2} }}{{l^2 }}+\frac{1}{{b\log^2
(b \cdot l)}}, \frac{{-2t^2 \cdot e^{3b + 2} }}{{l^3 }}+\frac{1}{{l\log^2
(b \cdot l)}}, \frac{{2t \cdot e^{3b + 2} }}{{l^2 }} \right).
\]

\item Para ver si se satisface la igualdad calculamos ambas derivadas:
\begin{align*}
\frac{\partial ^2 C}{\partial t \partial b} & = \frac{\partial}{\partial
t}\left(\frac{\partial C}{\partial b}\right) = \frac{\partial}{\partial t}\left(
\frac{{3t^2 \cdot e^{3b + 2} }}{{l^2 }}+\frac{1}{{b\log^2
(b \cdot l)}} \right) = \frac{{6t \cdot e^{3b + 2} }}{{l^2 }} \\
\frac{\partial ^2 C}{\partial b \partial t} & = \frac{\partial}{\partial
b}\left(\frac{\partial C}{\partial t}\right) = \frac{\partial}{\partial b}\left(
\frac{{2t \cdot e^{3b + 2} }}{{l^2 }}\right) = \frac{{6t \cdot e^{3b + 2} }}{{l^2 }}
\end{align*}
Por tanto, la igualdad es cierta.
\end{enumerate}
}


\newproblem*{par-7}{amb}{*}
%ENUNCIADO
{Supongamos que la cantidad de agua almacenada en un pantano al final del año hidrológico, $A$ en hectómetros cúbicos, viene dada por:
\[
A = \sqrt {\frac{{p^3 }}{{t - 1}} - c^2 e^{cpt}}
\]
donde $p$ es la precipitación en litros/m$^2$ caí­da durante el año hidrológico, $t$ es la temperatura media del año hidrológico en ºC y $c$ el consumo debido a abastecimiento de poblaciones cercanas y riego, en hectómetros cúbicos.
Se pide:
\begin{enumerate}
\item Calcular el gradiente de la cantidad de agua almacenada.
\item Suponiendo que hubiese algún año en el que el consumo fuese nulo, ¿qué condición tendría­ que cumplir la temperatura para que la derivada del agua almacenada con respecto a la temperatura fuese igual a la derivada con respecto a la precipitación?
\end{enumerate}
}


\newproblem*{par-8}{gen}{*}
%ENUNCIADO
{Dada la función $f(x)=e^{2xy}\sen(x+3z)$, se pide:
\begin{enumerate}
  \item ¿Calcular el vector gradiente en el origen de coordenadas?
  \item ¿Es cierto que $\dfrac{\partial^3f}{\partial y^2\partial z}=\dfrac{\partial^3f}{\partial y\partial z\partial y}?$
\end{enumerate}
}


\newproblem{par-9}{gen}{*}
%ENUNCIADO
{La variable aleatoria bidimensional $(X,Y)$ con función de densidad
\[
f(x,y) = \frac{1}{\sqrt{2\pi}\, \sigma_x\sigma_y} e^{-\frac{1}{2}\left(\frac{(x-\mu_x)^2}{\sigma_x^2}+\frac{(y-\mu_y)^2}{\sigma_y^2}\right)}
\]
se conoce como normal bidimensional con $X$ e $Y$ independientes, de parámetros $\mathbf{\mu}=(\mu_x,\mu_y)$ y $\mathbf{\sigma}=(\sigma_x,\sigma_y)$.
Calcular el gradiente de $f$ e interpretarlo. ¿En qué punto se anula el gradiente? ¿Qué conclusiones sacas? ¿Cuál es la tasa de crecimiento de $f$ cuando $x\rightarrow \infty$?
}
%SOLUCIÓN
{$\nabla f(x,y) = -\frac{1}{\sqrt{2\pi}\, \sigma_x\sigma_y} e^{-\frac{1}{2}\left(\frac{(x-\mu_x)^2}{\sigma_x^2}+\frac{(y-\mu_y)^2}{\sigma_y^2}\right)} \left(\frac{x-\mu_x}{\sigma_x^2}, \frac{y-\mu_y}{\sigma_y^2}\right)$.\\
El gradiente se anula en $(x=\mu_x, y=\mu_y)$.\\
$\lim_{x\rightarrow \infty}f(x,y) = 0$.
}
%RESOLUCIÓN
{
}


\newproblem{par-10}{gen}{*}
%ENUNCIADO
{La ecuación diferencial parcial
\[
\displaystyle{\frac{\partial^2 u}{\partial x^2}} + \ \displaystyle{\frac{\partial^2 u}{\partial y^2}} + \displaystyle{\frac{\partial^2 u}{\partial z^2}} = 0,
\]
se conoce como ecuación de Laplace se aplica a multitud de fenómenos relacionadas con conducción de calor, flujo de fluidos y potencial eléctrico.

Dada la función $u(x,y,z)=\dfrac{1}{ \sqrt{x^2 + y^2 + z^2}},$
\begin{enumerate}
\item Comprobar que $f$ satisface la ecuación de Laplace.
\item ¿Existe algún punto en el que el crecimiento de la función sea nulo?
\item Si fijamos $z=1$, calcular
\[
\frac{\partial^4u}{\partial x^2\partial y^2}.
\]
\end{enumerate}
}
%SOLUCIÓN
{
\begin{enumerate}[start=2]
\item No hay ningún punto donde se el crecimiento es nulo.
\item $\frac{{\partial ^4 u}}{{\partial x^2 \partial y^2 }} =3\left( {x^2  + y^2  + 1} \right)^{ - 5/2}  - 15\left( {x^2  + y^2
} \right)\left( {x^2  + y^2 + 1} \right)^{ - 7/2}  + 105x^2 y\left({x^2  + y^2  + 1} \right)^{ - 9/2}$.
\end{enumerate}
}
%RESOLUCIÓN
{\begin{enumerate}
\item Para comprobar que $u(x,y,z)$ satisface la ecuación de Laplace
calculamos las tres derivadas parciales segundas que intervienen en
la ecuación. Comenzando con las derivadas parciales con respecto a
la variable $x$, obtenemos:
\[
u(x,y,z) = \frac{1}{{\sqrt {x^2  + y^2  + z^2 } }} = \left( {x^2  +
y^2  + z^2 } \right)^{ - 1/2}
\]
\[
\frac{{\partial u}}{{\partial x}} =  - \frac{1}{2}\left( {x^2  + y^2
+ z^2 } \right)^{ - 3/2} 2x =  - x\left( {x^2  + y^2  + z^2 }
\right)^{ - 3/2}
\]
\[
\frac{{\partial ^2 u}}{{\partial x^2 }} = \frac{\partial }{{\partial
x}}\left( { - x\left( {x^2  + y^2  + z^2 } \right)^{ - 3/2} }
\right) =  - \left( {x^2  + y^2  + z^2 } \right)^{ - 3/2}  + 3x^2
\left( {x^2  + y^2  + z^2 } \right)^{ - 5/2}
\]
e igualmente para las variables $y$ y $z$, tenemos:
\[
\frac{{\partial u}}{{\partial y}} =  - y\left( {x^2  + y^2  + z^2 }
\right)^{ - 3/2}
\]
\[
\frac{{\partial ^2 u}}{{\partial y^2 }} =  - \left( {x^2  + y^2  +
z^2 } \right)^{ - 3/2}  + 3y^2 \left( {x^2  + y^2  + z^2 } \right)^{
- 5/2}
\]
\[
\frac{{\partial u}}{{\partial z}} =  - z\left( {x^2  + y^2  + z^2 }
\right)^{ - 3/2}
\]
\[
\frac{{\partial ^2 u}}{{\partial z^2 }} =  - \left( {x^2  + y^2  +
z^2 } \right)^{ - 3/2}  + 3z^2 \left( {x^2  + y^2  + z^2 } \right)^{
- 5/2}
\]
Por lo tanto:
\[
\frac{{\partial ^2 u}}{{\partial x^2 }} + \frac{{\partial ^2
u}}{{\partial y^2 }} + \frac{{\partial ^2 u}}{{\partial z^2 }} =  -
3\left( {x^2  + y^2  + z^2 } \right)^{ - 3/2}  + 3\left( {x^2  + y^2
+ z^2 } \right)\left( {x^2  + y^2  + z^2 } \right)^{ - 5/2}  =
\]
\[
=- 3\left( {x^2  + y^2  + z^2 } \right)^{ - 3/2}  + 3\left( {x^2  +
y^2 + z^2 } \right)^{ - 3/2}  = 0
\]

\item Una condición necesaria para que el crecimiento de una función
de varias variables en un punto sea nulo es que el gradiente en
dicho punto se anule, y el gradiente se anula si se anulan sus tres
componentes:
\[
\vec \nabla u = \vec 0 \Leftrightarrow \left( {\frac{{\partial
u}}{{\partial x}},\frac{{\partial u}}{{\partial y}},\frac{{\partial
u}}{{\partial z}}} \right) = \left( {0,0,0} \right)
\]
Por lo tanto, tenemos un sistema no lineal de tres ecuaciones con
tres incógnitas:
\[
 - x\left( {x^2  + y^2  + z^2 } \right)^{ - 3/2}  = 0
\]
\[
 - y\left( {x^2  + y^2  + z^2 } \right)^{ - 3/2}  = 0
\]
\[
 - z\left( {x^2  + y^2  + z^2 } \right)^{ - 3/2}  = 0
\]
Y teniendo en cuenta que el término $(x^2+y^2+z^2)$, por tratarse de
una suma de cuadrados, únicamente puede ser 0 si $x=y=z=0$; y a
igual conclusión llegamos si suponemos que es distinto de 0, ya que
entonces la primera ecuación implica que necesariamente $x=0$, la
segunda implica que $y=0$, y la tercera implica que $z=0$. Por lo
tanto, concluimos que el único punto en el que el crecimiento puede
ser nulo es $(x,y,z)=(0,0,0)$, pero dicho punto no pertenece al
dominio de definición de la función (tendríamos un cero como
denominador de una fracción), por lo que no hay ningún punto en el
que la función presente un crecimiento nulo.

\item Suponiendo $z=1$, la función resultante presenta únicamente
dos variables:
\[
u(x,y,1) = \frac{1}{{\sqrt {x^2  + y^2  + 1} }} = \left( {x^2  + y^2
+ 1} \right)^{ - 1/2}
\]
La derivada propuesta es:
\[
\frac{{\partial ^4 u}}{{\partial x^2 \partial y^2 }} =
\frac{\partial }{{\partial x}}\left( {\frac{\partial }{{\partial
x}}\left( {\frac{\partial }{{\partial y}}\left( {\frac{{\partial
u}}{{\partial y}}} \right)} \right)} \right)
\]
en donde, como ya sabemos, se puede cambiar el orden de derivación
sin que afecte al resultado final, aunque nunca el número total de
derivadas con respecto a cada variable.

Operando como ya hicimos en los cálculos previos de las derivadas
segundas, obtenemos:
\[
\frac{{\partial u}}{{\partial y}} =  - y\left( {x^2  + y^2  + 1}
\right)^{ - 3/2}
\]
\[
\frac{\partial }{{\partial y}}\left( {\frac{{\partial u}}{{\partial
y}}} \right) = \frac{{\partial ^2 u}}{{\partial y^2 }} =  - \left(
{x^2  + y^2  + 1} \right)^{ - 3/2}  + 3y^2 \left( {x^2  + y^2  + 1}
\right)^{ - 5/2}
\]
\[
\frac{\partial }{{\partial x}}\left( {\frac{{\partial ^2
u}}{{\partial y^2 }}} \right) = \frac{{\partial ^3 u}}{{\partial
x\partial y^2 }} = 3x\left( {x^2  + y^2  + 1} \right)^{ - 5/2}  -
15y^2 x\left( {x^2  + y^2  + 1} \right)^{ - 7/2}
\]
\[
\frac{\partial }{{\partial x}}\left( {\frac{{\partial ^3
u}}{{\partial x\partial y^2 }}} \right) = \frac{{\partial ^4
u}}{{\partial x^2 \partial y^2 }} =
\]
\[
=3\left( {x^2  + y^2  + 1} \right)^{ - 5/2}  - 15\left( {x^2  + y^2
} \right)\left( {x^2  + y^2 + 1} \right)^{ - 7/2}  + 105x^2 y\left(
{x^2  + y^2  + 1} \right)^{ - 9/2}
\]
\end{enumerate}
}


\newproblem{par-11}{qui}{*}
%ENUNCIADO
{La siguiente función determina la temperatura en cada punto del plano real:
\[f(x,y)=e^{x+2y}\cos(x^2+y^2).\]
Se pide:
\begin{enumerate}
  \item Calcular el gradiente de $f$.
  \item Si estamos situados en el origen de coordenadas, ¿en qué dirección aumentará más rápidamente la temperatura? ¿Y si estuviésemos en el punto $(0,1)$?
\item Calcular la matriz Hessiana y el Hessiano de $f$ en el origen de coordenadas.
\end{enumerate}
}
%SOLUCIÓN
{\begin{enumerate}
\item $\nabla f(x,y) = e^{x+2y}\left(\cos(x^{2}+y^{2})-2x\sen(x^{2}+y^{2}), 2\cos(x^{2}+y^{2})-2y\sen(x^{2}+y^{2})\right)$.
\item $\nabla f(0,0) = (1,2)$ y $\nabla f(0,1) = (3.99\,,\,-4.45)$.
\item $Hf(0,0)=\left(
\begin{array}[]{cc}
1 & 2 \\
2 & 4
\end{array}
\right)
\quad |Hf(0,0)|= 0$.
\end{enumerate}
}
%RESOLUCIÓN
{\begin{enumerate}
\item Para calcular el vector gradiente de $f$ necesitamos calcular sus derivadas parciales de primer orden.
\begin{align*}
\frac{\partial}{\partial x}f(x,y) &= \frac{\partial}{\partial x}\left(e^{x+2y}\cos(x^{2}+y^{2})\right) = \frac{\partial}{\partial x}e^{x+2y}\cos(x^{2}+y^{2}) + e^{x+2y}\frac{\partial}{\partial x}\cos(x^{2}+y^{2}) = \\
&= e^{x+2y}\frac{\partial}{\partial x}(x+2y)\cos(x^{2}+y^{2})+e^{x+2y}(-\sen(x^{2}+y^{2})\frac{\partial}{\partial x}(x^{2}+y^{2}) =\\
&= e^{x+2y}\cos(x^{2}+y^{2})-e^{x+2y}\sen(x^{2}+y^{2})2x = e^{x+2y}(\cos(x^{2}+y^{2})-2x\sen(x^{2}+y^{2}),
\\
\frac{\partial}{\partial y}f(x,y) &= \frac{\partial}{\partial y}\left(e^{x+2y}\cos(x^{2}+y^{2})\right) = \frac{\partial}{\partial y}e^{x+2y}\cos(x^{2}+y^{2}) + e^{x+2y}\frac{\partial}{\partial y}\cos(x^{2}+y^{2}) = \\
&= e^{x+2y}\frac{\partial}{\partial y}(x+2y)\cos(x^{2}+y^{2})+e^{x+2y}(-\sen(x^{2}+y^{2})\frac{\partial}{\partial y}(x^{2}+y^{2}) =\\
&= e^{x+2y}\cos(x^{2}+y^{2})2-e^{x+2y}\sen(x^{2}+y^{2})2y= e^{x+2y}(2\cos(x^{2}+y^{2})-2y\sen(x^{2}+y^{2}),
\end{align*}
Así pues, el vector gradiente es
\begin{align*}
\nabla f(x,y) &= \left(\dfrac{\partial}{\partial x}f(x,y),\dfrac{\partial}{\partial y}f(x,y)\right) =\\
&= e^{x+2y}\left(\cos(x^{2}+y^{2})-2x\sen(x^{2}+y^{2}), 2\cos(x^{2}+y^{2})-2y\sen(x^{2}+y^{2})\right).
\end{align*}

\item La dirección en que más rápidamente aumenta la temperatura es la dirección del vector gradiente. Si estamos en el origen de coordenadas, dicha dirección es
\[
\nabla f(0,0) = e^{0+2\cdot 0}\left(\cos(0^{2}+0^{2})-2\cdot 0\sen(0^{2}+0^{2}), 2\cos(0^{2}+0^{2})-2\cdot 0\sen(0^{2}+0^{2})\right) = (1,2).
\]
Y si estamos en el punto $(0,1)$, la dirección de máximo crecimiento de la temperatura es
\begin{align*}
\nabla f(0,1) &= e^{0+2\cdot 1}\left(\cos(0^{2}+1^{2})-2\cdot 0\sen(0^{2}+1^{2}), 2\cos(0^{2}+1^{2})-2\cdot 1\sen(0^{2}+1^{2})\right) =\\
&= e^{2}(\cos 1, 2\cos 1-2\sen 1) = (3.99\,,\,-4.45).
\end{align*}

\item Para calcular la matriz Hessiana necesitamos calcular las derivadas parciales de segundo orden de $f$.
\begin{align*}
\frac{\partial^{2}}{\partial x^{2}}f(x,y) &= \frac{\partial}{\partial x}\left(\frac{\partial}{\partial x}f(x,y)\right) = \frac{\partial}{\partial x}\left(e^{x+2y}(\cos(x^{2}+y^{2})-2x\sen(x^{2}+y^{2})\right) = \\
&= \frac{\partial}{\partial x}e^{x+2y}(\cos(x^{2}+y^{2})-2x\sen(x^{2}+y^{2})+\\
&+ e^{x+2y}\frac{\partial}{\partial x}(\cos(x^{2}+y^{2})-2x\sen(x^{2}+y^{2}) = \\
&= e^{x+2y}(\cos(x^{2}+y^{2})-2x\sen(x^{2}+y^{2})+\\
&+ e^{x+2y}(-\sen(x^{2}+y^{2})2x-2\sen(x^{2}+y^{2})-2x\cos(x^{2}+y^{2}))2x = \\
&= e^{x+2y}((1-4x^{2})\cos(x^{2}+y^{2})-(4x+2)\sen(x^{2}+y^{2})),\\
\frac{\partial^{2}}{\partial y\partial x}f(x,y) &= \frac{\partial}{\partial y}\left(\frac{\partial}{\partial x}f(x,y)\right) = \frac{\partial}{\partial y}\left(e^{x+2y}(\cos(x^{2}+y^{2})-2x\sen(x^{2}+y^{2})\right) = \\
&= \frac{\partial}{\partial y}e^{x+2y}(\cos(x^{2}+y^{2})-2x\sen(x^{2}+y^{2})+\\
&+ e^{x+2y}\frac{\partial}{\partial y}(\cos(x^{2}+y^{2})-2x\sen(x^{2}+y^{2}) = \\
&= e^{x+2y}2(\cos(x^{2}+y^{2})-2x\sen(x^{2}+y^{2})+\\
&+ e^{x+2y}(-\sen(x^{2}+y^{2})2y-2x\cos(x^{2}+y^{2}))2y = \\
&= e^{x+2y}((2-4xy)\cos(x^{2}+y^{2})-(4x+2y)\sen(x^{2}+y^{2})),\\
\end{align*}
\begin{align*}
\frac{\partial^{2}}{\partial x\partial y}f(x,y) &= \frac{\partial^{2}}{\partial y\partial x}\quad \mbox{(Igualdad de derivadas cruzadas),}\\
%
\frac{\partial^{2}}{\partial y^{2}}f(x,y) &= \frac{\partial}{\partial y}\left(\frac{\partial}{\partial y}f(x,y)\right) = \frac{\partial}{\partial y}\left(e^{x+2y}(2\cos(x^{2}+y^{2})-2y\sen(x^{2}+y^{2})\right) = \\
&= \frac{\partial}{\partial y}e^{x+2y}(2\cos(x^{2}+y^{2})-2y\sen(x^{2}+y^{2})+\\
&+ e^{x+2y}\frac{\partial}{\partial y}(2\cos(x^{2}+y^{2})-2y\sen(x^{2}+y^{2}) = \\
&= e^{x+2y}2(2\cos(x^{2}+y^{2})-2y\sen(x^{2}+y^{2})+\\
&+ e^{x+2y}(-2\sen(x^{2}+y^{2})2y-2\sen(x^{2}+y^{2})-2y\cos(x^{2}+y^{2}))2y = \\
&= e^{x+2y}((4-4y^{2})\cos(x^{2}+y^{2})-(8y+2)\sen(x^{2}+y^{2})).
\end{align*}
\end{enumerate}
Así pues la matriz hessiana es
\[Hf(x,y)= \left(
\begin{array}{cc}
\frac{\partial^{2}}{\partial x^{2}}f(x,y) & \frac{\partial^{2}}{\partial x\partial y}f(x,y)\\
\frac{\partial^{2}}{\partial y\partial x}f(x,y) & \frac{\partial^{2}}{\partial y^{2}}f(x,y)
\end{array}
\right) =
\]
\[=
e^{x+2y} \left(
\begin{array}[]{cc}
(1-4x^{2})\cos(x^{2}+y^{2})-(4x+2)\sen(x^{2}+y^{2}) & (2-4xy)\cos(x^{2}+y^{2})-(4x+2y)\sen(x^{2}+y^{2})\\
(2-4xy)\cos(x^{2}+y^{2})-(4x+2y)\sen(x^{2}+y^{2}) & (4-4y^{2})\cos(x^{2}+y^{2})-(8y+2)\sen(x^{2}+y^{2})
\end{array}
\right)
\]
En el origen de coordenadas, la matriz Hessiana es
\[
Hf(0,0)=\left(
\begin{array}[]{cc}
1 & 2 \\
2 & 4
\end{array}
\right)
\]
y el hessiano vale
\[
|Hf(0,0)|=\left|
\begin{array}[]{cc}
1 & 2 \\
2 & 4
\end{array}
\right| =
4-4 = 0.
\]
}


\newproblem{par-12}{gen}{*}
%ENUNCIADO
{Se  dice que la función $z(t,x,y)$ satisface la ecuación de ondas si verifica la ecuación en derivadas parciales:
\[
\frac{{\partial ^2 z}} {{\partial t^2 }} = k^2 \left(
{\frac{{\partial ^2 z}} {{\partial x^2 }} + \frac{{\partial ^2 z}}
{{\partial y^2 }}} \right)
\]
para algún $k\in \mathbb{R}$.

Comprobar que la función:
\[
z\left( {t,x,y} \right) = \cos (ax)\sen(by)\sen\left( {kt\sqrt
{a^2 + b^2 } } \right)
\]
donde $a,b,k \in \mathbb{R}$, satisface la ecuación de ondas.
}
%SOLUCIÓN
{Si la satisface.
}
%	RESOLUCIÓN
{Para comprobar que $z(t,x,y)$ satisface la ecuación de ondas vamos a calcular primero las derivadas parciales de segundo orden que aparecen en dicha ecuación:
\begin{align*}
\frac{\partial^2 z}{\partial t^2} &=
\frac{\partial}{\partial t}\left(\frac{\partial z}{\partial t}\right) =
\frac{\partial}{\partial t}\left(\frac{\partial}{\partial t}\left(\cos(ax)\sen(by)\sen(kt\sqrt{a^2+b^2})\right)\right)= \\
&= \frac{\partial}{\partial t}\left(\cos(ax)\sen(by)\frac{\partial}{\partial t}\left(\sen(kt\sqrt{a^2+b^2})\right)\right)=\\
&= \frac{\partial}{\partial t}\left(\cos(ax)\sen(by)\cos(kt\sqrt{a^2+b^2})\frac{\partial}{\partial t}(kt\sqrt{a^2+b^2})\right)=\\
&= \frac{\partial}{\partial t}\left(\cos(ax)\sen(by)\cos(kt\sqrt{a^2+b^2}) k\sqrt{a^2+b^2}\right)=\\
&= k\sqrt{a^2+b^2}\cos(ax)\sen(by)\frac{\partial}{\partial t}\left(\cos(kt\sqrt{a^2+b^2}) \right)=\\
&= k\sqrt{a^2+b^2}\cos(ax)\sen(by)(-\sen(kt\sqrt{a^2+b^2}))\frac{\partial}{\partial t}\left(kt\sqrt{a^2+b^2}\right)=\\
&=k\sqrt{a^2+b^2}\cos(ax)\sen(by)(-\sen(kt\sqrt{a^2+b^2}))k\sqrt{a^2+b^2}=\\
&= -k^2(a^2+b^2)\cos(ax)\sen(by)\sen(kt\sqrt{a^2+b^2}),\\[.5cm]
\frac{\partial^2 z}{\partial x^2} &=
\frac{\partial}{\partial x}\left(\frac{\partial z}{\partial x}\right)
= \frac{\partial}{\partial x}\left(\frac{\partial}{\partial x}\left(\cos(ax)\sen(by)\sen(kt\sqrt{a^2+b^2})\right)\right)= \\
&= \frac{\partial}{\partial x}\left(\frac{\partial}{\partial x}\left(\cos(ax)\right)\sen(by)\sen(kt\sqrt{a^2+b^2})\right)=\\
&= \frac{\partial}{\partial x}\left(-\sen(ax)a\sen(by)\sen(kt\sqrt{a^2+b^2})\right)=\\
&= \frac{\partial}{\partial x}\left(-\sen(ax)\right)a\sen(by)\cos(kt\sqrt{a^2+b^2}) =\\
&= -a^2\cos(ax)\sen(by)\cos(kt\sqrt{a^2+b^2}),\\[.5cm]
\frac{\partial^2 z}{\partial y^2} &=
\frac{\partial}{\partial y}\left(\frac{\partial z}{\partial y}\right)
= \frac{\partial}{\partial y}\left(\frac{\partial}{\partial y}\left(\cos(ax)\sen(by)\sen(kt\sqrt{a^2+b^2})\right)\right)= \\
&= \frac{\partial}{\partial y}\left(\cos(ax)\frac{\partial}{\partial y}\left(\sen(by)\right)\sen(kt\sqrt{a^2+b^2})\right)=\\
&= \frac{\partial}{\partial y}\left(\cos(ax)\cos(by)b\sen(kt\sqrt{a^2+b^2})\right)=\\
&= \cos(ax)\frac{\partial}{\partial y}\left(\cos(by)\right)b\cos(kt\sqrt{a^2+b^2}) =\\
&= -b^2\cos(ax)\sen(by)\cos(kt\sqrt{a^2+b^2}).
\end{align*}
Para terminar, sustituimos estas derivadas en la ecuación de ondas y constatamos que efectivamente se cumple
\begin{align*}
& -k^2(a^2+b^2)\cos(ax)\sen(by)\sen(kt\sqrt{a^2+b^2}) =\\
&= k^2\left(-a^2\cos(ax)\sen(by)\cos(kt\sqrt{a^2+b^2})-b^2\cos(ax)\sen(by)\cos(kt\sqrt{a^2+b^2})\right).
\end{align*}
}


\newproblem{par-13}{gen}{*}
%ENUNCIADO
{Dadas las siguientes funciones de dos variables:
\[
\begin{array}{*{20}c}
   {f(x,y) = x^2  - 2xy^2  + \sen(xy)}  \\
   {g(x,y) = \left( {2x+ 3y^2 } \right)e^{\left( {1 - x^2  - y^2 } \right)} }  \\

 \end{array}
\]
\begin{enumerate}
\item Calcular el gradiente de cada una de ellas.
\item ¿A cuál de las funciones corresponde el siguiente dibujo del gradiente en los puntos $(1,0)$, $(0,1)$, $(-1,0)$ y $(0,-1)$?
\begin{center}
\includegraphics[scale=0.45,angle=270]{img/vectores-par-13}
\end{center}
\end{enumerate}
}
%SOLUCIÓN
{\begin{enumerate}
\item $\nabla f(x,y) = \left(2x-2y^2+\cos(xy)y\, ,\, -4xy+\cos(xy)x\right)$  y
$\nabla g(x,y) = \left((-4x^2-6xy^2+2)\, ,\, (-4xy-6y^3+6y)\right)e^{1-x^2-y^2}$.
\item Los vectores gradientes son de la función $f$.
\end{enumerate}
}
%RESOLUCIÓN
{\begin{enumerate}
\item Para calcular el gradiente necesitamos calcular las derivadas parciales de $f$ y $g$ con respecto a sus variables:
\begin{align*}
\frac{\partial f}{\partial x}(x,y) &= \frac{\partial}{\partial
x}\left(x^2- 2xy^2+\sen(xy)\right)=
2x-2y^2+\cos(xy)\frac{\partial}{\partial x}(xy)=2x-2y^2+\cos(xy)y,\\
\frac{\partial f}{\partial y}(x,y) &= \frac{\partial}{\partial
y}\left(x^2- 2xy^2+\sen(xy)\right)=
-4xy+\cos(xy)\frac{\partial}{\partial y}(xy)=-4xy+\cos(xy)x,\\
\frac{\partial g}{\partial x}(x,y) &= \frac{\partial}{\partial x}\left((2x+3y^2)e^{1-x^2-y^2}\right)=
\frac{\partial}{\partial x}(2x+3y^2)e^{1-x^2-y^2}+(2x+3y^2)\frac{\partial}{\partial x}e^{1-x^2-y^2}=\\
&= 2e^{1-x^2-y^2}+(2x+3y^2)e^{1-x^2-y^2}\frac{\partial}{\partial x}\left(1-x^2-y^2\right) = \\
&= 2e^{1-x^2-y^2}+(2x+3y^2)e^{1-x^2-y^2}(-2x)= (-4x^2-6xy^2+2)e^{1-x^2-y^2},\\
\frac{\partial g}{\partial y}(x,y) &= \frac{\partial}{\partial y}\left((2x+3y^2)e^{1-x^2-y^2}\right)=
\frac{\partial}{\partial x}(2x+3y^2)e^{1-x^2-y^2}+(2x+3y^2)\frac{\partial}{\partial y}e^{1-x^2-y^2}=\\
&= 6y e^{1-x^2-y^2}+(2x+3y^2)e^{1-x^2-y^2}\frac{\partial}{\partial y}\left(1-x^2-y^2\right) =\\
&=6ye^{1-x^2-y^2}+(2x+3y^2)e^{1-x^2-y^2}(-2y)= (-4xy-6y^3+6y)e^{1-x^2-y^2}.
\end{align*}
Así pues, los gradientes son
\begin{align*}
\nabla f(x,y) &=\left(\frac{\partial f}{\partial x}(x,y),\frac{\partial
f}{\partial y}(x,y)\right) = \left(2x-2y^2+\cos(xy)y\, ,\, -4xy+\cos(xy)x\right) \\
\nabla g(x,y) &=\left(\frac{\partial g}{\partial x}(x,y),\frac{\partial
g}{\partial y}(x,y)\right) = \left((-4x^2-6xy^2+2)\, ,\, (-4xy-6y^3+6y)\right)e^{1-x^2-y^2}
\end{align*}

\item Para ver a qué función corresponde la gráfica, calculamos el gradiente en los puntos que nos dan
\begin{align*}
\nabla f(1,0) &= \left(2\cdot 1-2\cdot 0^2+\cos(1\cdot 0)\cdot 0\, ,\, -4\cdot1\cdot0+\cos(1\cdot 0)\cdot1\right) =(2,1),\\
\nabla g(1,0) &= \left((-4\cdot1^2-6\cdot 1\cdot 0^2+2)\, ,\, (-4\cdot 1\cdot 0-6\cdot 0^3+6\cdot 0)\right)e^{1-1^2-0^2}= (-2,0).
\end{align*}
Como el vector libre situado en el punto $(1,0)$ es el $(2,1)$, la gráfica no puede pertenecer a la función $g(x,y)$. Para asegurarnos que se corresponde con la $f(x,y)$, calculamos el gradiente de esta función en el resto de los puntos:
\begin{align*}
\nabla f(0,1) &= \left(2\cdot 0-2\cdot 1^2+\cos(0\cdot 1)\cdot 1\, ,\, -4\cdot0\cdot1+\cos(0\cdot 1)\cdot0\right) =(-1,0),\\
\nabla f(-1,0) &= \left(2\cdot (-1)-2\cdot 0^2+\cos(-1\cdot 0)\cdot 0\, ,\, -4\cdot-1\cdot0+\cos(-1\cdot 0)\cdot(-1)\right) =(-2,-1),\\
\nabla f(0,-1) &= \left(2\cdot 0-2\cdot (-1)^2+\cos(0\cdot (-1))\cdot (-1)\, ,\, -4\cdot0\cdot(-1)+\cos(0\cdot (-1))\cdot0\right) =(-3,0).
\end{align*}
Luego los vectores de la gráfica se corresponden con los vectores gradientes de $f(x,y)$.
\end{enumerate}
}


\newproblem{par-14}{gen}{}
%ENUNCIADO
{Tenemos dos objetos de masas $m_1$ y $m_2$ unidas por una cuerda que pasa a través de una polea como la de la figura.
\begin{center}
  \includegraphics[scale=0.5]{img/polea-par-14}
\end{center}
Si $m_1\geq m_2$, la aceleración de $m_1$ viene dada por la ecuación
\[
a=\frac{m_1-m_2}{m_1+m_2}g,
\]
siendo $g$ la aceleración de la gravedad.
Demostrar que se cumple la ecuación
\[
m_1\frac{\partial a}{\partial m_1}+m_2\frac{\partial a}{\partial m_2}=0.
\]
}
%SOLUCIÓN
{$\dfrac{\partial a}{\partial m_1} = \dfrac{2gm_2}{(m_1+m_2)^2}$ y $\dfrac{\partial a}{\partial m_2} = \dfrac{-2gm_1}{(m_1+m_2)^2}$.
}
%RESOLUCIÓN
{
}


\newproblem{par-15}{gen}{*}
%ENUNCIADO
{La relación que modeliza el potencial eléctrico $V$ de un punto del plano en función de su distancia, es $V=\log D$, donde $D=\sqrt{x^2+y^2}$.

Se pide:
\begin{enumerate}
\item Calcular el gradiente de $V$.
\item Hallar la dirección de máxima variación del potencial
eléctrico en el punto $(x,y)=(\sqrt{3},\sqrt{3})$.
\item Calcular la matriz Hessiana y el Hessiano de $V$ en el punto anterior.
\item Si nos movemos a lo largo de la curva $y=x+1$, cuál será el mínimo potencial alcanzado?
\end{enumerate}
}
%SOLUCIÓN
{\begin{enumerate}
\item $\nabla V(x,y) = \left( \frac{x}{x^2+y^2},\frac{y}{x^2+y^2}\right)$.
\item $\nabla V(\sqrt 3, \sqrt 3) = \sqrt 3 /6(1,1)$.
\item $
HV(x,y) = \left(
\begin{array}{cc}
\frac{y^2-x^2}{y^4+2x^2y^2+x^4} & \frac{-2xy}{y^4+2x^2y^2+x^4} \\
\frac{-2xy}{y^4+2x^2y^2+x^4} & \frac{x^2-y^2}{y^4+2x^2y^2+x^4}
\end{array}
\right),\quad
\left(
\begin{array}{cc}
0 & -1/6 \\
-1/6 & 0
\end{array}
\right),\quad \mbox{y }
|H(\sqrt 3,\sqrt 3)| = -1/36.
$
\item El potencial máximo se alcanza en $(x=-1/2, y=1/2)$ y vale $V(-1/2,1/2) = -\dfrac{\log 2}{2}$.
\end{enumerate}
}
%RESOLUCIÓN
{
}


\newproblem*{par-16}{qui}{}
%ENUNCIADO
{La ecuación unidimensional del calor es
\[
\frac{\partial q}{\partial t}=c^2\frac{\partial^2q}{\partial x^2},
\]
donde $c$ es una constante y $q(x,t)$ es la temperatura de una varilla en un punto que ocupa la posición $x$ en el instante $t$. Demostrar que $q(x,t)=e^{ax+bt}$, con $a\neq 0$, satisface dicha ecuación para un valor apropiado de $c$.
}


\newproblem*{par-17}{amb}{*}
%ENUNCIADO
{Suponiendo que la temperatura, $T$ en ºC, de una zona de la atmósfera es función de la densidad del aire, $d$, en g por cm$^3$, la altura, $h$, en kilómetros, y de la concentración de un determinado elemento, $c$, en mg por cm$^3$, viene dada por la expresión:
\[
T(d,h,c) = \frac{{\ln (dh)}}{c} + c^2 3^{hd}
\]
\begin{enumerate}
\item Suponiendo que la altura a la que medimos la temperatura es de un kilómetro, y que la temperatura medida es de 0 ºC, dar la expresión de la concentración en función de la densidad.
\item Calcular el gradiente de la temperatura en el punto $(d_0,h_0,c_0)=(1,1,2)$.
\item Comprobar que se cumple que:
\[
\frac{{\partial ^2 T}}{{\partial d\partial h}} = \frac{{\partial ^2
T}}{{\partial h\partial d}}
\]
\end{enumerate}
}


\newproblem{par-18}{gen}{}
%ENUNCIADO
{Sea $z(x,y)=\dfrac{x^{2}}{y}+\dfrac{y^{2}}{x}.$ Calcular todas sus derivadas parciales de primer y segundo orden.
}
%SOLUCIÓN
{$\frac{\partial z}{\partial x} = \frac{2x}{y}-\frac{y^2}{x^2}$, $\dfrac{\partial z}{\partial x} = \frac{2y}{x}-\frac{x^2}{y^2}$,\\
$\frac{\partial^2 z}{\partial x^2} = \frac{2y^2}{x^3}+\frac{2}{y}$, $\frac{\partial^2 z}{\partial y\partial x} = -\frac{2y}{x^2}-\frac{2x}{y^2}$, $\frac{\partial^2 z}{\partial x\partial y} = -\frac{2y}{x^2}-\frac{2x}{y^2}$, $\frac{\partial^2 z}{\partial x^2} = \frac{2x^2}{y^3}+\frac{2}{x}$.
}
%RESOLUCIÓN
{
}


\newproblem*{par-19}{gen}{}
%ENUNCIADO
{Dada la función $f(x,y)=\dfrac{x-y}{x+y}$, hallar $\dfrac{\partial f}{\partial x}$ y $\dfrac{\partial f}{\partial y}$ en el punto $(2,-1)$.
}


\newproblem*{par-20}{amb}{*}
%ENUNCIADO
{Supongamos la función de varias variables $f(x,y,z)=x^{3}+\sqrt{xyz}$ que da la presión en un recipiente en función de la posición $(x,y,z)$. Suponiendo que en el recipiente hay un insecto y que se encuentra en el punto de coordenadas $(2,1,3)$, ¿en qué dirección debe moverse si busca ir lo más rápidamente posible hacia zonas de menor presión?
}


\newproblem*{par-21}{gen}{}
%ENUNCIADO
{Dado el siguiente campo escalar expresado en coordenadas cartesianas:
\[
f(x,y,z)=3xy\ln \left( \dfrac{1}{z}\right)
\]
Calcular:
\begin{enumerate}
\item  Su vector gradiente.
\item  Su matriz Hessiana.
\end{enumerate}
}


\newproblem*{par-22}{gen}{*}
%ENUNCIADO
{La definición del polinomio de Taylor de grado 2 de una función de dos variables, $f(x,y)$, centrado en el punto $(x_{0},y_{0})$, es
\begin{align*}
P_{f,(x_0,y_0)}^{2}(x,y)&= f(x_{0},y_{0})+\dfrac{\partial f(x_{0},y_{0})}{\partial x}(x-x_{0})+\dfrac{\partial f(x_{0},y_{0})}{\partial y}(y-y_{0})+\\
&+\dfrac{1}{2}\dfrac{\partial ^{2}f(x_{0},y_{0})}{\partial x^{2}}(x-x_{0})^{2}+\dfrac{1}{2}\dfrac{\partial ^{2}f(x_{0},y_{0})}{\partial y^{2}}(y-y_{0})^{2}+\dfrac{\partial ^{2}f(x_{0},y_{0})}{\partial x\partial y}(x-x_{0})(y-y_{0})
\end{align*}
\begin{enumerate}
\item  Utilizar la fórmula anterior para calcular el polinomio de Taylor de grado 2 de la función $f(x,y)=e^{(x+2y)}$ centrado en $(x_{0},y_{0})=(0,0)$.
\item  Utilizar el polinomio anterior para dar el valor aproximado de $e^{(0.1+2\cdot 0.1)}$.
\end{enumerate}
}


\newproblem*{par-23}{fis}{*}
%ENUNCIADO
{Suponiendo que el potencial eléctrico en un punto de coordenadas cartesianas $(x,y,z)$ viene dado por:
\[
V(x,y,z) = \frac{1} {{x{\kern 1pt} e^y \ln z}},
\]
calcular en el punto $(1,0,e)$:
\begin{enumerate}
\item El campo eléctrico (recordar que el campo eléctrico es el gradiente del potencial cambiado de signo: $\vec E =  - \vec\nabla V$).
\item La divergencia del campo eléctrico.
\end{enumerate}
}


\newproblem*{par-24}{gen}{*}
%ENUNCIADO
{Para la función de 2 variables $f(x,y) = x^{y^2}$
\begin{enumerate}
\item Calcular su dirección y sentido de máximo crecimiento en el punto $(1,1)$.
\item Calcular su matriz Hessiana.
\end{enumerate}
}


\newproblem{par-25}{amb}{*}
%ENUNCIADO
{La Quimiotaxis es el movimiento de los organismos dirigido por un gradiente de concentración, es decir, en la dirección
en la que la concentración aumenta con más rapidez. El moho del cieno Dictyoselium discoideum muestra este
comportamiento. En esta caso, las amebas unicelulares de esta especie se mueven según el gradiente de concentración de
una sustancia química denominada adenosina monofosfato (AMP cíclico). Si suponemos que la expresión que da la
concentración de AMP cíclico en un punto de coordenadas $(x,y,z)$ es:
\[
C(x,y,z) = \frac{4} {{\sqrt {x^2  + y^2  + z^4  + 1} }}
\]
y se sitúa una ameba de moho del cieno en el punto $(-1,0,1)$, ¿en qué dirección se moverá la ameba?
}
%SOLUCIÓN
{$(4/\sqrt{27}, 0, -8/\sqrt{27})$.
}
%RESOLUCIÓN
{
}



%%%%%%% Pendiente 26



\newproblem{par-27}{qui}{*}
%ENUNCIADO
{Supongamos que tenemos una superficie plana, y que la cantidad de una sustancia, $C$ en g/cm$^2$,
depositada sobre cada punto de coordenadas $x$ e $y$, en metros, es función del punto y del tiempo $t$, en horas, y
viene dada por la expresión:
\[
C(x,y,t) = \sqrt{e^{-\frac{3ty}{x^2+1}}}
\]
\begin{enumerate}
\item Calcular la dirección y sentido de máximo crecimiento de la
función en el punto $(x_0,y_0,t_0)=(1,0,1)$.
\item Calcular: $\dfrac{{\partial ^2 C}}{{\partial y\partial x}}$.
\item ¿En qué puntos se anulará el gradiente de $C$?
\end{enumerate}
}
%SOLUCIÓN
{\begin{enumerate}
\item $\nabla C(1,0,1) =\frac{1}{4}(0,-3,0)$.
\item $\displaystyle \frac{\partial^2 C}{\partial y\partial x} =
\frac{e^{-\frac{3ty}{2x^2+2}}}{(2x^2+2)^2}\left(\frac{-36t^2yx}{2x^2+2}+12tx\right)$.
\item En los puntos de la forma $(x,0,0)\ \forall x\in \mathbb{R}$.
\end{enumerate}
}
%RESOLUCIÓN
{Antes de nada conviene simplificar la función:
\[
C(x,y,t) = \sqrt{e^{-\frac{3ty}{x^2+1}}} = \left(e^{-\frac{3ty}{x^2+1}}\right)^{1/2} = e^{-\frac{3ty}{2x^2+2}}
\]
\begin{enumerate}
\item La dirección y sentido de máximo crecimiento de una función de varias variables la da el vector gradiente, en este caso,
\[
\nabla C(x,y,t) =\left(\frac{\partial C}{\partial x}, \frac{\partial C}{\partial y}, \frac{\partial C}{\partial t} \right)
\]
Calulamos las tres derivadas parciales:
\begin{align*}
\frac{\partial C}{\partial x} &= \frac{\partial}{\partial x} e^{-\frac{3ty}{2x^2+2}} = e^{-\frac{3ty}{2x^2+2}} \frac{\partial}{\partial x}\left(-\frac{3ty}{2x^2+2}\right) = e^{-\frac{3ty}{2x^2+2}}\frac{3ty\cdot 4x}{(2x^2+2)^2} \\
\frac{\partial C}{\partial y} &= \frac{\partial}{\partial y} e^{-\frac{3ty}{2x^2+2}} = e^{-\frac{3ty}{2x^2+2}} \frac{\partial}{\partial y}\left(-\frac{3ty}{2x^2+2}\right) = e^{-\frac{3ty}{2x^2+2}}\frac{-3t}{2x^2+2} \\
\frac{\partial C}{\partial t} &= \frac{\partial}{\partial t} e^{-\frac{3ty}{2x^2+2}} = e^{-\frac{3ty}{2x^2+2}} \frac{\partial}{\partial t}\left(-\frac{3ty}{2x^2+2}\right) = e^{-\frac{3ty}{2x^2+2}}\frac{-3y}{2x^2+2}
\end{align*}
De modo que el vector gradiente es
\[
\nabla C(x,y,t) =\frac{e^{-\frac{3ty}{2x^2+2}}}{2x^2+2}\left(\frac{12tyx}{2x^2+2}, -3t, -3y\right),
\]
y en el punto $(x_0,y_0,t_0)=(1,0,1)$ vale
\[
\nabla C(1,0,1) =\frac{e^{-\frac{3\cdot 1\cdot 0}{2\cdot 1^2+2}}}{2\cdot 1^2+2}\left(\frac{12\cdot 1\cdot 0\cdot 1}{2\cdot 1^2+2}, -3\cdot 1, -3\cdot 0\right) = \frac{1}{4}(0,-3,0).
\]

\item
\begin{align*}
\frac{\partial^2 C}{\partial y\partial x} &= \frac{\partial}{\partial y}\frac{\partial C}{\partial x} e^{-\frac{3ty}{2x^2+2}} = \frac{\partial}{\partial y}  \left(e^{-\frac{3ty}{2x^2+2}}\frac{12tyx}{(2x^2+2)^2}\right) = \\
&= \frac{\partial}{\partial y} \left(e^{-\frac{3ty}{2x^2+2}}\right)\frac{12tyx}{(2x^2+2)^2}+e^{-\frac{3ty}{2x^2+2}}\frac{\partial}{\partial y}\left(\frac{12tyx}{(2x^2+2)^2}\right) = \\
&= e^{-\frac{3ty}{2x^2+2}}\frac{\partial}{\partial y}\left(-\frac{3ty}{2x^2+2}\right)\frac{3ty\cdot 4x}{(2x^2+2)^2}+e^{-\frac{3ty}{2x^2+2}}\frac{12tx}{(2x^2+2)^2} = \\
&= e^{-\frac{3ty}{2x^2+2}}\frac{-3t}{2x^2+2}\frac{3ty\cdot 4x}{(2x^2+2)^2}+e^{-\frac{3ty}{2x^2+2}}\frac{12tx}{(2x^2+2)^2} = \\
&= \frac{e^{-\frac{3ty}{2x^2+2}}}{(2x^2+2)^2}\left(\frac{-36t^2yx}{2x^2+2}+12tx\right).
\end{align*}

\item
\[
\nabla C(x,y,t) =\frac{e^{-\frac{3ty}{2x^2+2}}}{2x^2+2}\left(\frac{12tyx}{2x^2+2}, -3t, -3y\right) = (0,0,0) \Leftrightarrow
\left\{
\begin{array}{l}
12txy =0 \\
-3t = 0\\
-3y = 0
\end{array}
\right.
\]
de donde se deduce que $t=0$, $y=0$ y $x$ puede tomar cualquier valor. Así pues, los puntos que anulan el gradiente son de la forma $(x,0,0)$, $x\in\mathbb{R}$.
\end{enumerate}
}


\newproblem{par-28}{fis}{*}
%ENUNCIADO
{Una barra de metal de un metro de largo se calienta de manera irregular y de forma tal que a $x$ metros de su extremo izquierdo y en el instante $t$ minutos, su temperatura en grados centígrados esta dada por $H(x,t) = 100e^{-0.1t}\sen(\pi xt)$ con $0\leq x \leq 1$.
\begin{enumerate}
\item Calcular $\dfrac{\partial H}{\partial x}(0.2, 1)$ y $\dfrac{\partial H}{\partial x}(0.8, 1).$ ¿Cuál es la interpretación práctica (en términos de temperatura) de estas derivadas parciales? Explicar por qué cada una tiene el signo que tiene.
\item Calcular la matriz hessiana de $H$.
\end{enumerate}
}
%SOLUCIÓN
{\begin{enumerate}
\item $\frac{\partial H}{\partial x}(0.2,\, 1) = 100e^{-0.1}\cos(0.2\pi) \pi = 229.9736$ \\
$\frac{\partial H}{\partial x}(0.8,\, 1) = 100e^{-0.1}\cos(0.8\pi) \pi = -229.9736$.
\item $
\left(
\begin{array}{cc}
-100e^{-0.1t}\pi^2 t^2\sen(\pi xt) & 100e^{-0.1t}\left((-0.1\pi t+\pi)\cos(\pi xt) - \pi^2 xt \sen(\pi xt)\right) \\
100e^{-0.1t}\left((-0.1\pi t+\pi)\cos(\pi xt) - \pi^2 xt \sen(\pi xt)\right) & 100e^{-0.1t}\left(0.01\sen(\pi xt) -(0.2+\pi^2x^2) \cos(\pi xt)\right)
\end{array}
\right)$
\end{enumerate}
}
%RESOLUCIÓN
{\begin{enumerate}
\item La derivada parcial de $H$ con respecto a $x$ es
\begin{align*}
\frac{\partial H}{\partial x}(x,t) &= 100e^{-0.1t}\cos(\pi xt) \pi t \\
\end{align*}
y en los puntos que nos piden vale
\begin{align*}
\frac{\partial H}{\partial x}(0.2,\, 1) &= 100e^{-0.1}\cos(0.2\pi) \pi =
229.9736\\
\frac{\partial H}{\partial x}(0.8,\, 1) &= 100e^{-0.1}\cos(0.8\pi) \pi =
-229.9736
\end{align*}
La derivada parcial $\dfrac{\partial H}{\partial x}(x_0,t_0)$ indica la variación instantánea que experimenta la temperatura con respecto a la variación de la distancia al extremo izquierdo en el punto. El signo de la derviada parcial indica si la variación de la temperatura es creciente (aumenta la temperatura) o decreciente (disminuye). Así en el punto $(0.2,\, 1)$ la temperatura aumentará a razón de $229.9736$ grados centígrados por cada metro que nos alejemos del extremo izquierdo de la barra de metal, mientras que en el $(0.8,\,1)$ la temperatura disminuirá a razón de $229.9736$ grados centígrados por cada metro que nos alejemos del extremo izquierdo de la barra de metal.

\item Para calcular la matriz Hessiana necesitamos las derivadas parciales de
segundo orden:
\begin{align*}
\frac{\partial H}{\partial t} (x,t) &= 100\left(\frac{\partial}{\partial x} e^{-0.1 t} \sen (\pi xt) + e^{-0.1t}\frac{\partial}{\partial x}\sen(\pi xt)\right)=\\
&= 100\left(-0.1e^{-0.1t}\sen(\pi xt) +e^{-0.1t}\cos(\pi xt)\pi x\right) =\\
&= 100 e^{-0.1t}\left(-0.1 \sen(\pi xt) + \pi x \cos(\pi xt)\right),\\
\frac{\partial^2 H}{\partial x^2}(x,t) &= \frac{\partial}{\partial x}\left(100e^{-0.1t}\pi t\cos(\pi xt) \right) = 100e^{-0.1t}\pi t(-\sen(\pi xt) \pi t) =\\
&= -100e^{-0.1t}\pi^2 t^2\sen(\pi xt),\\
\frac{\partial^2 H}{\partial t\partial x}(x,t) &= \frac{\partial}{\partial t}\left(100e^{-0.1t}\pi t\cos(\pi xt) \right) =\\
&= 100\left(\frac{\partial}{\partial t}e^{-0.1t}\pi t\cos(\pi xt) + e^{-0.1t}\left(\frac{\partial}{\partial t}(\pi t)\cos(\pi xt) + \pi t \frac{\partial}{\partial t}\cos(\pi xt)\right) \right) =\\
&= 100\left(-0.1e^{-0.1t}\pi t\cos(\pi xt) + e^{-0.1t}\left(\pi \cos(\pi xt) - \pi t \sen(\pi xt)\pi x\right) \right) =\\
&= 100e^{-0.1t}\left(-0.1\pi t\cos(\pi xt)+\pi \cos(\pi xt) - \pi^2 xt \sen(\pi xt)\right) = \\
&= 100e^{-0.1t}\left((-0.1\pi t+\pi)\cos(\pi xt) - \pi^2 xt \sen(\pi xt)\right),\\
\end{align*}

\begin{align*}
\frac{\partial^2 H}{\partial x\partial t}(x,t) &= \frac{\partial^2 H}{\partial t\partial x}(x,t) \quad (\mbox{igualdad de las derivadas cruzadas por el teorema de Schwartz})\\
\frac{\partial^2 H}{\partial t^2}(x,t) &= \frac{\partial}{\partial t} \left(100 e^{-0.1t}\left(-0.1 \sen(\pi xt) + \pi x \cos(\pi xt)\right)\right)  =\\
&= 100\left(\frac{\partial}{\partial t} e^{-0.1t}\left(-0.1 \sen(\pi xt) + \pi x \cos(\pi xt)\right) +\right.\\
&\left. \qquad + e^{0.1t}\left(\frac{\partial}{\partial t}\left(-0.1\sen(\pi xt)\right) + \frac{\partial}{\partial t}\left(\pi x \cos(\pi xt)\right)\right)\right) =\\
&= 100\left(-0.1 e^{-0.1t}\left(-0.1 \sen(\pi xt) + \pi x \cos(\pi xt)\right)\right. +\\
&\left. \qquad + e^{0.1t}\left(-0.1\cos(\pi xt)\pi x - \pi x \cos(\pi xt)\pi x\right)\right) =\\
&= 100e^{-0.1t}\left(0.01\sen(\pi xt) -0.1 \pi x \cos(\pi xt) -0.1\pi x\cos(\pi xt) - \pi^2 x^2 \cos(\pi xt)\right) = \\
&= 100e^{-0.1t}\left(0.01\sen(\pi xt) -(0.2+\pi^2x^2) \cos(\pi xt)\right).
\end{align*}
Así pues, la matriz Hessiana es
\[
\left(
\begin{array}{cc}
-100e^{-0.1t}\pi^2 t^2\sen(\pi xt) & 100e^{-0.1t}\left((-0.1\pi t+\pi)\cos(\pi xt) - \pi^2 xt \sen(\pi xt)\right) \\
100e^{-0.1t}\left((-0.1\pi t+\pi)\cos(\pi xt) - \pi^2 xt \sen(\pi xt)\right) & 100e^{-0.1t}\left(0.01\sen(\pi xt) -(0.2+\pi^2x^2) \cos(\pi xt)\right)
\end{array}
\right)
\]
\end{enumerate}
}


\newproblem{par-29}{gen}{*}
%ENUNCIADO
{Dar la dirección de máximo crecimiento de la función
\[
f(x,y,z) = \frac{\log(zx)}z-xe^{-zxy}
\]
en el punto $(1,1,1)$.
}
%SOLUCIÓN
{$\nabla f(1,1,1)=(1,e^{-1},1+e^{-1})$.
}
%RESOLUCIÓN
{La dirección de máximo crecimiento de una función de varias variables la da el vector gradiente:
\[
\nabla f(x,y,z) = \left(\frac{\partial f}{\partial x}(x,y,z),\frac{\partial f}{\partial y}(x,y,z),\frac{\partial f}{\partial z}(x,y,z)\right)
\]
Calculamos por tanto cada una de las derivadas parciales que aparecen en las componentes del vector:
\begin{align*}
\frac{\partial f}{\partial x}(x,y,z) &= \frac{\partial}{\partial x}(\frac{\log (zx)}z-xe^{-zxy}) = \frac{\partial}{\partial x}(\frac{\log (zx)}z)-\frac{\partial}{\partial x}(xe^{-zxy})= \\
&= \frac{1}{z}\frac{\partial}{\partial x}(\log (zx))-(\frac{\partial}{\partial x}(x)e^{-zxy}+x\frac{\partial}{\partial x}(e^{-zxy}))= \\
&= \frac{1}{z}\frac{1}{zx}\frac{\partial}{\partial x}(zx)-(e^{-zxy}+xe^{-zxy}\frac{\partial}{\partial x}(-zxy))= \\
&= \frac{1}{z}\frac{1}{zx}z-(e^{-zxy}+xe^{-zxy}(-zy)) = \frac{1}{zx}-e^{-zxy}(1-xyz),\\
\frac{\partial f}{\partial y}(x,y,z) &= \frac{\partial}{\partial y}(\frac{\log(zx)}z-xe^{-zxy}) = \frac{\partial}{\partial y}(\frac{\log (zx)}z)-\frac{\partial}{\partial y}(xe^{-zxy})= \\
&= -x\frac{\partial}{\partial y}(e^{-zxy}) = -xe^{-zxy}\frac{\partial}{\partial y}(-zxy)=x^2ze^{-zxy},\\
\frac{\partial f}{\partial z}(x,y,z) &= \frac{\partial}{\partial z}(\frac{\log(zx)}z-xe^{-zxy}) = \frac{\partial}{\partial z}(\frac{\log (zx)}z)-\frac{\partial}{\partial z}(xe^{-zxy})= \\
&= \frac{\frac{\partial}{\partial z}(\log (zx))z-\log (zx)\frac{\partial}{\partial z}(z)}{z^2}-x\frac{\partial}{\partial z}(e^{-zxy}))= \\
&= \frac{\frac 1{zx}\frac \partial {\partial z}(zx)z-\log (zx)}{z^2}-xe^{-zxy}\frac{\partial}{\partial z}(-zxy))= \\
&= \frac{\frac 1{zx}xz-\log (zx)}{z^2}-xe^{-zxy}-xy=\frac{1-\log (zx)}{z^2}+x^2ye^{-zxy}.
\end{align*}
Por lo tanto, el vector gradiente será:
\[
\nabla f(x,y,z)=(\frac{1}{zx}-e^{-zxy}(1-xyz), x^2ze^{-zxy}, \frac{1-\log (zx)}{z^2}+x^2ye^{-zxy})
\]

Finalmente, como nos pieden la dirección de máximo crecimiento en el punto $(1,1,1)$, tendremos que particularizar el vector gradiente en dicho punto, es decir:
\[
\nabla f(1,1,1)=(1,e^{-1},1+e^{-1}).
\]
}


\newproblem{par-30}{gen}{*}
%ENUNCIADO
{Calcular el gradiente de la función
\[
f(x,y,z)=e^{\sqrt{x^2+2yz}}+\ln (\frac{xy}z)
\]
en el punto $(1,-2,-2)$.
}
%SOLUCIÓN
{$\nabla f(x,y,z)=\left(\frac{xe^{\sqrt{x^2+2yz}}}{\sqrt{x^2+2yz}}+\frac{1}{x}, \frac{ze^{\sqrt{x^2+2yz}}}{\sqrt{x^2+2yz}}+\frac{1}{y}, \frac{ye^{\sqrt{x^2+2yz}}}{\sqrt{x^2+2yz}}-\frac{1}{z}\right)$\\ y $\nabla f(1,-2,-2)=\left(\frac{e^3}{3}+1,\frac{-2e^3}{3}-\frac{1}{2},\frac{-2e^3}{3}+\frac{1}{2}\right)$.}
%RESOLUCIÓN
{El gradiente de $f(x,y,z)$ se define como el vector $\nabla f(x,y,z)=\left(\dfrac{\partial f}{\partial x}(x,y,z),\dfrac{\partial f}{\partial y}(x,y,z),\dfrac{\partial f}{\partial z}(x,y,z)\right).$ Por tanto, tenemos que calcular las tres derivadas parciales siguientes:
\begin{align*}
\dfrac{\partial f}{\partial x}(x,y,z) &= \dfrac{\partial}{\partial x}(e^{\sqrt{x^2+2yz}}+\ln (\frac{xy}z)) = \dfrac{\partial}{\partial x}(e^{\sqrt{x^2+2yz}})+\dfrac{\partial}{\partial x}(\ln (\frac{xy}z))= \\
&= e^{\sqrt{x^2+2yz}}\dfrac \partial {\partial x}(\sqrt{x^2+2yz})+\frac{1}{xy/z}\dfrac{\partial}{\partial x}(\frac{xy}z)= \\
&= e^{\sqrt{x^2+2yz}}\frac{1}{2\sqrt{x^2+2yz}}\dfrac{\partial}{\partial x}(x^2+2yz)+\frac{z}{xy}\frac{y}{z}= \\
&= e^{\sqrt{x^2+2yz}}\frac{1}{2\sqrt{x^2+2yz}}2x+\frac{1}{x} = \frac{xe^{\sqrt{x^2+2yz}}}{\sqrt{x^2+2yz}}+\frac{1}{x}, \\
\dfrac{\partial f}{\partial y}(x,y,z) &= \dfrac{\partial}{\partial y}(e^{\sqrt{x^2+2yz}}+\ln (\frac{xy}z)) = \dfrac{\partial}{\partial y}(e^{\sqrt{x^2+2yz}})+\dfrac{\partial}{\partial y}(\ln (\frac{xy}z))= \\
&= e^{\sqrt{x^2+2yz}}\dfrac{\partial}{\partial y}(\sqrt{x^2+2yz})+\frac{1}{xy/z}\dfrac{\partial}{\partial y}(\frac{xy}z)= \\
&= e^{\sqrt{x^2+2yz}}\frac{1}{2\sqrt{x^2+2yz}}\dfrac{\partial}{\partial y}(x^2+2yz)+\frac{z}{xy}\frac{x}{z}= \\
&= e^{\sqrt{x^2+2yz}}\frac{1}{2\sqrt{x^2+2yz}}2z+\frac{1}{y}=\frac{ze^{\sqrt{x^2+2yz}}}{\sqrt{x^2+2yz}}+\frac{1}{y}, \\
\dfrac{\partial f}{\partial z}(x,y,z) &= \dfrac{\partial}{\partial z}(e^{\sqrt{x^2+2yz}}+\ln (\frac{xy}z)) = \dfrac{\partial}{\partial z}(e^{\sqrt{x^2+2yz}})+\dfrac{\partial}{\partial z}(\ln (\frac{xy}{z}))= \\
&= e^{\sqrt{x^2+2yz}}\dfrac{\partial}{\partial z}(\sqrt{x^2+2yz})+\frac{1}{xy/z}\dfrac{\partial}{\partial z}(\frac{xy}{z})= \\
&= e^{\sqrt{x^2+2yz}}\frac{1}{2\sqrt{x^2+2yz}}\dfrac{\partial}{\partial z}(x^2+2yz)+\frac{z}{xy}\frac{-xy}{z^2}= \\
&= e^{\sqrt{x^2+2yz}}\frac{1}{2\sqrt{x^2+2yz}}2y-\frac{1}{z} = \frac{ye^{\sqrt{x^2+2yz}}}{\sqrt{x^2+2yz}}-\frac{1}{z},
\end{align*}
y, en consecuencia tenemos
\[
\nabla f(x,y,z)=\left(\frac{xe^{\sqrt{x^2+2yz}}}{\sqrt{x^2+2yz}}+\frac{1}{x}, \frac{ze^{\sqrt{x^2+2yz}}}{\sqrt{x^2+2yz}}+\frac{1}{y}, \frac{ye^{\sqrt{x^2+2yz}}}{\sqrt{x^2+2yz}}-\frac{1}{z}\right).
\]
Como nos piden el gradiente en el punto $(1,-2,-2),$ sustituimos $x$ por 1, $y$ por -2, y $z$ por -2 en el vector anterior y obtenemos
\[
\nabla f(1,-2,-2)=\left(\frac{e^3}{3}+1,\frac{-2e^3}{3}-\frac{1}{2},\frac{-2e^3}{3}+\frac{1}{2}\right).
\]
}


\newproblem{par-31}{gen}{*}
%ENUNCIADO
{Calcular el vector gradiente de la función
\[
\log \left( \sqrt{x^{2}-z^{2}}\right) +3^{\tfrac{x^{2}}{y}}
\]
en el punto $(1,1,0)$.
}
%SOLUCIÓN
{$\nabla f(x,y,z)=(\frac{x}{x^{2}-z^{2}}+3^{\tfrac{x^{2}}{y}}\log 3\dfrac{2x}{y},3^{\tfrac{x^{2}}{y}}\log 3\dfrac{-x^{2}}{y^{2}},-\frac{z}{x^{2}-z^{2}})$, y $\nabla f(1,-2,-2)=(1+6\log 3,-3\log 3,0)$.
}
%RESOLUCIÓN
{El gradiente de $f(x,y,z)$ se define como el vector $\nabla f(x,y,z)=(\dfrac{\partial f}{\partial x}(x,y,z),\dfrac{\partial f}{\partial y}(x,y,z),\dfrac{\partial f}{\partial z}(x,y,z))$. Por tanto, tenemos que calcular las tres derivadas parciales siguientes:
\begin{align*}
\dfrac{\partial f}{\partial x}(x,y,z) &= \dfrac{\partial }{\partial x}(\log\left(\sqrt{x^{2}-z^{2}}\right) +3^{\tfrac{x^{2}}{y}}) = \dfrac{\partial }{\partial x}(\log \left(\sqrt{x^{2}-z^{2}}\right) )+\dfrac{\partial }{\partial x}(3^{\tfrac{x^{2}}{y}})= \\
&= \frac{1}{\sqrt{x^{2}-z^{2}}}\dfrac{\partial }{\partial x}(\sqrt{x^{2}-z^{2}})+3^{\tfrac{x^{2}}{y}}\log 3\dfrac{\partial }{\partial x}(\dfrac{x^{2}}{y}) = \\
&= \frac{1}{\sqrt{x^{2}-z^{2}}}\frac{1}{2\sqrt{x^{2}-z^{2}}}\dfrac{\partial}{\partial x}(x^{2}-z^{2})+3^{\tfrac{x^{2}}{y}}\log 3\dfrac{2x}{y}= \\
&= \frac{1}{2(x^{2}-z^{2})}2x+3^{\tfrac{x^{2}}{y}}\log 3\dfrac{2x}{y}=\frac{x}{x^{2}-z^{2}}+3^{\tfrac{x^{2}}{y}}\log 3\dfrac{2x}{y}, \\
\dfrac{\partial f}{\partial y}(x,y,z) &= \dfrac{\partial }{\partial y}(\log\left( \sqrt{x^{2}-z^{2}}\right) +3^{\tfrac{x^{2}}{y}}) = \dfrac{\partial }{\partial y}(\log \left( \sqrt{x^{2}-z^{2}}\right) )+\dfrac{\partial }{\partial y}(3^{\tfrac{x^{2}}{y}})= \\
&= 0+3^{\tfrac{x^{2}}{y}}\log 3\dfrac{\partial }{\partial y}(\dfrac{x^{2}}{y}) = 3^{\tfrac{x^{2}}{y}}\log 3\dfrac{-x^{2}}{y^{2}}, \\
\dfrac{\partial f}{\partial z}(x,y,z) &= \dfrac{\partial }{\partial z}(\log\left( \sqrt{x^{2}-z^{2}}\right) +3^{\tfrac{x^{2}}{y}}) = \dfrac{\partial }{\partial z}(\log \left( \sqrt{x^{2}-z^{2}}\right) )+\dfrac{\partial }{\partial z}(3^{\tfrac{x^{2}}{y}})= \\
&= \frac{1}{\sqrt{x^{2}-z^{2}}}\dfrac{\partial }{\partial x}(\sqrt{x^{2}-z^{2}})+0 = \frac{1}{\sqrt{x^{2}-z^{2}}}\frac{1}{2\sqrt{x^{2}-z^{2}}}\dfrac{\partial }{\partial x}(x^{2}-z^{2})= \\
&= \frac{1}{2(x^{2}-z^{2})}(-2z)=-\frac{z}{x^{2}-z^{2}}.
\end{align*}
y, en consecuencia tenemos
\[
\nabla f(x,y,z)=(\frac{x}{x^{2}-z^{2}}+3^{\tfrac{x^{2}}{y}}\log 3\dfrac{2x}{y},3^{\tfrac{x^{2}}{y}}\log 3\dfrac{-x^{2}}{y^{2}},-\frac{z}{x^{2}-z^{2}}).
\]
Como nos piden el gradiente en el punto $(1,1,0),$ sustituimos $x$ por 1, $y$
por 1, y $z$ por 0 en el vector anterior y obtenemos
\[
\nabla f(1,-2,-2)=(1+6\log 3,-3\log 3,0).
\]
}


\newproblem{par-32}{amb}{}
%ENUNCIADO
{La asimilación de CO$_2$ de una planta depende de la temperatura ambiente (t) y de la intensidad de la luz (l), según la función
\[
f(t,l) = ctl^2,
\]
donde $c$ es una constante.
Estudiar cómo evoluciona la asimilación de CO$_2$ para distintas intensidades de luz, cuando se mantiene la temperatura constante.
Estudiar también cómo evoluciona para distintas temperaturas cuando se mantiene la intensidad de la luz constante.
}
%SOLUCIÓN
{$\frac{\partial f}{\partial l}(t,l) = 2ctl$ y $\frac{\partial f}{\partial t}(t,l) = cl^2$.
}
%RESOLUCIÓN
{
}


\newproblem{par-33}{amb}{}
%ENUNCIADO
{La abundancia de una determinada especie de planta depende del nivel de nitrógeno en el suelo y del nivel de perturbaciones, de manera que un incremento del nivel de nitrógeno tiene un efecto negativo en la abundancia de esta especie, y un aumento de las perturbaciones también tiene un efecto negativo.
Si en un momento dado comienza a aumentar el nivel de nitrógeno en el suelo y también las perturbaciones debidas al pastoreo, ¿cómo se verá afectada la abundancia de la especie?
}
%SOLUCIÓN
{La abundancia de la especie disminuirá.
}
%RESOLUCIÓN
{
}


\newproblem{par-34}{amb}{}
%ENUNCIADO
{La velocidad de crecimiento de un organismo depende de la disponibilidad de alimento y del número de competidores en busca de alimento.
¿Cómo se verá afectada la velocidad de crecimiento si la disponibilidad de alimento aumenta con el tiempo y el número de competidores disminuye?}
%SOLUCIÓN
{La velocidad de crecimiento aumentará.
}
%RESOLUCIÓN
{
}


\newproblem{par-35}{amb}{}
%ENUNCIADO
{Un organismo se mueve sobre una superficie inclinada siguiendo la línea de máxima pendiente descendiente.
Si la expresión de la superficie es
\[
f(x,y) = x^2-y^2,
\]
calcule la dirección en la que se moverá el organismo en el punto $(2,3)$.
}
%SOLUCIÓN
{Se moverá en la dirección $-\nabla f(2,3)=(-4,6)$.}
%RESOLUCIÓN
{
}


\newproblem{par-36}{gen}{}
%ENUNCIADO
{Si $f(x,y,z)=x^3y^2z$ y $g(t)=(e^t,\cos t,\sen t)$, calcular $(f\circ g)'(t)$.
}
%SOLUCIÓN
{$(f\circ g)'(t)= e^{3t}(3\sen t\cos^2 t-2\sen^2 t\cos t+\cos^3 t)$.
}
%RESOLUCIÓN
{
}


\newproblem{par-37}{amb}{}
%ENUNCIADO
{Obtener los puntos críticos de $z=f(x,y)$ para:
\begin{enumerate}
\item $f(x,y)=x^2+y^2$.
\item $f(x,y)=x^2y+y^2x$.
\item $f(x,y)=x^2-2xy+2y^2$.
\end{enumerate}
}
%SOLUCIÓN
{\begin{enumerate}
\item $(0,0)$.
\item $(0,0)$.
\item $(0,0)$.
\end{enumerate}
}
%RESOLUCIÓN
{
}


\newproblem{par-38}{gen}{}
%ENUNCIADO
{La superficie de una montaña tiene la forma
\[
S:z=a-bx^2-cy^2,
\]
donde $a$, $b$ y $c$ son constantes, $x$ es la coordenada Este-Oeste e $y$ la coordenada Norte-Sur en el mapa, y $z$ la altura sobre el nivel del mar en metros.
En el punto $P=(1,1)$ del mapa, ¿en qué dirección crece más rápidamente la altura?
}
%SOLUCIÓN
{$(-2b,-2c)$.
}
%RESOLUCIÓN
{
}


\newproblem{par-39}{gen}{}
%ENUNCIADO
{Hallar las direcciones de máximo y mínimo crecimiento de las siguientes funciones en el punto $P$:
\begin{enumerate}
\item $f(x,y)=x^2+xy+y^2$, $P=(-1,1)$.
\item $f(x,y)=x^2y+e^{xy}\sen y$, $P=(1,0)$.
\item $f(x,y,z)=\log(xy)+\log(yz)+\log(xz)$, $P=(1,1,1)$.
\item $f(x,y,z)=\log(x^2+y^2-1)+y+6z$, $P=(1,1,0)$.
\end{enumerate}
}
%SOLUCIÓN
{\begin{enumerate}
\item Máximo crecimiento en la dirección $(-1,1)$ y máximo decrecimiento en la dirección $(1,-1)$.
\item Máximo crecimiento en la dirección $(0,2)$ y máximo decrecimiento en la dirección $(0,-2)$.
\item Máximo crecimiento en la dirección $(2,2,2)$ y máximo decrecimiento en la dirección $(-2,-2,-2)$.
\item Máximo crecimiento en la dirección $(2,3,6)$ y máximo decrecimiento en la dirección $(-2,-3,-6)$.
\end{enumerate}
}
%RESOLUCIÓN
{
}


\newproblem{par-40}{gen}{}
%ENUNCIADO
{¿En qué direcciones se anulará la derivada direccional de la función
\[
f(x,y)=\frac{x^2-y^2}{x^2+y^2}
\]
en el punto $P=(1,1)$?
}
%SOLUCIÓN
{En la dirección $(1/\sqrt{2},1/\sqrt{2})$.
}
%RESOLUCIÓN
{
}


\newproblem{par-41}{gen}{}
%ENUNCIADO
{¿Existe alguna dirección en la que la derivada direccional en el punto $P=(1,2)$ de la función
\[
f(x,y) = x^2-3xy+4y^2
\]
valga 14?
}
%SOLUCIÓN
{No.
}
%RESOLUCIÓN
{
}


\newproblem{par-42}{gen}{}
%ENUNCIADO
{La derivada direccional de una función $f$ en un punto $P$ es máxima en la dirección del vector $(1,1,-1)$ y su valor es $2\sqrt{3}$.
¿Cuánto vale la derivada direccional de $f$ en $P$ en la dirección del vector $(1,1,0)$?
}
%SOLUCIÓN
{$2\sqrt{2}$.
}
%RESOLUCIÓN
{
}


\newproblem{par-43}{gen}{}
%ENUNCIADO
{Dado el campo escalar
\[
f(x,y,z) = x^2-y^2+xyz^3-zx
\]
en el punto $P=(1,2,3)$, se pide:
\begin{enumerate}
\item Calcular la derivada direccional de $f$ en $P$ a lo largo del vector unitario $\mathbf{u}=\frac{1}{\sqrt2}(1,-1,0)$.
\item ¿En qué dirección es máxima la derivada direccional de $f$ en $P$? Obtener el valor de dicha derivada direccional.
\end{enumerate}
}
%SOLUCIÓN
{\begin{enumerate}
\item $15\sqrt{2}$.
\item La derivada direccional es máxima en la dirección del gradiente $(53,23,53)$ y vale $\sqrt{6147}$.
\end{enumerate}
}
%RESOLUCIÓN
{
}


\newproblem*{par-44}{gen}{}
%ENUNCIADO
{En el ajuste de regresión de una recta $y=a+bx$, se suele utilizar la técnica de mínimos cuadrados que consisten en buscar los valores
de $a$ y $b$ que hacen mínima la función
\[
f(a,b)= \sum_{i=1}^{n}(y_i-a-bx_i)^2,
\]
donde el sumatorio abarca a todos los pares de la muestra $(x_i,y_i)$ para $i=1,\ldots, n$, siendo $n$ el tamaño de la muestra.

Demostrar que esta función alcanza el mínimo en los puntos
\[
a=\bar y-b\bar x \quad \mbox{ y } b=\frac{s_{xy}}{s_x^2}.
\]
}
%SOLUCIÓN
{
}
%RESOLUCIÓN
{
}


\newproblem{par-45}{gen}{}
%ENUNCIADO
{La siguiente función mide la presión del aire en la posición $(x,y,z)$.
\[
f(x,y,z)= x^2+y^2-z^3.
\]
Sea $A$ un objeto que se mueve a lo largo de la trayectoria:
\[
\begin{cases}
x=t\\
y=1\\
z=1/t
\end{cases}
t>0.
\]
\begin{enumerate}
\item Calcular la ecuación de la recta tangente a la trayectoria de $A$ en el punto $(1,1,1)$.
\item Sigue la trayectoria de $A$ en el punto $(1,1,1)$ la dirección de máximo crecimiento de la función $f$?
\end{enumerate}
}
%SOLUCIÓN
{\begin{enumerate}
\item $(1+t, 1, 1-t)$.
\item No ya que la dirección de máximo crecimiento de $f$ es $\nabla f(1,1,1)=(2,2,-3)$ y la dirección de la trayectoria es $(1,0,-1)$.
\end{enumerate}
}
%RESOLUCIÓN
{
}


\newproblem{par-46}{gen}{*}
%STATEMENT
{Obtener la ecuación del plano tangente y de la recta normal a la superficie
\[
S:xyz=8
\]
en el punto $P=(4,-2,-1)$.
}
%SOLUTION
{Recta normal $l:(4+2t,-2-4t,-1-8t)$. Plano tangente $\pi: 2x-4y-8z+24=0$.
}
%RESOLUTION
{
}


%%%%%%% Pendiente 26
