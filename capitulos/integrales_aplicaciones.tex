% Autor: Alfredo Sánchez Alberca (asalber@ceu.es)

\newproblem*{intapl-1}{gen}{}
%ENUNCIADO
{Calcular el área comprendida entre la curva $y=x^{3}-6x^{2}+8x$ y el eje de abcisas.
}


\newproblem*{intapl-2}{gen}{}
%ENUNCIADO
{Calcular el área comprendida entre la parábola $y^{2}=4x$ y la recta $y=2x-4$.
}


\newproblem*{intapl-3}{gen}{}
%ENUNCIADO
{Calcular el área comprendida entre las parábolas $y=6x-x^{2}$ e $y=x^{2}-2x.$
}


\newproblem*{intapl-4}{gen}{*}
%ENUNCIADO
{Dibujar aproximadamente el recinto limitado por $f(x)=\cos x$, $g(x)=|x^2-1|$, $x=-1$ y $x=1$, y calcular el área de dicho recinto. 
}


\newproblem*{intapl-5}{gen}{*}
%ENUNCIADO
{Calcular el área del recinto limitado por las parábolas:
\[
\left\{
\begin{array}{l}
y_1 = x^2+2x+2,\\
y_2 = -x^2+2x+4.
\end{array}
\right.
\]
}


\newproblem*{intapl-6}{gen}{*}
%ENUNCIADO
{Calcular el área del recinto limitado por las funciones $f(x)= e^{-x}$, $g(x)=x^2-4x+1$ y la recta $x=2$.
}


\newproblem*{intapl-7}{gen}{}
%ENUNCIADO
{Dibujar aproximadamente el recinto limitado por la función $f(x)=\left| x^{2}-4x+3\right|$ y la recta $y=3.$ Calcular el área de dicho recinto.
}


\newproblem*{intapl-8}{gen}{}
%ENUNCIADO
{Calcular el área encerrada entre $y=e^{-\left|x\right| }$ y su asíntota.
}


\newproblem*{intapl-9}{gen}{*}
%ENUNCIADO
{Dada la función 
\[
f(x)=\frac{x}{(1+2x^2)^6}
\]
Calcular:
\begin{enumerate}
\item El área del recinto limitado por $f(x)$ y el eje de abscisas desde $x=1$ hasta $x=2$.
\item El área del recinto limitado por $f(x)$ y el eje de abscisas desde $x=0$ has el infinito.
\end{enumerate}
}


\newproblem*{intapl-10}{gen}{}
%ENUNCIADO
{Calcular el área entre las funciones siguientes y el eje de abscisas en el intervalo $[1,3]$:
\begin{multicols}{2}
\begin{enumerate}\setlength{\itemsep}{3mm}
\item $f(x)=\sqrt{x}$
\item $f(x)=\dfrac{1}{x^2}$
\item $f(x)=\sin x^2$
\item $f(x)=x^2-3x+2$
\end{enumerate}
\end{multicols}
}


\newproblem*{intapl-11}{amb}{*}
%ENUNCIADO
{Si la cantidad de dióxido de carbono, en toneladas/hora, que arroja a la atmósfera una empresa viene dada en función del tiempo, en horas, por la expresión:
\[
c(t)=20te^{-t}
\]
\begin{enumerate}
\item ¿Cuál es la cantidad total de monóxido de carbono arrojada a la atmósfera por la empresa desde $t=0$ hasta transcurridas 2 horas?
\item ¿Hacia dónde tiende dicha cantidad cuando el tiempo tiende a infinito?
\end{enumerate}
}


\newproblem*{intapl-12}{gen}{*}
%ENUNCIADO
{Dadas las funciones: $f(t) = t$ y $g(t) = \dfrac{t} {{\sqrt{1 + 3t} }}$
\begin{enumerate}
\item Calcular el área del recinto limitado por la funciones entre $t=0$ y $t=1$.
\item Si $f(t)$ es el volumen de agua por unidad de tiempo, en m$^3$/s, que llega a un depósito, y $g(t)$ la que sale del mismo, también en m$^3$/s, ¿qué volumen de agua habrá ganado, o perdido, dicho depósito entre $t=0$ y $t=1/2$?
\end{enumerate}
}


\newproblem*{intapl-13}{amb}{*}
%ENUNCIADO
{Supongamos que el caudal de agua de una fuente natural (en metros cúbicos/día)
viene dado por la expresión:
\[
C(t) = \frac{t}{{\left( {2 + 0,01 t^2 } \right)^3 }}
\]
donde $t$ es el tiempo, expresado en días, desde que hemos comenzado a medir el caudal.
\begin{enumerate}
\item ¿Qué cantidad de metros cúbicos de agua podemos recoger en esa fuente desde el momento en el que hemos comenzado a medir hasta transcurridos 10 días?
\item ¿Y desde el momento en el que hemos comenzado a medir hasta transcurrido un tiempo muy grande?
\end{enumerate}
}


\newproblem*{intapl-14}{amb}{*}
%ENUNCIADO
{Suponiendo que el caudal de agua, $C$ en metros m$^3$/día, que un arroyo vierte en un río, viene dado en función del tiempo, $t$ en días, por la expresión:
\[
C(t) = t^2  \cdot \sqrt {t^2  + 9}
\]
\begin{enumerate}
\item Calcular el total de agua que el arroyo ha vertido en el río desde $t=0$ hasta transcurridos 10 días.
\item Teniendo en cuenta que se define el caudal medio como el total del agua vertida dividida entre el total del tiempo transcurrido, ¿cuál ha sido el caudal medio del arroyo en los dos primeros días?
\end{enumerate}
}


\newproblem*{intapl-15}{gen}{}
%ENUNCIADO
{Calcular el área delimitada por las funciones $y=e^x$, $y=e^{-x}$, $x=-1$, $x=1$ y el eje de abscisas.
}


\newproblem*{intapl-16}{gen}{}
%ENUNCIADO
{Calcular el área encerrada entre las funciones $f(x)=x+2$ y $g(x)=4-x^2$ entre sus puntos de corte.
}


\newproblem*{intapl-17}{gen}{}
%ENUNCIADO
{Calcular el área que queda entre las funciones $f(x)=\sen x$ y $g(x)=\cos x$ en el intervalo $\pi/4$ y $5\pi/4$.
}