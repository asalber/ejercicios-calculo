% Autor: Alfredo Sánchez Alberca (asalber@ceu.es)

\newproblem{extn-1}{gen}{*}
%ENUNCIADO
{Hallar los extremos relativos y los puntos de silla de la función:
\[
f(x,y) = (x^2+y^2)^2-2a^2(x^2-y^2),
\]
con $a\neq 0$.
}
%SOLUCIÓN
{No tiene máximos relativos. Mínimos relativos en $(-a,0)$ y $(a,0)$. Punto de silla en $(0,0)$. 
}
%RESOLUCIÓN
{
}


\newproblem{extn-2}{gen}{*}
%ENUNCIADO
{Dado el campo escalar
\[
h(x,y) = xy+\frac{xy^2}{2}-2x^2,
\]
determinar sus extremos relativos y sus puntos de silla.
}
%SOLUCIÓN
{Máximo relativo en $(-1/8,-1)$. No tiene mínimos relativos. Puntos de silla en $(0,0)$ y $(0,-2)$.
$(0,0)$.
}
%RESOLUCIÓN
{
}

\newproblem{extn-3}{gen}{}
%ENUNCIADO
{Estudiar los extremos y los puntos de silla de $f$ en los siguientes casos:
\begin{enumerate}
\item $f(x,y) = x^2+y^2$.
\item $f(x,y) = x^2-y^2$.
\item $f(x,y) = x^2-2xy+2y^2$.
\item $f(x,y) = \log(x^2+y^2+1)$.
\end{enumerate}
}
%SOLUCIÓN
{\begin{enumerate}
\item Mínimo en $(0,0)$.
\item Punto de silla en $(0,0)$.
\item No se puede saber con el hessiano.
\item Mínimo en $(0,0)$.
\end{enumerate} 
}
%RESOLUCIÓN
{
}


\newproblem{extn-4}{gen}{}
%ENUNCIADO
{La función 
\[
f(x,y) = \frac{x^3}{3}-x-\left(\frac{y^3}{3}-y\right)
\]
tiene un máximo, un mínimo y dos puntos de silla. Encontrarlos.
}
%SOLUCIÓN
{Máximo en $(-1,1)$, mínimo en $(1,-1)$ y puntos de silla en $(1,1)$ y $(-1,-1)$.
}
%RESOLUCIÓN
{
}



