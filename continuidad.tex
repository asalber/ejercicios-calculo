% Autor: Alfredo Sánchez Alberca (asalber@ceu.es)

\newproblem*{con-1}{gen}{}
%ENUNCIADO
{Las siguientes funciones no están definidas en $x=0$.
Determinar, cuando sea posible, su valor en dicho punto de modo que
sean continuas.
\begin{multicols}{2}
\begin{enumerate}
\item  $f(x)=\dfrac{(1+x)^n-1}{x}$.
\item  $h(x)=\dfrac{e^x-e^{-x}}{x}$.
\item  $j(x)=\dfrac{\log(1+x)-\log(1-x)}{x}$.
\item  $k(x)=x^2\sen\dfrac{1}{x}$.
\end{enumerate}
\end{multicols}
}


\newproblem*{con-2}{gen}{}
%ENUNCIADO
{Estudiar la continuidad de las siguientes funciones en los puntos que se indican:
\begin{enumerate}
\item  $g(x)=\dfrac{e^x-1}{e^x+1}$ \quad en el punto $x=0$.
\item  $j(x)=
\begin{cases}
\dfrac{1+e^{1/x}}{1-e^{1/x}} & \mbox{si $x\neq 0$,} \\
1 &  \mbox{si $x=0$.}
\end{cases}$
\quad
en el punto $x=0$.
\item  $h(x)=
\begin{cases}
x\sen\dfrac{\pi}{x} & \mbox{si $x\neq 0$,} \\
0 & \mbox{si $x=0$.}
\end{cases}$
\quad
en el punto $x=0$.
\item  $k(x)=
\begin{cases}
\dfrac{1}{2^{1/x}} &  \mbox{si $x\neq 0$,} \\
0 & \mbox{si $x=0$.}
\end{cases}$
\quad
en el punto $x=0$.
\end{enumerate}
}


\newproblem*{con-3}{gen}{*}
%ENUNCIADO
{Hallar los puntos de discontinuidad y estudiar el carácter de dichas discontinuidades en la función:
\[
f(x)= 
\begin{cases}
\dfrac{x+1}{x^2-1} & \mbox{si $x<0$,} \\
\dfrac{1}{e^{1/(x^2-1)}} & \mbox{si $x\geq 0$.} \\
\end{cases} 
\]
}


\newproblem{con-4}{gen}{*}
%ENUNCIADO
{Clasificar las discontinuidades de la función
\[
f(x)=
\begin{cases}
\dfrac{\sen x}{x} & \text{si $x<0$,} \\ 
e^{\frac{1}{x-1}} & \text{si $x\geq 0$.}
\end{cases}
\]
}
%SOLUCIÓN
{Hay una discontinuidad de salto en $x=0$ y una esencial de primera especie en $x=1$.
}
%RESOLUCIÓN
{La función $\frac{\sen x}{x}$ es continua en $\mathbb{R}\setminus{\{0\}}$ y en consecuencia es continua en la región donde está definida, es decir $(-\infty ,0)$. Por su parte, la función $e^{\frac 1{x-1}}$ es continua en todos los puntos en que sea continuo el exponente $\frac{1}{x-1},$ es decir en $\mathbb{R}\setminus{\{1\}},$ en consecuencia, es continua en toda la región en donde está definida, menos en el 1. Así pues, reduciremos el estudio de la continuidad a dos puntos, el 0 por ser donde cambia la definición de la función y el 1, por no estar definida la función $e^{\frac{1}{x-1}}.$

Estudiamos primero la continuidad en el punto $x=0$:
\begin{align*}
\lim_{x\rightarrow 0^{-}}f(x) &= \lim_{x\rightarrow 0^{-}}\frac{\sen x}{x}\stackrel{\text{L}^{\prime }\text{H\^{o}pital}}{=} \lim_{x\rightarrow 0^{-}}\frac{\cos x}{1} = 1,\\
\lim_{x\rightarrow 0^{+}}f(x) &= \lim_{x\rightarrow 0^{+}}e^{\frac{1}{x-1}} = e^{-1}.\\
\end{align*}
Como ambos límites laterales son distintos, en $x=0$ hay una discontinuidad de salto.

Estudiamos ahora la continuidad en el punto $x=1$:
\begin{align*}
\lim_{x\rightarrow 1^{-}}f(x) &= \lim_{x\rightarrow 1^{-}}e^{\frac{1}{x-1}}=e^{-\infty }=0, \\
\lim_{x\rightarrow 1^{+}}f(x) &= \lim_{x\rightarrow 1^{+}}e^{\frac{1}{x-1}}=e^\infty = \infty.\\ 
\end{align*}
Como el límite lateral por la derecha no existe, en $x=1$ hay una discontinuidad esencial de primera especie.
}


\newproblem{con-5}{gen}{*}
%ENUNCIADO
{Clasificar las discontinuidades de la función
\[
f(x)=\frac{\frac{1}{x}-\frac{1}{x+1}}{\frac{3}{x-1}-\frac{1}{x}}
\]
}
%SOLUCIÓN
{Hay una discontinuidad evitable en $x=0$ y $x=1$ y una discontinuidad esencial en $x=-1$ y $x=-1/2$.}
%RESOLUCIÓN
{A simple vista, podemos ver que se trata de una función racional y estará definida en todo $\mathbb{R}$ salvo en los puntos que anulen alguno de los denominadores. Dichos puntos son fáciles de obtener igualando a 0 los denominadores:
\begin{align*}
x &= 0 \\
x+1 &= 0\Leftrightarrow x=-1 \\
x-1 &= 0\Leftrightarrow x=1 \\
\frac{3}{x-1}-\frac{1}{x} &= 0\Leftrightarrow x=-\frac{1}{2}
\end{align*}

Por tanto, obtenemos 4 punto de discontinuidad, que son: $x=0,$ $x=1$, $x=-1$ y $x=-\frac{1}{2}$.

Para clasificar estas cuatro discontinuidades, tenemos que estudiar los correspondientes límites por la izquierda y por la derecha.

\begin{itemize}
\item Discontinuidad en $x=0$:
\begin{align*}
\lim_{x\rightarrow 0^{-}}f(x) &= \lim_{x\rightarrow 0^{-}}\frac{\frac{1}{x}-\frac{1}{x+1}}{\frac{3}{x-1}-\frac{1}{x}} = \lim_{x\rightarrow 0^{-}}\frac{x-1}{\left( x+1\right) \left( 2x+1\right) } = \lim_{x\rightarrow 0^{-}}\frac{-1}{1}=-1, \\
\lim_{x\rightarrow 0^{+}}f(x) &= \lim_{x\rightarrow 0^{+}}\frac{\frac{1}{x}-\frac{1}{x+1}}{\frac{3}{x-1}-\frac{1}{x}} = \lim_{x\rightarrow 0^{+}}\frac{x-1}{\left( x+1\right) \left( 2x+1\right) } = \lim_{x\rightarrow 0^{+}}\frac{-1}{1}=-1.
\end{align*}
Como ambos límites coinciden, se trata de una discontinuidad evitable.

\item Discontinuidad en $x=1$:
\begin{align*}
\lim_{x\rightarrow 1^{-}}f(x) &= \lim_{x\rightarrow 1^{-}}\frac{\frac{1}{x}-\frac{1}{x+1}}{\frac{3}{x-1}-\frac{1}{x}} = \lim_{x\rightarrow 1^{-}}\frac{x-1}{\left( x+1\right) \left( 2x+1\right) } = \lim_{x\rightarrow 1^{-}}\frac{0}{6}=0, \\
\lim_{x\rightarrow 1^{+}}f(x) &= \lim_{x\rightarrow 1^{+}}\frac{\frac{1}{x}-\frac{1}{x+1}}{\frac{3}{x-1}-\frac{1}{x}} = \lim_{x\rightarrow 1^{+}}\frac{x-1}{\left( x+1\right) \left( 2x+1\right) } = \lim_{x\rightarrow 1^{+}}\frac{0}{6}=0.
\end{align*}
De nuevo, como ambos límites coinciden, se trata de una discontinuidad evitable.

\item Discontinuidad en $x=-1$:
\begin{align*}
\lim_{x\rightarrow -1^{-}}f(x) &= \lim_{x\rightarrow -1^{-}}\frac{\frac{1}{x}-\frac{1}{x+1}}{\frac{3}{x-1}-\frac{1}{x}} = \lim_{x\rightarrow -1^{-}}\frac{x-1}{\left( x+1\right) \left( 2x+1\right) } = \lim_{x\rightarrow -1^{-}}\frac{-2}{-0\cdot -1}=-\infty, \\
\lim_{x\rightarrow -1^{+}}f(x) &= \lim_{x\rightarrow -1^{+}}\frac{\frac{1}{x}-\frac{1}{x+1}}{\frac{3}{x-1}-\frac{1}{x}} = \lim_{x\rightarrow -1^{+}}\frac{x-1}{\left( x+1\right) \left( 2x+1\right) } = \lim_{x\rightarrow -1^{+}}\frac{-2}{+0\cdot -1}=\infty.
\end{align*}
Como ambos límites divergen, se trata de una discontinuidad de tipo esencial.

\item Discontinuidad en $x=-\frac{1}{2}$:
\begin{align*}
\lim_{x\rightarrow -1/2^{-}}f(x) &= \lim_{x\rightarrow -1/2^{-}}\frac{\frac{1}{x}-\frac{1}{x+1}}{\frac{3}{x-1}-\frac{1}{x}} = \lim_{x\rightarrow -1/2^{-}}\frac{x-1}{\left( x+1\right) \left( 2x+1\right) } = \lim_{x\rightarrow-1/2^{-}}\frac{-3/2}{1/2\cdot -0} = +\infty, \\
\lim_{x\rightarrow -1/2^{+}}f(x) &= \lim_{x\rightarrow -1/2^{+}}\frac{\frac{1}{x}-\frac{1}{x+1}}{\frac{3}{x-1}-\frac{1}{x}} = \lim_{x\rightarrow -1/2^{+}}\frac{x-1}{\left( x+1\right) \left( 2x+1\right) } = \lim_{x\rightarrow-1/2^{+}}\frac{-2}{1/2\cdot +0} = -\infty.
\end{align*}
Por último, como ambos límites divergen, se trata también de una discontinuidad de tipo esencial.
\end{itemize}
}