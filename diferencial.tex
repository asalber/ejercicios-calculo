% Autor: Alfredo Sánchez Alberca (asalber@ceu.es)

\newproblem{dif-1}{gen}{}
%ENUNCIADO
{Un cilindro de 4 cm de radio ($r$) y 3 cm de altura ($h$) se somete a un proceso de calentamiento con el que varían sus dimensiones de tal forma que $\dfrac{dr}{dt}=\dfrac{dh}{dt}= 1$ cm/s. Hallar la variación de su volumen a los 5 segundos y a los 10 segundos.
}
%SOLUCIÓN
{$dV = 2\pi r h dt + \pi r^2 dt$ y en el instante inicial tenemos $dV = 40\pi dt$. A los 5 segundos la variación aproximada será $dV(5) = 40\pi 5 = 200\pi$ cm$^3$/s, y a los 10 segundos $dV(10) = 40\pi 10 = 400\pi$ cm$^3$/s.
}
%RESOLUCIÓN
{
}


\newproblem{dif-2}{qui}{*}
%ENUNCIADO
{La ecuación de los gases perfectos es $PV=CT$ donde $C$ es constante. Si en un cierto instante el volumen es 0.3 m$^3$, la presión es 90 Pa, la temperatura 290 K, y comenzamos a aumentar el volumen a razón de 0.01 m$^3$/s:
\begin{enumerate}
   \item  Hallar la variación de la presión en dicho instante si
   la temperatura se mantiene constante.

   \item  Hallar la variación de la temperatura en dicho instante si
   la presión se mantiene constante.
\end{enumerate}
}
%SOLUCIÓN
{\begin{enumerate}
\item $dP = \frac{-CT}{V^2}dV$ y en el instante indicado vale $dP = -3 $ Pa/s.
\item $dT = \frac{P}{C} dV$ y en el instante indicado vale $dT = 9.67$ K/s.  
\end{enumerate}
}
%RESOLUCIÓN
{
}
