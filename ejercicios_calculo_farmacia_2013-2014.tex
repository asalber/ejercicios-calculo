% Author Alfredo Sánchez Alberca (asalber@ceu.es)

\documentclass[a4paper,titlepage,dvips]{article}
%===============================================
\usepackage[spanish]{babel}
\usepackage[utf8x]{inputenc}
\usepackage{amsmath}
\usepackage{amsfonts}
\usepackage{amssymb}
\usepackage{macros}
%\usepackage[amsmath]{maxiplot}
\usepackage{graphicx}
\usepackage{eurosym}
\usepackage{multicol}
\usepackage{fancybox}
\usepackage[usenames,dvipsnames]{pstricks}
\usepackage{pst-all,pst-math,pst-plot,pst-infixplot,pst-xkey,pstricks-add}
\usepackage{pst-solides3d}
\usepackage{enumitem}
\usepackage{probsoln-alf}
%\usepackage{times}
\usepackage[colorlinks=true]{hyperref}
\hypersetup{pdfauthor={Alfredo S\'anchez Alberca (asalber@ceu.es)}, pdftitle={Ejercicios de C\'alculo}} 
\usepackage{url}
\usepackage[top=3cm, bottom=3cm, left=2.54cm, right=2.54cm, marginparwidth=2mm]{geometry}
\usepackage{fancyhdr}
\pagestyle{fancy}

\lhead{\textsc{\textcolor[rgb]{0.00,0.00,0.50}{Universidad San Pablo CEU}}} \rhead{\textsl{\textsf{\textcolor[rgb]{0.00,0.00,0.50}{Departamento de Matemática Aplicada y Estadística}}}}
\renewcommand{\headrulewidth}{0pt}
\renewcommand{\floatpagefraction}{.8}
\renewcommand{\textfraction}{.1}

\makeatletter
\let\savees@listquot\es@listquot
\def\es@listquot{\protect\savees@listquot}
\makeatletter

\reversemarginpar
\showshortanswers
%\showanswers
%\PSNrandseed{2007}


\begin{document}
\sloppy

\title{\vskip 2cm
\shadowbox{\Huge \textbf{\textsf{\quad \textcolor[rgb]{0.00,0.00,0.50}{EJERCICIOS DE CÁLCULO}\quad}}}\\
   \vskip 1cm
   {\Large \textsf{\textcolor[rgb]{0.50,0.00,0.25}{Asignatura: Matemáticas }}}\\
   {\Large \textsf{\textcolor[rgb]{0.50,0.00,0.25}{Curso: 1º de Grado en Farmacia}}}
   }
\author{
   Eduardo López Ramírez (\url{elopez@ceu.es})
   \and
   José Rojo Montijano (\url{jrojo.eps@ceu.es})
   \and
   Alfredo Sánchez Alberca (\url{asalber@ceu.es})
}
\date{Curso 2013-2014\\[1cm]
\includegraphics[scale=0.3]{img/logo_uspceu_01}}

\maketitle
\tableofcontents
\newpage

\input{derivadas}
% Autor: Alfredo Sánchez Alberca (asalber@ceu.es)

\newproblem{dif-1}{gen}{}
%ENUNCIADO
{Un cilindro de 4 cm de radio ($r$) y 3 cm de altura ($h$) se somete a un proceso de calentamiento con el que varían sus dimensiones de tal
forma que $\dfrac{dr}{dt}=\dfrac{dh}{dt}= 1$ cm/s. Hallar de forma aproximada la variación de su volumen a los 5 segundos y a los
10 segundos.
}
%SOLUCIÓN
{$dV = 2\pi r h dt + \pi r^2 dt$ y en el instante inicial tenemos $dV = 40\pi dt$. A los 5 segundos la variación aproximada será $dV(5) = 40\pi 5 = 200\pi$ cm$^3$/s, y a los 10 segundos $dV(10) = 40\pi 10 = 400\pi$ cm$^3$/s.
}
%RESOLUCIÓN
{
}


\newproblem{dif-2}{qui}{*}
%ENUNCIADO
{La ecuación de los gases perfectos es $PV=CT$ donde $C$ es constante. Si en un cierto instante el volumen es 0.3 m$^3$, la presión es 90 Pa, la temperatura 290 K, y comenzamos a aumentar el volumen a razón de 0.01 m$^3$/s:
\begin{enumerate}
   \item  Hallar la variación de la presión en dicho instante si
   la temperatura se mantiene constante.

   \item  Hallar la variación de la temperatura en dicho instante si
   la presión se mantiene constante.
\end{enumerate}
}
%SOLUCIÓN
{\begin{enumerate}
\item $dP = \frac{-CT}{V^2}dV$ y en el instante indicado vale $dP = -3 $ Pa/s.
\item $dT = \frac{P}{C} dV$ y en el instante indicado vale $dT = 9.67$ K/s.  
\end{enumerate}
}
%RESOLUCIÓN
{
}

\input{derivadas_implicitas}
% Version control information:
%$HeadURL: https://ejercicioscalculo.googlecode.com/svn/trunk/derivadas_parametricas.tex $
%$LastChangedDate: 2010-01-28 20:28:03 +0100 (jue, 28 ene 2010) $
%$LastChangedRevision: 11 $
%$LastChangedBy: asalber $

\newproblem{derpar-1}{gen}{}
%ENUNCIADO
{Una partícula se mueve a lo largo de una curva $y=\cos(2x+1)$, siendo $x=t^2+1$ y $t$ el tiempo. ¿Con qué velocidad está desplazándose respecto a las direcciones vertical y horizontal cuando $t=2$?
}
%SOLUCIÓN
{Velocidad horizontal: $\frac{dx}{dt} = 2t$ y en el instante $t=2$, $\frac{dx}{dt}(t=2)=4$.\\
Velocidad vertical: $\frac{dy}{dt}=-\sen(2t^2+3)4t$ y en el instante $t=2$, $\frac{dy}{dt}=-8\sen 11$. 
}
%RESOLUCIÓN
{
}


\newproblem{derpar-2}{gen}{}
%ENUNCIADO
{Un punto se mueve en el plano siguiendo una trayectoria
\[ \left\{
\begin{array}{rcl}
    x & = & \sen t,\\
    y & = & t^2-1.
\end{array} \right.\]
Se pide:
\begin{enumerate}
    \item  Hallar la derivada de la función $y(x)$ (es decir,
    $\dfrac{dy}{dx}$) para los puntos $t=0$ y $t=2$.

    \item  Hallar la tangente a la trayectoria en el punto (0,-1).
\end{enumerate}
}
%SOLUCIÓN
{\begin{enumerate}
\item $\frac{dy}{dx} = \frac{2t}{\cos t}$. $\frac{y}{dx}(t=0) = 0$ y $\frac{dy}{dx}(t=2) = 4/\cos 2$.
\item Tantenge: $y=-1$. 
\end{enumerate}
}
%RESOLUCIÓN
{
}


\newproblem*{derpar-3}{gen}{}
%ENUNCIADO
{Una partícula se mueve a lo largo de la curva
\[ \left\{
 \begin{array}{rcl}
   x & = & 2\sen t,  \\
   y & = & \sqrt{3}\cos t,
 \end{array}\right.
  \]
donde $x$ e $y$ están medidos en metros y el tiempo $t$ en
segundos.
\begin{enumerate}
   \item  Hallar la ecuación de la recta tangente a la trayectoria
   en el punto (1,3/2).

   \item  ¿Con qué velocidad se mueve la partícula respecto a
   las direcciones vertical y horizontal en dicho punto?
\end{enumerate}
}


\newproblem{derpar-4}{gen}{*}
%ENUNCIADO
{Las coordenadas paramétricas de un punto material lanzado bajo un ángulo respecto al horizonte son
\[
\left\{
  \begin{array}{ll}
    x=v_0t \\
    y=-\frac{1}{2}gt^2
  \end{array}
\right.
\]
donde $t$ es el tiempo contado a partir del instante en que el punto llega a la posición más alta, $v_0$ es la velocidad horizontal en el instante $t=0$ y $g=9.8$ m$^2$/s es la aceleración de la gravedad. ¿En qué instante la magnitud de la velocidad horizontal será igual a la de la velocidad vertical? ¿Cuánto debería valer $v_0$ para que en dicho instante el punto haya recorrido 100 m horizontalmente? Calcular la ecuación de la recta tangente en dicho instante con el valor de $v_0$ calculado.
}
%SOLUCION
{
Las velocidades serán iguales en el instante $ t=\frac{v_0}{9.8}$. Para que en dicho instante el punto haya recorrido 10  m horizontalmente, la velocidad inicial debería ser $v_0 = 31.3$ m/s.

La ecuación de la recta tangente en dicho instante es $y =-x+50.14$.
}
%RESOLUCIÓN
{La velocidad horizontal es la derivada del espacio recorrido horizontalmente (componente $x$) con respecto al tiempo, es decir,
\[
\frac{dx}{dt} = \frac{d}{dt}(v_0t)=v_0.
\]
Del mismo modo, la velocidad vertical es la derivada del espacio recorrido verticalmente (componente $y$) en relación al tiempo,
\[
\frac{dy}{dt} = \frac{d}{dt}(-\frac{1}{2}gt^2)=-gt
\]
Para ver en qué instante ambas magnitudes serán iguales, las igualamos y resolvemos la ecuación:
\[
|\frac{dx}{dt}|=|\frac{dy}{dt}| \Leftrightarrow v_0 = gt \Leftrightarrow t=\frac{v_0}{g}=\frac{v_0}{9.8}.
\]

Para que en dicho instante el punto haya recorrido 100 m horizontalmente, debe cumplirse que $x(v_0/9.8)=100$, de lo que se deduce:
\[
x(v_0/9.8)=v_0\frac{v_0}{9.8} = \frac{v_0^2}{9.8}=100 \Leftrightarrow v_0^2 = 980 \Leftrightarrow v_0 = +\sqrt{980}= 31.3.
\]
Por tanto, el instance en cuestión es $t=v_0/9.8= 31.3/9.8 = 3.19$.

Por último, la ecuación de la recta tangente en dicho instante, para el valor de $v_0$ calculado es:
\[
y = y(3.19) + \frac{dy}{dx}(3.19) (x-x(3.19))
\]
Ya hemos visto que $x(3.19)=100$, y que en dicho instante la velocidad horizontal y vertical coinciden, de manera que
\[
\frac{dy}{dx}(3.19)=\frac{dy/dt}{dx/dt}=-1,
\]
de modo que sólo nos queda calcular el espacio vertical recorrido en dicho instante, que es
\[
y(3.19)=-\frac{1}{2}9.8\cdot 3.19^2= -49.86.
\]
Sustituyendo en la ecuación anterior llegamos a la recta tangente:
\[
y = -49.86-(x-100) \Leftrightarrow y=-x+50.14.
\]
}


\newproblem{derpar-5}{gen}{*}
%ENUNCIADO
{Dada la función paramétrica
\[
\left(
    x =\frac{(t-2)^2}{t^2+1},\, y=\dfrac{2t}{t^2+1}
\right)
\]
Calcular los valores máximos y mínimos de $x$ y de $y$. ¿En qué instante la tasa de crecimiento de $y$ coincide con la de $x$?
}
%SOLUCIÓN
{$\frac{dx}{dt}=\frac{4t^2-6t-4}{(t^2+1)^2}$. Puntos críticos: $t=-1/2$ (máximo) y $t=2$ (mínimo).\\
$\frac{dy}{dt}=\frac{-2t^2+2}{(t^2+1)^2}$. Puntos críticos: $t=-1$ (mínimo) y $t=1$ (máximo).\\
$\frac{dx}{dt}=\frac{dy}{dt}$ en los puntos $t=\frac{1-\sqrt 5}{2}$ y $t=\frac{1+\sqrt 5}{2}$.
}
%RESOLUCIÓN
{
}


\newproblem*{derpar-6}{gen}{*}
%ENUNCIADO
{Una mosca se mueve en un plano siguiendo la trayectoria
\[
\left\{
\begin{array}{lll}
x & = & \sen t
\; ,
\\
y & = & \cos t + t^2 - 1
\; .
\end{array}
\right.
\]
Se pide
\begin{enumerate}
\item Hallar la derivada de la función $y(x)$, es decir $dy/dx$,
en los puntos $t=0$ y $t=\pi/2$.
\item Hallar la ecuación de la recta tangente y normal a la trayectoria
en el punto $(x,y)=(0,0)$.
\end{enumerate}
}


\newproblem*{derpar-7}{gen}{*}
%ENUNCIADO
{Dadas las siguientes ecuaciones paramétricas:
\[
\left\{
\begin{array}{l}
x(t)=e^{at}t \\
y(t)=\ln t\cos (t-1)
\end{array}
\right.
\] 
calcular la ecuación de la recta tangente a la gráfica de $y$ como función de $x$ en el punto que corresponde a $t=1$.
}


\newproblem*{derpar-8}{gen}{*}
%ENUNCIADO
{La cantidad de árboles en un ecosistema, $a$, depende del tiempo según la expresión:
\[
a(t)=100\ln(t^2+1)
\]
Y la cantidad de un determinado parásito de los árboles, $p$, que también depende del tiempo, viene dada por:
\[
p(t) = \sqrt[3]{{t^2  + 2}}
\]
Y se pide:
\begin{enumerate}
\item Calcular el número de parásitos cuando el número de árboles sea 500.
\item La derivada del número de parásitos con respecto al número de árboles cuando el número de parásitos sea 3.
\end{enumerate}
}


\newproblem*{derpar-9}{gen}{*}
%ENUNCIADO
{Supongamos un ecosistema en el que hay una especie ``presa", $p$, y otra ``depredador", $d$, y que la cantidad de individuos de una y otra dependen del tiempo, en años, según las siguientes expresiones ($t>0$):
\[
\renewcommand{\arraystretch}{2.2}
\begin{array}{*{20}c}
   {p(t) = \dfrac{{\ln (t^2  + 1)}}{{t + 1}}}  \\
   {d(t) = te^{ - 2t} }  \\
\end{array}
\]
\begin{enumerate}
\item Calcular el número de presas y depredadores para tiempos muy grandes.
\item Calcular la derivada del número de presas con respecto a los depredadores cuando $d=2/e^4$.
\end{enumerate}
}


\newproblem{derpar-10}{gen}{*}
%ENUNCIADO
{Un punto se mueve en el plano siguiendo una trayectoria 
\[
\begin{cases}
x = \tg t,  \\ 
y = t^2-2t+3. \\
\end{cases}
\]

\begin{enumerate}
\item  Hallar $\frac{\partial y}{\partial x}$ en $t=0$.
\item  Hallar la tangente a la trayectoria en el punto $(0,3)$.
\end{enumerate}
}
%SOLUCIÓN
{\begin{enumerate}
\item $\dfrac{\partial y}{\partial x}(t) = \frac{2t-2}{1+\tg^2t}$ y $\dfrac{\partial y}{\partial x}(0) = -2$.
\item $y = 3-2x$.
\end{enumerate}
}
%RESOLUCIÓN
{Se trata de la ecuación de una trayectoria en coordenadas paramétricas.
\begin{enumerate}
\item  Aplicando la regla de la cadena se tiene que 
\[
\dfrac{\partial y}{\partial t} = \dfrac{\partial y}{\partial x}\dfrac{\partial x}{\partial t},
\] 
en consecuencia,
\[
\dfrac{\partial y}{\partial x}(t) = \frac{\partial y/\partial t}{\partial x/\partial t}(t)=\frac{2t-2}{1+\tg^2t}.
\]
En el punto $t=0$ tendremos
\[
\dfrac{\partial y}{\partial x}(0) = \frac{-2}{1+\tg^20} = -2.
\]

\item  La ecuación de la recta tangente a la trayectoria en el punto $(x(t_0),y(t_0))$ correspondiente al instante $t_0,$ viene dada por la expresión
\[
y-y(t_0) = \dfrac{\partial y}{\partial x}(t_0)(x-x(t_0)).
\]
Como el punto $(0,3)$ se alcanza precisamente en el instante $t=0$ tenemos que la ecuación de la recta tangente a la trayectoria en dicho instante es:
\[
y-y(0) = \dfrac{\partial y}{\partial x}(0)(x-x(0)),
\]
es decir,
\[
y-3 = -2(x-0),
\]
y simplificando obtenemos:
\[
y = 3-2x.
\]
\end{enumerate}
}



\newproblem{derpar-11}{gen}{}
%ENUNCIADO
{Hallar las ecuaciones de la rectas tangente y normal a la curva $C$ en el punto $P$ en cada uno de los casos siguientes:
\begin{enumerate}
\item $C: y=x^2$, $P=(0,0)$
\item $C: \begin{cases}
x=2\cos t,\\
y=2\sen t,
\end{cases}
$ $0\leq t\leq 2\pi$, $P=(0,2)$
\item $C:x^2+y^2=1$, $P=(\sqrt{2}/2),\sqrt{2}/2)$
\item $C:(x-1)^2+y^2=4$, $P=(3,0)$
\item $C:x^2-y^2=1$, $P=(1,0)$
\item $C:\begin{cases}
x=e^t\cos t,\\
y=e^t\sen t,
\end{cases}
$, $t\in \mathbb{R}$, $P=(1,0)$
\end{enumerate}
}
%SOLUCIÓN
{
}
%RESOLUCIÓN
{
}


\newproblem{derpar-12}{gen}{}
%ENUNCIADO
{Hallar al ecuación de la recta tangente y el plano normal a la línea
\[
C: 
\begin{cases}
x=\cos t \\
y=\sen t\\
z= t,
\end{cases}
\quad t\in \mathbb{R},
\]
en el punto $P=(1,0,0)$.
}
%SOLUCIÓN
{
}
%RESOLUCIÓN
{
}


\newproblem{derpar-13}{gen}{}
%ENUNCIADO
{Una trayectoria pasa por el punto $(3,6,5)$ en el instante $t=0$ con velocidad $\mathbf{i}-\mathbf{k}$. 
Hallar la ecuación del plano normal y de la recta tangente en ese instante.
}
%SOLUCIÓN
{
}
%RESOLUCIÓN
{
}


\newproblem{derpar-14}{gen}{}
%ENUNCIADO
{Una partícula sigue la trayectoria
\[
\begin{cases}
x=e^t,\\
y=e^{-t},\\
z=\cos t,
\end{cases}
\quad t\in \mathbb{R}
\]
hasta que se sale por la tangente en el instante $t=1$. ¿Dónde estará en el instante $t=3$?
}
%SOLUCIÓN
{
}
%RESOLUCIÓN
{
}
% Autor: Alfredo Sánchez Alberca (asalber@ceu.es)

\newproblem{dertray-1}{gen}{*}
%ENUNCIADO
{La presión en la posición $(x,y,z)$ de un espacio es 
\[
f(x,y,z)= x^2+y^2-z^3
\]
y la trayectoria de un observador $A$ es 
\[
\begin{cases}
x=t\\
y=1\\
z=1/t
\end{cases}
t>0.
\]
Se pide:
\begin{enumerate}
\item Calcular la ecuación de la recta tangente a la trayectoria de $A$ en el punto $(1,1,1)$.
\item ¿Es la dirección de esta trayectoria al pasar por el punto $(1,1,1)$ aquella en la que el crecimiento de $g$
es máximo? 
Justificar la respuesta. 
\end{enumerate}
}
%SOLUCIÓN
{\begin{enumerate}
\item $l:(1+t,1,1-t)$.
\item La dirección de la trayectoria $A$ en $(1,1,1)$ es $(1,0,-1)$ y la dirección de máximo crecimiento de $g$ en
$(1,1,1)$ es $(2,2,-3)$, luego no coinciden. 
\end{enumerate}
}
%RESOLUCIÓN
{
}

% Autor: Alfredo Sánchez Alberca (asalber@ceu.es)

\newproblem{ext-1}{gen}{*}
%ENUNCIADO
{La figura adjunta es la de la derivada de una función.  Estudiar el comportamiento de la función (crecimiento, decrecimiento, extremos, concavidad y convexidad).
\[
\scalebox{0.8}{\psset{unit=1,algebraic}
\begin{pspicture*}(-0.62,-2.61)(8.56,3.1)
\psaxes[labelFontSize=\scriptstyle,ticksize=-3pt 0,labelsep=2pt,ticks=none,labels=none]{<->}(0,0)(-0.62,-2.61)(8.56,3.1)
\psplot[plotpoints=200]{-0.621375292779696}{8.55937801199009}{1.7+3*(x-2)-4.3*(x-2)^2+(x-2)^3}
\rput[bl](6,2.5){$f'(x)$}
\psxTick(1.64){a}
\psxTick(2.41){b}
\psxTick(3.47){c}
\psxTick(4.46){d}
\psxTick(5.19){e}
\end{pspicture*}}
\]
}
%SOLUCIÓN
{Crecimiento: Decreciente en $(-\infty,a)$ y $(c,e)$, y creciente en $(a,c)$ y $(e,\infty)$.\\
Extremos: Mínimos en $x=a$ y $x=e$, y máximo en $x=c$.\\
Concavidad: Cóncava en $(-\infty,b)$ y $(d,\infty)$, y convexa en $(b,d)$. 
}
%RESOLUCIÓN
{
}

\newproblem{ext-2}{gen}{}
%ENUNCIADO
{Hallar $a$, $b$ y $c$ en la función  $f(x)=x^3+bx^2+cx+d$ para que tenga un punto de inflexión en $x=3$, pase por el punto $(1,0)$ y alcance un máximo en $x=1$.
}
%SOLUCIÓN
{$b=-9$, $c=15$ y $d=-7$.
}
%RESOLUCIÓN
{
}

\newproblem{ext-3}{far}{}
{Se ha diseñado un envoltorio cilíndrico para unas cápsulas. Si el contenido de las cápsulas debe ser de $0.15$ ml, hallar las dimensiones del cilindro para que el material empleado en el envoltorio sea mínimo.
}
%SOLUCIÓN
{Radio $0.2879$ cm y altura $0.5760$ cm.
}
%RESOLUCIÓN
{
}


\newproblem*{ext-4}{gen}{*}
%ENUNCIADO
{La variable aleatoria bidimensional $(X,Y)$ con función de densidad
\[
f(x,y) = \frac{1}{\sqrt{2\pi}\, \sigma_x\sigma_y} e^{-\frac{1}{2}\left(\frac{(x-\mu_x)^2}{\sigma_x^2}+\frac{(y-\mu_y)^2}{\sigma_y^2}\right)}
\]
se conoce como normal bidimensional con $X$ e $Y$ independientes, de parámetros $\mathbf{\mu}=(\mu_x,\mu_y)$ y $\mathbf{\sigma}=(\sigma_x,\sigma_y)$.
Calcular los puntos de inflexión de la curva formada por la intersección de la superficie de $f$ con el plano $y=x$.
}


\newproblem{ext-5}{amb}{*}
%ENUNCIADO
{Mediante simulación por ordenador se ha podido cuantificar la cantidad de agua almacenada en un acuífero en función del tiempo, $m(t)$, en millones de metros cúbicos, y el tiempo $t$ en años transcurridos desde el instante en el que se ha hecho la simulación, teniendo en cuenta que la ecuación sólo tiene sentido para los $t$ mayores que 0:
\[
m(t) = 10 + \frac{{\sqrt t }} {{e^t }}
\]
\begin{enumerate}
\item En el límite, cuando $t$ tiende a infinito, qué cantidad de agua almacenada habrá en el acuífero?
\item Mediante derivadas, calcular el valor del tiempo en el que el agua almacenada ser máxima y cuál es su cantidad de agua correspondiente en millones de metros cúbicos.
\end{enumerate}
}
%SOLUCIÓN
{\begin{enumerate}
\item $\lim_{t\rightarrow \infty}m(t) = 10$.
\item $\frac{dm}{dt}=e^{-t}(\frac{1}{2}t^{-1/2}-t^{1/2})$. El instante en el que el agua almacenada será máxima es $t=0.5$ años y en dicho
instante habrá $10.429$ millones de m$^3$.
\end{enumerate}
}
%RESOLUCIÓN
{
}


\newproblem{ext-6}{far}{*}
%ENUNCIADO
{Se está estudiando fabricar unas cápsulas de cuerpo cilíndrico terminadas en sus extremos por dos semiesferas. El volumen de la cápsula debe ser $0.8$ cm$^3$ y se quiere que la superficie sea mínima. ¿Cuáles deben ser las dimensiones del radio y de la longitud de la parte cilíndrica? Comentar el resultado obtenido.
\begin{quote} 
    \textbf{Nota}:\\
    Volumen del cilindro $V=\pi r^2 h$\\
    Superficie lateral del cilindro $S=2\pi r h$\\
    Volumen de la esfera $V= \frac{4}{3}\pi r^3$\\
    Superficie de la esfera $S=4\pi r^2$\\
\end{quote}
}
%SOLUCIÓN
{$r = 0.5759$ y $h=0$.
}
%RESOLUCIÓN
{La capsula está formada por un cilindro de radio $r$ y altura $h$ mas una esfera (2 semiesferas) de radio $r$, 
\[
\includegraphics[scale=0.4]{img/capsula-ext-6}
\]
así que su volumen es
\[
V(r,h) = \pi r^2 h + \frac{4}{3}\pi r^3,
\]  
y su superficie
\[
S(r,h) = 2\pi r h +4 \pi r^2.
\]
Como el volumen  debe ser $0.8$ cm$^3$ tenemos que
\[
V(r,h) = \pi r^2 h + \frac{4}{3}\pi r^3 = 0.8 \Leftrightarrow h = \frac{0.8-4/3\pi r^3}{\pi r^2},
\]
y sustituyendo en la fórmula de superficie tenemos
\[
S(r) = 2\pi r \frac{0.8-4/3 \pi r^3}{\pi r^2} +4 \pi r^2 = \frac{1.6-8/3\pi r^2}{r}+4\pi r^2 = \frac{1.6}{r}-\frac{8}{3}\pi r^2 +4\pi r^2 = \frac{1.6}{r^2}+\frac{4}{3}\pi r^2
\]
Como queremos que la superficie de la cápsula sea mínima, tenemos que buscar el mínimo de esta función. Para ello, calculamos primero los puntos críticos que anulan su derivada:
\[
\frac{dS}{dr} = -\frac{1.6}{r^2}+\frac{4}{3}\pi2r = 0 \Leftrightarrow \frac{1.6}{r^2} = \frac{8}{3}\pi r \Leftrightarrow \frac{8}{3}\pi r^3 = 1.6 \Leftrightarrow r^3 = \frac{1.6}{8/3 \pi} = 0.1910 \Leftrightarrow r = \sqrt[3]{0.1910} = 0.5759
\]
y por tanto la altura será
\[
h = \frac{0.8-4/3\pi r^3}{\pi r^2} = \frac{0.8-4/3\pi 0.5759^3}{\pi 0.5759^2} = 0.
\]
Esto quiere decir, que realmente no habría cilindro, y por tanto para que la supercie sea mínima la cápsula debería tener forma de esfera. 

Sólo falta comprobar que el punto anterior es realmente un punto de mínimo. Para ello podemos utilizar la segunda derivada
\[
\frac{d^2S}{dr^2} = \frac{d}{dr}\left(-\frac{1.6}{r^2}+\frac{8}{3}\pi r\right) = \frac{1.6\cdot 2r}{r^4}+\frac{8}{3}\pi = \frac{3.2}{r^3}+\frac{8}{3}\pi. 
\]
y sustituyendo en el punto anterior tenemos
\[
\frac{d^2S}{dr^2}(0.5759) =  \frac{3.2}{0.5759^3}+\frac{8}{3}\pi = 25.13 >0,
\]
que al tener signo positivo indica que efectivamente se trata de un mínimo.
}

\newproblem{ext-7}{gen}{*}
%ENUNCIADO
{Sea $f(x)$ una función cuya derivada vale:
\[
f'(x) = \frac{(2-x) e^{-\frac{x^2}{2}+2x-2}}{\sqrt{2\pi}}
\]
Se pide:
\begin{enumerate}
\item Estudiar el crecimiento de $f$.
\item Calcular los valores de $x$ en los que $f$ tiene extremos relativos.
\item Estudiar la concavidad de $f$.
\item Calcular los valores de $x$ en los que $f$ tiene puntos de inflexión.
\end{enumerate}
}
%SOLUCIÓN
{\begin{enumerate}
\item Creciente en $x<2$ y decreciente en $x>2$.
\item Máximo relativo en $x=2$.
\item Cóncava en $(-\infty,1)$ y $(3,\infty)$. Convexa en $(1,3)$.
\item Puntos de inflexión en $x=1$ y $x=3$.
\end{enumerate}
}
%RESOLUCIÓN
{
}


\newproblem{ext-8}{qui}{}
%ENUNCIADO
{La velocidad $v$ de una reacción irreversible $A+B\rightarrow AB$ es función de la concentración $x$ del producto $AB$ y puede expresarse según la ecuación
\[
v(x) = 4(3-x)(5-x).
\]
¿Qué valor de $x$ maximiza la velocidad de reacción?
}
%SOLUCIÓN
{Ninguno.
}
%RESOLUCIÓN
{
}


\newproblem{ext-9}{amb}{}
%ENUNCIADO
{La cantidad de trigo en una cosecha $C$ depende del nivel de nitrógeno en el suelo $n$ según la ecuación
\[
C(n) = \frac{n}{1+n^2},\quad n\geq 0. 
\]
¿Para qué nivel de nitrógeno se obtendrá la mayor cosecha de trigo?
}
%SOLUCIÓN
{$n=1$.
}
%RESOLUCIÓN
{
}


\newproblem{ext-10}{amb}{}
%ENUNCIADO
{Existen organismos que se reproducen una sóla vez en su vida como por ejemplo los salmones. 
En este tipo de especies, la velocidad de incremento per cápita $v$, que mide la capacidad reproductiva, depende de la edad $x$ según la ecuación
\[
v(x) = \frac{\log(p(x)h(x))}{x},
\] 
donde $p(x)$ es la probabilidad de sobrevivir hasta la edad $x$ y $h(x)$ es el número de nacimientos de hembras a la edad $x$. 
Calcular la edad óptima de reproducción, es decir, el valor que maximice $v$, para $p(x)=e^{-0.1x}$ y $h(x)=4x^{0.9}$.}
%SOLUCIÓN
{$x=0.58$ años.
}
%RESOLUCIÓN
{
}


\newproblem{ext-12}{amb}{}
%ENUNCIADO
{La sensibilidad de $S$ de un organismo ante un fármaco depende de la dosis $x$ suministrada según la relación
\[
S(x) = x(C-x),
\]
siendo $C$ la cantidad máxima del fármaco que puede suministrarse, que depende de cada individuo. 
Hallar la dosis $x$ para la que la sensibilidad es máxima. 
}
%SOLUCIÓNsensibilidad
{$x=C/2$.
}
%RESOLUCIÓN
{
}

\input{taylor}
% Autor: Alfredo Sánchez Alberca (asalber@ceu.es)

\newproblem{par-1}{gen}{}
%ENUNCIADO
{Calcular las siguientes derivadas parciales:
\begin{multicols}{2}
\begin{enumerate}
\item $\dfrac{\partial}{\partial x}\ln \dfrac{x}{y}$.
\item $\dfrac{\partial}{\partial v}\dfrac{nRT}{v}$.
%\item $\dfrac{\partial^2}{\partial x \partial y}\left(e^{x+y}\sen\dfrac{x}{y}\right)$.
%\item $\dfrac{\partial^2}{\partial y \partial x}\left(e^{x+y}\sen\dfrac{x}{y}\right)$.
\end{enumerate}
\end{multicols}
}
%SOLUCIÓN
{\begin{enumerate}

\item $\frac{\partial}{\partial x}\,\log \left(\frac{x}{y}\right) = \frac{1}{x}$.
\item $\frac{\partial}{\partial v}\,\left(\frac{n\,R\,T}{v}\right) = -\frac{n\,R\,T}{v^2}$.
%\item $\frac{\partial^2}{\partial x \partial y}\,\left(\sin \left(\frac{x}{y}\right)\,e^{y+x}\right) = \frac{\left(\sen \left(\frac{x}{y}\right)\,y^3+\cos \left(\frac{x}{y}\right)\,y^2-x\,\cos \left(\frac{x}{y}\right)\,y-\cos \left(\frac{x}{y}\right)\,y+x\,\sen \left(\frac{x}{y}\right)\right)\,e^{y+x}}{y^3}$
%\item $\frac{\partial^2}{\partial y \partial x}\,\left(\sin \left(\frac{x}{y}\right)\,e^{y+x}\right) = \frac{\left(\sen \left(\frac{x}{y}\right)\,y^3+\cos \left(\frac{x}{y}\right)\,y^2-x\,\cos \left(\frac{x}{y}\right)\,y-\cos \left(\frac{x}{y}\right)\,y+x\,\sen \left(\frac{x}{y}\right)\right)\,e^{y+x}}{y^3}$
\end{enumerate}
}
%RESOLUCIÓN
{
}


\newproblem{par-2}{gen}{}
%ENUNCIADO
{Calcular el vector gradiente y la matriz Hessiana de las siguientes funciones:
\begin{multicols}{2}
\begin{enumerate}
\item $e^{x^2+y^2+z^2}$
\item $\sen((x^2-y^2)z)$
\end{enumerate}
\end{multicols}
}
%SOLUCIÓN
{\begin{enumerate}
\item $\nabla e^{x^2+y^2+z^2} = \left( 2\,x\,e^{z^2+y^2+x^2} , 2\,y\,e^{z^2+y^2+x^2} , 2\,z\,e^{z^2 +y^2+x^2} \right)$,\\
$
H e^{x^2+y^2+z^2} =
\left(
\begin{array}{ccc}
(4x^2+2)e^{x^2+y^2+z^2} & 4xye^{x^2+y^2+z^2} & 4xze^{x^2+y^2+z^2} \\
4xye^{x^2+y^2+z^2} & (4y^2+2)e^{x^2+y^2+z^2} & 4yze^{x^2+y^2+z^2} \\
4xze^{x^2+y^2+z^2} & 4yze^{x^2+y^2+z^2} & (4z^2+2)e^{x^2+y^2+z^2}
\end{array}
\right).
$
\item $\nabla \sen((x^2-y^2)z) = \left( 2\,x\,z\,\cos \left(\left(x^2-y^2\right)\,z\right) , -2\,y\, z\,\cos \left(\left(x^2-y^2\right)\,z\right) , \left(x^2-y^2\right) \,\cos \left(\left(x^2-y^2\right)\,z\right) \right) $\\
$H \sen((x^2-y^2)z) =$\\
\resizebox{\linewidth}{!}{
$
\left(
\begin{array}{ccc}
4x^2\sen((x^2-y^2)z)+2\cos((x^2-y^2)z) & 4xy\sen((x^2-y^2)z) & -2x(x^2-y^2)\sen((x^2-y^2)z) \\
4xy\sen((x^2-y^2)z) & -4y^2\sen((x^2-y^2)z)-2\cos((x^2-y^2)z) & 2y(x^2-y^2)\sen((x^2-y^2)z) \\
-2x(x^2-y^2)\sen((x^2-y^2)z) & 2y(x^2-y^2)\sen((x^2-y^2)z) & -(x^2-y^2)^2\sen((x^2-y^2)z)
\end{array}
\right).
$
}
\end{enumerate}
}
%RESOLUCIÓN
{
}


\newproblem{par-3}{gen}{*}
%ENUNCIADO
{Calcular el gradiente de la función
\[ f(x,y,z)=\log \frac{\sqrt{x}}{yz}+\arcsen (xz). \]
}
%SOLUCIÓN
{$\nabla f(x,y,z) = \left( \frac{z}{\sqrt{1-x^2z^2}}+\frac{1}{2x} ,\frac{-1}{y} , \frac{x}{\sqrt{1-x^2\,z^2}}-\frac{1}{z} \right) $.
}
%RESOLUCIÓN
{
}


\newproblem{par-4}{gen}{}
% ENUNCIADO
{Una nave espacial está en problemas cerca del sol.
Se encuentra en la posición $(1,1,1)$ y la temperatura de la nave cuando está en la posición $(x,y,z)$ viene dada por
$T(x,y,z)=\mbox{e}^{-x^2-2y^2-3z^2}$ donde $x,y,z$ se miden en metros.
¿En qué dirección debe moverse la nave para que la temperatura decrezca lo más rápidamente posible? }
%SOLUCIÓN
{Debe moverse en la dirección $-\nabla f(1,1,1)=e^{-6}(2,4,6)$.
}
%RESOLUCIÓN
{
}

\newproblem{par-5}{gen}{*}
%ENUNCIADO
{Dada la función
\[
f(x,y,z)=\log \sqrt{xy-\frac{z^2}{xy}}
\]
\begin{enumerate}
\item Hallar el vector gradiente.
\item Hallar un punto en el que el vector gradiente sea paralelo a la bisectriz del plano $XY$, y calcular el vector gradiente en dicho punto.
\end{enumerate}
}
%SOLUCIÓN
{\begin{enumerate}
\item $\nabla f(x,y,z) = \left( -\frac{z^2+x^2y^2}{2xz^2-2x^3y^2} , -\frac{z^2+x^2y^2}{2yz^2-2x^2y^3} , \frac{z}{z^2-x^2y^2}  \right) $.
\item El vector gradiente es paralelo a la bisectriz del plano $XY$ en cualquier punto de la forma $(a,a,0)$ con $a\in \mathbb{R}$.\\
$\nabla f(1,1,0) = \left(\frac{1}{2},\frac{1}{2},0\right)$.
\end{enumerate}
}
%RESOLUCIÓN
{
}


\newproblem{par-6}{far}{*}
%ENUNCIADO
{La cantidad $C$ de cierta toxina en sangre (en mg/dl) depende del número de bacterias, $b$ (bacterias/dl), del número de linfocitos, $l$ (linfocitos/dl), y del tiempo, $t$ (horas), según la ecuación:
\[
C(b,l,t) = \frac{{t^2  \cdot e^{3b + 2} }}{{l^2 }} - \frac{1}{{\log
(b \cdot l)}}
\]
\begin{enumerate}
\item Calcular su gradiente.

\item Comprobar que se cumple: $\dfrac{{\partial ^2 C}}{{\partial t\partial b}} = \dfrac{{\partial ^2 C}}{{\partial b\partial t}}$.
\end{enumerate}
}
%SOLUCIÓN
{
\begin{enumerate}
\item $\nabla C(b,l,t)=\left( \frac{{3t^2 \cdot e^{3b + 2} }}{{l^2 }}+\frac{1}{{b\log^2
(b \cdot l)}}, \frac{{-2t^2 \cdot e^{3b + 2} }}{{l^3 }}+\frac{1}{{l\log^2
(b \cdot l)}}, \frac{{2t \cdot e^{3b + 2} }}{{l^2 }} \right)$.

\item $\frac{\partial ^2 C}{\partial t \partial b}  = \frac{{6t \cdot e^{3b + 2} }}{{l^2 }}$.
\end{enumerate}
}
%RESOLUCIÓN
{
\begin{enumerate}
  \item La fórmula del gradiente es
\begin{equation}
\label{e:gradiente}
\nabla C(b,l,t)=\left(\frac{\partial C}{\partial b}, \frac{\partial C}{\partial l},\frac{\partial C}{\partial t}\right),
\end{equation}
de modo que necesitamos calcular las tres primeras derivadas parciales:
\begin{align*}
\frac{\partial C}{\partial b} &= \frac{\partial}{\partial b}\left(\frac{{t^2
\cdot e^{3b + 2} }}{{l^2 }}\right)-\frac{\partial}{\partial b}\left(\frac{1}{{\log
(b \cdot l)}}\right)= \frac{{3t^2 \cdot e^{3b + 2} }}{{l^2 }}+\frac{1}{{b\log^2
(b \cdot l)}}\\
\frac{\partial C}{\partial l} &= \frac{\partial}{\partial l}\left(\frac{{t^2
\cdot e^{3b + 2} }}{{l^2 }}\right)-\frac{\partial}{\partial l}\left(\frac{1}{{\log
(b \cdot l)}}\right)= \frac{{-2t^2 \cdot e^{3b + 2} }}{{l^3 }}+\frac{1}{{l\log^2
(b \cdot l)}}\\
\frac{\partial C}{\partial t} &= \frac{\partial}{\partial t}\left(\frac{{t^2
\cdot e^{3b + 2} }}{{l^2 }}\right)-\frac{\partial}{\partial t}\left(\frac{1}{{\log
(b \cdot l)}}\right)= \frac{{2t \cdot e^{3b + 2} }}{{l^2 }}\\
\end{align*}

Así que, sustituyendo en la fórmula \ref{e:gradiente} tenemos:
\[
\nabla C(b,l,t)=\left( \frac{{3t^2 \cdot e^{3b + 2} }}{{l^2 }}+\frac{1}{{b\log^2
(b \cdot l)}}, \frac{{-2t^2 \cdot e^{3b + 2} }}{{l^3 }}+\frac{1}{{l\log^2
(b \cdot l)}}, \frac{{2t \cdot e^{3b + 2} }}{{l^2 }} \right).
\]

\item Para ver si se satisface la igualdad calculamos ambas derivadas:
\begin{align*}
\frac{\partial ^2 C}{\partial t \partial b} & = \frac{\partial}{\partial
t}\left(\frac{\partial C}{\partial b}\right) = \frac{\partial}{\partial t}\left(
\frac{{3t^2 \cdot e^{3b + 2} }}{{l^2 }}+\frac{1}{{b\log^2
(b \cdot l)}} \right) = \frac{{6t \cdot e^{3b + 2} }}{{l^2 }} \\
\frac{\partial ^2 C}{\partial b \partial t} & = \frac{\partial}{\partial
b}\left(\frac{\partial C}{\partial t}\right) = \frac{\partial}{\partial b}\left(
\frac{{2t \cdot e^{3b + 2} }}{{l^2 }}\right) = \frac{{6t \cdot e^{3b + 2} }}{{l^2 }}
\end{align*}
Por tanto, la igualdad es cierta.
\end{enumerate}
}


\newproblem*{par-7}{amb}{*}
%ENUNCIADO
{Supongamos que la cantidad de agua almacenada en un pantano al final del año hidrológico, $A$ en hectómetros cúbicos, viene dada por:
\[
A = \sqrt {\frac{{p^3 }}{{t - 1}} - c^2 e^{cpt}}
\]
donde $p$ es la precipitación en litros/m$^2$ caí­da durante el año hidrológico, $t$ es la temperatura media del año hidrológico en ºC y $c$ el consumo debido a abastecimiento de poblaciones cercanas y riego, en hectómetros cúbicos.
Se pide:
\begin{enumerate}
\item Calcular el gradiente de la cantidad de agua almacenada.
\item Suponiendo que hubiese algún año en el que el consumo fuese nulo, ¿qué condición tendría­ que cumplir la temperatura para que la derivada del agua almacenada con respecto a la temperatura fuese igual a la derivada con respecto a la precipitación?
\end{enumerate}
}


\newproblem*{par-8}{gen}{*}
%ENUNCIADO
{Dada la función $f(x)=e^{2xy}\sen(x+3z)$, se pide:
\begin{enumerate}
  \item ¿Calcular el vector gradiente en el origen de coordenadas?
  \item ¿Es cierto que $\dfrac{\partial^3f}{\partial y^2\partial z}=\dfrac{\partial^3f}{\partial y\partial z\partial y}?$
\end{enumerate}
}


\newproblem{par-9}{gen}{*}
%ENUNCIADO
{La variable aleatoria bidimensional $(X,Y)$ con función de densidad
\[
f(x,y) = \frac{1}{\sqrt{2\pi}\, \sigma_x\sigma_y} e^{-\frac{1}{2}\left(\frac{(x-\mu_x)^2}{\sigma_x^2}+\frac{(y-\mu_y)^2}{\sigma_y^2}\right)}
\]
se conoce como normal bidimensional con $X$ e $Y$ independientes, de parámetros $\mathbf{\mu}=(\mu_x,\mu_y)$ y $\mathbf{\sigma}=(\sigma_x,\sigma_y)$.
Calcular el gradiente de $f$ e interpretarlo. ¿En qué punto se anula el gradiente? ¿Qué conclusiones sacas? ¿Cuál es la tasa de crecimiento de $f$ cuando $x\rightarrow \infty$?
}
%SOLUCIÓN
{$\nabla f(x,y) = -\frac{1}{\sqrt{2\pi}\, \sigma_x\sigma_y} e^{-\frac{1}{2}\left(\frac{(x-\mu_x)^2}{\sigma_x^2}+\frac{(y-\mu_y)^2}{\sigma_y^2}\right)} \left(\frac{x-\mu_x}{\sigma_x^2}, \frac{y-\mu_y}{\sigma_y^2}\right)$.\\
El gradiente se anula en $(x=\mu_x, y=\mu_y)$.\\
$\lim_{x\rightarrow \infty}f(x,y) = 0$.
}
%RESOLUCIÓN
{
}


\newproblem{par-10}{gen}{*}
%ENUNCIADO
{La ecuación diferencial parcial
\[
\displaystyle{\frac{\partial^2 u}{\partial x^2}} + \ \displaystyle{\frac{\partial^2 u}{\partial y^2}} + \displaystyle{\frac{\partial^2 u}{\partial z^2}} = 0,
\]
se conoce como ecuación de Laplace se aplica a multitud de fenómenos relacionadas con conducción de calor, flujo de fluidos y potencial eléctrico.

Dada la función $u(x,y,z)=\dfrac{1}{ \sqrt{x^2 + y^2 + z^2}},$
\begin{enumerate}
\item Comprobar que $f$ satisface la ecuación de Laplace.
\item ¿Existe algún punto en el que el crecimiento de la función sea nulo?
\item Si fijamos $z=1$, calcular
\[
\frac{\partial^4u}{\partial x^2\partial y^2}.
\]
\end{enumerate}
}
%SOLUCIÓN
{
\begin{enumerate}[start=2]
\item No hay ningún punto donde se el crecimiento es nulo.
\item $\frac{{\partial ^4 u}}{{\partial x^2 \partial y^2 }} =3\left( {x^2  + y^2  + 1} \right)^{ - 5/2}  - 15\left( {x^2  + y^2
} \right)\left( {x^2  + y^2 + 1} \right)^{ - 7/2}  + 105x^2 y\left({x^2  + y^2  + 1} \right)^{ - 9/2}$.
\end{enumerate}
}
%RESOLUCIÓN
{\begin{enumerate}
\item Para comprobar que $u(x,y,z)$ satisface la ecuación de Laplace
calculamos las tres derivadas parciales segundas que intervienen en
la ecuación. Comenzando con las derivadas parciales con respecto a
la variable $x$, obtenemos:
\[
u(x,y,z) = \frac{1}{{\sqrt {x^2  + y^2  + z^2 } }} = \left( {x^2  +
y^2  + z^2 } \right)^{ - 1/2}
\]
\[
\frac{{\partial u}}{{\partial x}} =  - \frac{1}{2}\left( {x^2  + y^2
+ z^2 } \right)^{ - 3/2} 2x =  - x\left( {x^2  + y^2  + z^2 }
\right)^{ - 3/2}
\]
\[
\frac{{\partial ^2 u}}{{\partial x^2 }} = \frac{\partial }{{\partial
x}}\left( { - x\left( {x^2  + y^2  + z^2 } \right)^{ - 3/2} }
\right) =  - \left( {x^2  + y^2  + z^2 } \right)^{ - 3/2}  + 3x^2
\left( {x^2  + y^2  + z^2 } \right)^{ - 5/2}
\]
e igualmente para las variables $y$ y $z$, tenemos:
\[
\frac{{\partial u}}{{\partial y}} =  - y\left( {x^2  + y^2  + z^2 }
\right)^{ - 3/2}
\]
\[
\frac{{\partial ^2 u}}{{\partial y^2 }} =  - \left( {x^2  + y^2  +
z^2 } \right)^{ - 3/2}  + 3y^2 \left( {x^2  + y^2  + z^2 } \right)^{
- 5/2}
\]
\[
\frac{{\partial u}}{{\partial z}} =  - z\left( {x^2  + y^2  + z^2 }
\right)^{ - 3/2}
\]
\[
\frac{{\partial ^2 u}}{{\partial z^2 }} =  - \left( {x^2  + y^2  +
z^2 } \right)^{ - 3/2}  + 3z^2 \left( {x^2  + y^2  + z^2 } \right)^{
- 5/2}
\]
Por lo tanto:
\[
\frac{{\partial ^2 u}}{{\partial x^2 }} + \frac{{\partial ^2
u}}{{\partial y^2 }} + \frac{{\partial ^2 u}}{{\partial z^2 }} =  -
3\left( {x^2  + y^2  + z^2 } \right)^{ - 3/2}  + 3\left( {x^2  + y^2
+ z^2 } \right)\left( {x^2  + y^2  + z^2 } \right)^{ - 5/2}  =
\]
\[
=- 3\left( {x^2  + y^2  + z^2 } \right)^{ - 3/2}  + 3\left( {x^2  +
y^2 + z^2 } \right)^{ - 3/2}  = 0
\]

\item Una condición necesaria para que el crecimiento de una función
de varias variables en un punto sea nulo es que el gradiente en
dicho punto se anule, y el gradiente se anula si se anulan sus tres
componentes:
\[
\vec \nabla u = \vec 0 \Leftrightarrow \left( {\frac{{\partial
u}}{{\partial x}},\frac{{\partial u}}{{\partial y}},\frac{{\partial
u}}{{\partial z}}} \right) = \left( {0,0,0} \right)
\]
Por lo tanto, tenemos un sistema no lineal de tres ecuaciones con
tres incógnitas:
\[
 - x\left( {x^2  + y^2  + z^2 } \right)^{ - 3/2}  = 0
\]
\[
 - y\left( {x^2  + y^2  + z^2 } \right)^{ - 3/2}  = 0
\]
\[
 - z\left( {x^2  + y^2  + z^2 } \right)^{ - 3/2}  = 0
\]
Y teniendo en cuenta que el término $(x^2+y^2+z^2)$, por tratarse de
una suma de cuadrados, únicamente puede ser 0 si $x=y=z=0$; y a
igual conclusión llegamos si suponemos que es distinto de 0, ya que
entonces la primera ecuación implica que necesariamente $x=0$, la
segunda implica que $y=0$, y la tercera implica que $z=0$. Por lo
tanto, concluimos que el único punto en el que el crecimiento puede
ser nulo es $(x,y,z)=(0,0,0)$, pero dicho punto no pertenece al
dominio de definición de la función (tendríamos un cero como
denominador de una fracción), por lo que no hay ningún punto en el
que la función presente un crecimiento nulo.

\item Suponiendo $z=1$, la función resultante presenta únicamente
dos variables:
\[
u(x,y,1) = \frac{1}{{\sqrt {x^2  + y^2  + 1} }} = \left( {x^2  + y^2
+ 1} \right)^{ - 1/2}
\]
La derivada propuesta es:
\[
\frac{{\partial ^4 u}}{{\partial x^2 \partial y^2 }} =
\frac{\partial }{{\partial x}}\left( {\frac{\partial }{{\partial
x}}\left( {\frac{\partial }{{\partial y}}\left( {\frac{{\partial
u}}{{\partial y}}} \right)} \right)} \right)
\]
en donde, como ya sabemos, se puede cambiar el orden de derivación
sin que afecte al resultado final, aunque nunca el número total de
derivadas con respecto a cada variable.

Operando como ya hicimos en los cálculos previos de las derivadas
segundas, obtenemos:
\[
\frac{{\partial u}}{{\partial y}} =  - y\left( {x^2  + y^2  + 1}
\right)^{ - 3/2}
\]
\[
\frac{\partial }{{\partial y}}\left( {\frac{{\partial u}}{{\partial
y}}} \right) = \frac{{\partial ^2 u}}{{\partial y^2 }} =  - \left(
{x^2  + y^2  + 1} \right)^{ - 3/2}  + 3y^2 \left( {x^2  + y^2  + 1}
\right)^{ - 5/2}
\]
\[
\frac{\partial }{{\partial x}}\left( {\frac{{\partial ^2
u}}{{\partial y^2 }}} \right) = \frac{{\partial ^3 u}}{{\partial
x\partial y^2 }} = 3x\left( {x^2  + y^2  + 1} \right)^{ - 5/2}  -
15y^2 x\left( {x^2  + y^2  + 1} \right)^{ - 7/2}
\]
\[
\frac{\partial }{{\partial x}}\left( {\frac{{\partial ^3
u}}{{\partial x\partial y^2 }}} \right) = \frac{{\partial ^4
u}}{{\partial x^2 \partial y^2 }} =
\]
\[
=3\left( {x^2  + y^2  + 1} \right)^{ - 5/2}  - 15\left( {x^2  + y^2
} \right)\left( {x^2  + y^2 + 1} \right)^{ - 7/2}  + 105x^2 y\left(
{x^2  + y^2  + 1} \right)^{ - 9/2}
\]
\end{enumerate}
}


\newproblem{par-11}{qui}{*}
%ENUNCIADO
{La siguiente función determina la temperatura en cada punto del plano real:
\[f(x,y)=e^{x+2y}\cos(x^2+y^2).\]
Se pide:
\begin{enumerate}
  \item Calcular el gradiente de $f$.
  \item Si estamos situados en el origen de coordenadas, ¿en qué dirección aumentará más rápidamente la temperatura? ¿Y si estuviésemos en el punto $(0,1)$?
\item Calcular la matriz Hessiana y el Hessiano de $f$ en el origen de coordenadas.
\end{enumerate}
}
%SOLUCIÓN
{\begin{enumerate}
\item $\nabla f(x,y) = e^{x+2y}\left(\cos(x^{2}+y^{2})-2x\sen(x^{2}+y^{2}), 2\cos(x^{2}+y^{2})-2y\sen(x^{2}+y^{2})\right)$.
\item $\nabla f(0,0) = (1,2)$ y $\nabla f(0,1) = (3.99\,,\,-4.45)$.
\item $Hf(0,0)=\left(
\begin{array}[]{cc}
1 & 2 \\
2 & 4
\end{array}
\right)
\quad |Hf(0,0)|= 0$.
\end{enumerate}
}
%RESOLUCIÓN
{\begin{enumerate}
\item Para calcular el vector gradiente de $f$ necesitamos calcular sus derivadas parciales de primer orden.
\begin{align*}
\frac{\partial}{\partial x}f(x,y) &= \frac{\partial}{\partial x}\left(e^{x+2y}\cos(x^{2}+y^{2})\right) = \frac{\partial}{\partial x}e^{x+2y}\cos(x^{2}+y^{2}) + e^{x+2y}\frac{\partial}{\partial x}\cos(x^{2}+y^{2}) = \\
&= e^{x+2y}\frac{\partial}{\partial x}(x+2y)\cos(x^{2}+y^{2})+e^{x+2y}(-\sen(x^{2}+y^{2})\frac{\partial}{\partial x}(x^{2}+y^{2}) =\\
&= e^{x+2y}\cos(x^{2}+y^{2})-e^{x+2y}\sen(x^{2}+y^{2})2x = e^{x+2y}(\cos(x^{2}+y^{2})-2x\sen(x^{2}+y^{2}),
\\
\frac{\partial}{\partial y}f(x,y) &= \frac{\partial}{\partial y}\left(e^{x+2y}\cos(x^{2}+y^{2})\right) = \frac{\partial}{\partial y}e^{x+2y}\cos(x^{2}+y^{2}) + e^{x+2y}\frac{\partial}{\partial y}\cos(x^{2}+y^{2}) = \\
&= e^{x+2y}\frac{\partial}{\partial y}(x+2y)\cos(x^{2}+y^{2})+e^{x+2y}(-\sen(x^{2}+y^{2})\frac{\partial}{\partial y}(x^{2}+y^{2}) =\\
&= e^{x+2y}\cos(x^{2}+y^{2})2-e^{x+2y}\sen(x^{2}+y^{2})2y= e^{x+2y}(2\cos(x^{2}+y^{2})-2y\sen(x^{2}+y^{2}),
\end{align*}
Así pues, el vector gradiente es
\begin{align*}
\nabla f(x,y) &= \left(\dfrac{\partial}{\partial x}f(x,y),\dfrac{\partial}{\partial y}f(x,y)\right) =\\
&= e^{x+2y}\left(\cos(x^{2}+y^{2})-2x\sen(x^{2}+y^{2}), 2\cos(x^{2}+y^{2})-2y\sen(x^{2}+y^{2})\right).
\end{align*}

\item La dirección en que más rápidamente aumenta la temperatura es la dirección del vector gradiente. Si estamos en el origen de coordenadas, dicha dirección es
\[
\nabla f(0,0) = e^{0+2\cdot 0}\left(\cos(0^{2}+0^{2})-2\cdot 0\sen(0^{2}+0^{2}), 2\cos(0^{2}+0^{2})-2\cdot 0\sen(0^{2}+0^{2})\right) = (1,2).
\]
Y si estamos en el punto $(0,1)$, la dirección de máximo crecimiento de la temperatura es
\begin{align*}
\nabla f(0,1) &= e^{0+2\cdot 1}\left(\cos(0^{2}+1^{2})-2\cdot 0\sen(0^{2}+1^{2}), 2\cos(0^{2}+1^{2})-2\cdot 1\sen(0^{2}+1^{2})\right) =\\
&= e^{2}(\cos 1, 2\cos 1-2\sen 1) = (3.99\,,\,-4.45).
\end{align*}

\item Para calcular la matriz Hessiana necesitamos calcular las derivadas parciales de segundo orden de $f$.
\begin{align*}
\frac{\partial^{2}}{\partial x^{2}}f(x,y) &= \frac{\partial}{\partial x}\left(\frac{\partial}{\partial x}f(x,y)\right) = \frac{\partial}{\partial x}\left(e^{x+2y}(\cos(x^{2}+y^{2})-2x\sen(x^{2}+y^{2})\right) = \\
&= \frac{\partial}{\partial x}e^{x+2y}(\cos(x^{2}+y^{2})-2x\sen(x^{2}+y^{2})+\\
&+ e^{x+2y}\frac{\partial}{\partial x}(\cos(x^{2}+y^{2})-2x\sen(x^{2}+y^{2}) = \\
&= e^{x+2y}(\cos(x^{2}+y^{2})-2x\sen(x^{2}+y^{2})+\\
&+ e^{x+2y}(-\sen(x^{2}+y^{2})2x-2\sen(x^{2}+y^{2})-2x\cos(x^{2}+y^{2}))2x = \\
&= e^{x+2y}((1-4x^{2})\cos(x^{2}+y^{2})-(4x+2)\sen(x^{2}+y^{2})),\\
\frac{\partial^{2}}{\partial y\partial x}f(x,y) &= \frac{\partial}{\partial y}\left(\frac{\partial}{\partial x}f(x,y)\right) = \frac{\partial}{\partial y}\left(e^{x+2y}(\cos(x^{2}+y^{2})-2x\sen(x^{2}+y^{2})\right) = \\
&= \frac{\partial}{\partial y}e^{x+2y}(\cos(x^{2}+y^{2})-2x\sen(x^{2}+y^{2})+\\
&+ e^{x+2y}\frac{\partial}{\partial y}(\cos(x^{2}+y^{2})-2x\sen(x^{2}+y^{2}) = \\
&= e^{x+2y}2(\cos(x^{2}+y^{2})-2x\sen(x^{2}+y^{2})+\\
&+ e^{x+2y}(-\sen(x^{2}+y^{2})2y-2x\cos(x^{2}+y^{2}))2y = \\
&= e^{x+2y}((2-4xy)\cos(x^{2}+y^{2})-(4x+2y)\sen(x^{2}+y^{2})),\\
\end{align*}
\begin{align*}
\frac{\partial^{2}}{\partial x\partial y}f(x,y) &= \frac{\partial^{2}}{\partial y\partial x}\quad \mbox{(Igualdad de derivadas cruzadas),}\\
%
\frac{\partial^{2}}{\partial y^{2}}f(x,y) &= \frac{\partial}{\partial y}\left(\frac{\partial}{\partial y}f(x,y)\right) = \frac{\partial}{\partial y}\left(e^{x+2y}(2\cos(x^{2}+y^{2})-2y\sen(x^{2}+y^{2})\right) = \\
&= \frac{\partial}{\partial y}e^{x+2y}(2\cos(x^{2}+y^{2})-2y\sen(x^{2}+y^{2})+\\
&+ e^{x+2y}\frac{\partial}{\partial y}(2\cos(x^{2}+y^{2})-2y\sen(x^{2}+y^{2}) = \\
&= e^{x+2y}2(2\cos(x^{2}+y^{2})-2y\sen(x^{2}+y^{2})+\\
&+ e^{x+2y}(-2\sen(x^{2}+y^{2})2y-2\sen(x^{2}+y^{2})-2y\cos(x^{2}+y^{2}))2y = \\
&= e^{x+2y}((4-4y^{2})\cos(x^{2}+y^{2})-(8y+2)\sen(x^{2}+y^{2})).
\end{align*}
\end{enumerate}
Así pues la matriz hessiana es
\[Hf(x,y)= \left(
\begin{array}{cc}
\frac{\partial^{2}}{\partial x^{2}}f(x,y) & \frac{\partial^{2}}{\partial x\partial y}f(x,y)\\
\frac{\partial^{2}}{\partial y\partial x}f(x,y) & \frac{\partial^{2}}{\partial y^{2}}f(x,y)
\end{array}
\right) =
\]
\[=
e^{x+2y} \left(
\begin{array}[]{cc}
(1-4x^{2})\cos(x^{2}+y^{2})-(4x+2)\sen(x^{2}+y^{2}) & (2-4xy)\cos(x^{2}+y^{2})-(4x+2y)\sen(x^{2}+y^{2})\\
(2-4xy)\cos(x^{2}+y^{2})-(4x+2y)\sen(x^{2}+y^{2}) & (4-4y^{2})\cos(x^{2}+y^{2})-(8y+2)\sen(x^{2}+y^{2})
\end{array}
\right)
\]
En el origen de coordenadas, la matriz Hessiana es
\[
Hf(0,0)=\left(
\begin{array}[]{cc}
1 & 2 \\
2 & 4
\end{array}
\right)
\]
y el hessiano vale
\[
|Hf(0,0)|=\left|
\begin{array}[]{cc}
1 & 2 \\
2 & 4
\end{array}
\right| =
4-4 = 0.
\]
}


\newproblem{par-12}{gen}{*}
%ENUNCIADO
{Se  dice que la función $z(t,x,y)$ satisface la ecuación de ondas si verifica la ecuación en derivadas parciales:
\[
\frac{{\partial ^2 z}} {{\partial t^2 }} = k^2 \left(
{\frac{{\partial ^2 z}} {{\partial x^2 }} + \frac{{\partial ^2 z}}
{{\partial y^2 }}} \right)
\]
para algún $k\in \mathbb{R}$.

Comprobar que la función:
\[
z\left( {t,x,y} \right) = \cos (ax)\sen(by)\sen\left( {kt\sqrt
{a^2 + b^2 } } \right)
\]
donde $a,b,k \in \mathbb{R}$, satisface la ecuación de ondas.
}
%SOLUCIÓN
{Si la satisface.
}
%	RESOLUCIÓN
{Para comprobar que $z(t,x,y)$ satisface la ecuación de ondas vamos a calcular primero las derivadas parciales de segundo orden que aparecen en dicha ecuación:
\begin{align*}
\frac{\partial^2 z}{\partial t^2} &=
\frac{\partial}{\partial t}\left(\frac{\partial z}{\partial t}\right) =
\frac{\partial}{\partial t}\left(\frac{\partial}{\partial t}\left(\cos(ax)\sen(by)\sen(kt\sqrt{a^2+b^2})\right)\right)= \\
&= \frac{\partial}{\partial t}\left(\cos(ax)\sen(by)\frac{\partial}{\partial t}\left(\sen(kt\sqrt{a^2+b^2})\right)\right)=\\
&= \frac{\partial}{\partial t}\left(\cos(ax)\sen(by)\cos(kt\sqrt{a^2+b^2})\frac{\partial}{\partial t}(kt\sqrt{a^2+b^2})\right)=\\
&= \frac{\partial}{\partial t}\left(\cos(ax)\sen(by)\cos(kt\sqrt{a^2+b^2}) k\sqrt{a^2+b^2}\right)=\\
&= k\sqrt{a^2+b^2}\cos(ax)\sen(by)\frac{\partial}{\partial t}\left(\cos(kt\sqrt{a^2+b^2}) \right)=\\
&= k\sqrt{a^2+b^2}\cos(ax)\sen(by)(-\sen(kt\sqrt{a^2+b^2}))\frac{\partial}{\partial t}\left(kt\sqrt{a^2+b^2}\right)=\\
&=k\sqrt{a^2+b^2}\cos(ax)\sen(by)(-\sen(kt\sqrt{a^2+b^2}))k\sqrt{a^2+b^2}=\\
&= -k^2(a^2+b^2)\cos(ax)\sen(by)\sen(kt\sqrt{a^2+b^2}),\\[.5cm]
\frac{\partial^2 z}{\partial x^2} &=
\frac{\partial}{\partial x}\left(\frac{\partial z}{\partial x}\right)
= \frac{\partial}{\partial x}\left(\frac{\partial}{\partial x}\left(\cos(ax)\sen(by)\sen(kt\sqrt{a^2+b^2})\right)\right)= \\
&= \frac{\partial}{\partial x}\left(\frac{\partial}{\partial x}\left(\cos(ax)\right)\sen(by)\sen(kt\sqrt{a^2+b^2})\right)=\\
&= \frac{\partial}{\partial x}\left(-\sen(ax)a\sen(by)\sen(kt\sqrt{a^2+b^2})\right)=\\
&= \frac{\partial}{\partial x}\left(-\sen(ax)\right)a\sen(by)\cos(kt\sqrt{a^2+b^2}) =\\
&= -a^2\cos(ax)\sen(by)\cos(kt\sqrt{a^2+b^2}),\\[.5cm]
\frac{\partial^2 z}{\partial y^2} &=
\frac{\partial}{\partial y}\left(\frac{\partial z}{\partial y}\right)
= \frac{\partial}{\partial y}\left(\frac{\partial}{\partial y}\left(\cos(ax)\sen(by)\sen(kt\sqrt{a^2+b^2})\right)\right)= \\
&= \frac{\partial}{\partial y}\left(\cos(ax)\frac{\partial}{\partial y}\left(\sen(by)\right)\sen(kt\sqrt{a^2+b^2})\right)=\\
&= \frac{\partial}{\partial y}\left(\cos(ax)\cos(by)b\sen(kt\sqrt{a^2+b^2})\right)=\\
&= \cos(ax)\frac{\partial}{\partial y}\left(\cos(by)\right)b\cos(kt\sqrt{a^2+b^2}) =\\
&= -b^2\cos(ax)\sen(by)\cos(kt\sqrt{a^2+b^2}).
\end{align*}
Para terminar, sustituimos estas derivadas en la ecuación de ondas y constatamos que efectivamente se cumple
\begin{align*}
& -k^2(a^2+b^2)\cos(ax)\sen(by)\sen(kt\sqrt{a^2+b^2}) =\\
&= k^2\left(-a^2\cos(ax)\sen(by)\cos(kt\sqrt{a^2+b^2})-b^2\cos(ax)\sen(by)\cos(kt\sqrt{a^2+b^2})\right).
\end{align*}
}


\newproblem{par-13}{gen}{*}
%ENUNCIADO
{Dadas las siguientes funciones de dos variables:
\[
\begin{array}{*{20}c}
   {f(x,y) = x^2  - 2xy^2  + \sen(xy)}  \\
   {g(x,y) = \left( {2x+ 3y^2 } \right)e^{\left( {1 - x^2  - y^2 } \right)} }  \\

 \end{array}
\]
\begin{enumerate}
\item Calcular el gradiente de cada una de ellas.
\item ¿A cuál de las funciones corresponde el siguiente dibujo del gradiente en los puntos $(1,0)$, $(0,1)$, $(-1,0)$ y $(0,-1)$?
\begin{center}
\includegraphics[scale=0.45,angle=270]{img/vectores-par-13}
\end{center}
\end{enumerate}
}
%SOLUCIÓN
{\begin{enumerate}
\item $\nabla f(x,y) = \left(2x-2y^2+\cos(xy)y\, ,\, -4xy+\cos(xy)x\right)$  y
$\nabla g(x,y) = \left((-4x^2-6xy^2+2)\, ,\, (-4xy-6y^3+6y)\right)e^{1-x^2-y^2}$.
\item Los vectores gradientes son de la función $f$.
\end{enumerate}
}
%RESOLUCIÓN
{\begin{enumerate}
\item Para calcular el gradiente necesitamos calcular las derivadas parciales de $f$ y $g$ con respecto a sus variables:
\begin{align*}
\frac{\partial f}{\partial x}(x,y) &= \frac{\partial}{\partial
x}\left(x^2- 2xy^2+\sen(xy)\right)=
2x-2y^2+\cos(xy)\frac{\partial}{\partial x}(xy)=2x-2y^2+\cos(xy)y,\\
\frac{\partial f}{\partial y}(x,y) &= \frac{\partial}{\partial
y}\left(x^2- 2xy^2+\sen(xy)\right)=
-4xy+\cos(xy)\frac{\partial}{\partial y}(xy)=-4xy+\cos(xy)x,\\
\frac{\partial g}{\partial x}(x,y) &= \frac{\partial}{\partial x}\left((2x+3y^2)e^{1-x^2-y^2}\right)=
\frac{\partial}{\partial x}(2x+3y^2)e^{1-x^2-y^2}+(2x+3y^2)\frac{\partial}{\partial x}e^{1-x^2-y^2}=\\
&= 2e^{1-x^2-y^2}+(2x+3y^2)e^{1-x^2-y^2}\frac{\partial}{\partial x}\left(1-x^2-y^2\right) = \\
&= 2e^{1-x^2-y^2}+(2x+3y^2)e^{1-x^2-y^2}(-2x)= (-4x^2-6xy^2+2)e^{1-x^2-y^2},\\
\frac{\partial g}{\partial y}(x,y) &= \frac{\partial}{\partial y}\left((2x+3y^2)e^{1-x^2-y^2}\right)=
\frac{\partial}{\partial x}(2x+3y^2)e^{1-x^2-y^2}+(2x+3y^2)\frac{\partial}{\partial y}e^{1-x^2-y^2}=\\
&= 6y e^{1-x^2-y^2}+(2x+3y^2)e^{1-x^2-y^2}\frac{\partial}{\partial y}\left(1-x^2-y^2\right) =\\
&=6ye^{1-x^2-y^2}+(2x+3y^2)e^{1-x^2-y^2}(-2y)= (-4xy-6y^3+6y)e^{1-x^2-y^2}.
\end{align*}
Así pues, los gradientes son
\begin{align*}
\nabla f(x,y) &=\left(\frac{\partial f}{\partial x}(x,y),\frac{\partial
f}{\partial y}(x,y)\right) = \left(2x-2y^2+\cos(xy)y\, ,\, -4xy+\cos(xy)x\right) \\
\nabla g(x,y) &=\left(\frac{\partial g}{\partial x}(x,y),\frac{\partial
g}{\partial y}(x,y)\right) = \left((-4x^2-6xy^2+2)\, ,\, (-4xy-6y^3+6y)\right)e^{1-x^2-y^2}
\end{align*}

\item Para ver a qué función corresponde la gráfica, calculamos el gradiente en los puntos que nos dan
\begin{align*}
\nabla f(1,0) &= \left(2\cdot 1-2\cdot 0^2+\cos(1\cdot 0)\cdot 0\, ,\, -4\cdot1\cdot0+\cos(1\cdot 0)\cdot1\right) =(2,1),\\
\nabla g(1,0) &= \left((-4\cdot1^2-6\cdot 1\cdot 0^2+2)\, ,\, (-4\cdot 1\cdot 0-6\cdot 0^3+6\cdot 0)\right)e^{1-1^2-0^2}= (-2,0).
\end{align*}
Como el vector libre situado en el punto $(1,0)$ es el $(2,1)$, la gráfica no puede pertenecer a la función $g(x,y)$. Para asegurarnos que se corresponde con la $f(x,y)$, calculamos el gradiente de esta función en el resto de los puntos:
\begin{align*}
\nabla f(0,1) &= \left(2\cdot 0-2\cdot 1^2+\cos(0\cdot 1)\cdot 1\, ,\, -4\cdot0\cdot1+\cos(0\cdot 1)\cdot0\right) =(-1,0),\\
\nabla f(-1,0) &= \left(2\cdot (-1)-2\cdot 0^2+\cos(-1\cdot 0)\cdot 0\, ,\, -4\cdot-1\cdot0+\cos(-1\cdot 0)\cdot(-1)\right) =(-2,-1),\\
\nabla f(0,-1) &= \left(2\cdot 0-2\cdot (-1)^2+\cos(0\cdot (-1))\cdot (-1)\, ,\, -4\cdot0\cdot(-1)+\cos(0\cdot (-1))\cdot0\right) =(-3,0).
\end{align*}
Luego los vectores de la gráfica se corresponden con los vectores gradientes de $f(x,y)$.
\end{enumerate}
}


\newproblem{par-14}{gen}{}
%ENUNCIADO
{Tenemos dos objetos de masas $m_1$ y $m_2$ unidas por una cuerda que pasa a través de una polea como la de la figura.
\begin{center}
  \includegraphics[scale=0.5]{img/polea-par-14}
\end{center}
Si $m_1\geq m_2$, la aceleración de $m_1$ viene dada por la ecuación
\[
a=\frac{m_1-m_2}{m_1+m_2}g,
\]
siendo $g$ la aceleración de la gravedad.
Demostrar que se cumple la ecuación
\[
m_1\frac{\partial a}{\partial m_1}+m_2\frac{\partial a}{\partial m_2}=0.
\]
}
%SOLUCIÓN
{$\dfrac{\partial a}{\partial m_1} = \dfrac{2gm_2}{(m_1+m_2)^2}$ y $\dfrac{\partial a}{\partial m_2} = \dfrac{-2gm_1}{(m_1+m_2)^2}$.
}
%RESOLUCIÓN
{
}


\newproblem{par-15}{gen}{*}
%ENUNCIADO
{La relación que modeliza el potencial eléctrico $V$ de un punto del plano en función de su distancia, es $V=\log D$, donde $D=\sqrt{x^2+y^2}$.

Se pide:
\begin{enumerate}
\item Calcular el gradiente de $V$.
\item Hallar la dirección de máxima variación del potencial
eléctrico en el punto $(x,y)=(\sqrt{3},\sqrt{3})$.
\item Calcular la matriz Hessiana y el Hessiano de $V$ en el punto anterior.
\item Si nos movemos a lo largo de la curva $y=x+1$, cuál será el mínimo potencial alcanzado?
\end{enumerate}
}
%SOLUCIÓN
{\begin{enumerate}
\item $\nabla V(x,y) = \left( \frac{x}{x^2+y^2},\frac{y}{x^2+y^2}\right)$.
\item $\nabla V(\sqrt 3, \sqrt 3) = \sqrt 3 /6(1,1)$.
\item $
HV(x,y) = \left(
\begin{array}{cc}
\frac{y^2-x^2}{y^4+2x^2y^2+x^4} & \frac{-2xy}{y^4+2x^2y^2+x^4} \\
\frac{-2xy}{y^4+2x^2y^2+x^4} & \frac{x^2-y^2}{y^4+2x^2y^2+x^4}
\end{array}
\right),\quad
\left(
\begin{array}{cc}
0 & -1/6 \\
-1/6 & 0
\end{array}
\right),\quad \mbox{y }
|H(\sqrt 3,\sqrt 3)| = -1/36.
$
\item El potencial máximo se alcanza en $(x=-1/2, y=1/2)$ y vale $V(-1/2,1/2) = -\dfrac{\log 2}{2}$.
\end{enumerate}
}
%RESOLUCIÓN
{
}


\newproblem*{par-16}{qui}{}
%ENUNCIADO
{La ecuación unidimensional del calor es
\[
\frac{\partial q}{\partial t}=c^2\frac{\partial^2q}{\partial x^2},
\]
donde $c$ es una constante y $q(x,t)$ es la temperatura de una varilla en un punto que ocupa la posición $x$ en el instante $t$. Demostrar que $q(x,t)=e^{ax+bt}$, con $a\neq 0$, satisface dicha ecuación para un valor apropiado de $c$.
}


\newproblem*{par-17}{amb}{*}
%ENUNCIADO
{Suponiendo que la temperatura, $T$ en ºC, de una zona de la atmósfera es función de la densidad del aire, $d$, en g por cm$^3$, la altura, $h$, en kilómetros, y de la concentración de un determinado elemento, $c$, en mg por cm$^3$, viene dada por la expresión:
\[
T(d,h,c) = \frac{{\ln (dh)}}{c} + c^2 3^{hd}
\]
\begin{enumerate}
\item Suponiendo que la altura a la que medimos la temperatura es de un kilómetro, y que la temperatura medida es de 0 ºC, dar la expresión de la concentración en función de la densidad.
\item Calcular el gradiente de la temperatura en el punto $(d_0,h_0,c_0)=(1,1,2)$.
\item Comprobar que se cumple que:
\[
\frac{{\partial ^2 T}}{{\partial d\partial h}} = \frac{{\partial ^2
T}}{{\partial h\partial d}}
\]
\end{enumerate}
}


\newproblem{par-18}{gen}{}
%ENUNCIADO
{Sea $z(x,y)=\dfrac{x^{2}}{y}+\dfrac{y^{2}}{x}.$ Calcular todas sus derivadas parciales de primer y segundo orden.
}
%SOLUCIÓN
{$\frac{\partial z}{\partial x} = \frac{2x}{y}-\frac{y^2}{x^2}$, $\dfrac{\partial z}{\partial x} = \frac{2y}{x}-\frac{x^2}{y^2}$,\\
$\frac{\partial^2 z}{\partial x^2} = \frac{2y^2}{x^3}+\frac{2}{y}$, $\frac{\partial^2 z}{\partial y\partial x} = -\frac{2y}{x^2}-\frac{2x}{y^2}$, $\frac{\partial^2 z}{\partial x\partial y} = -\frac{2y}{x^2}-\frac{2x}{y^2}$, $\frac{\partial^2 z}{\partial x^2} = \frac{2x^2}{y^3}+\frac{2}{x}$.
}
%RESOLUCIÓN
{
}


\newproblem*{par-19}{gen}{}
%ENUNCIADO
{Dada la función $f(x,y)=\dfrac{x-y}{x+y}$, hallar $\dfrac{\partial f}{\partial x}$ y $\dfrac{\partial f}{\partial y}$ en el punto $(2,-1)$.
}


\newproblem*{par-20}{amb}{*}
%ENUNCIADO
{Supongamos la función de varias variables $f(x,y,z)=x^{3}+\sqrt{xyz}$ que da la presión en un recipiente en función de la posición $(x,y,z)$. Suponiendo que en el recipiente hay un insecto y que se encuentra en el punto de coordenadas $(2,1,3)$, ¿en qué dirección debe moverse si busca ir lo más rápidamente posible hacia zonas de menor presión?
}


\newproblem*{par-21}{gen}{}
%ENUNCIADO
{Dado el siguiente campo escalar expresado en coordenadas cartesianas:
\[
f(x,y,z)=3xy\ln \left( \dfrac{1}{z}\right)
\]
Calcular:
\begin{enumerate}
\item  Su vector gradiente.
\item  Su matriz Hessiana.
\end{enumerate}
}


\newproblem*{par-22}{gen}{*}
%ENUNCIADO
{La definición del polinomio de Taylor de grado 2 de una función de dos variables, $f(x,y)$, centrado en el punto $(x_{0},y_{0})$, es
\begin{align*}
P_{f,(x_0,y_0)}^{2}(x,y)&= f(x_{0},y_{0})+\dfrac{\partial f(x_{0},y_{0})}{\partial x}(x-x_{0})+\dfrac{\partial f(x_{0},y_{0})}{\partial y}(y-y_{0})+\\
&+\dfrac{1}{2}\dfrac{\partial ^{2}f(x_{0},y_{0})}{\partial x^{2}}(x-x_{0})^{2}+\dfrac{1}{2}\dfrac{\partial ^{2}f(x_{0},y_{0})}{\partial y^{2}}(y-y_{0})^{2}+\dfrac{\partial ^{2}f(x_{0},y_{0})}{\partial x\partial y}(x-x_{0})(y-y_{0})
\end{align*}
\begin{enumerate}
\item  Utilizar la fórmula anterior para calcular el polinomio de Taylor de grado 2 de la función $f(x,y)=e^{(x+2y)}$ centrado en $(x_{0},y_{0})=(0,0)$.
\item  Utilizar el polinomio anterior para dar el valor aproximado de $e^{(0.1+2\cdot 0.1)}$.
\end{enumerate}
}


\newproblem*{par-23}{fis}{*}
%ENUNCIADO
{Suponiendo que el potencial eléctrico en un punto de coordenadas cartesianas $(x,y,z)$ viene dado por:
\[
V(x,y,z) = \frac{1} {{x{\kern 1pt} e^y \ln z}},
\]
calcular en el punto $(1,0,e)$:
\begin{enumerate}
\item El campo eléctrico (recordar que el campo eléctrico es el gradiente del potencial cambiado de signo: $\vec E =  - \vec\nabla V$).
\item La divergencia del campo eléctrico.
\end{enumerate}
}


\newproblem*{par-24}{gen}{*}
%ENUNCIADO
{Para la función de 2 variables $f(x,y) = x^{y^2}$
\begin{enumerate}
\item Calcular su dirección y sentido de máximo crecimiento en el punto $(1,1)$.
\item Calcular su matriz Hessiana.
\end{enumerate}
}


\newproblem{par-25}{amb}{*}
%ENUNCIADO
{La Quimiotaxis es el movimiento de los organismos dirigido por un gradiente de concentración, es decir, en la dirección
en la que la concentración aumenta con más rapidez. El moho del cieno Dictyoselium discoideum muestra este
comportamiento. En esta caso, las amebas unicelulares de esta especie se mueven según el gradiente de concentración de
una sustancia química denominada adenosina monofosfato (AMP cíclico). Si suponemos que la expresión que da la
concentración de AMP cíclico en un punto de coordenadas $(x,y,z)$ es:
\[
C(x,y,z) = \frac{4} {{\sqrt {x^2  + y^2  + z^4  + 1} }}
\]
y se sitúa una ameba de moho del cieno en el punto $(-1,0,1)$, ¿en qué dirección se moverá la ameba?
}
%SOLUCIÓN
{$(4/\sqrt{27}, 0, -8/\sqrt{27})$.
}
%RESOLUCIÓN
{
}



%%%%%%% Pendiente 26



\newproblem{par-27}{qui}{*}
%ENUNCIADO
{Supongamos que tenemos una superficie plana, y que la cantidad de una sustancia, $C$ en g/cm$^2$,
depositada sobre cada punto de coordenadas $x$ e $y$, en metros, es función del punto y del tiempo $t$, en horas, y
viene dada por la expresión:
\[
C(x,y,t) = \sqrt{e^{-\frac{3ty}{x^2+1}}}
\]
\begin{enumerate}
\item Calcular la dirección y sentido de máximo crecimiento de la
función en el punto $(x_0,y_0,t_0)=(1,0,1)$.
\item Calcular: $\dfrac{{\partial ^2 C}}{{\partial y\partial x}}$.
\item ¿En qué puntos se anulará el gradiente de $C$?
\end{enumerate}
}
%SOLUCIÓN
{\begin{enumerate}
\item $\nabla C(1,0,1) =\frac{1}{4}(0,-3,0)$.
\item $\displaystyle \frac{\partial^2 C}{\partial y\partial x} =
\frac{e^{-\frac{3ty}{2x^2+2}}}{(2x^2+2)^2}\left(\frac{-36t^2yx}{2x^2+2}+12tx\right)$.
\item En los puntos de la forma $(x,0,0)\ \forall x\in \mathbb{R}$.
\end{enumerate}
}
%RESOLUCIÓN
{Antes de nada conviene simplificar la función:
\[
C(x,y,t) = \sqrt{e^{-\frac{3ty}{x^2+1}}} = \left(e^{-\frac{3ty}{x^2+1}}\right)^{1/2} = e^{-\frac{3ty}{2x^2+2}}
\]
\begin{enumerate}
\item La dirección y sentido de máximo crecimiento de una función de varias variables la da el vector gradiente, en este caso,
\[
\nabla C(x,y,t) =\left(\frac{\partial C}{\partial x}, \frac{\partial C}{\partial y}, \frac{\partial C}{\partial t} \right)
\]
Calulamos las tres derivadas parciales:
\begin{align*}
\frac{\partial C}{\partial x} &= \frac{\partial}{\partial x} e^{-\frac{3ty}{2x^2+2}} = e^{-\frac{3ty}{2x^2+2}} \frac{\partial}{\partial x}\left(-\frac{3ty}{2x^2+2}\right) = e^{-\frac{3ty}{2x^2+2}}\frac{3ty\cdot 4x}{(2x^2+2)^2} \\
\frac{\partial C}{\partial y} &= \frac{\partial}{\partial y} e^{-\frac{3ty}{2x^2+2}} = e^{-\frac{3ty}{2x^2+2}} \frac{\partial}{\partial y}\left(-\frac{3ty}{2x^2+2}\right) = e^{-\frac{3ty}{2x^2+2}}\frac{-3t}{2x^2+2} \\
\frac{\partial C}{\partial t} &= \frac{\partial}{\partial t} e^{-\frac{3ty}{2x^2+2}} = e^{-\frac{3ty}{2x^2+2}} \frac{\partial}{\partial t}\left(-\frac{3ty}{2x^2+2}\right) = e^{-\frac{3ty}{2x^2+2}}\frac{-3y}{2x^2+2}
\end{align*}
De modo que el vector gradiente es
\[
\nabla C(x,y,t) =\frac{e^{-\frac{3ty}{2x^2+2}}}{2x^2+2}\left(\frac{12tyx}{2x^2+2}, -3t, -3y\right),
\]
y en el punto $(x_0,y_0,t_0)=(1,0,1)$ vale
\[
\nabla C(1,0,1) =\frac{e^{-\frac{3\cdot 1\cdot 0}{2\cdot 1^2+2}}}{2\cdot 1^2+2}\left(\frac{12\cdot 1\cdot 0\cdot 1}{2\cdot 1^2+2}, -3\cdot 1, -3\cdot 0\right) = \frac{1}{4}(0,-3,0).
\]

\item
\begin{align*}
\frac{\partial^2 C}{\partial y\partial x} &= \frac{\partial}{\partial y}\frac{\partial C}{\partial x} e^{-\frac{3ty}{2x^2+2}} = \frac{\partial}{\partial y}  \left(e^{-\frac{3ty}{2x^2+2}}\frac{12tyx}{(2x^2+2)^2}\right) = \\
&= \frac{\partial}{\partial y} \left(e^{-\frac{3ty}{2x^2+2}}\right)\frac{12tyx}{(2x^2+2)^2}+e^{-\frac{3ty}{2x^2+2}}\frac{\partial}{\partial y}\left(\frac{12tyx}{(2x^2+2)^2}\right) = \\
&= e^{-\frac{3ty}{2x^2+2}}\frac{\partial}{\partial y}\left(-\frac{3ty}{2x^2+2}\right)\frac{3ty\cdot 4x}{(2x^2+2)^2}+e^{-\frac{3ty}{2x^2+2}}\frac{12tx}{(2x^2+2)^2} = \\
&= e^{-\frac{3ty}{2x^2+2}}\frac{-3t}{2x^2+2}\frac{3ty\cdot 4x}{(2x^2+2)^2}+e^{-\frac{3ty}{2x^2+2}}\frac{12tx}{(2x^2+2)^2} = \\
&= \frac{e^{-\frac{3ty}{2x^2+2}}}{(2x^2+2)^2}\left(\frac{-36t^2yx}{2x^2+2}+12tx\right).
\end{align*}

\item
\[
\nabla C(x,y,t) =\frac{e^{-\frac{3ty}{2x^2+2}}}{2x^2+2}\left(\frac{12tyx}{2x^2+2}, -3t, -3y\right) = (0,0,0) \Leftrightarrow
\left\{
\begin{array}{l}
12txy =0 \\
-3t = 0\\
-3y = 0
\end{array}
\right.
\]
de donde se deduce que $t=0$, $y=0$ y $x$ puede tomar cualquier valor. Así pues, los puntos que anulan el gradiente son de la forma $(x,0,0)$, $x\in\mathbb{R}$.
\end{enumerate}
}


\newproblem{par-28}{fis}{*}
%ENUNCIADO
{Una barra de metal de un metro de largo se calienta de manera irregular y de forma tal que a $x$ metros de su extremo izquierdo y en el instante $t$ minutos, su temperatura en grados centígrados esta dada por $H(x,t) = 100e^{-0.1t}\sen(\pi xt)$ con $0\leq x \leq 1$.
\begin{enumerate}
\item Calcular $\dfrac{\partial H}{\partial x}(0.2, 1)$ y $\dfrac{\partial H}{\partial x}(0.8, 1).$ ¿Cuál es la interpretación práctica (en términos de temperatura) de estas derivadas parciales? Explicar por qué cada una tiene el signo que tiene.
\item Calcular la matriz hessiana de $H$.
\end{enumerate}
}
%SOLUCIÓN
{\begin{enumerate}
\item $\frac{\partial H}{\partial x}(0.2,\, 1) = 100e^{-0.1}\cos(0.2\pi) \pi = 229.9736$ \\
$\frac{\partial H}{\partial x}(0.8,\, 1) = 100e^{-0.1}\cos(0.8\pi) \pi = -229.9736$.
\item $
\left(
\begin{array}{cc}
-100e^{-0.1t}\pi^2 t^2\sen(\pi xt) & 100e^{-0.1t}\left((-0.1\pi t+\pi)\cos(\pi xt) - \pi^2 xt \sen(\pi xt)\right) \\
100e^{-0.1t}\left((-0.1\pi t+\pi)\cos(\pi xt) - \pi^2 xt \sen(\pi xt)\right) & 100e^{-0.1t}\left(0.01\sen(\pi xt) -(0.2+\pi^2x^2) \cos(\pi xt)\right)
\end{array}
\right)$
\end{enumerate}
}
%RESOLUCIÓN
{\begin{enumerate}
\item La derivada parcial de $H$ con respecto a $x$ es
\begin{align*}
\frac{\partial H}{\partial x}(x,t) &= 100e^{-0.1t}\cos(\pi xt) \pi t \\
\end{align*}
y en los puntos que nos piden vale
\begin{align*}
\frac{\partial H}{\partial x}(0.2,\, 1) &= 100e^{-0.1}\cos(0.2\pi) \pi =
229.9736\\
\frac{\partial H}{\partial x}(0.8,\, 1) &= 100e^{-0.1}\cos(0.8\pi) \pi =
-229.9736
\end{align*}
La derivada parcial $\dfrac{\partial H}{\partial x}(x_0,t_0)$ indica la variación instantánea que experimenta la temperatura con respecto a la variación de la distancia al extremo izquierdo en el punto. El signo de la derviada parcial indica si la variación de la temperatura es creciente (aumenta la temperatura) o decreciente (disminuye). Así en el punto $(0.2,\, 1)$ la temperatura aumentará a razón de $229.9736$ grados centígrados por cada metro que nos alejemos del extremo izquierdo de la barra de metal, mientras que en el $(0.8,\,1)$ la temperatura disminuirá a razón de $229.9736$ grados centígrados por cada metro que nos alejemos del extremo izquierdo de la barra de metal.

\item Para calcular la matriz Hessiana necesitamos las derivadas parciales de
segundo orden:
\begin{align*}
\frac{\partial H}{\partial t} (x,t) &= 100\left(\frac{\partial}{\partial x} e^{-0.1 t} \sen (\pi xt) + e^{-0.1t}\frac{\partial}{\partial x}\sen(\pi xt)\right)=\\
&= 100\left(-0.1e^{-0.1t}\sen(\pi xt) +e^{-0.1t}\cos(\pi xt)\pi x\right) =\\
&= 100 e^{-0.1t}\left(-0.1 \sen(\pi xt) + \pi x \cos(\pi xt)\right),\\
\frac{\partial^2 H}{\partial x^2}(x,t) &= \frac{\partial}{\partial x}\left(100e^{-0.1t}\pi t\cos(\pi xt) \right) = 100e^{-0.1t}\pi t(-\sen(\pi xt) \pi t) =\\
&= -100e^{-0.1t}\pi^2 t^2\sen(\pi xt),\\
\frac{\partial^2 H}{\partial t\partial x}(x,t) &= \frac{\partial}{\partial t}\left(100e^{-0.1t}\pi t\cos(\pi xt) \right) =\\
&= 100\left(\frac{\partial}{\partial t}e^{-0.1t}\pi t\cos(\pi xt) + e^{-0.1t}\left(\frac{\partial}{\partial t}(\pi t)\cos(\pi xt) + \pi t \frac{\partial}{\partial t}\cos(\pi xt)\right) \right) =\\
&= 100\left(-0.1e^{-0.1t}\pi t\cos(\pi xt) + e^{-0.1t}\left(\pi \cos(\pi xt) - \pi t \sen(\pi xt)\pi x\right) \right) =\\
&= 100e^{-0.1t}\left(-0.1\pi t\cos(\pi xt)+\pi \cos(\pi xt) - \pi^2 xt \sen(\pi xt)\right) = \\
&= 100e^{-0.1t}\left((-0.1\pi t+\pi)\cos(\pi xt) - \pi^2 xt \sen(\pi xt)\right),\\
\end{align*}

\begin{align*}
\frac{\partial^2 H}{\partial x\partial t}(x,t) &= \frac{\partial^2 H}{\partial t\partial x}(x,t) \quad (\mbox{igualdad de las derivadas cruzadas por el teorema de Schwartz})\\
\frac{\partial^2 H}{\partial t^2}(x,t) &= \frac{\partial}{\partial t} \left(100 e^{-0.1t}\left(-0.1 \sen(\pi xt) + \pi x \cos(\pi xt)\right)\right)  =\\
&= 100\left(\frac{\partial}{\partial t} e^{-0.1t}\left(-0.1 \sen(\pi xt) + \pi x \cos(\pi xt)\right) +\right.\\
&\left. \qquad + e^{0.1t}\left(\frac{\partial}{\partial t}\left(-0.1\sen(\pi xt)\right) + \frac{\partial}{\partial t}\left(\pi x \cos(\pi xt)\right)\right)\right) =\\
&= 100\left(-0.1 e^{-0.1t}\left(-0.1 \sen(\pi xt) + \pi x \cos(\pi xt)\right)\right. +\\
&\left. \qquad + e^{0.1t}\left(-0.1\cos(\pi xt)\pi x - \pi x \cos(\pi xt)\pi x\right)\right) =\\
&= 100e^{-0.1t}\left(0.01\sen(\pi xt) -0.1 \pi x \cos(\pi xt) -0.1\pi x\cos(\pi xt) - \pi^2 x^2 \cos(\pi xt)\right) = \\
&= 100e^{-0.1t}\left(0.01\sen(\pi xt) -(0.2+\pi^2x^2) \cos(\pi xt)\right).
\end{align*}
Así pues, la matriz Hessiana es
\[
\left(
\begin{array}{cc}
-100e^{-0.1t}\pi^2 t^2\sen(\pi xt) & 100e^{-0.1t}\left((-0.1\pi t+\pi)\cos(\pi xt) - \pi^2 xt \sen(\pi xt)\right) \\
100e^{-0.1t}\left((-0.1\pi t+\pi)\cos(\pi xt) - \pi^2 xt \sen(\pi xt)\right) & 100e^{-0.1t}\left(0.01\sen(\pi xt) -(0.2+\pi^2x^2) \cos(\pi xt)\right)
\end{array}
\right)
\]
\end{enumerate}
}


\newproblem{par-29}{gen}{*}
%ENUNCIADO
{Dar la dirección de máximo crecimiento de la función
\[
f(x,y,z) = \frac{\log(zx)}z-xe^{-zxy}
\]
en el punto $(1,1,1)$.
}
%SOLUCIÓN
{$\nabla f(1,1,1)=(1,e^{-1},1+e^{-1})$.
}
%RESOLUCIÓN
{La dirección de máximo crecimiento de una función de varias variables la da el vector gradiente:
\[
\nabla f(x,y,z) = \left(\frac{\partial f}{\partial x}(x,y,z),\frac{\partial f}{\partial y}(x,y,z),\frac{\partial f}{\partial z}(x,y,z)\right)
\]
Calculamos por tanto cada una de las derivadas parciales que aparecen en las componentes del vector:
\begin{align*}
\frac{\partial f}{\partial x}(x,y,z) &= \frac{\partial}{\partial x}(\frac{\log (zx)}z-xe^{-zxy}) = \frac{\partial}{\partial x}(\frac{\log (zx)}z)-\frac{\partial}{\partial x}(xe^{-zxy})= \\
&= \frac{1}{z}\frac{\partial}{\partial x}(\log (zx))-(\frac{\partial}{\partial x}(x)e^{-zxy}+x\frac{\partial}{\partial x}(e^{-zxy}))= \\
&= \frac{1}{z}\frac{1}{zx}\frac{\partial}{\partial x}(zx)-(e^{-zxy}+xe^{-zxy}\frac{\partial}{\partial x}(-zxy))= \\
&= \frac{1}{z}\frac{1}{zx}z-(e^{-zxy}+xe^{-zxy}(-zy)) = \frac{1}{zx}-e^{-zxy}(1-xyz),\\
\frac{\partial f}{\partial y}(x,y,z) &= \frac{\partial}{\partial y}(\frac{\log(zx)}z-xe^{-zxy}) = \frac{\partial}{\partial y}(\frac{\log (zx)}z)-\frac{\partial}{\partial y}(xe^{-zxy})= \\
&= -x\frac{\partial}{\partial y}(e^{-zxy}) = -xe^{-zxy}\frac{\partial}{\partial y}(-zxy)=x^2ze^{-zxy},\\
\frac{\partial f}{\partial z}(x,y,z) &= \frac{\partial}{\partial z}(\frac{\log(zx)}z-xe^{-zxy}) = \frac{\partial}{\partial z}(\frac{\log (zx)}z)-\frac{\partial}{\partial z}(xe^{-zxy})= \\
&= \frac{\frac{\partial}{\partial z}(\log (zx))z-\log (zx)\frac{\partial}{\partial z}(z)}{z^2}-x\frac{\partial}{\partial z}(e^{-zxy}))= \\
&= \frac{\frac 1{zx}\frac \partial {\partial z}(zx)z-\log (zx)}{z^2}-xe^{-zxy}\frac{\partial}{\partial z}(-zxy))= \\
&= \frac{\frac 1{zx}xz-\log (zx)}{z^2}-xe^{-zxy}-xy=\frac{1-\log (zx)}{z^2}+x^2ye^{-zxy}.
\end{align*}
Por lo tanto, el vector gradiente será:
\[
\nabla f(x,y,z)=(\frac{1}{zx}-e^{-zxy}(1-xyz), x^2ze^{-zxy}, \frac{1-\log (zx)}{z^2}+x^2ye^{-zxy})
\]

Finalmente, como nos pieden la dirección de máximo crecimiento en el punto $(1,1,1)$, tendremos que particularizar el vector gradiente en dicho punto, es decir:
\[
\nabla f(1,1,1)=(1,e^{-1},1+e^{-1}).
\]
}


\newproblem{par-30}{gen}{*}
%ENUNCIADO
{Calcular el gradiente de la función
\[
f(x,y,z)=e^{\sqrt{x^2+2yz}}+\ln (\frac{xy}z)
\]
en el punto $(1,-2,-2)$.
}
%SOLUCIÓN
{$\nabla f(x,y,z)=\left(\frac{xe^{\sqrt{x^2+2yz}}}{\sqrt{x^2+2yz}}+\frac{1}{x}, \frac{ze^{\sqrt{x^2+2yz}}}{\sqrt{x^2+2yz}}+\frac{1}{y}, \frac{ye^{\sqrt{x^2+2yz}}}{\sqrt{x^2+2yz}}-\frac{1}{z}\right)$\\ y $\nabla f(1,-2,-2)=\left(\frac{e^3}{3}+1,\frac{-2e^3}{3}-\frac{1}{2},\frac{-2e^3}{3}+\frac{1}{2}\right)$.}
%RESOLUCIÓN
{El gradiente de $f(x,y,z)$ se define como el vector $\nabla f(x,y,z)=\left(\dfrac{\partial f}{\partial x}(x,y,z),\dfrac{\partial f}{\partial y}(x,y,z),\dfrac{\partial f}{\partial z}(x,y,z)\right).$ Por tanto, tenemos que calcular las tres derivadas parciales siguientes:
\begin{align*}
\dfrac{\partial f}{\partial x}(x,y,z) &= \dfrac{\partial}{\partial x}(e^{\sqrt{x^2+2yz}}+\ln (\frac{xy}z)) = \dfrac{\partial}{\partial x}(e^{\sqrt{x^2+2yz}})+\dfrac{\partial}{\partial x}(\ln (\frac{xy}z))= \\
&= e^{\sqrt{x^2+2yz}}\dfrac \partial {\partial x}(\sqrt{x^2+2yz})+\frac{1}{xy/z}\dfrac{\partial}{\partial x}(\frac{xy}z)= \\
&= e^{\sqrt{x^2+2yz}}\frac{1}{2\sqrt{x^2+2yz}}\dfrac{\partial}{\partial x}(x^2+2yz)+\frac{z}{xy}\frac{y}{z}= \\
&= e^{\sqrt{x^2+2yz}}\frac{1}{2\sqrt{x^2+2yz}}2x+\frac{1}{x} = \frac{xe^{\sqrt{x^2+2yz}}}{\sqrt{x^2+2yz}}+\frac{1}{x}, \\
\dfrac{\partial f}{\partial y}(x,y,z) &= \dfrac{\partial}{\partial y}(e^{\sqrt{x^2+2yz}}+\ln (\frac{xy}z)) = \dfrac{\partial}{\partial y}(e^{\sqrt{x^2+2yz}})+\dfrac{\partial}{\partial y}(\ln (\frac{xy}z))= \\
&= e^{\sqrt{x^2+2yz}}\dfrac{\partial}{\partial y}(\sqrt{x^2+2yz})+\frac{1}{xy/z}\dfrac{\partial}{\partial y}(\frac{xy}z)= \\
&= e^{\sqrt{x^2+2yz}}\frac{1}{2\sqrt{x^2+2yz}}\dfrac{\partial}{\partial y}(x^2+2yz)+\frac{z}{xy}\frac{x}{z}= \\
&= e^{\sqrt{x^2+2yz}}\frac{1}{2\sqrt{x^2+2yz}}2z+\frac{1}{y}=\frac{ze^{\sqrt{x^2+2yz}}}{\sqrt{x^2+2yz}}+\frac{1}{y}, \\
\dfrac{\partial f}{\partial z}(x,y,z) &= \dfrac{\partial}{\partial z}(e^{\sqrt{x^2+2yz}}+\ln (\frac{xy}z)) = \dfrac{\partial}{\partial z}(e^{\sqrt{x^2+2yz}})+\dfrac{\partial}{\partial z}(\ln (\frac{xy}{z}))= \\
&= e^{\sqrt{x^2+2yz}}\dfrac{\partial}{\partial z}(\sqrt{x^2+2yz})+\frac{1}{xy/z}\dfrac{\partial}{\partial z}(\frac{xy}{z})= \\
&= e^{\sqrt{x^2+2yz}}\frac{1}{2\sqrt{x^2+2yz}}\dfrac{\partial}{\partial z}(x^2+2yz)+\frac{z}{xy}\frac{-xy}{z^2}= \\
&= e^{\sqrt{x^2+2yz}}\frac{1}{2\sqrt{x^2+2yz}}2y-\frac{1}{z} = \frac{ye^{\sqrt{x^2+2yz}}}{\sqrt{x^2+2yz}}-\frac{1}{z},
\end{align*}
y, en consecuencia tenemos
\[
\nabla f(x,y,z)=\left(\frac{xe^{\sqrt{x^2+2yz}}}{\sqrt{x^2+2yz}}+\frac{1}{x}, \frac{ze^{\sqrt{x^2+2yz}}}{\sqrt{x^2+2yz}}+\frac{1}{y}, \frac{ye^{\sqrt{x^2+2yz}}}{\sqrt{x^2+2yz}}-\frac{1}{z}\right).
\]
Como nos piden el gradiente en el punto $(1,-2,-2),$ sustituimos $x$ por 1, $y$ por -2, y $z$ por -2 en el vector anterior y obtenemos
\[
\nabla f(1,-2,-2)=\left(\frac{e^3}{3}+1,\frac{-2e^3}{3}-\frac{1}{2},\frac{-2e^3}{3}+\frac{1}{2}\right).
\]
}


\newproblem{par-31}{gen}{*}
%ENUNCIADO
{Calcular el vector gradiente de la función
\[
\log \left( \sqrt{x^{2}-z^{2}}\right) +3^{\tfrac{x^{2}}{y}}
\]
en el punto $(1,1,0)$.
}
%SOLUCIÓN
{$\nabla f(x,y,z)=(\frac{x}{x^{2}-z^{2}}+3^{\tfrac{x^{2}}{y}}\log 3\dfrac{2x}{y},3^{\tfrac{x^{2}}{y}}\log 3\dfrac{-x^{2}}{y^{2}},-\frac{z}{x^{2}-z^{2}})$, y $\nabla f(1,-2,-2)=(1+6\log 3,-3\log 3,0)$.
}
%RESOLUCIÓN
{El gradiente de $f(x,y,z)$ se define como el vector $\nabla f(x,y,z)=(\dfrac{\partial f}{\partial x}(x,y,z),\dfrac{\partial f}{\partial y}(x,y,z),\dfrac{\partial f}{\partial z}(x,y,z))$. Por tanto, tenemos que calcular las tres derivadas parciales siguientes:
\begin{align*}
\dfrac{\partial f}{\partial x}(x,y,z) &= \dfrac{\partial }{\partial x}(\log\left(\sqrt{x^{2}-z^{2}}\right) +3^{\tfrac{x^{2}}{y}}) = \dfrac{\partial }{\partial x}(\log \left(\sqrt{x^{2}-z^{2}}\right) )+\dfrac{\partial }{\partial x}(3^{\tfrac{x^{2}}{y}})= \\
&= \frac{1}{\sqrt{x^{2}-z^{2}}}\dfrac{\partial }{\partial x}(\sqrt{x^{2}-z^{2}})+3^{\tfrac{x^{2}}{y}}\log 3\dfrac{\partial }{\partial x}(\dfrac{x^{2}}{y}) = \\
&= \frac{1}{\sqrt{x^{2}-z^{2}}}\frac{1}{2\sqrt{x^{2}-z^{2}}}\dfrac{\partial}{\partial x}(x^{2}-z^{2})+3^{\tfrac{x^{2}}{y}}\log 3\dfrac{2x}{y}= \\
&= \frac{1}{2(x^{2}-z^{2})}2x+3^{\tfrac{x^{2}}{y}}\log 3\dfrac{2x}{y}=\frac{x}{x^{2}-z^{2}}+3^{\tfrac{x^{2}}{y}}\log 3\dfrac{2x}{y}, \\
\dfrac{\partial f}{\partial y}(x,y,z) &= \dfrac{\partial }{\partial y}(\log\left( \sqrt{x^{2}-z^{2}}\right) +3^{\tfrac{x^{2}}{y}}) = \dfrac{\partial }{\partial y}(\log \left( \sqrt{x^{2}-z^{2}}\right) )+\dfrac{\partial }{\partial y}(3^{\tfrac{x^{2}}{y}})= \\
&= 0+3^{\tfrac{x^{2}}{y}}\log 3\dfrac{\partial }{\partial y}(\dfrac{x^{2}}{y}) = 3^{\tfrac{x^{2}}{y}}\log 3\dfrac{-x^{2}}{y^{2}}, \\
\dfrac{\partial f}{\partial z}(x,y,z) &= \dfrac{\partial }{\partial z}(\log\left( \sqrt{x^{2}-z^{2}}\right) +3^{\tfrac{x^{2}}{y}}) = \dfrac{\partial }{\partial z}(\log \left( \sqrt{x^{2}-z^{2}}\right) )+\dfrac{\partial }{\partial z}(3^{\tfrac{x^{2}}{y}})= \\
&= \frac{1}{\sqrt{x^{2}-z^{2}}}\dfrac{\partial }{\partial x}(\sqrt{x^{2}-z^{2}})+0 = \frac{1}{\sqrt{x^{2}-z^{2}}}\frac{1}{2\sqrt{x^{2}-z^{2}}}\dfrac{\partial }{\partial x}(x^{2}-z^{2})= \\
&= \frac{1}{2(x^{2}-z^{2})}(-2z)=-\frac{z}{x^{2}-z^{2}}.
\end{align*}
y, en consecuencia tenemos
\[
\nabla f(x,y,z)=(\frac{x}{x^{2}-z^{2}}+3^{\tfrac{x^{2}}{y}}\log 3\dfrac{2x}{y},3^{\tfrac{x^{2}}{y}}\log 3\dfrac{-x^{2}}{y^{2}},-\frac{z}{x^{2}-z^{2}}).
\]
Como nos piden el gradiente en el punto $(1,1,0),$ sustituimos $x$ por 1, $y$
por 1, y $z$ por 0 en el vector anterior y obtenemos
\[
\nabla f(1,-2,-2)=(1+6\log 3,-3\log 3,0).
\]
}


\newproblem{par-32}{amb}{}
%ENUNCIADO
{La asimilación de CO$_2$ de una planta depende de la temperatura ambiente (t) y de la intensidad de la luz (l), según la función
\[
f(t,l) = ctl^2,
\]
donde $c$ es una constante.
Estudiar cómo evoluciona la asimilación de CO$_2$ para distintas intensidades de luz, cuando se mantiene la temperatura constante.
Estudiar también cómo evoluciona para distintas temperaturas cuando se mantiene la intensidad de la luz constante.
}
%SOLUCIÓN
{$\frac{\partial f}{\partial l}(t,l) = 2ctl$ y $\frac{\partial f}{\partial t}(t,l) = cl^2$.
}
%RESOLUCIÓN
{
}


\newproblem{par-33}{amb}{}
%ENUNCIADO
{La abundancia de una determinada especie de planta depende del nivel de nitrógeno en el suelo y del nivel de perturbaciones, de manera que un incremento del nivel de nitrógeno tiene un efecto negativo en la abundancia de esta especie, y un aumento de las perturbaciones también tiene un efecto negativo.
Si en un momento dado comienza a aumentar el nivel de nitrógeno en el suelo y también las perturbaciones debidas al pastoreo, ¿cómo se verá afectada la abundancia de la especie?
}
%SOLUCIÓN
{La abundancia de la especie disminuirá.
}
%RESOLUCIÓN
{
}


\newproblem{par-34}{amb}{}
%ENUNCIADO
{La velocidad de crecimiento de un organismo depende de la disponibilidad de alimento y del número de competidores en busca de alimento.
¿Cómo se verá afectada la velocidad de crecimiento si la disponibilidad de alimento aumenta con el tiempo y el número de competidores disminuye?}
%SOLUCIÓN
{La velocidad de crecimiento aumentará.
}
%RESOLUCIÓN
{
}


\newproblem{par-35}{amb}{}
%ENUNCIADO
{Un organismo se mueve sobre una superficie inclinada siguiendo la línea de máxima pendiente descendiente.
Si la expresión de la superficie es
\[
f(x,y) = x^2-y^2,
\]
calcule la dirección en la que se moverá el organismo en el punto $(2,3)$.
}
%SOLUCIÓN
{Se moverá en la dirección $-\nabla f(2,3)=(-4,6)$.}
%RESOLUCIÓN
{
}


\newproblem{par-36}{gen}{}
%ENUNCIADO
{Si $f(x,y,z)=x^3y^2z$ y $g(t)=(e^t,\cos t,\sen t)$, calcular $(f\circ g)'(t)$.
}
%SOLUCIÓN
{$(f\circ g)'(t)= e^{3t}(3\sen t\cos^2 t-2\sen^2 t\cos t+\cos^3 t)$.
}
%RESOLUCIÓN
{
}


\newproblem{par-37}{amb}{}
%ENUNCIADO
{Obtener los puntos críticos de $z=f(x,y)$ para:
\begin{enumerate}
\item $f(x,y)=x^2+y^2$.
\item $f(x,y)=x^2y+y^2x$.
\item $f(x,y)=x^2-2xy+2y^2$.
\end{enumerate}
}
%SOLUCIÓN
{\begin{enumerate}
\item $(0,0)$.
\item $(0,0)$.
\item $(0,0)$.
\end{enumerate}
}
%RESOLUCIÓN
{
}


\newproblem{par-38}{gen}{}
%ENUNCIADO
{La superficie de una montaña tiene la forma
\[
S:z=a-bx^2-cy^2,
\]
donde $a$, $b$ y $c$ son constantes, $x$ es la coordenada Este-Oeste e $y$ la coordenada Norte-Sur en el mapa, y $z$ la altura sobre el nivel del mar en metros.
En el punto $P=(1,1)$ del mapa, ¿en qué dirección crece más rápidamente la altura?
}
%SOLUCIÓN
{$(-2b,-2c)$.
}
%RESOLUCIÓN
{
}


\newproblem{par-39}{gen}{}
%ENUNCIADO
{Hallar las direcciones de máximo y mínimo crecimiento de las siguientes funciones en el punto $P$:
\begin{enumerate}
\item $f(x,y)=x^2+xy+y^2$, $P=(-1,1)$.
\item $f(x,y)=x^2y+e^{xy}\sen y$, $P=(1,0)$.
\item $f(x,y,z)=\log(xy)+\log(yz)+\log(xz)$, $P=(1,1,1)$.
\item $f(x,y,z)=\log(x^2+y^2-1)+y+6z$, $P=(1,1,0)$.
\end{enumerate}
}
%SOLUCIÓN
{\begin{enumerate}
\item Máximo crecimiento en la dirección $(-1,1)$ y máximo decrecimiento en la dirección $(1,-1)$.
\item Máximo crecimiento en la dirección $(0,2)$ y máximo decrecimiento en la dirección $(0,-2)$.
\item Máximo crecimiento en la dirección $(2,2,2)$ y máximo decrecimiento en la dirección $(-2,-2,-2)$.
\item Máximo crecimiento en la dirección $(2,3,6)$ y máximo decrecimiento en la dirección $(-2,-3,-6)$.
\end{enumerate}
}
%RESOLUCIÓN
{
}


\newproblem{par-40}{gen}{}
%ENUNCIADO
{¿En qué direcciones se anulará la derivada direccional de la función
\[
f(x,y)=\frac{x^2-y^2}{x^2+y^2}
\]
en el punto $P=(1,1)$?
}
%SOLUCIÓN
{En la dirección $(1/\sqrt{2},1/\sqrt{2})$.
}
%RESOLUCIÓN
{
}


\newproblem{par-41}{gen}{}
%ENUNCIADO
{¿Existe alguna dirección en la que la derivada direccional en el punto $P=(1,2)$ de la función
\[
f(x,y) = x^2-3xy+4y^2
\]
valga 14?
}
%SOLUCIÓN
{No.
}
%RESOLUCIÓN
{
}


\newproblem{par-42}{gen}{}
%ENUNCIADO
{La derivada direccional de una función $f$ en un punto $P$ es máxima en la dirección del vector $(1,1,-1)$ y su valor es $2\sqrt{3}$.
¿Cuánto vale la derivada direccional de $f$ en $P$ en la dirección del vector $(1,1,0)$?
}
%SOLUCIÓN
{$2\sqrt{2}$.
}
%RESOLUCIÓN
{
}


\newproblem{par-43}{gen}{}
%ENUNCIADO
{Dado el campo escalar
\[
f(x,y,z) = x^2-y^2+xyz^3-zx
\]
en el punto $P=(1,2,3)$, se pide:
\begin{enumerate}
\item Calcular la derivada direccional de $f$ en $P$ a lo largo del vector unitario $\mathbf{u}=\frac{1}{\sqrt2}(1,-1,0)$.
\item ¿En qué dirección es máxima la derivada direccional de $f$ en $P$? Obtener el valor de dicha derivada direccional.
\end{enumerate}
}
%SOLUCIÓN
{\begin{enumerate}
\item $15\sqrt{2}$.
\item La derivada direccional es máxima en la dirección del gradiente $(53,23,53)$ y vale $\sqrt{6147}$.
\end{enumerate}
}
%RESOLUCIÓN
{
}


\newproblem*{par-44}{gen}{}
%ENUNCIADO
{En el ajuste de regresión de una recta $y=a+bx$, se suele utilizar la técnica de mínimos cuadrados que consisten en buscar los valores
de $a$ y $b$ que hacen mínima la función
\[
f(a,b)= \sum_{i=1}^{n}(y_i-a-bx_i)^2,
\]
donde el sumatorio abarca a todos los pares de la muestra $(x_i,y_i)$ para $i=1,\ldots, n$, siendo $n$ el tamaño de la muestra.

Demostrar que esta función alcanza el mínimo en los puntos
\[
a=\bar y-b\bar x \quad \mbox{ y } b=\frac{s_{xy}}{s_x^2}.
\]
}
%SOLUCIÓN
{
}
%RESOLUCIÓN
{
}


\newproblem{par-45}{gen}{}
%ENUNCIADO
{La siguiente función mide la presión del aire en la posición $(x,y,z)$.
\[
f(x,y,z)= x^2+y^2-z^3.
\]
Sea $A$ un objeto que se mueve a lo largo de la trayectoria:
\[
\begin{cases}
x=t\\
y=1\\
z=1/t
\end{cases}
t>0.
\]
\begin{enumerate}
\item Calcular la ecuación de la recta tangente a la trayectoria de $A$ en el punto $(1,1,1)$.
\item Sigue la trayectoria de $A$ en el punto $(1,1,1)$ la dirección de máximo crecimiento de la función $f$?
\end{enumerate}
}
%SOLUCIÓN
{\begin{enumerate}
\item $(1+t, 1, 1-t)$.
\item No ya que la dirección de máximo crecimiento de $f$ es $\nabla f(1,1,1)=(2,2,-3)$ y la dirección de la trayectoria es $(1,0,-1)$.
\end{enumerate}
}
%RESOLUCIÓN
{
}


\newproblem{par-46}{gen}{*}
%STATEMENT
{Obtener la ecuación del plano tangente y de la recta normal a la superficie
\[
S:xyz=8
\]
en el punto $P=(4,-2,-1)$.
}
%SOLUTION
{Recta normal $l:(4+2t,-2-4t,-1-8t)$. Plano tangente $\pi: 2x-4y-8z+24=0$.
}
%RESOLUTION
{
}


%%%%%%% Pendiente 26

% Autor: Alfredo Sánchez Alberca (asalber@ceu.es)

\newproblem{par-1}{gen}{}
%ENUNCIADO
{Calcular las siguientes derivadas parciales:
\begin{multicols}{2}
\begin{enumerate}
\item $\dfrac{\partial}{\partial x}\ln \dfrac{x}{y}$.
\item $\dfrac{\partial}{\partial v}\dfrac{nRT}{v}$.
%\item $\dfrac{\partial^2}{\partial x \partial y}\left(e^{x+y}\sen\dfrac{x}{y}\right)$.
%\item $\dfrac{\partial^2}{\partial y \partial x}\left(e^{x+y}\sen\dfrac{x}{y}\right)$.
\end{enumerate}
\end{multicols}
}
%SOLUCIÓN
{\begin{enumerate}

\item $\frac{\partial}{\partial x}\,\log \left(\frac{x}{y}\right) = \frac{1}{x}$.
\item $\frac{\partial}{\partial v}\,\left(\frac{n\,R\,T}{v}\right) = -\frac{n\,R\,T}{v^2}$.
%\item $\frac{\partial^2}{\partial x \partial y}\,\left(\sin \left(\frac{x}{y}\right)\,e^{y+x}\right) = \frac{\left(\sen \left(\frac{x}{y}\right)\,y^3+\cos \left(\frac{x}{y}\right)\,y^2-x\,\cos \left(\frac{x}{y}\right)\,y-\cos \left(\frac{x}{y}\right)\,y+x\,\sen \left(\frac{x}{y}\right)\right)\,e^{y+x}}{y^3}$
%\item $\frac{\partial^2}{\partial y \partial x}\,\left(\sin \left(\frac{x}{y}\right)\,e^{y+x}\right) = \frac{\left(\sen \left(\frac{x}{y}\right)\,y^3+\cos \left(\frac{x}{y}\right)\,y^2-x\,\cos \left(\frac{x}{y}\right)\,y-\cos \left(\frac{x}{y}\right)\,y+x\,\sen \left(\frac{x}{y}\right)\right)\,e^{y+x}}{y^3}$
\end{enumerate}
}
%RESOLUCIÓN
{
}


\newproblem{par-2}{gen}{}
%ENUNCIADO
{Calcular el vector gradiente y la matriz Hessiana de las siguientes funciones:
\begin{multicols}{2}
\begin{enumerate}
\item $e^{x^2+y^2+z^2}$
\item $\sen((x^2-y^2)z)$
\end{enumerate}
\end{multicols}
}
%SOLUCIÓN
{\begin{enumerate}
\item $\nabla e^{x^2+y^2+z^2} = \left( 2\,x\,e^{z^2+y^2+x^2} , 2\,y\,e^{z^2+y^2+x^2} , 2\,z\,e^{z^2 +y^2+x^2} \right)$,\\
$
H e^{x^2+y^2+z^2} =
\left(
\begin{array}{ccc}
(4x^2+2)e^{x^2+y^2+z^2} & 4xye^{x^2+y^2+z^2} & 4xze^{x^2+y^2+z^2} \\
4xye^{x^2+y^2+z^2} & (4y^2+2)e^{x^2+y^2+z^2} & 4yze^{x^2+y^2+z^2} \\
4xze^{x^2+y^2+z^2} & 4yze^{x^2+y^2+z^2} & (4z^2+2)e^{x^2+y^2+z^2}
\end{array}
\right).
$
\item $\nabla \sen((x^2-y^2)z) = \left( 2\,x\,z\,\cos \left(\left(x^2-y^2\right)\,z\right) , -2\,y\, z\,\cos \left(\left(x^2-y^2\right)\,z\right) , \left(x^2-y^2\right) \,\cos \left(\left(x^2-y^2\right)\,z\right) \right) $\\
$H \sen((x^2-y^2)z) =$\\
\resizebox{\linewidth}{!}{
$
\left(
\begin{array}{ccc}
4x^2\sen((x^2-y^2)z)+2\cos((x^2-y^2)z) & 4xy\sen((x^2-y^2)z) & -2x(x^2-y^2)\sen((x^2-y^2)z) \\
4xy\sen((x^2-y^2)z) & -4y^2\sen((x^2-y^2)z)-2\cos((x^2-y^2)z) & 2y(x^2-y^2)\sen((x^2-y^2)z) \\
-2x(x^2-y^2)\sen((x^2-y^2)z) & 2y(x^2-y^2)\sen((x^2-y^2)z) & -(x^2-y^2)^2\sen((x^2-y^2)z)
\end{array}
\right).
$
}
\end{enumerate}
}
%RESOLUCIÓN
{
}


\newproblem{par-3}{gen}{*}
%ENUNCIADO
{Calcular el gradiente de la función
\[ f(x,y,z)=\log \frac{\sqrt{x}}{yz}+\arcsen (xz). \]
}
%SOLUCIÓN
{$\nabla f(x,y,z) = \left( \frac{z}{\sqrt{1-x^2z^2}}+\frac{1}{2x} ,\frac{-1}{y} , \frac{x}{\sqrt{1-x^2\,z^2}}-\frac{1}{z} \right) $.
}
%RESOLUCIÓN
{
}


\newproblem{par-4}{gen}{}
% ENUNCIADO
{Una nave espacial está en problemas cerca del sol.
Se encuentra en la posición $(1,1,1)$ y la temperatura de la nave cuando está en la posición $(x,y,z)$ viene dada por
$T(x,y,z)=\mbox{e}^{-x^2-2y^2-3z^2}$ donde $x,y,z$ se miden en metros.
¿En qué dirección debe moverse la nave para que la temperatura decrezca lo más rápidamente posible? }
%SOLUCIÓN
{Debe moverse en la dirección $-\nabla f(1,1,1)=e^{-6}(2,4,6)$.
}
%RESOLUCIÓN
{
}

\newproblem{par-5}{gen}{*}
%ENUNCIADO
{Dada la función
\[
f(x,y,z)=\log \sqrt{xy-\frac{z^2}{xy}}
\]
\begin{enumerate}
\item Hallar el vector gradiente.
\item Hallar un punto en el que el vector gradiente sea paralelo a la bisectriz del plano $XY$, y calcular el vector gradiente en dicho punto.
\end{enumerate}
}
%SOLUCIÓN
{\begin{enumerate}
\item $\nabla f(x,y,z) = \left( -\frac{z^2+x^2y^2}{2xz^2-2x^3y^2} , -\frac{z^2+x^2y^2}{2yz^2-2x^2y^3} , \frac{z}{z^2-x^2y^2}  \right) $.
\item El vector gradiente es paralelo a la bisectriz del plano $XY$ en cualquier punto de la forma $(a,a,0)$ con $a\in \mathbb{R}$.\\
$\nabla f(1,1,0) = \left(\frac{1}{2},\frac{1}{2},0\right)$.
\end{enumerate}
}
%RESOLUCIÓN
{
}


\newproblem{par-6}{far}{*}
%ENUNCIADO
{La cantidad $C$ de cierta toxina en sangre (en mg/dl) depende del número de bacterias, $b$ (bacterias/dl), del número de linfocitos, $l$ (linfocitos/dl), y del tiempo, $t$ (horas), según la ecuación:
\[
C(b,l,t) = \frac{{t^2  \cdot e^{3b + 2} }}{{l^2 }} - \frac{1}{{\log
(b \cdot l)}}
\]
\begin{enumerate}
\item Calcular su gradiente.

\item Comprobar que se cumple: $\dfrac{{\partial ^2 C}}{{\partial t\partial b}} = \dfrac{{\partial ^2 C}}{{\partial b\partial t}}$.
\end{enumerate}
}
%SOLUCIÓN
{
\begin{enumerate}
\item $\nabla C(b,l,t)=\left( \frac{{3t^2 \cdot e^{3b + 2} }}{{l^2 }}+\frac{1}{{b\log^2
(b \cdot l)}}, \frac{{-2t^2 \cdot e^{3b + 2} }}{{l^3 }}+\frac{1}{{l\log^2
(b \cdot l)}}, \frac{{2t \cdot e^{3b + 2} }}{{l^2 }} \right)$.

\item $\frac{\partial ^2 C}{\partial t \partial b}  = \frac{{6t \cdot e^{3b + 2} }}{{l^2 }}$.
\end{enumerate}
}
%RESOLUCIÓN
{
\begin{enumerate}
  \item La fórmula del gradiente es
\begin{equation}
\label{e:gradiente}
\nabla C(b,l,t)=\left(\frac{\partial C}{\partial b}, \frac{\partial C}{\partial l},\frac{\partial C}{\partial t}\right),
\end{equation}
de modo que necesitamos calcular las tres primeras derivadas parciales:
\begin{align*}
\frac{\partial C}{\partial b} &= \frac{\partial}{\partial b}\left(\frac{{t^2
\cdot e^{3b + 2} }}{{l^2 }}\right)-\frac{\partial}{\partial b}\left(\frac{1}{{\log
(b \cdot l)}}\right)= \frac{{3t^2 \cdot e^{3b + 2} }}{{l^2 }}+\frac{1}{{b\log^2
(b \cdot l)}}\\
\frac{\partial C}{\partial l} &= \frac{\partial}{\partial l}\left(\frac{{t^2
\cdot e^{3b + 2} }}{{l^2 }}\right)-\frac{\partial}{\partial l}\left(\frac{1}{{\log
(b \cdot l)}}\right)= \frac{{-2t^2 \cdot e^{3b + 2} }}{{l^3 }}+\frac{1}{{l\log^2
(b \cdot l)}}\\
\frac{\partial C}{\partial t} &= \frac{\partial}{\partial t}\left(\frac{{t^2
\cdot e^{3b + 2} }}{{l^2 }}\right)-\frac{\partial}{\partial t}\left(\frac{1}{{\log
(b \cdot l)}}\right)= \frac{{2t \cdot e^{3b + 2} }}{{l^2 }}\\
\end{align*}

Así que, sustituyendo en la fórmula \ref{e:gradiente} tenemos:
\[
\nabla C(b,l,t)=\left( \frac{{3t^2 \cdot e^{3b + 2} }}{{l^2 }}+\frac{1}{{b\log^2
(b \cdot l)}}, \frac{{-2t^2 \cdot e^{3b + 2} }}{{l^3 }}+\frac{1}{{l\log^2
(b \cdot l)}}, \frac{{2t \cdot e^{3b + 2} }}{{l^2 }} \right).
\]

\item Para ver si se satisface la igualdad calculamos ambas derivadas:
\begin{align*}
\frac{\partial ^2 C}{\partial t \partial b} & = \frac{\partial}{\partial
t}\left(\frac{\partial C}{\partial b}\right) = \frac{\partial}{\partial t}\left(
\frac{{3t^2 \cdot e^{3b + 2} }}{{l^2 }}+\frac{1}{{b\log^2
(b \cdot l)}} \right) = \frac{{6t \cdot e^{3b + 2} }}{{l^2 }} \\
\frac{\partial ^2 C}{\partial b \partial t} & = \frac{\partial}{\partial
b}\left(\frac{\partial C}{\partial t}\right) = \frac{\partial}{\partial b}\left(
\frac{{2t \cdot e^{3b + 2} }}{{l^2 }}\right) = \frac{{6t \cdot e^{3b + 2} }}{{l^2 }}
\end{align*}
Por tanto, la igualdad es cierta.
\end{enumerate}
}


\newproblem*{par-7}{amb}{*}
%ENUNCIADO
{Supongamos que la cantidad de agua almacenada en un pantano al final del año hidrológico, $A$ en hectómetros cúbicos, viene dada por:
\[
A = \sqrt {\frac{{p^3 }}{{t - 1}} - c^2 e^{cpt}}
\]
donde $p$ es la precipitación en litros/m$^2$ caí­da durante el año hidrológico, $t$ es la temperatura media del año hidrológico en ºC y $c$ el consumo debido a abastecimiento de poblaciones cercanas y riego, en hectómetros cúbicos.
Se pide:
\begin{enumerate}
\item Calcular el gradiente de la cantidad de agua almacenada.
\item Suponiendo que hubiese algún año en el que el consumo fuese nulo, ¿qué condición tendría­ que cumplir la temperatura para que la derivada del agua almacenada con respecto a la temperatura fuese igual a la derivada con respecto a la precipitación?
\end{enumerate}
}


\newproblem*{par-8}{gen}{*}
%ENUNCIADO
{Dada la función $f(x)=e^{2xy}\sen(x+3z)$, se pide:
\begin{enumerate}
  \item ¿Calcular el vector gradiente en el origen de coordenadas?
  \item ¿Es cierto que $\dfrac{\partial^3f}{\partial y^2\partial z}=\dfrac{\partial^3f}{\partial y\partial z\partial y}?$
\end{enumerate}
}


\newproblem{par-9}{gen}{*}
%ENUNCIADO
{La variable aleatoria bidimensional $(X,Y)$ con función de densidad
\[
f(x,y) = \frac{1}{\sqrt{2\pi}\, \sigma_x\sigma_y} e^{-\frac{1}{2}\left(\frac{(x-\mu_x)^2}{\sigma_x^2}+\frac{(y-\mu_y)^2}{\sigma_y^2}\right)}
\]
se conoce como normal bidimensional con $X$ e $Y$ independientes, de parámetros $\mathbf{\mu}=(\mu_x,\mu_y)$ y $\mathbf{\sigma}=(\sigma_x,\sigma_y)$.
Calcular el gradiente de $f$ e interpretarlo. ¿En qué punto se anula el gradiente? ¿Qué conclusiones sacas? ¿Cuál es la tasa de crecimiento de $f$ cuando $x\rightarrow \infty$?
}
%SOLUCIÓN
{$\nabla f(x,y) = -\frac{1}{\sqrt{2\pi}\, \sigma_x\sigma_y} e^{-\frac{1}{2}\left(\frac{(x-\mu_x)^2}{\sigma_x^2}+\frac{(y-\mu_y)^2}{\sigma_y^2}\right)} \left(\frac{x-\mu_x}{\sigma_x^2}, \frac{y-\mu_y}{\sigma_y^2}\right)$.\\
El gradiente se anula en $(x=\mu_x, y=\mu_y)$.\\
$\lim_{x\rightarrow \infty}f(x,y) = 0$.
}
%RESOLUCIÓN
{
}


\newproblem{par-10}{gen}{*}
%ENUNCIADO
{La ecuación diferencial parcial
\[
\displaystyle{\frac{\partial^2 u}{\partial x^2}} + \ \displaystyle{\frac{\partial^2 u}{\partial y^2}} + \displaystyle{\frac{\partial^2 u}{\partial z^2}} = 0,
\]
se conoce como ecuación de Laplace se aplica a multitud de fenómenos relacionadas con conducción de calor, flujo de fluidos y potencial eléctrico.

Dada la función $u(x,y,z)=\dfrac{1}{ \sqrt{x^2 + y^2 + z^2}},$
\begin{enumerate}
\item Comprobar que $f$ satisface la ecuación de Laplace.
\item ¿Existe algún punto en el que el crecimiento de la función sea nulo?
\item Si fijamos $z=1$, calcular
\[
\frac{\partial^4u}{\partial x^2\partial y^2}.
\]
\end{enumerate}
}
%SOLUCIÓN
{
\begin{enumerate}[start=2]
\item No hay ningún punto donde se el crecimiento es nulo.
\item $\frac{{\partial ^4 u}}{{\partial x^2 \partial y^2 }} =3\left( {x^2  + y^2  + 1} \right)^{ - 5/2}  - 15\left( {x^2  + y^2
} \right)\left( {x^2  + y^2 + 1} \right)^{ - 7/2}  + 105x^2 y\left({x^2  + y^2  + 1} \right)^{ - 9/2}$.
\end{enumerate}
}
%RESOLUCIÓN
{\begin{enumerate}
\item Para comprobar que $u(x,y,z)$ satisface la ecuación de Laplace
calculamos las tres derivadas parciales segundas que intervienen en
la ecuación. Comenzando con las derivadas parciales con respecto a
la variable $x$, obtenemos:
\[
u(x,y,z) = \frac{1}{{\sqrt {x^2  + y^2  + z^2 } }} = \left( {x^2  +
y^2  + z^2 } \right)^{ - 1/2}
\]
\[
\frac{{\partial u}}{{\partial x}} =  - \frac{1}{2}\left( {x^2  + y^2
+ z^2 } \right)^{ - 3/2} 2x =  - x\left( {x^2  + y^2  + z^2 }
\right)^{ - 3/2}
\]
\[
\frac{{\partial ^2 u}}{{\partial x^2 }} = \frac{\partial }{{\partial
x}}\left( { - x\left( {x^2  + y^2  + z^2 } \right)^{ - 3/2} }
\right) =  - \left( {x^2  + y^2  + z^2 } \right)^{ - 3/2}  + 3x^2
\left( {x^2  + y^2  + z^2 } \right)^{ - 5/2}
\]
e igualmente para las variables $y$ y $z$, tenemos:
\[
\frac{{\partial u}}{{\partial y}} =  - y\left( {x^2  + y^2  + z^2 }
\right)^{ - 3/2}
\]
\[
\frac{{\partial ^2 u}}{{\partial y^2 }} =  - \left( {x^2  + y^2  +
z^2 } \right)^{ - 3/2}  + 3y^2 \left( {x^2  + y^2  + z^2 } \right)^{
- 5/2}
\]
\[
\frac{{\partial u}}{{\partial z}} =  - z\left( {x^2  + y^2  + z^2 }
\right)^{ - 3/2}
\]
\[
\frac{{\partial ^2 u}}{{\partial z^2 }} =  - \left( {x^2  + y^2  +
z^2 } \right)^{ - 3/2}  + 3z^2 \left( {x^2  + y^2  + z^2 } \right)^{
- 5/2}
\]
Por lo tanto:
\[
\frac{{\partial ^2 u}}{{\partial x^2 }} + \frac{{\partial ^2
u}}{{\partial y^2 }} + \frac{{\partial ^2 u}}{{\partial z^2 }} =  -
3\left( {x^2  + y^2  + z^2 } \right)^{ - 3/2}  + 3\left( {x^2  + y^2
+ z^2 } \right)\left( {x^2  + y^2  + z^2 } \right)^{ - 5/2}  =
\]
\[
=- 3\left( {x^2  + y^2  + z^2 } \right)^{ - 3/2}  + 3\left( {x^2  +
y^2 + z^2 } \right)^{ - 3/2}  = 0
\]

\item Una condición necesaria para que el crecimiento de una función
de varias variables en un punto sea nulo es que el gradiente en
dicho punto se anule, y el gradiente se anula si se anulan sus tres
componentes:
\[
\vec \nabla u = \vec 0 \Leftrightarrow \left( {\frac{{\partial
u}}{{\partial x}},\frac{{\partial u}}{{\partial y}},\frac{{\partial
u}}{{\partial z}}} \right) = \left( {0,0,0} \right)
\]
Por lo tanto, tenemos un sistema no lineal de tres ecuaciones con
tres incógnitas:
\[
 - x\left( {x^2  + y^2  + z^2 } \right)^{ - 3/2}  = 0
\]
\[
 - y\left( {x^2  + y^2  + z^2 } \right)^{ - 3/2}  = 0
\]
\[
 - z\left( {x^2  + y^2  + z^2 } \right)^{ - 3/2}  = 0
\]
Y teniendo en cuenta que el término $(x^2+y^2+z^2)$, por tratarse de
una suma de cuadrados, únicamente puede ser 0 si $x=y=z=0$; y a
igual conclusión llegamos si suponemos que es distinto de 0, ya que
entonces la primera ecuación implica que necesariamente $x=0$, la
segunda implica que $y=0$, y la tercera implica que $z=0$. Por lo
tanto, concluimos que el único punto en el que el crecimiento puede
ser nulo es $(x,y,z)=(0,0,0)$, pero dicho punto no pertenece al
dominio de definición de la función (tendríamos un cero como
denominador de una fracción), por lo que no hay ningún punto en el
que la función presente un crecimiento nulo.

\item Suponiendo $z=1$, la función resultante presenta únicamente
dos variables:
\[
u(x,y,1) = \frac{1}{{\sqrt {x^2  + y^2  + 1} }} = \left( {x^2  + y^2
+ 1} \right)^{ - 1/2}
\]
La derivada propuesta es:
\[
\frac{{\partial ^4 u}}{{\partial x^2 \partial y^2 }} =
\frac{\partial }{{\partial x}}\left( {\frac{\partial }{{\partial
x}}\left( {\frac{\partial }{{\partial y}}\left( {\frac{{\partial
u}}{{\partial y}}} \right)} \right)} \right)
\]
en donde, como ya sabemos, se puede cambiar el orden de derivación
sin que afecte al resultado final, aunque nunca el número total de
derivadas con respecto a cada variable.

Operando como ya hicimos en los cálculos previos de las derivadas
segundas, obtenemos:
\[
\frac{{\partial u}}{{\partial y}} =  - y\left( {x^2  + y^2  + 1}
\right)^{ - 3/2}
\]
\[
\frac{\partial }{{\partial y}}\left( {\frac{{\partial u}}{{\partial
y}}} \right) = \frac{{\partial ^2 u}}{{\partial y^2 }} =  - \left(
{x^2  + y^2  + 1} \right)^{ - 3/2}  + 3y^2 \left( {x^2  + y^2  + 1}
\right)^{ - 5/2}
\]
\[
\frac{\partial }{{\partial x}}\left( {\frac{{\partial ^2
u}}{{\partial y^2 }}} \right) = \frac{{\partial ^3 u}}{{\partial
x\partial y^2 }} = 3x\left( {x^2  + y^2  + 1} \right)^{ - 5/2}  -
15y^2 x\left( {x^2  + y^2  + 1} \right)^{ - 7/2}
\]
\[
\frac{\partial }{{\partial x}}\left( {\frac{{\partial ^3
u}}{{\partial x\partial y^2 }}} \right) = \frac{{\partial ^4
u}}{{\partial x^2 \partial y^2 }} =
\]
\[
=3\left( {x^2  + y^2  + 1} \right)^{ - 5/2}  - 15\left( {x^2  + y^2
} \right)\left( {x^2  + y^2 + 1} \right)^{ - 7/2}  + 105x^2 y\left(
{x^2  + y^2  + 1} \right)^{ - 9/2}
\]
\end{enumerate}
}


\newproblem{par-11}{qui}{*}
%ENUNCIADO
{La siguiente función determina la temperatura en cada punto del plano real:
\[f(x,y)=e^{x+2y}\cos(x^2+y^2).\]
Se pide:
\begin{enumerate}
  \item Calcular el gradiente de $f$.
  \item Si estamos situados en el origen de coordenadas, ¿en qué dirección aumentará más rápidamente la temperatura? ¿Y si estuviésemos en el punto $(0,1)$?
\item Calcular la matriz Hessiana y el Hessiano de $f$ en el origen de coordenadas.
\end{enumerate}
}
%SOLUCIÓN
{\begin{enumerate}
\item $\nabla f(x,y) = e^{x+2y}\left(\cos(x^{2}+y^{2})-2x\sen(x^{2}+y^{2}), 2\cos(x^{2}+y^{2})-2y\sen(x^{2}+y^{2})\right)$.
\item $\nabla f(0,0) = (1,2)$ y $\nabla f(0,1) = (3.99\,,\,-4.45)$.
\item $Hf(0,0)=\left(
\begin{array}[]{cc}
1 & 2 \\
2 & 4
\end{array}
\right)
\quad |Hf(0,0)|= 0$.
\end{enumerate}
}
%RESOLUCIÓN
{\begin{enumerate}
\item Para calcular el vector gradiente de $f$ necesitamos calcular sus derivadas parciales de primer orden.
\begin{align*}
\frac{\partial}{\partial x}f(x,y) &= \frac{\partial}{\partial x}\left(e^{x+2y}\cos(x^{2}+y^{2})\right) = \frac{\partial}{\partial x}e^{x+2y}\cos(x^{2}+y^{2}) + e^{x+2y}\frac{\partial}{\partial x}\cos(x^{2}+y^{2}) = \\
&= e^{x+2y}\frac{\partial}{\partial x}(x+2y)\cos(x^{2}+y^{2})+e^{x+2y}(-\sen(x^{2}+y^{2})\frac{\partial}{\partial x}(x^{2}+y^{2}) =\\
&= e^{x+2y}\cos(x^{2}+y^{2})-e^{x+2y}\sen(x^{2}+y^{2})2x = e^{x+2y}(\cos(x^{2}+y^{2})-2x\sen(x^{2}+y^{2}),
\\
\frac{\partial}{\partial y}f(x,y) &= \frac{\partial}{\partial y}\left(e^{x+2y}\cos(x^{2}+y^{2})\right) = \frac{\partial}{\partial y}e^{x+2y}\cos(x^{2}+y^{2}) + e^{x+2y}\frac{\partial}{\partial y}\cos(x^{2}+y^{2}) = \\
&= e^{x+2y}\frac{\partial}{\partial y}(x+2y)\cos(x^{2}+y^{2})+e^{x+2y}(-\sen(x^{2}+y^{2})\frac{\partial}{\partial y}(x^{2}+y^{2}) =\\
&= e^{x+2y}\cos(x^{2}+y^{2})2-e^{x+2y}\sen(x^{2}+y^{2})2y= e^{x+2y}(2\cos(x^{2}+y^{2})-2y\sen(x^{2}+y^{2}),
\end{align*}
Así pues, el vector gradiente es
\begin{align*}
\nabla f(x,y) &= \left(\dfrac{\partial}{\partial x}f(x,y),\dfrac{\partial}{\partial y}f(x,y)\right) =\\
&= e^{x+2y}\left(\cos(x^{2}+y^{2})-2x\sen(x^{2}+y^{2}), 2\cos(x^{2}+y^{2})-2y\sen(x^{2}+y^{2})\right).
\end{align*}

\item La dirección en que más rápidamente aumenta la temperatura es la dirección del vector gradiente. Si estamos en el origen de coordenadas, dicha dirección es
\[
\nabla f(0,0) = e^{0+2\cdot 0}\left(\cos(0^{2}+0^{2})-2\cdot 0\sen(0^{2}+0^{2}), 2\cos(0^{2}+0^{2})-2\cdot 0\sen(0^{2}+0^{2})\right) = (1,2).
\]
Y si estamos en el punto $(0,1)$, la dirección de máximo crecimiento de la temperatura es
\begin{align*}
\nabla f(0,1) &= e^{0+2\cdot 1}\left(\cos(0^{2}+1^{2})-2\cdot 0\sen(0^{2}+1^{2}), 2\cos(0^{2}+1^{2})-2\cdot 1\sen(0^{2}+1^{2})\right) =\\
&= e^{2}(\cos 1, 2\cos 1-2\sen 1) = (3.99\,,\,-4.45).
\end{align*}

\item Para calcular la matriz Hessiana necesitamos calcular las derivadas parciales de segundo orden de $f$.
\begin{align*}
\frac{\partial^{2}}{\partial x^{2}}f(x,y) &= \frac{\partial}{\partial x}\left(\frac{\partial}{\partial x}f(x,y)\right) = \frac{\partial}{\partial x}\left(e^{x+2y}(\cos(x^{2}+y^{2})-2x\sen(x^{2}+y^{2})\right) = \\
&= \frac{\partial}{\partial x}e^{x+2y}(\cos(x^{2}+y^{2})-2x\sen(x^{2}+y^{2})+\\
&+ e^{x+2y}\frac{\partial}{\partial x}(\cos(x^{2}+y^{2})-2x\sen(x^{2}+y^{2}) = \\
&= e^{x+2y}(\cos(x^{2}+y^{2})-2x\sen(x^{2}+y^{2})+\\
&+ e^{x+2y}(-\sen(x^{2}+y^{2})2x-2\sen(x^{2}+y^{2})-2x\cos(x^{2}+y^{2}))2x = \\
&= e^{x+2y}((1-4x^{2})\cos(x^{2}+y^{2})-(4x+2)\sen(x^{2}+y^{2})),\\
\frac{\partial^{2}}{\partial y\partial x}f(x,y) &= \frac{\partial}{\partial y}\left(\frac{\partial}{\partial x}f(x,y)\right) = \frac{\partial}{\partial y}\left(e^{x+2y}(\cos(x^{2}+y^{2})-2x\sen(x^{2}+y^{2})\right) = \\
&= \frac{\partial}{\partial y}e^{x+2y}(\cos(x^{2}+y^{2})-2x\sen(x^{2}+y^{2})+\\
&+ e^{x+2y}\frac{\partial}{\partial y}(\cos(x^{2}+y^{2})-2x\sen(x^{2}+y^{2}) = \\
&= e^{x+2y}2(\cos(x^{2}+y^{2})-2x\sen(x^{2}+y^{2})+\\
&+ e^{x+2y}(-\sen(x^{2}+y^{2})2y-2x\cos(x^{2}+y^{2}))2y = \\
&= e^{x+2y}((2-4xy)\cos(x^{2}+y^{2})-(4x+2y)\sen(x^{2}+y^{2})),\\
\end{align*}
\begin{align*}
\frac{\partial^{2}}{\partial x\partial y}f(x,y) &= \frac{\partial^{2}}{\partial y\partial x}\quad \mbox{(Igualdad de derivadas cruzadas),}\\
%
\frac{\partial^{2}}{\partial y^{2}}f(x,y) &= \frac{\partial}{\partial y}\left(\frac{\partial}{\partial y}f(x,y)\right) = \frac{\partial}{\partial y}\left(e^{x+2y}(2\cos(x^{2}+y^{2})-2y\sen(x^{2}+y^{2})\right) = \\
&= \frac{\partial}{\partial y}e^{x+2y}(2\cos(x^{2}+y^{2})-2y\sen(x^{2}+y^{2})+\\
&+ e^{x+2y}\frac{\partial}{\partial y}(2\cos(x^{2}+y^{2})-2y\sen(x^{2}+y^{2}) = \\
&= e^{x+2y}2(2\cos(x^{2}+y^{2})-2y\sen(x^{2}+y^{2})+\\
&+ e^{x+2y}(-2\sen(x^{2}+y^{2})2y-2\sen(x^{2}+y^{2})-2y\cos(x^{2}+y^{2}))2y = \\
&= e^{x+2y}((4-4y^{2})\cos(x^{2}+y^{2})-(8y+2)\sen(x^{2}+y^{2})).
\end{align*}
\end{enumerate}
Así pues la matriz hessiana es
\[Hf(x,y)= \left(
\begin{array}{cc}
\frac{\partial^{2}}{\partial x^{2}}f(x,y) & \frac{\partial^{2}}{\partial x\partial y}f(x,y)\\
\frac{\partial^{2}}{\partial y\partial x}f(x,y) & \frac{\partial^{2}}{\partial y^{2}}f(x,y)
\end{array}
\right) =
\]
\[=
e^{x+2y} \left(
\begin{array}[]{cc}
(1-4x^{2})\cos(x^{2}+y^{2})-(4x+2)\sen(x^{2}+y^{2}) & (2-4xy)\cos(x^{2}+y^{2})-(4x+2y)\sen(x^{2}+y^{2})\\
(2-4xy)\cos(x^{2}+y^{2})-(4x+2y)\sen(x^{2}+y^{2}) & (4-4y^{2})\cos(x^{2}+y^{2})-(8y+2)\sen(x^{2}+y^{2})
\end{array}
\right)
\]
En el origen de coordenadas, la matriz Hessiana es
\[
Hf(0,0)=\left(
\begin{array}[]{cc}
1 & 2 \\
2 & 4
\end{array}
\right)
\]
y el hessiano vale
\[
|Hf(0,0)|=\left|
\begin{array}[]{cc}
1 & 2 \\
2 & 4
\end{array}
\right| =
4-4 = 0.
\]
}


\newproblem{par-12}{gen}{*}
%ENUNCIADO
{Se  dice que la función $z(t,x,y)$ satisface la ecuación de ondas si verifica la ecuación en derivadas parciales:
\[
\frac{{\partial ^2 z}} {{\partial t^2 }} = k^2 \left(
{\frac{{\partial ^2 z}} {{\partial x^2 }} + \frac{{\partial ^2 z}}
{{\partial y^2 }}} \right)
\]
para algún $k\in \mathbb{R}$.

Comprobar que la función:
\[
z\left( {t,x,y} \right) = \cos (ax)\sen(by)\sen\left( {kt\sqrt
{a^2 + b^2 } } \right)
\]
donde $a,b,k \in \mathbb{R}$, satisface la ecuación de ondas.
}
%SOLUCIÓN
{Si la satisface.
}
%	RESOLUCIÓN
{Para comprobar que $z(t,x,y)$ satisface la ecuación de ondas vamos a calcular primero las derivadas parciales de segundo orden que aparecen en dicha ecuación:
\begin{align*}
\frac{\partial^2 z}{\partial t^2} &=
\frac{\partial}{\partial t}\left(\frac{\partial z}{\partial t}\right) =
\frac{\partial}{\partial t}\left(\frac{\partial}{\partial t}\left(\cos(ax)\sen(by)\sen(kt\sqrt{a^2+b^2})\right)\right)= \\
&= \frac{\partial}{\partial t}\left(\cos(ax)\sen(by)\frac{\partial}{\partial t}\left(\sen(kt\sqrt{a^2+b^2})\right)\right)=\\
&= \frac{\partial}{\partial t}\left(\cos(ax)\sen(by)\cos(kt\sqrt{a^2+b^2})\frac{\partial}{\partial t}(kt\sqrt{a^2+b^2})\right)=\\
&= \frac{\partial}{\partial t}\left(\cos(ax)\sen(by)\cos(kt\sqrt{a^2+b^2}) k\sqrt{a^2+b^2}\right)=\\
&= k\sqrt{a^2+b^2}\cos(ax)\sen(by)\frac{\partial}{\partial t}\left(\cos(kt\sqrt{a^2+b^2}) \right)=\\
&= k\sqrt{a^2+b^2}\cos(ax)\sen(by)(-\sen(kt\sqrt{a^2+b^2}))\frac{\partial}{\partial t}\left(kt\sqrt{a^2+b^2}\right)=\\
&=k\sqrt{a^2+b^2}\cos(ax)\sen(by)(-\sen(kt\sqrt{a^2+b^2}))k\sqrt{a^2+b^2}=\\
&= -k^2(a^2+b^2)\cos(ax)\sen(by)\sen(kt\sqrt{a^2+b^2}),\\[.5cm]
\frac{\partial^2 z}{\partial x^2} &=
\frac{\partial}{\partial x}\left(\frac{\partial z}{\partial x}\right)
= \frac{\partial}{\partial x}\left(\frac{\partial}{\partial x}\left(\cos(ax)\sen(by)\sen(kt\sqrt{a^2+b^2})\right)\right)= \\
&= \frac{\partial}{\partial x}\left(\frac{\partial}{\partial x}\left(\cos(ax)\right)\sen(by)\sen(kt\sqrt{a^2+b^2})\right)=\\
&= \frac{\partial}{\partial x}\left(-\sen(ax)a\sen(by)\sen(kt\sqrt{a^2+b^2})\right)=\\
&= \frac{\partial}{\partial x}\left(-\sen(ax)\right)a\sen(by)\cos(kt\sqrt{a^2+b^2}) =\\
&= -a^2\cos(ax)\sen(by)\cos(kt\sqrt{a^2+b^2}),\\[.5cm]
\frac{\partial^2 z}{\partial y^2} &=
\frac{\partial}{\partial y}\left(\frac{\partial z}{\partial y}\right)
= \frac{\partial}{\partial y}\left(\frac{\partial}{\partial y}\left(\cos(ax)\sen(by)\sen(kt\sqrt{a^2+b^2})\right)\right)= \\
&= \frac{\partial}{\partial y}\left(\cos(ax)\frac{\partial}{\partial y}\left(\sen(by)\right)\sen(kt\sqrt{a^2+b^2})\right)=\\
&= \frac{\partial}{\partial y}\left(\cos(ax)\cos(by)b\sen(kt\sqrt{a^2+b^2})\right)=\\
&= \cos(ax)\frac{\partial}{\partial y}\left(\cos(by)\right)b\cos(kt\sqrt{a^2+b^2}) =\\
&= -b^2\cos(ax)\sen(by)\cos(kt\sqrt{a^2+b^2}).
\end{align*}
Para terminar, sustituimos estas derivadas en la ecuación de ondas y constatamos que efectivamente se cumple
\begin{align*}
& -k^2(a^2+b^2)\cos(ax)\sen(by)\sen(kt\sqrt{a^2+b^2}) =\\
&= k^2\left(-a^2\cos(ax)\sen(by)\cos(kt\sqrt{a^2+b^2})-b^2\cos(ax)\sen(by)\cos(kt\sqrt{a^2+b^2})\right).
\end{align*}
}


\newproblem{par-13}{gen}{*}
%ENUNCIADO
{Dadas las siguientes funciones de dos variables:
\[
\begin{array}{*{20}c}
   {f(x,y) = x^2  - 2xy^2  + \sen(xy)}  \\
   {g(x,y) = \left( {2x+ 3y^2 } \right)e^{\left( {1 - x^2  - y^2 } \right)} }  \\

 \end{array}
\]
\begin{enumerate}
\item Calcular el gradiente de cada una de ellas.
\item ¿A cuál de las funciones corresponde el siguiente dibujo del gradiente en los puntos $(1,0)$, $(0,1)$, $(-1,0)$ y $(0,-1)$?
\begin{center}
\includegraphics[scale=0.45,angle=270]{img/vectores-par-13}
\end{center}
\end{enumerate}
}
%SOLUCIÓN
{\begin{enumerate}
\item $\nabla f(x,y) = \left(2x-2y^2+\cos(xy)y\, ,\, -4xy+\cos(xy)x\right)$  y
$\nabla g(x,y) = \left((-4x^2-6xy^2+2)\, ,\, (-4xy-6y^3+6y)\right)e^{1-x^2-y^2}$.
\item Los vectores gradientes son de la función $f$.
\end{enumerate}
}
%RESOLUCIÓN
{\begin{enumerate}
\item Para calcular el gradiente necesitamos calcular las derivadas parciales de $f$ y $g$ con respecto a sus variables:
\begin{align*}
\frac{\partial f}{\partial x}(x,y) &= \frac{\partial}{\partial
x}\left(x^2- 2xy^2+\sen(xy)\right)=
2x-2y^2+\cos(xy)\frac{\partial}{\partial x}(xy)=2x-2y^2+\cos(xy)y,\\
\frac{\partial f}{\partial y}(x,y) &= \frac{\partial}{\partial
y}\left(x^2- 2xy^2+\sen(xy)\right)=
-4xy+\cos(xy)\frac{\partial}{\partial y}(xy)=-4xy+\cos(xy)x,\\
\frac{\partial g}{\partial x}(x,y) &= \frac{\partial}{\partial x}\left((2x+3y^2)e^{1-x^2-y^2}\right)=
\frac{\partial}{\partial x}(2x+3y^2)e^{1-x^2-y^2}+(2x+3y^2)\frac{\partial}{\partial x}e^{1-x^2-y^2}=\\
&= 2e^{1-x^2-y^2}+(2x+3y^2)e^{1-x^2-y^2}\frac{\partial}{\partial x}\left(1-x^2-y^2\right) = \\
&= 2e^{1-x^2-y^2}+(2x+3y^2)e^{1-x^2-y^2}(-2x)= (-4x^2-6xy^2+2)e^{1-x^2-y^2},\\
\frac{\partial g}{\partial y}(x,y) &= \frac{\partial}{\partial y}\left((2x+3y^2)e^{1-x^2-y^2}\right)=
\frac{\partial}{\partial x}(2x+3y^2)e^{1-x^2-y^2}+(2x+3y^2)\frac{\partial}{\partial y}e^{1-x^2-y^2}=\\
&= 6y e^{1-x^2-y^2}+(2x+3y^2)e^{1-x^2-y^2}\frac{\partial}{\partial y}\left(1-x^2-y^2\right) =\\
&=6ye^{1-x^2-y^2}+(2x+3y^2)e^{1-x^2-y^2}(-2y)= (-4xy-6y^3+6y)e^{1-x^2-y^2}.
\end{align*}
Así pues, los gradientes son
\begin{align*}
\nabla f(x,y) &=\left(\frac{\partial f}{\partial x}(x,y),\frac{\partial
f}{\partial y}(x,y)\right) = \left(2x-2y^2+\cos(xy)y\, ,\, -4xy+\cos(xy)x\right) \\
\nabla g(x,y) &=\left(\frac{\partial g}{\partial x}(x,y),\frac{\partial
g}{\partial y}(x,y)\right) = \left((-4x^2-6xy^2+2)\, ,\, (-4xy-6y^3+6y)\right)e^{1-x^2-y^2}
\end{align*}

\item Para ver a qué función corresponde la gráfica, calculamos el gradiente en los puntos que nos dan
\begin{align*}
\nabla f(1,0) &= \left(2\cdot 1-2\cdot 0^2+\cos(1\cdot 0)\cdot 0\, ,\, -4\cdot1\cdot0+\cos(1\cdot 0)\cdot1\right) =(2,1),\\
\nabla g(1,0) &= \left((-4\cdot1^2-6\cdot 1\cdot 0^2+2)\, ,\, (-4\cdot 1\cdot 0-6\cdot 0^3+6\cdot 0)\right)e^{1-1^2-0^2}= (-2,0).
\end{align*}
Como el vector libre situado en el punto $(1,0)$ es el $(2,1)$, la gráfica no puede pertenecer a la función $g(x,y)$. Para asegurarnos que se corresponde con la $f(x,y)$, calculamos el gradiente de esta función en el resto de los puntos:
\begin{align*}
\nabla f(0,1) &= \left(2\cdot 0-2\cdot 1^2+\cos(0\cdot 1)\cdot 1\, ,\, -4\cdot0\cdot1+\cos(0\cdot 1)\cdot0\right) =(-1,0),\\
\nabla f(-1,0) &= \left(2\cdot (-1)-2\cdot 0^2+\cos(-1\cdot 0)\cdot 0\, ,\, -4\cdot-1\cdot0+\cos(-1\cdot 0)\cdot(-1)\right) =(-2,-1),\\
\nabla f(0,-1) &= \left(2\cdot 0-2\cdot (-1)^2+\cos(0\cdot (-1))\cdot (-1)\, ,\, -4\cdot0\cdot(-1)+\cos(0\cdot (-1))\cdot0\right) =(-3,0).
\end{align*}
Luego los vectores de la gráfica se corresponden con los vectores gradientes de $f(x,y)$.
\end{enumerate}
}


\newproblem{par-14}{gen}{}
%ENUNCIADO
{Tenemos dos objetos de masas $m_1$ y $m_2$ unidas por una cuerda que pasa a través de una polea como la de la figura.
\begin{center}
  \includegraphics[scale=0.5]{img/polea-par-14}
\end{center}
Si $m_1\geq m_2$, la aceleración de $m_1$ viene dada por la ecuación
\[
a=\frac{m_1-m_2}{m_1+m_2}g,
\]
siendo $g$ la aceleración de la gravedad.
Demostrar que se cumple la ecuación
\[
m_1\frac{\partial a}{\partial m_1}+m_2\frac{\partial a}{\partial m_2}=0.
\]
}
%SOLUCIÓN
{$\dfrac{\partial a}{\partial m_1} = \dfrac{2gm_2}{(m_1+m_2)^2}$ y $\dfrac{\partial a}{\partial m_2} = \dfrac{-2gm_1}{(m_1+m_2)^2}$.
}
%RESOLUCIÓN
{
}


\newproblem{par-15}{gen}{*}
%ENUNCIADO
{La relación que modeliza el potencial eléctrico $V$ de un punto del plano en función de su distancia, es $V=\log D$, donde $D=\sqrt{x^2+y^2}$.

Se pide:
\begin{enumerate}
\item Calcular el gradiente de $V$.
\item Hallar la dirección de máxima variación del potencial
eléctrico en el punto $(x,y)=(\sqrt{3},\sqrt{3})$.
\item Calcular la matriz Hessiana y el Hessiano de $V$ en el punto anterior.
\item Si nos movemos a lo largo de la curva $y=x+1$, cuál será el mínimo potencial alcanzado?
\end{enumerate}
}
%SOLUCIÓN
{\begin{enumerate}
\item $\nabla V(x,y) = \left( \frac{x}{x^2+y^2},\frac{y}{x^2+y^2}\right)$.
\item $\nabla V(\sqrt 3, \sqrt 3) = \sqrt 3 /6(1,1)$.
\item $
HV(x,y) = \left(
\begin{array}{cc}
\frac{y^2-x^2}{y^4+2x^2y^2+x^4} & \frac{-2xy}{y^4+2x^2y^2+x^4} \\
\frac{-2xy}{y^4+2x^2y^2+x^4} & \frac{x^2-y^2}{y^4+2x^2y^2+x^4}
\end{array}
\right),\quad
\left(
\begin{array}{cc}
0 & -1/6 \\
-1/6 & 0
\end{array}
\right),\quad \mbox{y }
|H(\sqrt 3,\sqrt 3)| = -1/36.
$
\item El potencial máximo se alcanza en $(x=-1/2, y=1/2)$ y vale $V(-1/2,1/2) = -\dfrac{\log 2}{2}$.
\end{enumerate}
}
%RESOLUCIÓN
{
}


\newproblem*{par-16}{qui}{}
%ENUNCIADO
{La ecuación unidimensional del calor es
\[
\frac{\partial q}{\partial t}=c^2\frac{\partial^2q}{\partial x^2},
\]
donde $c$ es una constante y $q(x,t)$ es la temperatura de una varilla en un punto que ocupa la posición $x$ en el instante $t$. Demostrar que $q(x,t)=e^{ax+bt}$, con $a\neq 0$, satisface dicha ecuación para un valor apropiado de $c$.
}


\newproblem*{par-17}{amb}{*}
%ENUNCIADO
{Suponiendo que la temperatura, $T$ en ºC, de una zona de la atmósfera es función de la densidad del aire, $d$, en g por cm$^3$, la altura, $h$, en kilómetros, y de la concentración de un determinado elemento, $c$, en mg por cm$^3$, viene dada por la expresión:
\[
T(d,h,c) = \frac{{\ln (dh)}}{c} + c^2 3^{hd}
\]
\begin{enumerate}
\item Suponiendo que la altura a la que medimos la temperatura es de un kilómetro, y que la temperatura medida es de 0 ºC, dar la expresión de la concentración en función de la densidad.
\item Calcular el gradiente de la temperatura en el punto $(d_0,h_0,c_0)=(1,1,2)$.
\item Comprobar que se cumple que:
\[
\frac{{\partial ^2 T}}{{\partial d\partial h}} = \frac{{\partial ^2
T}}{{\partial h\partial d}}
\]
\end{enumerate}
}


\newproblem{par-18}{gen}{}
%ENUNCIADO
{Sea $z(x,y)=\dfrac{x^{2}}{y}+\dfrac{y^{2}}{x}.$ Calcular todas sus derivadas parciales de primer y segundo orden.
}
%SOLUCIÓN
{$\frac{\partial z}{\partial x} = \frac{2x}{y}-\frac{y^2}{x^2}$, $\dfrac{\partial z}{\partial x} = \frac{2y}{x}-\frac{x^2}{y^2}$,\\
$\frac{\partial^2 z}{\partial x^2} = \frac{2y^2}{x^3}+\frac{2}{y}$, $\frac{\partial^2 z}{\partial y\partial x} = -\frac{2y}{x^2}-\frac{2x}{y^2}$, $\frac{\partial^2 z}{\partial x\partial y} = -\frac{2y}{x^2}-\frac{2x}{y^2}$, $\frac{\partial^2 z}{\partial x^2} = \frac{2x^2}{y^3}+\frac{2}{x}$.
}
%RESOLUCIÓN
{
}


\newproblem*{par-19}{gen}{}
%ENUNCIADO
{Dada la función $f(x,y)=\dfrac{x-y}{x+y}$, hallar $\dfrac{\partial f}{\partial x}$ y $\dfrac{\partial f}{\partial y}$ en el punto $(2,-1)$.
}


\newproblem*{par-20}{amb}{*}
%ENUNCIADO
{Supongamos la función de varias variables $f(x,y,z)=x^{3}+\sqrt{xyz}$ que da la presión en un recipiente en función de la posición $(x,y,z)$. Suponiendo que en el recipiente hay un insecto y que se encuentra en el punto de coordenadas $(2,1,3)$, ¿en qué dirección debe moverse si busca ir lo más rápidamente posible hacia zonas de menor presión?
}


\newproblem*{par-21}{gen}{}
%ENUNCIADO
{Dado el siguiente campo escalar expresado en coordenadas cartesianas:
\[
f(x,y,z)=3xy\ln \left( \dfrac{1}{z}\right)
\]
Calcular:
\begin{enumerate}
\item  Su vector gradiente.
\item  Su matriz Hessiana.
\end{enumerate}
}


\newproblem*{par-22}{gen}{*}
%ENUNCIADO
{La definición del polinomio de Taylor de grado 2 de una función de dos variables, $f(x,y)$, centrado en el punto $(x_{0},y_{0})$, es
\begin{align*}
P_{f,(x_0,y_0)}^{2}(x,y)&= f(x_{0},y_{0})+\dfrac{\partial f(x_{0},y_{0})}{\partial x}(x-x_{0})+\dfrac{\partial f(x_{0},y_{0})}{\partial y}(y-y_{0})+\\
&+\dfrac{1}{2}\dfrac{\partial ^{2}f(x_{0},y_{0})}{\partial x^{2}}(x-x_{0})^{2}+\dfrac{1}{2}\dfrac{\partial ^{2}f(x_{0},y_{0})}{\partial y^{2}}(y-y_{0})^{2}+\dfrac{\partial ^{2}f(x_{0},y_{0})}{\partial x\partial y}(x-x_{0})(y-y_{0})
\end{align*}
\begin{enumerate}
\item  Utilizar la fórmula anterior para calcular el polinomio de Taylor de grado 2 de la función $f(x,y)=e^{(x+2y)}$ centrado en $(x_{0},y_{0})=(0,0)$.
\item  Utilizar el polinomio anterior para dar el valor aproximado de $e^{(0.1+2\cdot 0.1)}$.
\end{enumerate}
}


\newproblem*{par-23}{fis}{*}
%ENUNCIADO
{Suponiendo que el potencial eléctrico en un punto de coordenadas cartesianas $(x,y,z)$ viene dado por:
\[
V(x,y,z) = \frac{1} {{x{\kern 1pt} e^y \ln z}},
\]
calcular en el punto $(1,0,e)$:
\begin{enumerate}
\item El campo eléctrico (recordar que el campo eléctrico es el gradiente del potencial cambiado de signo: $\vec E =  - \vec\nabla V$).
\item La divergencia del campo eléctrico.
\end{enumerate}
}


\newproblem*{par-24}{gen}{*}
%ENUNCIADO
{Para la función de 2 variables $f(x,y) = x^{y^2}$
\begin{enumerate}
\item Calcular su dirección y sentido de máximo crecimiento en el punto $(1,1)$.
\item Calcular su matriz Hessiana.
\end{enumerate}
}


\newproblem{par-25}{amb}{*}
%ENUNCIADO
{La Quimiotaxis es el movimiento de los organismos dirigido por un gradiente de concentración, es decir, en la dirección
en la que la concentración aumenta con más rapidez. El moho del cieno Dictyoselium discoideum muestra este
comportamiento. En esta caso, las amebas unicelulares de esta especie se mueven según el gradiente de concentración de
una sustancia química denominada adenosina monofosfato (AMP cíclico). Si suponemos que la expresión que da la
concentración de AMP cíclico en un punto de coordenadas $(x,y,z)$ es:
\[
C(x,y,z) = \frac{4} {{\sqrt {x^2  + y^2  + z^4  + 1} }}
\]
y se sitúa una ameba de moho del cieno en el punto $(-1,0,1)$, ¿en qué dirección se moverá la ameba?
}
%SOLUCIÓN
{$(4/\sqrt{27}, 0, -8/\sqrt{27})$.
}
%RESOLUCIÓN
{
}



%%%%%%% Pendiente 26



\newproblem{par-27}{qui}{*}
%ENUNCIADO
{Supongamos que tenemos una superficie plana, y que la cantidad de una sustancia, $C$ en g/cm$^2$,
depositada sobre cada punto de coordenadas $x$ e $y$, en metros, es función del punto y del tiempo $t$, en horas, y
viene dada por la expresión:
\[
C(x,y,t) = \sqrt{e^{-\frac{3ty}{x^2+1}}}
\]
\begin{enumerate}
\item Calcular la dirección y sentido de máximo crecimiento de la
función en el punto $(x_0,y_0,t_0)=(1,0,1)$.
\item Calcular: $\dfrac{{\partial ^2 C}}{{\partial y\partial x}}$.
\item ¿En qué puntos se anulará el gradiente de $C$?
\end{enumerate}
}
%SOLUCIÓN
{\begin{enumerate}
\item $\nabla C(1,0,1) =\frac{1}{4}(0,-3,0)$.
\item $\displaystyle \frac{\partial^2 C}{\partial y\partial x} =
\frac{e^{-\frac{3ty}{2x^2+2}}}{(2x^2+2)^2}\left(\frac{-36t^2yx}{2x^2+2}+12tx\right)$.
\item En los puntos de la forma $(x,0,0)\ \forall x\in \mathbb{R}$.
\end{enumerate}
}
%RESOLUCIÓN
{Antes de nada conviene simplificar la función:
\[
C(x,y,t) = \sqrt{e^{-\frac{3ty}{x^2+1}}} = \left(e^{-\frac{3ty}{x^2+1}}\right)^{1/2} = e^{-\frac{3ty}{2x^2+2}}
\]
\begin{enumerate}
\item La dirección y sentido de máximo crecimiento de una función de varias variables la da el vector gradiente, en este caso,
\[
\nabla C(x,y,t) =\left(\frac{\partial C}{\partial x}, \frac{\partial C}{\partial y}, \frac{\partial C}{\partial t} \right)
\]
Calulamos las tres derivadas parciales:
\begin{align*}
\frac{\partial C}{\partial x} &= \frac{\partial}{\partial x} e^{-\frac{3ty}{2x^2+2}} = e^{-\frac{3ty}{2x^2+2}} \frac{\partial}{\partial x}\left(-\frac{3ty}{2x^2+2}\right) = e^{-\frac{3ty}{2x^2+2}}\frac{3ty\cdot 4x}{(2x^2+2)^2} \\
\frac{\partial C}{\partial y} &= \frac{\partial}{\partial y} e^{-\frac{3ty}{2x^2+2}} = e^{-\frac{3ty}{2x^2+2}} \frac{\partial}{\partial y}\left(-\frac{3ty}{2x^2+2}\right) = e^{-\frac{3ty}{2x^2+2}}\frac{-3t}{2x^2+2} \\
\frac{\partial C}{\partial t} &= \frac{\partial}{\partial t} e^{-\frac{3ty}{2x^2+2}} = e^{-\frac{3ty}{2x^2+2}} \frac{\partial}{\partial t}\left(-\frac{3ty}{2x^2+2}\right) = e^{-\frac{3ty}{2x^2+2}}\frac{-3y}{2x^2+2}
\end{align*}
De modo que el vector gradiente es
\[
\nabla C(x,y,t) =\frac{e^{-\frac{3ty}{2x^2+2}}}{2x^2+2}\left(\frac{12tyx}{2x^2+2}, -3t, -3y\right),
\]
y en el punto $(x_0,y_0,t_0)=(1,0,1)$ vale
\[
\nabla C(1,0,1) =\frac{e^{-\frac{3\cdot 1\cdot 0}{2\cdot 1^2+2}}}{2\cdot 1^2+2}\left(\frac{12\cdot 1\cdot 0\cdot 1}{2\cdot 1^2+2}, -3\cdot 1, -3\cdot 0\right) = \frac{1}{4}(0,-3,0).
\]

\item
\begin{align*}
\frac{\partial^2 C}{\partial y\partial x} &= \frac{\partial}{\partial y}\frac{\partial C}{\partial x} e^{-\frac{3ty}{2x^2+2}} = \frac{\partial}{\partial y}  \left(e^{-\frac{3ty}{2x^2+2}}\frac{12tyx}{(2x^2+2)^2}\right) = \\
&= \frac{\partial}{\partial y} \left(e^{-\frac{3ty}{2x^2+2}}\right)\frac{12tyx}{(2x^2+2)^2}+e^{-\frac{3ty}{2x^2+2}}\frac{\partial}{\partial y}\left(\frac{12tyx}{(2x^2+2)^2}\right) = \\
&= e^{-\frac{3ty}{2x^2+2}}\frac{\partial}{\partial y}\left(-\frac{3ty}{2x^2+2}\right)\frac{3ty\cdot 4x}{(2x^2+2)^2}+e^{-\frac{3ty}{2x^2+2}}\frac{12tx}{(2x^2+2)^2} = \\
&= e^{-\frac{3ty}{2x^2+2}}\frac{-3t}{2x^2+2}\frac{3ty\cdot 4x}{(2x^2+2)^2}+e^{-\frac{3ty}{2x^2+2}}\frac{12tx}{(2x^2+2)^2} = \\
&= \frac{e^{-\frac{3ty}{2x^2+2}}}{(2x^2+2)^2}\left(\frac{-36t^2yx}{2x^2+2}+12tx\right).
\end{align*}

\item
\[
\nabla C(x,y,t) =\frac{e^{-\frac{3ty}{2x^2+2}}}{2x^2+2}\left(\frac{12tyx}{2x^2+2}, -3t, -3y\right) = (0,0,0) \Leftrightarrow
\left\{
\begin{array}{l}
12txy =0 \\
-3t = 0\\
-3y = 0
\end{array}
\right.
\]
de donde se deduce que $t=0$, $y=0$ y $x$ puede tomar cualquier valor. Así pues, los puntos que anulan el gradiente son de la forma $(x,0,0)$, $x\in\mathbb{R}$.
\end{enumerate}
}


\newproblem{par-28}{fis}{*}
%ENUNCIADO
{Una barra de metal de un metro de largo se calienta de manera irregular y de forma tal que a $x$ metros de su extremo izquierdo y en el instante $t$ minutos, su temperatura en grados centígrados esta dada por $H(x,t) = 100e^{-0.1t}\sen(\pi xt)$ con $0\leq x \leq 1$.
\begin{enumerate}
\item Calcular $\dfrac{\partial H}{\partial x}(0.2, 1)$ y $\dfrac{\partial H}{\partial x}(0.8, 1).$ ¿Cuál es la interpretación práctica (en términos de temperatura) de estas derivadas parciales? Explicar por qué cada una tiene el signo que tiene.
\item Calcular la matriz hessiana de $H$.
\end{enumerate}
}
%SOLUCIÓN
{\begin{enumerate}
\item $\frac{\partial H}{\partial x}(0.2,\, 1) = 100e^{-0.1}\cos(0.2\pi) \pi = 229.9736$ \\
$\frac{\partial H}{\partial x}(0.8,\, 1) = 100e^{-0.1}\cos(0.8\pi) \pi = -229.9736$.
\item $
\left(
\begin{array}{cc}
-100e^{-0.1t}\pi^2 t^2\sen(\pi xt) & 100e^{-0.1t}\left((-0.1\pi t+\pi)\cos(\pi xt) - \pi^2 xt \sen(\pi xt)\right) \\
100e^{-0.1t}\left((-0.1\pi t+\pi)\cos(\pi xt) - \pi^2 xt \sen(\pi xt)\right) & 100e^{-0.1t}\left(0.01\sen(\pi xt) -(0.2+\pi^2x^2) \cos(\pi xt)\right)
\end{array}
\right)$
\end{enumerate}
}
%RESOLUCIÓN
{\begin{enumerate}
\item La derivada parcial de $H$ con respecto a $x$ es
\begin{align*}
\frac{\partial H}{\partial x}(x,t) &= 100e^{-0.1t}\cos(\pi xt) \pi t \\
\end{align*}
y en los puntos que nos piden vale
\begin{align*}
\frac{\partial H}{\partial x}(0.2,\, 1) &= 100e^{-0.1}\cos(0.2\pi) \pi =
229.9736\\
\frac{\partial H}{\partial x}(0.8,\, 1) &= 100e^{-0.1}\cos(0.8\pi) \pi =
-229.9736
\end{align*}
La derivada parcial $\dfrac{\partial H}{\partial x}(x_0,t_0)$ indica la variación instantánea que experimenta la temperatura con respecto a la variación de la distancia al extremo izquierdo en el punto. El signo de la derviada parcial indica si la variación de la temperatura es creciente (aumenta la temperatura) o decreciente (disminuye). Así en el punto $(0.2,\, 1)$ la temperatura aumentará a razón de $229.9736$ grados centígrados por cada metro que nos alejemos del extremo izquierdo de la barra de metal, mientras que en el $(0.8,\,1)$ la temperatura disminuirá a razón de $229.9736$ grados centígrados por cada metro que nos alejemos del extremo izquierdo de la barra de metal.

\item Para calcular la matriz Hessiana necesitamos las derivadas parciales de
segundo orden:
\begin{align*}
\frac{\partial H}{\partial t} (x,t) &= 100\left(\frac{\partial}{\partial x} e^{-0.1 t} \sen (\pi xt) + e^{-0.1t}\frac{\partial}{\partial x}\sen(\pi xt)\right)=\\
&= 100\left(-0.1e^{-0.1t}\sen(\pi xt) +e^{-0.1t}\cos(\pi xt)\pi x\right) =\\
&= 100 e^{-0.1t}\left(-0.1 \sen(\pi xt) + \pi x \cos(\pi xt)\right),\\
\frac{\partial^2 H}{\partial x^2}(x,t) &= \frac{\partial}{\partial x}\left(100e^{-0.1t}\pi t\cos(\pi xt) \right) = 100e^{-0.1t}\pi t(-\sen(\pi xt) \pi t) =\\
&= -100e^{-0.1t}\pi^2 t^2\sen(\pi xt),\\
\frac{\partial^2 H}{\partial t\partial x}(x,t) &= \frac{\partial}{\partial t}\left(100e^{-0.1t}\pi t\cos(\pi xt) \right) =\\
&= 100\left(\frac{\partial}{\partial t}e^{-0.1t}\pi t\cos(\pi xt) + e^{-0.1t}\left(\frac{\partial}{\partial t}(\pi t)\cos(\pi xt) + \pi t \frac{\partial}{\partial t}\cos(\pi xt)\right) \right) =\\
&= 100\left(-0.1e^{-0.1t}\pi t\cos(\pi xt) + e^{-0.1t}\left(\pi \cos(\pi xt) - \pi t \sen(\pi xt)\pi x\right) \right) =\\
&= 100e^{-0.1t}\left(-0.1\pi t\cos(\pi xt)+\pi \cos(\pi xt) - \pi^2 xt \sen(\pi xt)\right) = \\
&= 100e^{-0.1t}\left((-0.1\pi t+\pi)\cos(\pi xt) - \pi^2 xt \sen(\pi xt)\right),\\
\end{align*}

\begin{align*}
\frac{\partial^2 H}{\partial x\partial t}(x,t) &= \frac{\partial^2 H}{\partial t\partial x}(x,t) \quad (\mbox{igualdad de las derivadas cruzadas por el teorema de Schwartz})\\
\frac{\partial^2 H}{\partial t^2}(x,t) &= \frac{\partial}{\partial t} \left(100 e^{-0.1t}\left(-0.1 \sen(\pi xt) + \pi x \cos(\pi xt)\right)\right)  =\\
&= 100\left(\frac{\partial}{\partial t} e^{-0.1t}\left(-0.1 \sen(\pi xt) + \pi x \cos(\pi xt)\right) +\right.\\
&\left. \qquad + e^{0.1t}\left(\frac{\partial}{\partial t}\left(-0.1\sen(\pi xt)\right) + \frac{\partial}{\partial t}\left(\pi x \cos(\pi xt)\right)\right)\right) =\\
&= 100\left(-0.1 e^{-0.1t}\left(-0.1 \sen(\pi xt) + \pi x \cos(\pi xt)\right)\right. +\\
&\left. \qquad + e^{0.1t}\left(-0.1\cos(\pi xt)\pi x - \pi x \cos(\pi xt)\pi x\right)\right) =\\
&= 100e^{-0.1t}\left(0.01\sen(\pi xt) -0.1 \pi x \cos(\pi xt) -0.1\pi x\cos(\pi xt) - \pi^2 x^2 \cos(\pi xt)\right) = \\
&= 100e^{-0.1t}\left(0.01\sen(\pi xt) -(0.2+\pi^2x^2) \cos(\pi xt)\right).
\end{align*}
Así pues, la matriz Hessiana es
\[
\left(
\begin{array}{cc}
-100e^{-0.1t}\pi^2 t^2\sen(\pi xt) & 100e^{-0.1t}\left((-0.1\pi t+\pi)\cos(\pi xt) - \pi^2 xt \sen(\pi xt)\right) \\
100e^{-0.1t}\left((-0.1\pi t+\pi)\cos(\pi xt) - \pi^2 xt \sen(\pi xt)\right) & 100e^{-0.1t}\left(0.01\sen(\pi xt) -(0.2+\pi^2x^2) \cos(\pi xt)\right)
\end{array}
\right)
\]
\end{enumerate}
}


\newproblem{par-29}{gen}{*}
%ENUNCIADO
{Dar la dirección de máximo crecimiento de la función
\[
f(x,y,z) = \frac{\log(zx)}z-xe^{-zxy}
\]
en el punto $(1,1,1)$.
}
%SOLUCIÓN
{$\nabla f(1,1,1)=(1,e^{-1},1+e^{-1})$.
}
%RESOLUCIÓN
{La dirección de máximo crecimiento de una función de varias variables la da el vector gradiente:
\[
\nabla f(x,y,z) = \left(\frac{\partial f}{\partial x}(x,y,z),\frac{\partial f}{\partial y}(x,y,z),\frac{\partial f}{\partial z}(x,y,z)\right)
\]
Calculamos por tanto cada una de las derivadas parciales que aparecen en las componentes del vector:
\begin{align*}
\frac{\partial f}{\partial x}(x,y,z) &= \frac{\partial}{\partial x}(\frac{\log (zx)}z-xe^{-zxy}) = \frac{\partial}{\partial x}(\frac{\log (zx)}z)-\frac{\partial}{\partial x}(xe^{-zxy})= \\
&= \frac{1}{z}\frac{\partial}{\partial x}(\log (zx))-(\frac{\partial}{\partial x}(x)e^{-zxy}+x\frac{\partial}{\partial x}(e^{-zxy}))= \\
&= \frac{1}{z}\frac{1}{zx}\frac{\partial}{\partial x}(zx)-(e^{-zxy}+xe^{-zxy}\frac{\partial}{\partial x}(-zxy))= \\
&= \frac{1}{z}\frac{1}{zx}z-(e^{-zxy}+xe^{-zxy}(-zy)) = \frac{1}{zx}-e^{-zxy}(1-xyz),\\
\frac{\partial f}{\partial y}(x,y,z) &= \frac{\partial}{\partial y}(\frac{\log(zx)}z-xe^{-zxy}) = \frac{\partial}{\partial y}(\frac{\log (zx)}z)-\frac{\partial}{\partial y}(xe^{-zxy})= \\
&= -x\frac{\partial}{\partial y}(e^{-zxy}) = -xe^{-zxy}\frac{\partial}{\partial y}(-zxy)=x^2ze^{-zxy},\\
\frac{\partial f}{\partial z}(x,y,z) &= \frac{\partial}{\partial z}(\frac{\log(zx)}z-xe^{-zxy}) = \frac{\partial}{\partial z}(\frac{\log (zx)}z)-\frac{\partial}{\partial z}(xe^{-zxy})= \\
&= \frac{\frac{\partial}{\partial z}(\log (zx))z-\log (zx)\frac{\partial}{\partial z}(z)}{z^2}-x\frac{\partial}{\partial z}(e^{-zxy}))= \\
&= \frac{\frac 1{zx}\frac \partial {\partial z}(zx)z-\log (zx)}{z^2}-xe^{-zxy}\frac{\partial}{\partial z}(-zxy))= \\
&= \frac{\frac 1{zx}xz-\log (zx)}{z^2}-xe^{-zxy}-xy=\frac{1-\log (zx)}{z^2}+x^2ye^{-zxy}.
\end{align*}
Por lo tanto, el vector gradiente será:
\[
\nabla f(x,y,z)=(\frac{1}{zx}-e^{-zxy}(1-xyz), x^2ze^{-zxy}, \frac{1-\log (zx)}{z^2}+x^2ye^{-zxy})
\]

Finalmente, como nos pieden la dirección de máximo crecimiento en el punto $(1,1,1)$, tendremos que particularizar el vector gradiente en dicho punto, es decir:
\[
\nabla f(1,1,1)=(1,e^{-1},1+e^{-1}).
\]
}


\newproblem{par-30}{gen}{*}
%ENUNCIADO
{Calcular el gradiente de la función
\[
f(x,y,z)=e^{\sqrt{x^2+2yz}}+\ln (\frac{xy}z)
\]
en el punto $(1,-2,-2)$.
}
%SOLUCIÓN
{$\nabla f(x,y,z)=\left(\frac{xe^{\sqrt{x^2+2yz}}}{\sqrt{x^2+2yz}}+\frac{1}{x}, \frac{ze^{\sqrt{x^2+2yz}}}{\sqrt{x^2+2yz}}+\frac{1}{y}, \frac{ye^{\sqrt{x^2+2yz}}}{\sqrt{x^2+2yz}}-\frac{1}{z}\right)$\\ y $\nabla f(1,-2,-2)=\left(\frac{e^3}{3}+1,\frac{-2e^3}{3}-\frac{1}{2},\frac{-2e^3}{3}+\frac{1}{2}\right)$.}
%RESOLUCIÓN
{El gradiente de $f(x,y,z)$ se define como el vector $\nabla f(x,y,z)=\left(\dfrac{\partial f}{\partial x}(x,y,z),\dfrac{\partial f}{\partial y}(x,y,z),\dfrac{\partial f}{\partial z}(x,y,z)\right).$ Por tanto, tenemos que calcular las tres derivadas parciales siguientes:
\begin{align*}
\dfrac{\partial f}{\partial x}(x,y,z) &= \dfrac{\partial}{\partial x}(e^{\sqrt{x^2+2yz}}+\ln (\frac{xy}z)) = \dfrac{\partial}{\partial x}(e^{\sqrt{x^2+2yz}})+\dfrac{\partial}{\partial x}(\ln (\frac{xy}z))= \\
&= e^{\sqrt{x^2+2yz}}\dfrac \partial {\partial x}(\sqrt{x^2+2yz})+\frac{1}{xy/z}\dfrac{\partial}{\partial x}(\frac{xy}z)= \\
&= e^{\sqrt{x^2+2yz}}\frac{1}{2\sqrt{x^2+2yz}}\dfrac{\partial}{\partial x}(x^2+2yz)+\frac{z}{xy}\frac{y}{z}= \\
&= e^{\sqrt{x^2+2yz}}\frac{1}{2\sqrt{x^2+2yz}}2x+\frac{1}{x} = \frac{xe^{\sqrt{x^2+2yz}}}{\sqrt{x^2+2yz}}+\frac{1}{x}, \\
\dfrac{\partial f}{\partial y}(x,y,z) &= \dfrac{\partial}{\partial y}(e^{\sqrt{x^2+2yz}}+\ln (\frac{xy}z)) = \dfrac{\partial}{\partial y}(e^{\sqrt{x^2+2yz}})+\dfrac{\partial}{\partial y}(\ln (\frac{xy}z))= \\
&= e^{\sqrt{x^2+2yz}}\dfrac{\partial}{\partial y}(\sqrt{x^2+2yz})+\frac{1}{xy/z}\dfrac{\partial}{\partial y}(\frac{xy}z)= \\
&= e^{\sqrt{x^2+2yz}}\frac{1}{2\sqrt{x^2+2yz}}\dfrac{\partial}{\partial y}(x^2+2yz)+\frac{z}{xy}\frac{x}{z}= \\
&= e^{\sqrt{x^2+2yz}}\frac{1}{2\sqrt{x^2+2yz}}2z+\frac{1}{y}=\frac{ze^{\sqrt{x^2+2yz}}}{\sqrt{x^2+2yz}}+\frac{1}{y}, \\
\dfrac{\partial f}{\partial z}(x,y,z) &= \dfrac{\partial}{\partial z}(e^{\sqrt{x^2+2yz}}+\ln (\frac{xy}z)) = \dfrac{\partial}{\partial z}(e^{\sqrt{x^2+2yz}})+\dfrac{\partial}{\partial z}(\ln (\frac{xy}{z}))= \\
&= e^{\sqrt{x^2+2yz}}\dfrac{\partial}{\partial z}(\sqrt{x^2+2yz})+\frac{1}{xy/z}\dfrac{\partial}{\partial z}(\frac{xy}{z})= \\
&= e^{\sqrt{x^2+2yz}}\frac{1}{2\sqrt{x^2+2yz}}\dfrac{\partial}{\partial z}(x^2+2yz)+\frac{z}{xy}\frac{-xy}{z^2}= \\
&= e^{\sqrt{x^2+2yz}}\frac{1}{2\sqrt{x^2+2yz}}2y-\frac{1}{z} = \frac{ye^{\sqrt{x^2+2yz}}}{\sqrt{x^2+2yz}}-\frac{1}{z},
\end{align*}
y, en consecuencia tenemos
\[
\nabla f(x,y,z)=\left(\frac{xe^{\sqrt{x^2+2yz}}}{\sqrt{x^2+2yz}}+\frac{1}{x}, \frac{ze^{\sqrt{x^2+2yz}}}{\sqrt{x^2+2yz}}+\frac{1}{y}, \frac{ye^{\sqrt{x^2+2yz}}}{\sqrt{x^2+2yz}}-\frac{1}{z}\right).
\]
Como nos piden el gradiente en el punto $(1,-2,-2),$ sustituimos $x$ por 1, $y$ por -2, y $z$ por -2 en el vector anterior y obtenemos
\[
\nabla f(1,-2,-2)=\left(\frac{e^3}{3}+1,\frac{-2e^3}{3}-\frac{1}{2},\frac{-2e^3}{3}+\frac{1}{2}\right).
\]
}


\newproblem{par-31}{gen}{*}
%ENUNCIADO
{Calcular el vector gradiente de la función
\[
\log \left( \sqrt{x^{2}-z^{2}}\right) +3^{\tfrac{x^{2}}{y}}
\]
en el punto $(1,1,0)$.
}
%SOLUCIÓN
{$\nabla f(x,y,z)=(\frac{x}{x^{2}-z^{2}}+3^{\tfrac{x^{2}}{y}}\log 3\dfrac{2x}{y},3^{\tfrac{x^{2}}{y}}\log 3\dfrac{-x^{2}}{y^{2}},-\frac{z}{x^{2}-z^{2}})$, y $\nabla f(1,-2,-2)=(1+6\log 3,-3\log 3,0)$.
}
%RESOLUCIÓN
{El gradiente de $f(x,y,z)$ se define como el vector $\nabla f(x,y,z)=(\dfrac{\partial f}{\partial x}(x,y,z),\dfrac{\partial f}{\partial y}(x,y,z),\dfrac{\partial f}{\partial z}(x,y,z))$. Por tanto, tenemos que calcular las tres derivadas parciales siguientes:
\begin{align*}
\dfrac{\partial f}{\partial x}(x,y,z) &= \dfrac{\partial }{\partial x}(\log\left(\sqrt{x^{2}-z^{2}}\right) +3^{\tfrac{x^{2}}{y}}) = \dfrac{\partial }{\partial x}(\log \left(\sqrt{x^{2}-z^{2}}\right) )+\dfrac{\partial }{\partial x}(3^{\tfrac{x^{2}}{y}})= \\
&= \frac{1}{\sqrt{x^{2}-z^{2}}}\dfrac{\partial }{\partial x}(\sqrt{x^{2}-z^{2}})+3^{\tfrac{x^{2}}{y}}\log 3\dfrac{\partial }{\partial x}(\dfrac{x^{2}}{y}) = \\
&= \frac{1}{\sqrt{x^{2}-z^{2}}}\frac{1}{2\sqrt{x^{2}-z^{2}}}\dfrac{\partial}{\partial x}(x^{2}-z^{2})+3^{\tfrac{x^{2}}{y}}\log 3\dfrac{2x}{y}= \\
&= \frac{1}{2(x^{2}-z^{2})}2x+3^{\tfrac{x^{2}}{y}}\log 3\dfrac{2x}{y}=\frac{x}{x^{2}-z^{2}}+3^{\tfrac{x^{2}}{y}}\log 3\dfrac{2x}{y}, \\
\dfrac{\partial f}{\partial y}(x,y,z) &= \dfrac{\partial }{\partial y}(\log\left( \sqrt{x^{2}-z^{2}}\right) +3^{\tfrac{x^{2}}{y}}) = \dfrac{\partial }{\partial y}(\log \left( \sqrt{x^{2}-z^{2}}\right) )+\dfrac{\partial }{\partial y}(3^{\tfrac{x^{2}}{y}})= \\
&= 0+3^{\tfrac{x^{2}}{y}}\log 3\dfrac{\partial }{\partial y}(\dfrac{x^{2}}{y}) = 3^{\tfrac{x^{2}}{y}}\log 3\dfrac{-x^{2}}{y^{2}}, \\
\dfrac{\partial f}{\partial z}(x,y,z) &= \dfrac{\partial }{\partial z}(\log\left( \sqrt{x^{2}-z^{2}}\right) +3^{\tfrac{x^{2}}{y}}) = \dfrac{\partial }{\partial z}(\log \left( \sqrt{x^{2}-z^{2}}\right) )+\dfrac{\partial }{\partial z}(3^{\tfrac{x^{2}}{y}})= \\
&= \frac{1}{\sqrt{x^{2}-z^{2}}}\dfrac{\partial }{\partial x}(\sqrt{x^{2}-z^{2}})+0 = \frac{1}{\sqrt{x^{2}-z^{2}}}\frac{1}{2\sqrt{x^{2}-z^{2}}}\dfrac{\partial }{\partial x}(x^{2}-z^{2})= \\
&= \frac{1}{2(x^{2}-z^{2})}(-2z)=-\frac{z}{x^{2}-z^{2}}.
\end{align*}
y, en consecuencia tenemos
\[
\nabla f(x,y,z)=(\frac{x}{x^{2}-z^{2}}+3^{\tfrac{x^{2}}{y}}\log 3\dfrac{2x}{y},3^{\tfrac{x^{2}}{y}}\log 3\dfrac{-x^{2}}{y^{2}},-\frac{z}{x^{2}-z^{2}}).
\]
Como nos piden el gradiente en el punto $(1,1,0),$ sustituimos $x$ por 1, $y$
por 1, y $z$ por 0 en el vector anterior y obtenemos
\[
\nabla f(1,-2,-2)=(1+6\log 3,-3\log 3,0).
\]
}


\newproblem{par-32}{amb}{}
%ENUNCIADO
{La asimilación de CO$_2$ de una planta depende de la temperatura ambiente (t) y de la intensidad de la luz (l), según la función
\[
f(t,l) = ctl^2,
\]
donde $c$ es una constante.
Estudiar cómo evoluciona la asimilación de CO$_2$ para distintas intensidades de luz, cuando se mantiene la temperatura constante.
Estudiar también cómo evoluciona para distintas temperaturas cuando se mantiene la intensidad de la luz constante.
}
%SOLUCIÓN
{$\frac{\partial f}{\partial l}(t,l) = 2ctl$ y $\frac{\partial f}{\partial t}(t,l) = cl^2$.
}
%RESOLUCIÓN
{
}


\newproblem{par-33}{amb}{}
%ENUNCIADO
{La abundancia de una determinada especie de planta depende del nivel de nitrógeno en el suelo y del nivel de perturbaciones, de manera que un incremento del nivel de nitrógeno tiene un efecto negativo en la abundancia de esta especie, y un aumento de las perturbaciones también tiene un efecto negativo.
Si en un momento dado comienza a aumentar el nivel de nitrógeno en el suelo y también las perturbaciones debidas al pastoreo, ¿cómo se verá afectada la abundancia de la especie?
}
%SOLUCIÓN
{La abundancia de la especie disminuirá.
}
%RESOLUCIÓN
{
}


\newproblem{par-34}{amb}{}
%ENUNCIADO
{La velocidad de crecimiento de un organismo depende de la disponibilidad de alimento y del número de competidores en busca de alimento.
¿Cómo se verá afectada la velocidad de crecimiento si la disponibilidad de alimento aumenta con el tiempo y el número de competidores disminuye?}
%SOLUCIÓN
{La velocidad de crecimiento aumentará.
}
%RESOLUCIÓN
{
}


\newproblem{par-35}{amb}{}
%ENUNCIADO
{Un organismo se mueve sobre una superficie inclinada siguiendo la línea de máxima pendiente descendiente.
Si la expresión de la superficie es
\[
f(x,y) = x^2-y^2,
\]
calcule la dirección en la que se moverá el organismo en el punto $(2,3)$.
}
%SOLUCIÓN
{Se moverá en la dirección $-\nabla f(2,3)=(-4,6)$.}
%RESOLUCIÓN
{
}


\newproblem{par-36}{gen}{}
%ENUNCIADO
{Si $f(x,y,z)=x^3y^2z$ y $g(t)=(e^t,\cos t,\sen t)$, calcular $(f\circ g)'(t)$.
}
%SOLUCIÓN
{$(f\circ g)'(t)= e^{3t}(3\sen t\cos^2 t-2\sen^2 t\cos t+\cos^3 t)$.
}
%RESOLUCIÓN
{
}


\newproblem{par-37}{amb}{}
%ENUNCIADO
{Obtener los puntos críticos de $z=f(x,y)$ para:
\begin{enumerate}
\item $f(x,y)=x^2+y^2$.
\item $f(x,y)=x^2y+y^2x$.
\item $f(x,y)=x^2-2xy+2y^2$.
\end{enumerate}
}
%SOLUCIÓN
{\begin{enumerate}
\item $(0,0)$.
\item $(0,0)$.
\item $(0,0)$.
\end{enumerate}
}
%RESOLUCIÓN
{
}


\newproblem{par-38}{gen}{}
%ENUNCIADO
{La superficie de una montaña tiene la forma
\[
S:z=a-bx^2-cy^2,
\]
donde $a$, $b$ y $c$ son constantes, $x$ es la coordenada Este-Oeste e $y$ la coordenada Norte-Sur en el mapa, y $z$ la altura sobre el nivel del mar en metros.
En el punto $P=(1,1)$ del mapa, ¿en qué dirección crece más rápidamente la altura?
}
%SOLUCIÓN
{$(-2b,-2c)$.
}
%RESOLUCIÓN
{
}


\newproblem{par-39}{gen}{}
%ENUNCIADO
{Hallar las direcciones de máximo y mínimo crecimiento de las siguientes funciones en el punto $P$:
\begin{enumerate}
\item $f(x,y)=x^2+xy+y^2$, $P=(-1,1)$.
\item $f(x,y)=x^2y+e^{xy}\sen y$, $P=(1,0)$.
\item $f(x,y,z)=\log(xy)+\log(yz)+\log(xz)$, $P=(1,1,1)$.
\item $f(x,y,z)=\log(x^2+y^2-1)+y+6z$, $P=(1,1,0)$.
\end{enumerate}
}
%SOLUCIÓN
{\begin{enumerate}
\item Máximo crecimiento en la dirección $(-1,1)$ y máximo decrecimiento en la dirección $(1,-1)$.
\item Máximo crecimiento en la dirección $(0,2)$ y máximo decrecimiento en la dirección $(0,-2)$.
\item Máximo crecimiento en la dirección $(2,2,2)$ y máximo decrecimiento en la dirección $(-2,-2,-2)$.
\item Máximo crecimiento en la dirección $(2,3,6)$ y máximo decrecimiento en la dirección $(-2,-3,-6)$.
\end{enumerate}
}
%RESOLUCIÓN
{
}


\newproblem{par-40}{gen}{}
%ENUNCIADO
{¿En qué direcciones se anulará la derivada direccional de la función
\[
f(x,y)=\frac{x^2-y^2}{x^2+y^2}
\]
en el punto $P=(1,1)$?
}
%SOLUCIÓN
{En la dirección $(1/\sqrt{2},1/\sqrt{2})$.
}
%RESOLUCIÓN
{
}


\newproblem{par-41}{gen}{}
%ENUNCIADO
{¿Existe alguna dirección en la que la derivada direccional en el punto $P=(1,2)$ de la función
\[
f(x,y) = x^2-3xy+4y^2
\]
valga 14?
}
%SOLUCIÓN
{No.
}
%RESOLUCIÓN
{
}


\newproblem{par-42}{gen}{}
%ENUNCIADO
{La derivada direccional de una función $f$ en un punto $P$ es máxima en la dirección del vector $(1,1,-1)$ y su valor es $2\sqrt{3}$.
¿Cuánto vale la derivada direccional de $f$ en $P$ en la dirección del vector $(1,1,0)$?
}
%SOLUCIÓN
{$2\sqrt{2}$.
}
%RESOLUCIÓN
{
}


\newproblem{par-43}{gen}{}
%ENUNCIADO
{Dado el campo escalar
\[
f(x,y,z) = x^2-y^2+xyz^3-zx
\]
en el punto $P=(1,2,3)$, se pide:
\begin{enumerate}
\item Calcular la derivada direccional de $f$ en $P$ a lo largo del vector unitario $\mathbf{u}=\frac{1}{\sqrt2}(1,-1,0)$.
\item ¿En qué dirección es máxima la derivada direccional de $f$ en $P$? Obtener el valor de dicha derivada direccional.
\end{enumerate}
}
%SOLUCIÓN
{\begin{enumerate}
\item $15\sqrt{2}$.
\item La derivada direccional es máxima en la dirección del gradiente $(53,23,53)$ y vale $\sqrt{6147}$.
\end{enumerate}
}
%RESOLUCIÓN
{
}


\newproblem*{par-44}{gen}{}
%ENUNCIADO
{En el ajuste de regresión de una recta $y=a+bx$, se suele utilizar la técnica de mínimos cuadrados que consisten en buscar los valores
de $a$ y $b$ que hacen mínima la función
\[
f(a,b)= \sum_{i=1}^{n}(y_i-a-bx_i)^2,
\]
donde el sumatorio abarca a todos los pares de la muestra $(x_i,y_i)$ para $i=1,\ldots, n$, siendo $n$ el tamaño de la muestra.

Demostrar que esta función alcanza el mínimo en los puntos
\[
a=\bar y-b\bar x \quad \mbox{ y } b=\frac{s_{xy}}{s_x^2}.
\]
}
%SOLUCIÓN
{
}
%RESOLUCIÓN
{
}


\newproblem{par-45}{gen}{}
%ENUNCIADO
{La siguiente función mide la presión del aire en la posición $(x,y,z)$.
\[
f(x,y,z)= x^2+y^2-z^3.
\]
Sea $A$ un objeto que se mueve a lo largo de la trayectoria:
\[
\begin{cases}
x=t\\
y=1\\
z=1/t
\end{cases}
t>0.
\]
\begin{enumerate}
\item Calcular la ecuación de la recta tangente a la trayectoria de $A$ en el punto $(1,1,1)$.
\item Sigue la trayectoria de $A$ en el punto $(1,1,1)$ la dirección de máximo crecimiento de la función $f$?
\end{enumerate}
}
%SOLUCIÓN
{\begin{enumerate}
\item $(1+t, 1, 1-t)$.
\item No ya que la dirección de máximo crecimiento de $f$ es $\nabla f(1,1,1)=(2,2,-3)$ y la dirección de la trayectoria es $(1,0,-1)$.
\end{enumerate}
}
%RESOLUCIÓN
{
}


\newproblem{par-46}{gen}{*}
%STATEMENT
{Obtener la ecuación del plano tangente y de la recta normal a la superficie
\[
S:xyz=8
\]
en el punto $P=(4,-2,-1)$.
}
%SOLUTION
{Recta normal $l:(4+2t,-2-4t,-1-8t)$. Plano tangente $\pi: 2x-4y-8z+24=0$.
}
%RESOLUTION
{
}


%%%%%%% Pendiente 26

\input{derivadas_superficies}
% Autor: Alfredo Sánchez Alberca (asalber@ceu.es)

\newproblem{derimpn-1}{gen}{}
%ENUNCIADO
{Suponiendo que $z$ es función de $x$ e $y$ ($z=f(x,y)$), a partir de la ecuación $F(x,y,z)=0$, deducir que 
\[
\frac{\partial z}{\partial x} = \frac{-\dfrac{\partial F}{\partial x}}{\dfrac{\partial F}{\partial z}}
\quad \mbox{y} \quad
\frac{\partial z}{\partial y} = \frac{-\dfrac{\partial F}{\partial y}}{\dfrac{\partial F}{\partial z}}.
\]

Aplicarlo para obtener $\dfrac{\partial f}{\partial x}(2,1)$, sabiendo que $x^2yz=4$ y que $f(2,1)=1$.
}
%SOLUCIÓN
{$\frac{\partial f}{\partial x}(2,1)=-1.$
}
%RESOLUCIÓN
{
}

\newproblem{derimpn-2}{gen}{*}
%ENUNCIADO
{La ecuación 
\[
x\log y+\frac{2e^{y^2+z}}{x} - \frac{x}{z^2} = -1
\] 
define a $z$ como función de $x$ e $y$ alrededor del punto $(2,1,-1)$. 
Calcular el vector gradiente de $z$ en ese punto e interpretarlo.
}
%SOLUCIÓN
{$\nabla z(2,1,-1) = (-1/2,4/3)$.
}
%RESOLUCIÓN
{
}
% Autor: Alfredo Sánchez Alberca (asalber@ceu.es)

\newproblem{extn-1}{gen}{*}
%ENUNCIADO
{Hallar los extremos relativos y los puntos de silla de la función:
\[
f(x,y) = (x^2+y^2)^2-2a^2(x^2-y^2),
\]
con $a\neq 0$.
}
%SOLUCIÓN
{No tiene máximos relativos. Mínimos relativos en $(-a,0)$ y $(a,0)$. Punto de silla en $(0,0)$.
}
%RESOLUCIÓN
{
}


\newproblem{extn-2}{gen}{*}
%ENUNCIADO
{Dado el campo escalar
\[
h(x,y) = xy+\frac{xy^2}{2}-2x^2,
\]
determinar sus extremos relativos y sus puntos de silla.
}
%SOLUCIÓN
{Máximo relativo en $(-1/8,-1)$. No tiene mínimos relativos. Puntos de silla en $(0,0)$ y $(0,-2)$.
$(0,0)$.
}
%RESOLUCIÓN
{
}

\newproblem{extn-3}{gen}{}
%ENUNCIADO
{Estudiar los extremos y los puntos de silla de $f$ en los siguientes casos:
\begin{enumerate}
\item $f(x,y) = x^2+y^2$.
\item $f(x,y) = x^2-y^2$.
\item $f(x,y) = x^2-2xy+2y^2$.
\item $f(x,y) = \log(x^2+y^2+1)$.
\end{enumerate}
}
%SOLUCIÓN
{\begin{enumerate}
\item Mínimo en $(0,0)$.
\item Punto de silla en $(0,0)$.
\item No se puede saber con el hessiano.
\item Mínimo en $(0,0)$.
\end{enumerate}
}
%RESOLUCIÓN
{
}


\newproblem{extn-4}{gen}{}
%ENUNCIADO
{La función
\[
f(x,y) = \frac{x^3}{3}-x-\left(\frac{y^3}{3}-y\right)
\]
tiene un máximo, un mínimo y dos puntos de silla. Encontrarlos.
}
%SOLUCIÓN
{Máximo en $(-1,1)$, mínimo en $(1,-1)$ y puntos de silla en $(1,1)$ y $(-1,-1)$.
}
%RESOLUCIÓN
{
}


\newproblem{extn-5}{gen}{}
%STATEMENT
{Si suponemos que el rendimiento de una cosecha, $R$, depende de las concentraciones de nitrógeno, $n$, y fósforo, $p$, presentes en el suelo según la función:
\[
R(n,p) = n \cdot p \cdot e^{ - (n + p)}
\]
\begin{enumerate}
\item Calcular todas las derivadas parciales de primer y segundo orden de la función $R(n,p)$.
\item Teniendo en cuenta que una condición necesaria para que una función de varias variables presente un máximo en un
punto es que todas las derivadas parciales de primer orden se anulen en dicho punto, ¿cuánto deben valer las
concentraciones de nitrógeno y fósforo para que el rendimiento de la cosecha sea máximo?
\end{enumerate}
}
%SOLUCION
{El rendimiento de la cosecha será máximo para $n=p=1$.
}
%RESOLUTION
{
}


\newproblem{extn-6}{gen}{*}
%STATEMENT
{Dada la función $f(x,y)=\dfrac{ax^3}{3} + \dfrac{by^3}{3}-4ax-4by$, con $a$ y $b$ dos parámetros positivos, estudiar la existencia de extremos relativos y puntos de silla de $f$.
}
%SOLUTION
{Máximo relativo en $(-2,-2)$, mínimo relativo en $(2,2)$ y puntos de silla en $(-2,2)$ y $(2,-2)$.
}
%RESOLUTION
{
}

% Autor: Alfredo Sánchez Alberca (asalber@ceu.es)

\newproblem{tayn-1}{gen}{*}
%ENUNCIADO
{Dada la función $f(x,y)=\sqrt{xy}$, se pide:
\begin{enumerate}
\item Calcular el polinomio de Taylor de primer grado centrado en el punto $(4,9)$.
\item Calcular el valor aproximado de $f(4.01,\,8.99)$ a partir del polinomio anterior. 
\end{enumerate}
}
%SOLUCIÓN
{\begin{enumerate}
\item $P(x,y)= \frac{3}{4}x+\frac{1}{3}y$.
\item $f(4.01,\,8.99)\approx 6.00416667$.
\end{enumerate}
}
%RESOLUCIÓN
{
}


\newproblem{tayn-2}{gen}{}
%ENUNCIADO
{Obtener el desarrollo de Taylor de segundo orden de $f$ alrededor del punto $P$ en cada uno de los casos siguientes:
\begin{enumerate}
\item $f(x,y)=\sen(x+2y)$, $P=(0,0)$.
\item $f(x,y)=e^x\cos y$, $P(0,0)$.
\item $f(x,y)=\sen(xy)$, $P(1,\pi/2)$.
\end{enumerate}
}
%SOLUCIÓN
{\begin{enumerate}
\item $P^2_{f,P}(x,y)= x+2y$.
\item $P^2_{f,P}(x,y)= 1+x+\frac{x^2}{2}-\frac{y^2}{2}$.
\item $P^2_{f,P}(x,y)= 1+\frac{1}{2}\left(-\frac{\pi^2}{4}(x-1)^2-\pi(x-1)(y-\pi/2)-(y-\pi/2)^2\right)$.
\end{enumerate}
}
%RESOLUCIÓN
{
}


\newproblem{tayn-3}{gen}{}
%ENUNCIADO
{Dada la función 
\[
f(x,y,z)=e^x\sqrt{yz},
\]
estimar el valor de $f(0.01,24.8,1.02)$ mediante un desarrollo de Taylor lineal alrededor del punto $P=(0,25,1)$.
}
%SOLUCIÓN
{$5.08$.
}
%RESOLUCIÓN
{
}


\newproblem{tayn-4}{gen}{}
%ENUNCIADO
{Hallar las aproximaciones lineal y cuadrática de la expresión
\[
\frac{(3.98-1)^2}{(5.97-3)^2}
\]
usando desarrollos de Taylor. Comparar el resultado con el valor exacto.
}
%SOLUCIÓN
{La aproximación lineal es $1.00666666$ y la cuadrática $1.006744438$.
}
%RESOLUCIÓN
{
}
%% Autor: Alfredo Sánchez Alberca (asalber@ceu.es)

\newproblem*{int-1}{gen}{}
%ENUNCIADO
{Calcular por cambio de variable las integrales indefinidas siguientes:
\begin{multicols}{2}
\begin{enumerate}
\item $\dint e^{4x}\, dx$
\item $\dint \dfrac{x^{3}}{2+x^{8}}\, dx$
\item $\dint \dfrac{e^{\arcsen x}}{\sqrt{1-x^{2}}}\, dx$
\item $\dint \dfrac{1}{x\log x}\, dx$
\end{enumerate}
\end{multicols}
}


\newproblem*{int-2}{gen}{}
%ENUNCIADO
{Calcular las primitivas de las siguientes funciones:
\begin{multicols}{2}
\begin{enumerate}
\item $\dfrac{x+3}{\left( x^{2}+6x\right) ^{1/3}}$
\item $\dfrac{\arcsen x+x}{\sqrt{1-x^{2}}}$
\end{enumerate}
\end{multicols}
}


\newproblem*{int-3}{gen}{}
%ENUNCIADO
{Calcular las siguientes integrales por partes:
\begin{multicols}{2}
\begin{enumerate}
\item $\dint x^{5}\log x\, dx$
\item $\dint e^{x}\cos x\, dx$
\end{enumerate}
\end{multicols}
}


\newproblem*{int-4}{gen}{}
%ENUNCIADO
{Calcular las integrales:
\begin{multicols}{2}
\begin{enumerate}
\item $\dint x\sen 3x\,dx$
\item $\dint \dfrac{x}{\cos^{2}x}\,dx$
\end{enumerate}
\end{multicols}
}


\newproblem*{int-5}{gen}{}
%ENUNCIADO
{Calcular las siguientes integrales de funciones racionales:
\begin{multicols}{2}
\begin{enumerate}\setlength{\itemsep}{3mm}
\item $\dint \dfrac{x+1}{x^{2}-4x+8}\,dx$
\item $\dint \dfrac{x^{4}}{x^{4}-1}\,dx$
\item $\dint \dfrac{x}{x^{6}-1}\,dx$ 
\end{enumerate}
\end{multicols}
\textbf{Nota}: Para el apartado (c) hacer previamente la sustitución $x^{2}=t$.
}


\newproblem*{int-6}{gen}{}
%ENUNCIADO
{Calcular las integrales trigonométricas siguientes:
\begin{multicols}{2}
\begin{enumerate}\setlength{\itemsep}{3mm}
\item $\dint \dfrac{\sen x}{3\sen x+4\cos x}\,dx$
\item $\dint \sen^{4}x\,dx$
\item $\dint \dfrac{\tg x}{1+\cos x}\,dx$ 
\end{enumerate}
\end{multicols}
}


\newproblem*{int-7}{gen}{}
%ENUNCIADO
{Calcular las primitivas de las siguientes funciones:
\begin{multicols}{2}
\begin{enumerate}\setlength{\itemsep}{3mm}
\item $f(x)=x^3-3x^2+3$
\item $g(x)=\dfrac{x}{x^2-1}$
\item $h(x)=\dfrac{e^{1/x}}{x^2}$
\item $i(x)=\tg x$
\item $j(x)=\dfrac{x+3}{\sqrt{1-x^2}}$
\item $k(x)=\dfrac{x^2}{\sqrt{1-x^2}}$
\item $l(x)=(x^2-2x+5)e^{-x}$
\item $m(x)=\dfrac{\log x}{\sqrt x}$
\item $n(x)=3^x\cos x$
\item $o(x)=\sen(\log x)$
\item $p(x)=\dfrac{1}{x^3-4x^2+5x+2}$
\item $q(x)=\dfrac{1}{x^3+1}$
\end{enumerate}
\end{multicols}
}


\newproblem*{int-8}{gen}{}
%ENUNCIADO
{La función $e^{-x^2}$ no tiene una primitiva conocida. Calcular de manera aproximada $\dint_{-1/2}^{1/2} e^{-x^2}\, dx$ mediante aproximaciones de Taylor.
}


\newproblem*{int-9}{gen}{*}
%ENUNCIADO
{Dada la función
\[
f(x) = \ln x\left( {x^3  + 2x + 1} \right)
\]
Calcular $\dint_1^2 {f(x)\,dx}$.
}

%% Autor: Alfredo Sánchez Alberca (asalber@ceu.es)

\newproblem*{intimp-1}{gen}{}
%ENUNCIADO
{Calcular las siguientes integrales impropias:
\begin{multicols}{2}
\begin{enumerate}
\item $\dint\limits_{2}^{\infty }x^{2}e^{-x}\,dx$
\item $\dint\limits_{1}^{\infty }\dfrac{\ln x}{x^{2}}\,dx$
\end{enumerate}
\end{multicols}
}


\newproblem*{intimp-2}{gen}{}
%ENUNCIADO
{Calcular la siguiente integral: $\dint\limits_{1}^{\infty }\left(xe^{-x^{2}}+\dfrac{1}{x^{2}}\right) dx.$
}


\newproblem*{intimp-3}{gen}{*}
%ENUNCIADO
{Calcular la siguiente integral impropia:
\[
\int\limits_{ - \pi }^\infty  {e^{ - 2x} \cos (3x)\,dx}.
\]
}


\newproblem*{intimp-4}{gen}{}
%ENUNCIADO
{Calcular la siguiente integral impropia:
\[
\dint\limits_{0}^{\infty }(x+1)e^{-\tfrac{1}{2}x}dx.
\]
}


\newproblem*{intimp-5}{gen}{}
%ENUNCIADO
{Sea la función
\[
f(x) =
\begin{cases}
2^x & \mbox{si $x\leq -1$},\\
e^{-x} & \mbox{si $x > -1$}
\end{cases} 
\]
Calcular $\dint\limits_{-\infty }^\infty f(x)\,dx$.
}



%% Version control information:
\svnidlong
{$HeadURL: https://ejercicioscalculo.googlecode.com/svn/trunk/integrales_aplicaciones.tex $}
{$LastChangedDate: 2010-01-28 20:28:03 +0100 (jue, 28 ene 2010) $}
{$LastChangedRevision: 11 $}
{$LastChangedBy: asalber $}
%\svnid{$Id: integrales_aplicaciones.tex 11 2010-01-28 19:28:03Z asalber $
%
\newproblem*{intapl-1}{gen}{}
%ENUNCIADO
{Calcular el área comprendida entre la curva $y=x^{3}-6x^{2}+8x$ y el eje de abcisas.
}


\newproblem*{intapl-2}{gen}{}
%ENUNCIADO
{Calcular el área comprendida entre la parábola $y^{2}=4x$ y la recta $y=2x-4$.
}


\newproblem*{intapl-3}{gen}{}
%ENUNCIADO
{Calcular el área comprendida entre las parábolas $y=6x-x^{2}$ e $y=x^{2}-2x.$
}


\newproblem*{intapl-4}{gen}{*}
%ENUNCIADO
{Dibujar aproximadamente el recinto limitado por $f(x)=\cos x$, $g(x)=|x^2-1|$, $x=-1$ y $x=1$, y calcular el área de dicho recinto. 
}


\newproblem*{intapl-5}{gen}{*}
%ENUNCIADO
{Calcular el área del recinto limitado por las parábolas:
\[
\left\{
\begin{array}{l}
y_1 = x^2+2x+2,\\
y_2 = -x^2+2x+4.
\end{array}
\right.
\]
}


\newproblem*{intapl-6}{gen}{*}
%ENUNCIADO
{Calcular el área del recinto limitado por las funciones $f(x)= e^{-x}$, $g(x)=x^2-4x+1$ y la recta $x=2$.
}


\newproblem*{intapl-7}{gen}{}
%ENUNCIADO
{Dibujar aproximadamente el recinto limitado por la función $f(x)=\left| x^{2}-4x+3\right|$ y la recta $y=3.$ Calcular el área de dicho recinto.
}


\newproblem*{intapl-8}{gen}{}
%ENUNCIADO
{Calcular el área encerrada entre $y=e^{-\left|x\right| }$ y su asíntota.
}


\newproblem*{intapl-9}{gen}{*}
%ENUNCIADO
{Dada la función 
\[
f(x)=\frac{x}{(1+2x^2)^6}
\]
Calcular:
\begin{enumerate}
\item El área del recinto limitado por $f(x)$ y el eje de abscisas desde $x=1$ hasta $x=2$.
\item El área del recinto limitado por $f(x)$ y el eje de abscisas desde $x=0$ has el infinito.
\end{enumerate}
}


\newproblem*{intapl-10}{gen}{}
%ENUNCIADO
{Calcular el área entre las funciones siguientes y el eje de abscisas en el intervalo $[1,3]$:
\begin{multicols}{2}
\begin{enumerate}\setlength{\itemsep}{3mm}
\item $f(x)=\sqrt{x}$
\item $f(x)=\dfrac{1}{x^2}$
\item $f(x)=\sin x^2$
\item $f(x)=x^2-3x+2$
\end{enumerate}
\end{multicols}
}


\newproblem*{intapl-11}{amb}{*}
%ENUNCIADO
{Si la cantidad de dióxido de carbono, en toneladas/hora, que arroja a la atmósfera una empresa viene dada en función del tiempo, en horas, por la expresión:
\[
c(t)=20te^{-t}
\]
\begin{enumerate}
\item ¿Cuál es la cantidad total de monóxido de carbono arrojada a la atmósfera por la empresa desde $t=0$ hasta transcurridas 2 horas?
\item ¿Hacia dónde tiende dicha cantidad cuando el tiempo tiende a infinito?
\end{enumerate}
}


\newproblem*{intapl-12}{gen}{*}
%ENUNCIADO
{Dadas las funciones: $f(t) = t$ y $g(t) = \dfrac{t} {{\sqrt{1 + 3t} }}$
\begin{enumerate}
\item Calcular el área del recinto limitado por la funciones entre $t=0$ y $t=1$.
\item Si $f(t)$ es el volumen de agua por unidad de tiempo, en m$^3$/s, que llega a un depósito, y $g(t)$ la que sale del mismo, también en m$^3$/s, ¿qué volumen de agua habrá ganado, o perdido, dicho depósito entre $t=0$ y $t=1/2$?
\end{enumerate}
}


\newproblem*{intapl-13}{amb}{*}
%ENUNCIADO
{Supongamos que el caudal de agua de una fuente natural (en metros cúbicos/día)
viene dado por la expresión:
\[
C(t) = \frac{t}{{\left( {2 + 0,01 t^2 } \right)^3 }}
\]
donde $t$ es el tiempo, expresado en días, desde que hemos comenzado a medir el caudal.
\begin{enumerate}
\item ¿Qué cantidad de metros cúbicos de agua podemos recoger en esa fuente desde el momento en el que hemos comenzado a medir hasta transcurridos 10 días?
\item ¿Y desde el momento en el que hemos comenzado a medir hasta transcurrido un tiempo muy grande?
\end{enumerate}
}


\newproblem*{intapl-14}{amb}{*}
%ENUNCIADO
{Suponiendo que el caudal de agua, $C$ en metros m$^3$/día, que un arroyo vierte en un río, viene dado en función del tiempo, $t$ en días, por la expresión:
\[
C(t) = t^2  \cdot \sqrt {t^2  + 9}
\]
\begin{enumerate}
\item Calcular el total de agua que el arroyo ha vertido en el río desde $t=0$ hasta transcurridos 10 días.
\item Teniendo en cuenta que se define el caudal medio como el total del agua vertida dividida entre el total del tiempo transcurrido, ¿cuál ha sido el caudal medio del arroyo en los dos primeros días?
\end{enumerate}
}


\newproblem*{intapl-15}{gen}{}
%ENUNCIADO
{Calcular el área delimitada por las funciones $y=e^x$, $y=e^{-x}$, $x=-1$, $x=1$ y el eje de abscisas.
}


\newproblem*{intapl-16}{gen}{}
%ENUNCIADO
{Calcular el área encerrada entre las funciones $f(x)=x+2$ y $g(x)=4-x^2$ entre sus puntos de corte.
}


\newproblem*{intapl-17}{gen}{}
%ENUNCIADO
{Calcular el área que queda entre las funciones $f(x)=\sen x$ y $g(x)=\cos x$ en el intervalo $\pi/4$ y $5\pi/4$.
}
% Version control information:
\svnidlong
{$HeadURL: https://ejercicioscalculo.googlecode.com/svn/trunk/edo_separables.tex $}
{$LastChangedDate: 2010-01-28 20:28:03 +0100 (jue, 28 ene 2010) $}
{$LastChangedRevision: 11 $}
{$LastChangedBy: asalber $}
%\svnid{$Id: edo_separables.tex 11 2010-01-28 19:28:03Z asalber $
%
\newproblem{edosep-1}{gen}{}
%ENUNCIADO
{Integrar las siguientes ecuaciones de variables separables:
\begin{enumerate}
\item $x\sqrt{1-y^2}+y\sqrt{1-x^2}y'=0$ con la condición inicial $y(0)=1$.
\item $(1+e^x)yy'=e^y$ con la condición inicial $y(0)=0$.
\item e$^y(1+x^2)y'-2x(1+\mbox{e}^y)=0$.
\item $y-xy'=a(1+x^2y')$.
\end{enumerate}
}
%SOLUCIÓN
{
\begin{enumerate}
\item $-\sqrt{1-y^2}=\sqrt{1-x^2}-1$.
\item $e^{-y}(y+1)=\log(1+e^x)-x-\log 2+1$.
\item $y=\log(C(1+x^2)-1)$.
\item $y=C\frac{x}{ax+1}+a$.
\end{enumerate}
}
%RESOLUCIÓN
{}


\newproblem{edosep-2}{qui}{}
%ENUNCIADO
{La desintegración radioactiva está regida por la ecuación
diferencial
\[
\frac{\partial x}{\partial y}+ax=0,
\]
donde $x$ es la masa, $t$ el tiempo y $a$ es una constante positiva. La vida media $T$ es el tiempo durante el cual la
masa se desintegra a la mitad de su valor inicial. Expresar $T$ en función de $a$ y evaluar $a$ para el isótopo de
uranio $U^{238}$, para el cual $T=4'5\cdot10^9$ años. } 
%SOLUCIÓN
{$T = \frac{\log 2}{a}$ y $a=1.54\cdot 10^{-10}$ años$^{-1}$.
}
%RESOLUCIÓN
{}



\newproblem{edosep-3}{qui}{}
%ENUNCIADO
{El azúcar se disuelve en el agua con una velocidad proporcional a la cantidad que queda por disolver. Si inicialmente
había 13.6 kg de azúcar y al cabo de 4 horas quedan sin disolver 4.5 kg, ¿cuánto tardará en disolverse el 95\% del
azúcar contando desde el instante inicial? }
%SOLUCIÓN
{$C(t)= 13.6e^{-0.276 t}$ y el instante en que se habrá disuelto el 95\% del azúcar es $t_0=10.854$ horas.
}
%RESOLUCIÓN
{}



\newproblem{edosep-4}{qui}{*}
%ENUNCIADO
{Una reacción química sigue la siguiente ecuación diferencial
\[
y'-2y=4,
\]
donde $y=f(t)$ es la concentración de oxígeno en el instante $t$ (medido en segundos). Si la concentración de oxígeno
al comienzo de la reacción era nula, ¿cuál será la concentración (mg/lt) a los 3 segundos? ¿En qué instante la
concentración de oxígeno será de 200 mg/lt?}
%SOLUCIÓN
{$y(t)=2e^{2t}-2$. La concentración a los tres segundos será $y(3)=804$ mg/lt y el instante en que la concentración de
oxígeno será de 200 mg/lt es $t_0=2.3076$ s.}
%RESOLUCIÓN
{}



\newproblem{edosep-5}{med}{}
%ENUNCIADO
{La sala de disección de un forense se mantiene fría a una temperatura constante de $5^\circ C$. Mientras se encontraba
realizando la autopsia de una víctima de asesinato, el forense es asesinado y el cuerpo de la víctima robado. A las 10
de la mañana el ayudante el forense descubre su cadáver a una temperatura de $23^\circ C$ y llama a la policía. A medio
día llega ésta y comprueba que la temperatura del cadáver es de $18'5^\circ C$. Supuesto que el forense tenía en vida
una temperatura normal de $37^\circ C$, ¿a qué hora fue asesinado?}
%SOLUCIÓN
{Fue asesinado a las 6 de la mañana aproximadamente.
}
%RESOLUCIÓN
{}



\newproblem{edosep-6}{qui}{*}
%ENUNCIADO
{Sea la siguiente ecuación diferencial que relaciona la temperatura y el tiempo en un determinado sistema físico:
$x't^2  - x't + x' - 2xt + x = 0$, siendo $x$ la temperatura expresada en Kelvins y $t$ el tiempo en segundos. 

Sabiendo que la temperatura en el instante inicial del experimento es 100 K, calcular la temperatura en función del
tiempo, y dar la temperatura del sistema físico tres segundos después de comenzar el experimento.  }
%SOLUCIÓN
{$x(t)=100(t^2-t+1)$ y la temperatura del sistema a los tres segundos de comenzar el experimento es $x(3)=700$ K.
}
%RESOLUCIÓN
{En primer lugar, intentamos separar las variables para ver si se trata de una ecuación de variables separables:
\[\renewcommand{\arraystretch}{2}
\begin{array}{c}
x't^2  - x't + x' - 2xt + x = 0 \Leftrightarrow x'(t^2-t+1)+x(-2t+1)=0 \Leftrightarrow\\
\Leftrightarrow \dfrac{dx}{dt}(t^2-t+1)=x(2t-1) \Leftrightarrow \dfrac{dx}{x}=\dfrac{2t-1}{t^2-t+1} dt
\end{array}
\]
Así pues, se trata de una ecuación diferencial ordinaria de variables separables. Integándo en ambos lados de la
ecuación tenemos  
\[\renewcommand{\arraystretch}{2}
\begin{array}{c}
\dint \dfrac{dx}{x}=\dint \dfrac{2t-1}{t^2-t+1}\,dt \Leftrightarrow \log |x|= \log |t^2-t+1|+C \Leftrightarrow \\
\Leftrightarrow \exp(\log |x| )= \exp(\log |t^2-t+1|+C) \Leftrightarrow x=(t^2-t+1)e^C,
\end{array}
\]
Y renombrando $e^C$ como una constante $C$, llegamos a la solución general de la ecuación
\[
x(t)=C(t^2-t+1).
\]

Imponiendo ahora la condición inicial $x(0)=100 K$, tenemos
\[
x(0)=C(0^2-0+1)=C=100,
\]
de manera que la solución particular es
\[
x(t)=100(t^2-t+1).
\]

Por último, la temperatura del sistema a los 3 segundos de comenzar el experimento será
\[
x(3)=100(3^2-3+1)=700\textrm{ K}.
\]
}


\newproblem{edosep-7}{far}{*}
%ENUNCIADO
{Se tiene un medicamento en un frigorífico a 2ºC, y se debe administrar a 15ºC. A las 9 h se saca el medicamento del
frigorífico y se coloca en una habitación que se encuentra a 22ºC. A las 10 h se observa que el medicamento está a
10ºC. Suponiendo que la velocidad de calentamiento es proporcional a la diferencia entre la temperatura del medicamento
y la del ambiente, ¿en qué hora se deberá administrar dicho medicamento?}
%SOLUCIÓN
{A las $11.06$ horas.
}
%RESOLUCIÓN
{La ecuación diferencial que rige el enfriamiento de los cuerpos es
\[
\frac{dT}{dt}k(T-T_a),
\]
donde $T$ es la temperatura del cuerpo, $t$ es el tiempo, $T_a$ es la
temperatura del medio que se supone constante y en este caso es 22ºC, y $k$ es
una constante de proporcionalidad.

Como se trata de una ecuación de variables separables, procedemos a separar las
variables:
\[
\frac{dT}{dt}=k(T-22) \Leftrightarrow \frac{dT}{T-22}=kdt,
\]
e integrar:
\[
\int \frac{dT}{T-22}=\int kdt \Leftrightarrow \log|T-22|=kt+C \Leftrightarrow
T-22 = e^{kt+C}=e^{kt}e^C
\]
y reescribiendo $e^C$ como una constante $C$ llegamos a la solución general:
\[T(t)=Ce^{kt}+22.\]

Imponemos ahora las condiciones iniciales para llegar a la solución particular.
En primer lugar, sabemos que en el instante en que se saca el fármaco
del frigorífico la temperatura del mismo era de 2ºC. Fijaremos dicho instante
como el instante inicial $t=0$ (que en realidad son las 9 h). Así pues, se
tiene:
\[
T(0)=2 \Leftrightarrow Ce^{k\cdot 0}+22 = 2 \Leftrightarrow C = -20
\]

En segundo lugar, transcurrida una hora del instante inicial ($t=1$), la
temperatura del fármaco era de 10ºC, de manera que se tiene:
\[
T(1)=10 \Leftrightarrow -20e^{k\cdot 1}+22 =10 \Leftrightarrow -20e^k = -12
\Leftrightarrow e^k = 12/20 \Leftrightarrow k=\log (12/20)=-0.51.
\]
Por consiguiente, llegamos a la solución particular
\[
T(t)= 22-20e^{-0.51t}
\]

Para terminar calculamos el tiempo que debe transcurrir hasta que el medicamento
alcance los 15ºC a que debe administrarse:
\[
T(t)=15 \Leftrightarrow 22-20e^{-0.51t}=15 \Leftrightarrow
e^{-0.51t}=\frac{22-15}{20} \Leftrightarrow t=\frac{\log(7/20)}{-0.51}=2.06
\mbox{ h}.
\]
Por tanto, debe administrarse unas $2.06$ horas después del instante inicial,
aproximadamente a las $11.06$ h.
}


\newproblem{edosep-8}{qui}{*}
%ENUNCIADO
{Una cámara de 500 l está llena de aire en condiciones normales
cuando comienza a entrar oxí­geno puro a razón de 5 litros por minuto. 
Al mismo tiempo se extrae la misma cantidad de la mezcla uniforme. ¿Qué
concentración de oxí­geno habrá a los 10 minutos? Suponiendo que una
concentración de oxí­geno en el aire superior a 0.5 gr/l puede ser perjudicial,
¿cuándo será peligroso respirar el aire de la cámara? 

\textbf{Nota}: La concentración de oxí­geno en el aire en condiciones normales es
de $0.15$ gr/l, mientras que en el oxí­geno puro es de $0.71$ gr/l. La ecuación
diferencial que explica el fenómeno es
\[
\frac{dx}{dt}=c_ev_e-c_sv_s
\]
donde $x$ es la cantidad de oxí­geno en la cámara en el instante $t$, $c_e$ y
$c_s$ son las concentraciones de oxí­geno en el aire que entra y sale
respectivamente, y $v_e$ y $v_s$ son las velocidades de entrada y salida del
aire. 
}
%SOLUCIÓN
{
}
%RESOLUCIÓN
{}



\newproblem{edosep-9}{qui}{*}
%ENUNCIADO
{Sabiendo que el núcleo del Polonio 210 es radiactivo y que su tiempo de semidesintegración (tiempo necesario para que
la cantidad inicial se reduzca a la mitad) es de 138 días:
\begin{enumerate}
\item ¿Qué cantidad inicial de Polonio 210 teníamos si al cabo de 100 días nos quedan 20 gramos?

\item ¿Qué tiempo tendrá que transcurrir para que se desintegre un 10\% de la masa inicial?
\end{enumerate}
}
%SOLUCIÓN
{
}
%RESOLUCIÓN
{}


\newproblem{edosep-10}{amb}{*}
%ENUNCIADO
{Estudios científicos han demostrado que la longitud en función
del tiempo de muchas especies, entre ellas las de gran variedad de
peces, viene dada por la ecuación de Bertalanffy:
\[
\frac{{dL}}{{dt}} = k\left( {L_f  - L(t)} \right)
\]
donde $L_f$ es la longitud de la especie al final del periodo de
crecimiento, y $k$ es una constante. Suponiendo que la longitud de
una especie de peces al final de su periodo de crecimiento es de un
metro, y que con uno y dos meses mide, respectivamente, 20 y 40 cm:
\begin{enumerate}
\item ¿Cuál será la longitud de esa especie para todo tiempo $t$?
\item ¿Cuánto tiempo debe transcurrir desde su nacimiento hasta que la longitud sea de 95 cm?
\end{enumerate}
}
%SOLUCIÓN
{\begin{enumerate}
\item $L(t)=-1.0667e^{-0.2877t}+1$.
\item $t_0=10.637$ años. 
\end{enumerate}
}
%RESOLUCIÓN
{}


\newproblem{edosep-11}{qui}{*}
%ENUNCIADO
{La cantidad de masa de un determinado reactivo de una reacción
química, $M$, en gramos, es función del tiempo, en segundo, y se
rige mediante la siguiente ecuación diferencial:
\[
M' - (a + b)M = 0
\]
donde $a$ y $b$ son constantes. Si inicialmente tenemos 20 gramos de reactivo, al cabo de 10 segundos tenemos 40 gramos, calcular:
\begin{enumerate}
\item La cantidad de reactivo para todo tiempo $t$.
\item La cantidad de reactivo al cabo de medio minuto.
\item ¿Cuando será la cantidad de reactivo 100 g?
\end{enumerate}
}
%SOLUCIÓN
{\begin{enumerate}
\item $M(t) = 20\;e^{\frac{{\ln 2}}{{10}}t}.$
\item $M(30) = 160$ gr.
\item $t_0  = 23.22$ s.
\end{enumerate}
}
%RESOLUCIÓN
{
\begin{enumerate}
\item Para calcular la masa $M$ para todo tiempo $t$ debemos
resolver la ecuación diferencial separable del enunciado.
Procediendo a su separación obtenemos:
\[
M' - (a + b)M = 0 \Leftrightarrow \frac{{dM}}{{dt}} = (a + b)M
\Leftrightarrow \frac{{dM}}{M} = (a + b)dt
\]
e integrando la ecuación separada:
\[
\int {\frac{{dM}}{M}}  = \int {(a + b)dt}  \Leftrightarrow \ln M =
(a + b)t + C_0
\]
donde $C_0$ es una constante de integración.

Por último, tomando exponenciales en ambos miembros de la ecuación
integrada, y teniendo en cuenta que la exponencial de una constante
es una nueva constante a la que llamamos $C$, nos queda:
\[
M(t) = e^{(a + b)t + C_0 }  = e^{(a + b)t} e^{C_0 }  = Ce^{(a + b)t}
\]
Como, además, tenemos 2 datos iniciales, podemos calcular los
valores tanto de $C$ como de la suma $a+b$:
\[
M(0) = 20 = Ce^{(a + b)0}  = C
\]
\[
M(10) = 40 = Ce^{(a+b)10}=20e^{(a + b)10}  \Leftrightarrow e^{(a +
b)10} = 2 \Leftrightarrow a + b = \frac{{\ln 2}}{{10}}
\]
Por lo tanto, la masa $M$ para todo tiempo $t$ vale:
\[
M(t) = 20\;e^{\frac{{\ln 2}}{{10}}t}
\]

\item Una vez que tenemos la masa para todo tiempo $t$, a los 30 s
tendremos:
\[
M(30) = 20\;e^{\frac{{\ln 2}}{{10}}30}  = 20e^{3\ln 2}  = 160
\]
donde la cantidad viene dada en gramos.

\item Para calcular el tiempo $t_0$ que debe transcurrir hasta que
tengamos 100 g de masa, sustituimos de nuevo en la solución general:
\[
M(t_0 ) = 100 = 20\;e^{\frac{{\ln 2}}{{10}}t_0 }  \Leftrightarrow
\ln 5 = \frac{{\ln 2}}{{10}}t_0  \Leftrightarrow t_0  = \frac{{10\ln
5}}{{\ln 2}} = 23.22
\]
donde el tiempo viene dado en segundos.
\end{enumerate}
}


\newproblem{edosep-12}{qui}{*}
%ENUNCIADO
{Se sabe que en una reacción química una sustancia se transforma en otra a una velocidad  proporcional a la cantidad
sin transformar. Si a las 2 horas del comienzo de la reacción había 20 gr. de la sustancia original y a las 3 horas
quedaban 10 gr., ¿qué cantidad es sustancia había al comienzo de la reacción? ¿Cuándo se habrá transformado el 90\% de
la sustancia?}
%SOLUCIÓN
{La cantidad original de sustancia era $x(0)=80$  gr y el tiempo que tiene que pasar para que se transforme el $90\%$
es $3.32$ horas.  }
%RESOLUCIÓN
{Llamemos $x(t)$ a la función que mide la cantidad de sustancia original en el instante $t$. Según el enunciado, la
transformación química responde a la ecuación diferencial 
\[
\frac{dx}{dt}=kx.
\]
Se trata de una ecuación diferencial de variables separables, así que, para resolverla separamos las variables e
integramos: 
\[
\frac{dx}{dt}=kx \Leftrightarrow \frac{dx}{x}=kdt \Leftrightarrow \int \frac{dx}{x} = \int k\,dt \Leftrightarrow
\log|x| = kt+C \Leftrightarrow x(t)=Ce^{kt}, 
\]
que es la solución general de la ecuación.

Para determinar las constantes imponemos las condiciones inicales:
\begin{align*}
x(2)=20 &\Leftrightarrow Ce^{2k} = 20 \Leftrightarrow e^{2k} = 20/C \Leftrightarrow 2k =
\log(20/C) \Leftrightarrow k=\log(20/C)/2,\\
x(3)=10 &\Leftrightarrow Ce^{3k} = 10 \Leftrightarrow Ce^{\frac{3}{2}\log(20/C)} = Ce^{\log(20/C)^{3/2}}=
C\left(\frac{20}{C}\right)^{3/2}=10 \Leftrightarrow C^{1/2}=\frac{20^{3/2}}{10} \Leftrightarrow C = 80. 
\end{align*}
de donde se deduce $k=\log(20/80)/2 = -\log 2$, y en consecuencia, la solución particular de la ecuación es
\[
x(t)=80e^{-\log2\cdot t}.
\]

Según esta ecuación, la cantidad original de sustancia en el instante inicial $(t=0)$ era
\[
x(0)=80e^{-\log2\cdot 0} = 80 \mbox{ gr},
\]
y el tiempo necesario para que se transforme el 90\% de la sustancia, es decir, que quede el 10\% será
\[
x(t_{0})=80*0.1=8 \Leftrightarrow 80e^{-\log2\cdot t_{0}}=8 \Leftrightarrow 
e^{-\log2\cdot t_{0}}=8/80=0.1 \Leftrightarrow t_{0}=-\frac{\log0.1}{\log2}=3.32 \mbox{ horas.}
\]
}


\newproblem{edosep-13}{qui}{}
%ENUNCIADO
{Un depósito contiene 5 kg de sal disueltos en 500 litros de agua en el instante en que comienza entrar una solución
salina con 0.4 kg de sal por litro a razón de 10 litros por minuto. Si la mezcla se mantiene uniforme mediante
agitación y sale la misma cantidad de litros que entra, ¿cuánta sal quedará en el depósito después de 5 minutos? ¿y
después de 1 hora?   

\noindent\textbf{Nota:} La tasa de variación de la cantidad de sal en el tanque es la diferencia entre la cantidad de
sal que entra y la que sale del tanque en cada instante.}
%SOLUCIÓN
{$C(t)=-195e^{-t/50}+200$. La cantidad de sal a los 5 minutos será $C(5)=23.557$ kg y a la hora $C(60)=141.267$ kg.
}
%RESOLUCIÓN
{}


\newproblem{edosep-14}{qui}{*}
%ENUNCIADO
{En una reacción química, un compuesto se transforma en otra sustancia a un ritmo proporcional al cuadrado de la
cantidad no transformada. Si había inicialmente 20 gr de la sustancia original y tras 1 hora queda la mitad, ¿en qué
momento se habrá transformado el 75\% de dicho compuesto?}
%SOLUCIÓN
{$C(t)=\frac{20}{t+1}$ y el instante en que se habrá transformado el 75\% de la cantidad inicial es $t_0=3$ horas.
}
%RESOLUCIÓN
{}


\newproblem{edosep-15}{gen}{}
%ENUNCIADO
{Cuando el movimiento se produce en un medio en el que hay cierta resistencia, como en el aire, aparece una fuerza
proporcional a la velocidad que se opone al mismo. En este caso, las leyes de Newton conducen a la siguiente ecuación
diferencial para la velocidad de caída en el medio:
\[
m\frac{{dv}} {{dt}} =  - kv - mg
\]
donde $v$ es la velocidad, $m$ es la masa, $g$ es la gravedad, y $k$ es la constante de proporcionalidad.

Si se dispara un móvil directamente hacia arriba al nivel del suelo, con velocidad inicial $100$ m/s, una masa de
$0.05$ kg, una constante $k$ de $0,002$ kg/s y $g$ de $10$ m/s$^2$, ¿cuál será la máxima altura del móvil y cuándo la
alcanzará? ¿Cuándo y con qué velocidad golpeará el móvil en el suelo?}
%SOLUCIÓN
{
}
%RESOLUCIÓN
{}


\newproblem{edosep-16}{amb}{*}
%ENUNCIADO
{La cantidad de masa, $M$, expresada en Kg, de sustancias contaminantes en un depósito de aguas residuales, cumple la
ecuación diferencial:
\[
\frac{{dM}} {{dt}} =  - 0.5M + 1000
\]
donde $k$ es una constante y $t$ es el tiempo expresado en días (podemos imaginar que el depósito está conectado a una
depuradora que elimina sustancia contaminante con un ritmo proporcional a la propia cantidad de contaminante, lo cual
explicaría el sumando $-0.5M$, y que también hay un aporte constante de contaminante de 1000 kg/día, que puede provenir
de un desagüe, lo cual explicaría el sumando constante $+1000$).

Si la cantidad inicial de contaminante es de 10000 Kg:
\begin{enumerate}
\item ¿Cuál será la cantidad de contaminante para todo tiempo $t$?
\item ¿Cuál será la cantidad de contaminante al cabo de una semana?
\end{enumerate}
}
%SOLUCIÓN
{\begin{enumerate}
\item $M(t)=8000e^{-0.5t}+2000$.
\item $M(7)=2241.579$ kg. 
\end{enumerate}
}
%RESOLUCIÓN
{}


\newproblem{edosep-17}{gen}{}
%ENUNCIADO
{Si tenemos en cuenta que cualquier onda sonora que atraviesa un medio sufre un proceso de amortiguamiento, y que su
Intensidad $I$ (cantidad de energía por unidad de área y tiempo que atravesaría una superficie colocada de forma
perpendicular a la dirección de desplazamiento de la onda, en w/m$^2$) viene dada por la ley de Lamber-Beer:
\[
\frac{{dI}}{{dx}} =  - \alpha I
\]
donde $\alpha$ es el coeficiente de absorción, y suponemos una onda sonora que llega a una pared con una intensidad de 1 w/m$^2$, y atraviesa 10 cm de pared con un coeficiente de absorción del material de la pared de $0,1$ cm$^-1$. En estas condiciones:
\begin{enumerate}
\item ¿Cuál es la intensidad que llega al otro lado de la pared?
\item Teniendo en cuenta que en ondas sonoras más que la intensidad misma se utiliza el nivel de intensidad $\beta$, cuya unidad es el decibelio, que viene dado por:
\[
\beta  = 10\log _{10} \frac{I}{{I_0 }}
\]
donde $I_0$ es una intensidad de referencia asociada con la intensidad más débil que se puede oír e igual a $10^{-12}$ W/m$^2$, calcular cuál es el nivel de intensidad de la onda entrante en la pared, y cuál el de la saliente.
\end{enumerate}
}
%SOLUCIÓN
{
}
%RESOLUCIÓN
{}


\newproblem{edosep-18}{med}{}
%ENUNCIADO
{El plasma sanguíneo se conserva a 4ºC. Para poder utilizarse en una transfusión el plasma tiene que alcanzar la
temperatura del cuerpo (37ºC). Sabemos que se tardan 45 minutos en alcanzar dicha temperatura en un horno a 50ºC.
¿Cuánto se tardará si aumentamos la temperatura del horno a 60º?}
%SOLUCIÓN
{Con el horno a 50ºC se tiene $T(t)=-46e^{-0.02808t}+50$, con el horno a 60ºC se tiene $T(t)=-56e^{-0.02808t}+60$ y en
este horno tardará $31.69$ min.}
%RESOLUCIÓN
{}


\newproblem{edosep-19}{gen}{}
%ENUNCIADO
{Hallar las curvas tales que en cada punto $(x,y)$ la pendiente de la recta tangente sea igual al cubo de la abscisa en
dicho punto. ¿Cuál de estas curvas pasa por el origen?}
%SOLUCIÓN
{$y=x^4/4$.
}
%RESOLUCIÓN
{}


\newproblem{edosep-20}{med}{}
%ENUNCIADO
{Al introducir glucosa por vía intravenosa a velocidad constante, el cambio de concentración global de glucosa  con
respecto al tiempo $c(t)$ se explica mediante la siguiente ecuación diferencial 
\[
\frac{dc}{dt}=\frac{G}{100V}-kc,
\]
donde $G$ es la velocidad constante a la que se suministra la glucosa, $V$ es el volumen total de la sangre en el
cuerpo y $k$ es una constante positiva que depende del paciente. Se pide calcular $c(t)$.}
%SOLUCIÓN
{$c(t)=De^{kt}+\frac{G}{100Vk}$
}
%RESOLUCIÓN
{}


\newproblem{edosep-21}{amb}{}
%ENUNCIADO
{La temperatura $T$ de una habitación en un día de invierno varía con el tiempo de acuerdo a la ecuación:
\[
\frac{dT}{dt}=
\left\{
  \begin{array}{ll}
    40-T, & \hbox{si la calefacción está encendida;} \\
    -T, & \hbox{si la calefacción está apagada.}
  \end{array}
\right.
\]
Suponiendo que la temperatura del aula es de 5ºC  a las 9:00 de la mañana, y que a esa hora se enciende la
calefacción, pero que debido a una avería la calefacción permanece apagada de 11:00 a 12:00, ¿qué temperatura habrá en
la habitación a las 13:00?}
%SOLUCIÓN
{De 9 a 11 la temperatura es $T(t)=-35e^{-t}+40$ y la temperatura a las $11$ será de $35.263$ºC.\\
De 11 a 12 la temperatura es $T(t)=35.263e^{-t}$ y la temperatura a las $12$ será de $12.973$ºC.\\
De 12 a 13 la temperatura es $T(t)=-27.027e^{-t}+40$ y la temperatura a las $13$ será de $30.057$ºC.
}
%RESOLUCIÓN
{}


\newproblem{edosep-22}{amb}{}
%ENUNCIADO
{Se considera que la población de una determinada ciudad, $P(t)$, con índices constantes de natalidad y mortalidad,
$\beta$ y $\gamma$ respectivamente, pero en la que también ingresan por inmigración $I$ personas al año, sigue la
ecuación diferencial:
\[
\frac{{dP}} {{dt}} = \left( {\beta  - \gamma } \right)P + I
\]
Suponiendo que dicha población tenía $1,5$ millones de habitantes en $1980$, que la diferencia entre los índices de
natalidad y mortalidad es de $0.01$ (es decir, crece un $1\%$ anual), y también que absorbe $40000$ inmigrantes al año,
¿cuál será la población en el año $2005$?}
%SOLUCIÓN
{
}
%RESOLUCIÓN
{}


\newproblem{edosep-23}{qui}{}
%ENUNCIADO
{Una reacción química se comporta según la siguiente ecuación diferencial:
\[
y\sqrt {2x} \,dy - 2y^2 \,dx = 0
\]
donde $y$ es la energía liberada (en Kj) y $x$ es la cantidad de una determinada sustancia (en gr). Sabiendo que para 2
gr la energía liberada es de 50 Kj, ¿cuánta cantidad habrá que utilizar para obtener 1000 Kj?}
%SOLUCIÓN
{$6.12$  gr.
}
%RESOLUCIÓN
{Se trata de una ecuación diferencial ordinaria de variables separables, así que, para resolverla primero separamos las variables
\[
y\sqrt {2x} \,dy - 2y^2 \,dx = 0
\Leftrightarrow
y\sqrt {2x} \,dy =  2y^2 \,dx 
\Leftrightarrow
\frac{y}{y^2}dy =  \frac{2}{\sqrt{2x}}dx 
\Leftrightarrow
\frac{1}{y}dy =  \frac{2}{\sqrt{2x}}dx 
\]
y ahora integramos ambos miembros de la ecuación
\begin{align*}
\int \frac{1}{y}dy &= \ln y +C, \\
\int \frac{2}{\sqrt{2x}}dx &= 2\sqrt{2x}+C.
\end{align*}
Por tanto, la solución general de la ecuación es
\[
\ln y = 2\sqrt{2x}+C 
\Leftrightarrow
y(x) = e^{2\sqrt{2x}+C} = e^{2\sqrt{2x}}e^C = C e^{2\sqrt{2x}},
\]
renonbrando $e^C$ como una constante $C$. 

Para llegar a una solución particular, imponemos la condición inicial que nos dan, que es $y(2)=50$.
\[
y(2) = C e^{2\sqrt{2\cdot 2}} = C e^4 = 50  
\Leftrightarrow
C = \frac{50}{e^4}
\]
Así pues, la solución particular es
\[
y(t) = \frac{50}{e^4} e^{2\sqrt{2x}} = 50  e^{2\sqrt{2x}-4}.
\]

Por último, para ver la masa $x_0$ necesaria para generar 1000 Kj, sustituimos en la solución particular
\[
\renewcommand{\arraystretch}{2}
\begin{array}{c}
y(x_0) = 50  e^{2\sqrt{2x_0}-4} = 1000 
\Leftrightarrow
e^{2\sqrt{2x_0}-4} = 	\dfrac{1000}{50} = 20
\Leftrightarrow \\
\Leftrightarrow 
2\sqrt{2x_0}-4 = 	\ln 20 = 2.9957
\Leftrightarrow
\sqrt{2x_0} = \dfrac{2,9957+4}{2} = 3.4979
\Leftrightarrow
x_0 = \dfrac{3,4979 ^2}{2} = 6.12 \mbox{ gr}.
\end{array}
\]
}


\newproblem{edosep-24}{gen}{*}
%ENUNCIADO
{Resolver el problema del valor inicial
\[
\left\{
  \begin{array}{l}
    y\sqrt{2x}dy-2y^2dx=0\\
    y(0)=5
  \end{array}
\right.
\]
¿Para qué valor de $x$, se obtiene $y=1000$?
}
%SOLUCIÓN
{$y(x) = 5 e^{2\sqrt{2x}}$. El valor de $x$ para el que $y=1000$ es $x=3.509$.
}
%RESOLUCIÓN
{Se trata de una ecuación diferencial ordinaria de variables separables, por lo que, para resolverla primero debemos separar las variables
\[
y\sqrt{2x}dy-2y^2dx=0 \Leftrightarrow  y\sqrt{2x}dy = 2y^2dx \Leftrightarrow \frac{y}{y^2}dy = \frac{2}{\sqrt{2x}}dx \Leftrightarrow \frac{1}{y}dy = \sqrt{2}x^{-1/2}dx
\]
Una vez separadas las variables integramos ambos lados de la ecuación
\[
\int \frac{1}{y}dy = \int \sqrt{2}x^{-1/2}dx \Leftrightarrow \log y = 2\sqrt{2x} +C
\]
y despejando $y$  obtenemos la solución general de la ecuación
\[
y(x) = e^{2\sqrt{2x}+C} = Ce^{2\sqrt{2x}}.
\]
Para obtener la solución particular imponemos la condición inicial $y(0)=5$,
\[
y(0) = Ce^{2\sqrt{2\cdot 0}} = 5 \Leftrightarrow C e^{0} = 5 \Leftrightarrow C = 5,
\]
de modo que la solución del problema del valor inicial es
\[
y(x) = 5 e^{2\sqrt{2x}}.
\]

Finalmente, calculamos el valor $x$ para el que $y=1000$:
\[
y(x) = 5 e^{2\sqrt{2x}} = 1000 \Leftrightarrow e^{2\sqrt{2x}} = \frac{1000}{5}=200 \Leftrightarrow 2\sqrt{2x} = \log 200 \Leftrightarrow x = \frac{(\log 200/2)^2}{2} = 3.509.
\]
}


\newproblem{edosep-25}{amb}{*}
%ENUNCIADO
{La velocidad de aumento del número de bacterias en un cultivo es proporcional al número de bacterias presentes, siguiendo la ecuación: 
\[
\frac{dx}{dt}=ax
\]
siendo $x$ el número de bacterias presentes y $t$ el tiempo.
\begin{enumerate}
\item  ¿Por cuánto se habrá multiplicado el número de bacterias al cabo de $5$ horas, si se duplicó al cabo de $3$ horas?
\item  Si al cabo de $4$ horas hay $10000$ bacterias,  ¿cuántas había al principio?
\end{enumerate}
}
%SOLUCIÓN
{La solución general de la ecuación es $x(t)=Ce^{at}$.
\begin{enumerate}
\item $k=3.17$.
\item Al principio había $3968$ bacterias.
\end{enumerate}
}
%RESOLUCION
{Antes de contestar a los apartados resolvemos la ecuación diferencial que plantea el problema. Se trata de una ecuación diferencial de variables separadas que se resuelve fácilmente: 
\[
\frac{dx}{dt}=ax \Longleftrightarrow \frac{dx}{x}=a\,dt \Longleftrightarrow 
\int \frac{dx}{x} = \int a\,dt \Longleftrightarrow \ln x = at+C \Longleftrightarrow
e^{\ln x}=e^{at+C},
\]
y, aplicando la función exponencial a ambos lados de la última igualdad para simplificar, obtenemos la solución general 
\[
e^{\ln x}=e^{at+C} \Longleftrightarrow x=e^{at}e^{C} \Longleftrightarrow x=Ce^{at}.
\]
donde, para simplificar, hemos reescrito $C=e^C$ al ser una constante.
\begin{enumerate}
\item Para resolver el primer apartado, llamamos $x(0)$ al número inicial de bacterias en el cultivo. Como al cabo de 3 horas se había duplicado el número de bacterias en el cultivo, tenemos la ecuación $x(3)=2x(0),$ que al revolverla nos lleva a 
\[
x(3)=2x(0) \Longleftrightarrow Ce^{3a}=2Ce^{0a}=2C \Longleftrightarrow
e^{3a}=2 \Longleftrightarrow 3a=\ln 2 \Longleftrightarrow a=\frac{\ln 2}{3}.
\]

Para saber por cuanto se habrá multiplicado el número de bacterias al cabo de 5 horas, planteamos igual que antes la ecuación $x(5)=kx(0),$ donde $k$ es el factor de multiplicación. Al resolver esta ecuación obtenemos 
\[
x(5)=kx(0)\Longleftrightarrow Ce^{5\frac{\ln 2}3}=kCe^{0\frac{\ln 2}{3}}=kC \Longleftrightarrow e^{5\frac{\ln 2}{3}}=k \Longleftrightarrow k=3.17.
\]
Luego al cabo de 5 horas habrá aproximadamente tres veces más bacterias que al comienzo

\item El número de inicial de bacterias es
\[
x(0)=Ce^{0\frac{\ln 2}3}=C.
\]

Ahora bien, como al cabo de 4 horas había 1000 bacterias, planteamos la ecuación $x(4)=10000$, que al resolverla, nos proporciona el valor de $%
C.$
\[
x(4)=Ce^{4\frac{\ln 2}{3}}=10000 \Longleftrightarrow C=\frac{10000}{e^{4\frac{\ln 2}3}} = 3968.5.
\]
\end{enumerate}
}


\newproblem{edosep-26}{gen}{*}
%ENUNCIADO
{Dada la ecuación diferencial: $yy'+ e^{x^2}x = 2xy^2e^{x^2}$, calcular el valor de $y(1)$ sabiendo que $y(0)=-1$.}
%SOLUCIÓN
{$y(1)=4.005$.}
%RESOLUCIÓN
{}


\newproblem{edosep-27}{gen}{*}
%ENUNCIADO
{Dos figuras de cerámicas del mismo material se ponen en un horno para su cocción a $1000^\circ$C. 
Si en el instante en que se meten al horno la primera está a $40^\circ$C y la segunda a $5^\circ$C, y al minuto la temperatura de la primera ha aumentado hasta los $200^\circ$C, ¿cuales serán sus temperaturas a los 5 minutos?  
}
%SOLUCIÓN
{
}
%RESOLUCIÓN
{}


\newproblem{edosep-28}{qui}{*}
%ENUNCIADO
{El átomo de radio se desintegra dando helio y una emanación gaseosa, radón, que también es radioactiva. 
Sabiendo que la velocidad de desintegración es proporcional a la masa ($m$) en cada instante, se pide:
\begin{enumerate}
\item Resolver la ecuación diferencial que explica la desintegración del radio.
\item Calcular la constante de desintegración sabiendo que la masa del radio disminuye un $0.043\%$ cada año.
\item Calcular el periodo del radio, que es el instante $T$ tal que  $m(t+T)=\frac{1}{2}m(t)$ $\forall t\geq 0$.
\end{enumerate} 
}
%SOLUCIÓN
{
}
%RESOLUCIÓN
{}


\newproblem{edosep-29}{gen}{*}
%ENUNCIADO
{Obtener la ecuación de la curva que pasa por el punto $P=(1,1)$, tal que la pendiente de la tangente en cada punto coincida con el cuadrado de su ordenada.
}
%SOLUCIÓN
{$y=\frac{-1}{x-2}$.
}
%RESOLUCIÓN
{}


\newproblem{edosep-30}{med}{*}
%ENUNCIADO
{Un investigador constata que, tras una inyección intravenosa de glucosa, la tasa de glucosa en sangre $g(t)$ en cada instante $t$ sigue la ecuación diferencial
\[
g'+kg=0,
\] 
donde $k>0$ es una constante conocida como \emph{coeficiente de asimilación}. Se pide:
\begin{enumerate}
\item Resolver la ecuación diferencial para un sujeto cuya tasa de glucosa en el instante de aplicar la inyección es 80 mg/dl.
\item Si el valor del coeficiente de asimilación varía de $1.06\cdot 10^{-2}$ a $2.42\cdot 10^{-2}$ en los sujetos normales, estudiar si los resultados del sujeto anterior son normales tiendo en cuenta que a los 30 minutos la tasa de glucosa era de $1.2$ mg/dl.
\end{enumerate}
}
%SOLUCIÓN
{
}
%RESOLUCIÓN
{}


\newproblem{edosep-31}{gen}{*}
%ENUNCIADO
{El carbono contenido en la materia viva incluye una ínfima proporción del isótopo radioactivo $C^{14}$, que proviene de los rayos cósmicos de la parte superior de la atmósfera.
Gracias a un proceso de intercambio complejo, la materia viva mantiene una proporción constante de $C^{14}$ en su carbono total (esencialmente constituido por el isótopo estable $C^{12}$).
Después de morir, ese intercambio cesa y la cantidad de carbono radioactivo disminuye: pierde $1/8000$ de su masa al año.
Estos datos permiten determinar el año en que murió un individuo. 
Se pide:
\begin{enumerate}
\item Si el análisis de los fragmentos de un esqueleto de un hombre de Neandertal mostró que la proporción de $C^{14}$ era de $6.24\%$ de la que hubiera tenido al estar vivo.
¿Cuándo murió el individuo?
\item Calcular la vida media del carbono $C^{14}$, es decir, el tiempo a partir del cual se ha desintegrado la mitad del carbono inicial.  
\end{enumerate}
}
%SOLUCIÓN
{
}
%RESOLUCIÓN
{}


\newproblem{edosep-32}{amb}{}
%ENUNCIADO
{Una colonia de salmones vive tranquilamente en una zona costera.
La tasa de natalidad es del 2\% por día y la de mortalidad del 1\% por día. 
En el instante inicial, la colonia tiene 1000 salmones y ese día llega un tiburón a esa zona costera que se come 15 salmones todos los días.
¿Cuánto tiempo tarda el tiburón en hacer desaparecer a la colonia de salmones?
}
%SOLUCIÓN
{Aproximadamente 110 días.
}
%RESOLUCIÓN



% Autor: Alfredo Sánchez Alberca (asalber@ceu.es)

\newproblem{edohom-1}{gen}{}
%ENUNCIADO
{Comprobar que las siguientes ecuaciones diferenciales son homogeneas y resolverlas:
\begin{enumerate}
\item $xy'=\sqrt{x^2-y^2}+y$.
\item $4x^2−xy+y^2+y'(x^2−xy+4y^2)=0$.
\item $(x+y)dx + (x-y)dy = 0$.
\end{enumerate}
}
%SOLUCIÓN
{
\begin{enumerate}
\item $y=\sen(\log(Cx))$.
\item $\dfrac{1}{4}\log\left(\dfrac{x^2-xy+y^2}{x^2}\right)+\frac{1}{2}\log\left(\dfrac{x+y}{x}\right)=-\log x +C$.
\item $\arctg\left(\dfrac{y}{x}\right)-\dfrac{1}{2}\log\left(\dfrac{x^2+y^2}{x^2}\right)=\log x + C$. 
\end{enumerate}
}
%RESOLUCIÓN
{}

% Version control information:
\svnidlong
{$HeadURL: https://ejercicioscalculo.googlecode.com/svn/trunk/edo_separables.tex $}
{$LastChangedDate: 2010-01-28 20:28:03 +0100 (jue, 28 ene 2010) $}
{$LastChangedRevision: 11 $}
{$LastChangedBy: asalber $}
%\svnid{$Id: edo_separables.tex 11 2010-01-28 19:28:03Z asalber $
%
\newproblem{edolin-1}{gen}{}
%ENUNCIADO
{Integrar las siguientes ecuaciones diferenciales lineales:
\begin{enumerate}
\item $y'-2y=4$.
\item $y'-6xy=x$.
\item $\frac{dz}{dt}+\frac{3z}{10+3t}=6$ con la condición inical $z(2)=100$.
\item $y'+y\cos x=\sen x\cos x$ con la condición inicial $y(0)=1$.
\end{enumerate}
}
%SOLUCIÓN
{
\begin{enumerate}
\item $y=Ce^{2x}-2$.
\item $y=Ce^{3x^2}-\frac{1}{6}$.
\item $z=\dfrac{9t^2+60t+1444}{3t+10}$.
\item $y=2e^{-\sen x}+\sen x -1$.
\end{enumerate}
}
%RESOLUCIÓN
{}


\newproblem{edolin-2}{gen}{}
%ENUNCIADO
{Un tanque de 50 litros contiene inicialmente 10 litros de agua. En el instante inicial se vierte al tanque una disolución salina que
contiene 100 gr de sal por cada litro de agua, a razón de 4 litros por minuto, mientras que la mezcla bien agitada abandona el tanque a un
ritmo de 2 litros por minuto. ¿Cuánto tiempo transcurrirá hasta que se llene el depósito? En dicho instante, ¿qué cantidad de sal habrá en
el depósito?}
%SOLUCIÓN
{El depósito se llenará a los 20 minutos y contendrá $4.8$ kg de sal. 
}
%RESOLUCIÓN
{}


{}
%% Version control information:
\svnidlong
{$HeadURL: https://ejercicioscalculo.googlecode.com/svn/trunk/edo_separables.tex $}
{$LastChangedDate: 2008-07-09 20:02:39 +0200 (mié, 09 jul 2008) $}
{$LastChangedRevision: 9 $}
{$LastChangedBy: asalber $}
%\svnid{$Id: edo_separables.tex 9 2008-07-09 18:02:39Z asalber $
%
\newproblem{err-1}{gen}{}
%ENUNCIADO
{Si al medir el volumen de un objeto se obtiene como valor más probable $23618.87543$ cm$^3$ con un error de $302.432$
cm$^3$, ¿cuál será la expresión correcta del resultado de dicha medida?     
}
%SOLUCIÓN
{$23620\pm310$ cm$^3$.
}
%RESOLUCIÓN
{Redondeando el error por exceso a la segunda cifra significativa tenemos $310$ cm$^3$, y redondeando la medida a la
primera cifra afectada por el error, es decir a las decenas, tenemos $23620$ cm$^3$, luego la expresión se la medida
será $23620\pm310$ cm$^3$.}


\newproblem{err-2}{qui}{}
%ENUNCIADO
{Se ha medido el tiempo de caída de una esfera en un líquido cuatro veces, obteniéndose los valores: $30.5$; $29.6$;
$30.1$; $30.8$. ¿Cuál será el resultado del experimento?}
%SOLUCIÓN
{$\bar t=20.25$ s y $s_{\bar t}=0.2598$, de manera que suponiendo que la incertidumbre en el aparato de medida es
$0.1$ s, el tiempo de caída es $30.25\pm0.36$.
}
%RESOLUCIÓN
{}


\newproblem{err-3}{far}{}
%ENUNCIADO
{En el análisis de la orina de una atleta, en un control antidopaje, se detectan $16.5 \pm 0.8$ $\mu$g de cafeína por
cada cm$^3$ de orina. El resultado del contra-análisis es igual a $13.5 \pm 0.5$ $\mu$g de cafeína por cm$^3$. ¿Son
ambos resultados coincidentes o necesariamente ha existido algún fallo en uno de ellos? Sabiendo que el límite máximo
permitido es de $12$ $\mu$g de cafeína por cm$^3$, ¿ha dado o no positivo dicho atleta?
}
%SOLUCIÓN
{Según el análisis la concentración de cafeina está en el intervalo $(14.1\,,\,18.9)$ $\mu$g/cm$^3$ con un $99.7\%$ de
confianza y según el contra-análisis está en el intervalo $(12,15)$ $\mu$g/cm$^3$ con la misma confianza, luego como
ambos intervalos se solapan se puede decir que son medidas coincidentes. Como ambos intervalos están por encima de 12
$\mu$g/cm$^3$ si se puede decir que el atleta ha dado positivo.
}
%RESOLUCIÓN
{}


\newproblem{err-4}{gen}{}
%ENUNCIADO
{Mediante una cinta métrica dividida en milímetros se ha obtenido que la longitud de una mesa es $1.5250$ m y que tiene
una anchura de $82.00$ cm. ¿Cuál es el área de dicha mesa?
}
%SOLUCIÓN
{$12505\pm12$ cm$^2$.
}
%RESOLUCIÓN
{}


\newproblem{err-5}{gen}{}
%ENUNCIADO
{Supongamos que nos piden calcular la densidad de una pieza homogénea de forma cónica sabiendo que su masa es $m=
300.23\pm 0.05$ g, su altura $h=12.3 \pm 0.1$ cm, y el radio de la base $r=7.44 \pm 0.01$ cm. Teniendo en cuenta que el
volumen de un cono es igual $\pi r^2 h/3$, calcular cuánto vale la densidad de la pieza cónica.
}
%SOLUCIÓN
{$0.4211\pm0.0047$ g/cm$^3$.
}
%RESOLUCIÓN
{}


\newproblem{err-6}{gen}{}
%ENUNCIADO
{Si suponemos dos cilindros concéntricos, cuyos radios son $r$ el interno y $R$ el externo, y los dos con altura $h$, y
consideramos el volumen de la pieza que queda entre los dos cilindros, calcular el volumen de dicha pieza estimando
correctamente el error cometido teniendo en cuenta que se han realizado 6 medidas diferentes de $r$, $R$ y $h$:
\[
\begin{array}{|c|c|c|c|}
\hline
\mbox{Medida} & \mbox{R (mm)} & \mbox{r (mm)} & \mbox{h(cm)} \\
\hline
1 & 48.51 & 43.42 & 29.12 \\
\hline
2 & 47.39 & 42.94 & 29.14 \\
\hline
3 & 48.81 & 42.59 & 28.99 \\
\hline
4 & 47.52 & 43.11 & 29.13 \\
\hline
5 & 47.93 & 42.45 & 29.13 \\
\hline
6 & 47.88 & 42.11 & 29.06 \\
\hline
\end{array}
\]
}
%SOLUCIÓN
{$\bar R=48.0067$ mm y $s_{\bar R}=0.2264$ mm con lo que se obtiene la medida $R=48.01\pm 0.24$mm.\\
$\bar r = 42.77$ mm y $s_{\bar r}=0.19471$ mm con lo que se obtiene la medida $r=42.77\pm 0.21$mm.\\
$\bar h= 292.45$ mm y $s_{\bar h}=1.44839$ mm con lo que se obtiene la medida $h=292.5\pm 1.6$ mm.
La medida indirecta del volumen es $440\pm 27$ cm$^3$. 
}
%RESOLUCIÓN
{}


\newproblem{err-7}{fis}{}
%ENUNCIADO
{En una fuente se ha llenado completamente de agua un recipiente de base cuadrada de lado $l$ y altura $h$ en un tiempo
$t$, y se quiere medir el caudal de agua $q$ que mana de la fuente ($q=(l^2h)/t$). Además, el experimento lo han
realizado consecutivamente 4 alumnos obteniendo:
\[
\begin{array}{|c|c|c|c|}
\hline
\mbox{Alumno} & l & h & t \\
\hline
1 & 220.4 \pm 0,1 \mbox{ mm} & 535.3 \pm 0.1 \mbox{ mm} & 314.6 \pm 0.1 \mbox{ s} \\
\hline
2 & 220.6 \pm 0.1 \mbox{ mm} & 535.2 \pm 0.1 \mbox{ mm} & 313.9 \pm 0.1 \mbox{ s} \\
\hline
3 & 220.8 \pm 0.1 \mbox{ mm} & 535.9 \pm 0.1 \mbox{ mm} & 314.2 \pm 0.1 \mbox{ s} \\
\hline
4 & 221.0 \pm 0.1 \mbox{ mm} & 535.6 \pm 0.1 \mbox{ mm} & 314.8 \pm 0.1 \mbox{ s} \\
\hline
\end{array}
\]
}
%SOLUCIÓN
{$q_1=82650\pm 120$ mm$^3$/s, $q_2=82970\pm 120$ mm$^3$/s, $q_3=83150\pm 120$ mm$^3$/s y $q_4=83010\pm 120$ mm$^3$/s.
La medida final del caudal es $q=82969\pm 59$ mm$^3$/s.
}
%RESOLUCIÓN
{}


\newproblem{err-8}{gen}{*}
%ENUNCIADO
{El volumen de un cono es $V(r,h)=\frac{1}{3}\pi r^2 h$ donde $r$ es el radio de la base y $h$ la altura. Si se ha medido un cono y se ha
obtenido un radio de 2 m y una altura de 1 m, se pide:
\begin{enumerate}
\item Calcular el gradiente del volumen del cono anterior. ¿Cómo se interpretaría?
\item Calcular el hessiano del volumen del cono.
\item Si el radio se ha medido una única vez con un error de $\pm 0,01$ m y la altura también una única vez con un error de $\pm 0,001$ m,
dar la expresión de la medida del volumen con su error.
\end{enumerate}
}
%SOLUCIÓN
{\begin{enumerate}
\item $\grad V(r,h)=(\frac{2}{3}\pi r h,\frac{1}{3}\pi r^2)$, $\grad V(2,1) = (\frac{4}{3}\pi,\frac{4}{3}\pi)$.
\item $H V(r,h)=\left(
\begin{array}{cc}
\frac{2}{3}\pi h & \frac{2}{3}\pi r\\
\frac{2}{3}\pi r & 0
\end{array}
\right)$,
$H V(2,1)=\left(
\begin{array}{cc}
\frac{2}{3}\pi & \frac{4}{3}\pi\\
\frac{4}{3}\pi r & 0
\end{array}
\right)$
y
$|H V(2,1)| = -\frac{16}{9}\pi^2$.
\item $V=4.189\pm 0.047$ $m^3$.
\end{enumerate}
}
%RESOLUCIÓN
{}


\newproblem{err-9}{gen}{*}
%ENUNCIADO
{Una magnitud física $m$, que se mide de forma indirecta, depende de otras dos $x$ e $y$ según la fórmula:
\[
m(x,y) = \sqrt{xy}\log _4 \left(\frac{x}{y}\right)
\]
Si se han medido directamente $x$ e $y$ con una imprecisión igual en ambos casos $\triangle x=\triangle y=0.1$, y después de un tratamiento
estadístico de los datos se han obtenido las siguientes medias y errores típicos de las medias: $\bar{x}=10$, $\bar{y}=15$,
$\sigma_{\bar{x}}=0,2$ y $\sigma_{\bar{y}}=0,3$, se pide:

\begin{enumerate}
\item Calcular el gradiente de la magnitud $m(x,y)$.
\item ¿Cuánto valdría el error cometido en la medida de $m(x,y)$?
\item ¿Cómo se expresaría el resultado obtenido para $m$?
\end{enumerate}
}
%SOLUCIÓN
{\begin{enumerate}
\item $\grad m(x,y)=\left(\dfrac{\sqrt y}{\sqrt x \ln 4}\left(\dfrac{\ln x-\ln y}{2}+1\right), \dfrac{\sqrt x}{\sqrt y \ln
4}\left(\dfrac{\ln x-\ln y}{2}-1\right) \right)$
\item $\varepsilon = 0.35347$.
\item $m=-3.58\pm 0.36$.
\end{enumerate}
}
%RESOLUCIÓN
{}



\section{Cálculo diferencial en una variable}
\begin{enumerate}[leftmargin=*]
\item \useproblem{der-9}
\item \useproblem{der-10}
\item \useproblem{der-12}
\item \useproblem{der-23}
\item \useproblem{der-27}
\item \useproblem{der-28}
\item \useproblem{der-25}
\item \useproblem{der-26}
\item \useproblem{der-29}
\item \useproblem{der-30}
\item \useproblem{der-31}
\item \useproblem{der-33}
\item \useproblem{der-32}
\item \useproblem{derpar-1}
\item \useproblem{derpar-2}
\item \useproblem{derpar-4}
\item \useproblem{derpar-5}
\item \useproblem{derpar-11}
\item \useproblem{derpar-12}
\item \useproblem{derpar-13}
\item \useproblem{derpar-14}
\item \useproblem{ext-1}
\item \useproblem{ext-2}
\item \useproblem{ext-12}
\item \useproblem{ext-8}
\item \useproblem{ext-9}
\item \useproblem{ext-3}
\item \useproblem{ext-5}
\item \useproblem{ext-7}
\item \useproblem{ext-10}
\end{enumerate}


\section{Cálculo diferencial en varias variables}
\begin{enumerate}[leftmargin=*,resume]
\item \useproblem{par-1}
\item \useproblem{par-32}
\item \useproblem{par-33}
\item \useproblem{par-34}
\item \useproblem{par-3}
\item \useproblem{par-5}
\item \useproblem{par-4}
\item \useproblem{par-35}
\item \useproblem{par-37}
\item \useproblem{par-38}
\item \useproblem{par-39}
\item \useproblem{par-36}
\item \useproblem{par-25}
\item \useproblem{dertray-1}
\item \useproblem{dersup-1}
\item \useproblem{derimp-1}
\item \useproblem{derimp-2}
\item \useproblem{derimp-3}
\item \useproblem{derimp-13}
\item \useproblem{derimp-14}
\item \useproblem{derimpn-1}
\item \useproblem{derimpn-2}
\item \useproblem{par-40}
\item \useproblem{par-41}
\item \useproblem{par-42}
\item \useproblem{par-43}
\item \useproblem{par-2}
\item \useproblem{par-11}
\item \useproblem{par-15}
\item \useproblem{par-28}
\item \useproblem{tayn-2}
\item \useproblem{tayn-3}
\item \useproblem{tayn-4}
\item \useproblem{tayn-1}
\item \useproblem{par-26}
\item \useproblem{extn-3}
\item \useproblem{extn-4}
\item \useproblem{extn-1}
\item \useproblem{extn-2}
\end{enumerate}

\section{Ecuaciones Diferenciales}
\begin{enumerate}[leftmargin=*,resume]
\item \useproblem{edosep-1}
\item \useproblem{edosep-2}
\item \useproblem{edosep-3}
\item \useproblem{edosep-4}
\item \useproblem{edosep-13}
\item \useproblem{edosep-6}
\item \useproblem{edosep-7}
\item \useproblem{edosep-11}
\item \useproblem{edosep-14}
\item \useproblem{edosep-16}
\item \useproblem{edosep-18}
\item \useproblem{edosep-19}
\item \useproblem{edosep-29}
\item \useproblem{edosep-20}
\item \useproblem{edosep-21}
\item \useproblem{edosep-27}
\item \useproblem{edosep-28}
\item \useproblem{edosep-31}
\item \useproblem{edolin-1}
\item \useproblem{edolin-2}
\item \useproblem{edosep-32}
\item \useproblem{edolin-3}
\end{enumerate}

\vspace{2cm}

\textsc{Nota}: Los problemas marcados con una estrella ($\bigstar$) son problemas de exámenes de otros años.

\end{document}
