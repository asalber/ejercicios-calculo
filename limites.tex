% Autor: Alfredo Sánchez Alberca (asalber@ceu.es)

\newproblem*{lim-1}{gen}{}
%ENUNCIADO
{Dar ejemplo de funciones que cumplan las siguientes condiciones:
\begin{enumerate}
  \item $\lim_{x\rightarrow 0}f(x)=+\infty$ y $\lim_{x\rightarrow
  \infty}f(x)=0$.

  \item $\lim_{x\rightarrow 3^+}f(x)=+\infty$, $\lim_{x\rightarrow
  3^-}f(x)=-\infty$ y $\lim_{x\rightarrow \infty}f(x)=1$.

  \item $\lim_{x\rightarrow 0^+}f(x)=+\infty$ y $\lim_{x\rightarrow
  \infty}f(x)=-\infty$.
\end{enumerate}
}


\newproblem*{lim-2}{gen}{}
%ENUNCIADO
{Determinar el valor de $a$ para que exista $\lim_{x\rightarrow 1}f(x)$ sabiendo que
\[ f(x)=\left\{
  \begin{array}{lll}
    x+1 & & \mbox{si }x\leq 1,\\
    3-ax^2 & & \mbox{si } x>1.
  \end{array}
   \right.
\]
}


\newproblem*{lim-3}{gen}{}
%ENUNCIADO
{Calcular los siguientes límites si existen:
\begin{multicols}{2}
\begin{enumerate}
    \item  $\displaystyle \lim_{x\rightarrow 1}\dfrac{x^3-3x+2}{x^4-4x+3}$.

    \item  $\displaystyle \lim_{x\rightarrow a}\dfrac{\sen x-\sen a}{x-a}$.

    \item $\displaystyle \lim_{x\rightarrow\infty}\dfrac{x^2-3x+2}{e^{2x}}$.

    \item $\displaystyle \lim_{x\rightarrow\infty}\dfrac{\log(x^2-1)}{x+2}$.

    \item $\displaystyle \lim_{x\rightarrow 1}\dfrac{\log(1/x)}{\tg(x+\dfrac{\pi}{2})}$.

    \item $\displaystyle \lim_{x\rightarrow a}\dfrac{x^n-a^n}{x-a}\quad n\in \mathbb{N}$.

    \item $ \displaystyle \lim_{x\rightarrow
    1}\dfrac{\sqrt[n]{x}-1}{\sqrt[m]{x}-1}\quad n,m \in \mathbb{Z}$.

    \item $\displaystyle \lim_{x\rightarrow 0}\dfrac{\tg x-\sen x}{x^3}$.

    \item $\displaystyle \lim_{x\rightarrow \pi/4}\dfrac{\sen x-\cos x}{1-\tg x}$.

    \item $\displaystyle \lim_{x\rightarrow 0}x^2e^{1/x^2}$.

    \item $\displaystyle \lim_{x\rightarrow \infty}\left(1+\dfrac{a}{x}\right)^x$.

    \item $\displaystyle \lim_{x\rightarrow \infty} \sqrt[x]{x^2}$.

    \item $\displaystyle \lim_{x\rightarrow 0}\left(\dfrac{1}{x}\right)^{\tg x}$.

    \item $\displaystyle \lim_{x\rightarrow 0}(\cos x)^{1/\mbox{\footnotesize sen}\, x}$.

    \item $\displaystyle \lim_{x\rightarrow 0}\dfrac{6}{4+e^{-1/x}}$.

    \item $\displaystyle \lim_{x\rightarrow
    \infty}\left(\sqrt{x^2+x+1}-\sqrt{x^2-2x-1}\right)$.

    \item $\displaystyle \lim_{x\rightarrow \pi/2}\sec x-\tg x$.
\end{enumerate}
\end{multicols}
}


\newproblem*{lim-4}{gen}{}
%ENUNCIADO
{Calcular los límites laterales de la función
$f(x)=\dfrac{1}{1+e^{\frac{1}{1-x}}}$ en el punto $x=1$.
}


\newproblem*{lim-5}{gen}{}
%ENUNCIADO
{Obtener las asíntotas de las siguientes funciones:
\begin{multicols}{2}
\begin{enumerate}
    \item  $f(x)=xe^{1/x}$.

    \item  $g(x)=\log(x^2+3x+2)$.

    \item  $j(x)=\dfrac{x^3}{(x+1)^2}$.
    
    \item  $h(x)=\cos x-\log(\cos x)$.
\end{enumerate}
\end{multicols}
}


\newproblem*{lim-6}{gen}{}
%ENUNCIADO
{Determinar el dominio y las asíntotas de la función:
\[f(x)=\sqrt{\frac{x^3}{x-1}}.\]

\noindent\textbf{Nota}: Calcular las asíntotas horizontales y oblicuas sólo para $x\rightarrow +\infty$.
}


\newproblem*{lim-7}{amb}{*}
%ENUNCIADO
{Se ha analizado la cantidad de un nutriente en suelo (mg/K), necesario para el crecimiento de las plantas, en función del tiempo (en días) desde que se procedió a su reposición mediante abonado, y viene dada por la siguiente función:
\[
\renewcommand{\arraystretch}{2.2}
C(t) = \left\{ {\begin{array}{*{20}c}
   {\dfrac{{a \cdot e^{ - \dfrac{1}{2}\cdot t} }}{{\sqrt {t + 1} }}} & {{\rm si}} & {0 \le t < 3}  \\
   {\dfrac{{b\ln \left( {t^2  + 1} \right)}}{{t^2  + 1}}} & {{\rm si}} & {t \ge 3}  \\
\end{array}} \right.
\]
Se pide:
\begin{enumerate}
\item Calcular el valor de las constantes $a$ y $b$ sabiendo que la concentración en el instante inicial era de 50 mg/K, y que la función es continua en $t=3$ días.
\item Teniendo en cuenta los valores de $a$ y $b$ obtenidos en el apartado anterior, ¿puede la función concentración ser derivable en $t=3$?
\item ¿Qué valor tendrá la concentración del nutriente transcurridos muchos días desde el abonado?
\end{enumerate}
}


\newproblem{lim-8}{gen}{*}
%ENUNCIADO
{Calcular $\displaystyle \lim\limits_{x\rightarrow 1}\left(\frac{1}{1-x}-\frac{3}{1-x^{3}}\right)$.
}
%SOLUCIÓN
{-1.
}
%RESOLUCIÓN
{Si sustituimos $x$ por $1$ en la expresión, obtenemos una indeterminación del tipo $\infty -\infty .$ Para resolverla sacamos común denominador y realizamos la resta: 
\[
\lim\limits_{x\rightarrow 1}\left( \frac{1}{1-x}-\frac{3}{1-x^{3}}\right) = \lim\limits_{x\rightarrow 1}\frac{-x^{3}+3x-2}{x^{4}-x^{3}-x+1}.
\]
Si ahora susitutimos $x$ por $1$ en la expresión obtenemos una indeterminación del tipo $\frac{0}{0}$. Para resolver esta indeterminación aplicamos la regla de L'Hôpital dos veces: 
\[
\lim\limits_{x\rightarrow 1}\frac{-x^{3}+3x-2}{x^{4}-x^{3}-x+1} \stackrel{\text{L}^{\prime }\text{H\^{o}pital}}{=} \lim\limits_{x\rightarrow 1}\frac{-3x^{2}+3}{4x^{3}-3x^{2}-1} \stackrel{\text{L}^{\prime }\text{Hôpital}}{=} \lim\limits_{x\rightarrow 1}\frac{-6x}{12x^{2}-6x}=\frac{-6}{6}=-1.
\]
}